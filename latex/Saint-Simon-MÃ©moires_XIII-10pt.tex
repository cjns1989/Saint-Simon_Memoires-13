\PassOptionsToPackage{unicode=true}{hyperref} % options for packages loaded elsewhere
\PassOptionsToPackage{hyphens}{url}
%
\documentclass[oneside,10pt,french,]{extbook} % cjns1989 - 27112019 - added the oneside option: so that the text jumps left & right when reading on a tablet/ereader
\usepackage{lmodern}
\usepackage{amssymb,amsmath}
\usepackage{ifxetex,ifluatex}
\usepackage{fixltx2e} % provides \textsubscript
\ifnum 0\ifxetex 1\fi\ifluatex 1\fi=0 % if pdftex
  \usepackage[T1]{fontenc}
  \usepackage[utf8]{inputenc}
  \usepackage{textcomp} % provides euro and other symbols
\else % if luatex or xelatex
  \usepackage{unicode-math}
  \defaultfontfeatures{Ligatures=TeX,Scale=MatchLowercase}
%   \setmainfont[]{EBGaramond-Regular}
    \setmainfont[Numbers={OldStyle,Proportional}]{EBGaramond-Regular}      % cjns1989 - 20191129 - old style numbers 
\fi
% use upquote if available, for straight quotes in verbatim environments
\IfFileExists{upquote.sty}{\usepackage{upquote}}{}
% use microtype if available
\IfFileExists{microtype.sty}{%
\usepackage[]{microtype}
\UseMicrotypeSet[protrusion]{basicmath} % disable protrusion for tt fonts
}{}
\usepackage{hyperref}
\hypersetup{
            pdftitle={SAINT-SIMON},
            pdfauthor={Mémoires XIII},
            pdfborder={0 0 0},
            breaklinks=true}
\urlstyle{same}  % don't use monospace font for urls
\usepackage[papersize={4.80 in, 6.40  in},left=.5 in,right=.5 in]{geometry}
\setlength{\emergencystretch}{3em}  % prevent overfull lines
\providecommand{\tightlist}{%
  \setlength{\itemsep}{0pt}\setlength{\parskip}{0pt}}
\setcounter{secnumdepth}{0}

% set default figure placement to htbp
\makeatletter
\def\fps@figure{htbp}
\makeatother

\usepackage{ragged2e}
\usepackage{epigraph}
\renewcommand{\textflush}{flushepinormal}

\usepackage{indentfirst}
\usepackage{relsize}

\usepackage{fancyhdr}
\pagestyle{fancy}
\fancyhf{}
\fancyhead[R]{\thepage}
\renewcommand{\headrulewidth}{0pt}
\usepackage{quoting}
\usepackage{ragged2e}

\newlength\mylen
\settowidth\mylen{...................}

\usepackage{stackengine}
\usepackage{graphicx}
\def\asterism{\par\vspace{1em}{\centering\scalebox{.9}{%
  \stackon[-0.6pt]{\bfseries*~*}{\bfseries*}}\par}\vspace{.8em}\par}

\usepackage{titlesec}
\titleformat{\chapter}[display]
  {\normalfont\bfseries\filcenter}{}{0pt}{\Large}
\titleformat{\section}[display]
  {\normalfont\bfseries\filcenter}{}{0pt}{\Large}
\titleformat{\subsection}[display]
  {\normalfont\bfseries\filcenter}{}{0pt}{\Large}

\setcounter{secnumdepth}{1}
\ifnum 0\ifxetex 1\fi\ifluatex 1\fi=0 % if pdftex
  \usepackage[shorthands=off,main=french]{babel}
\else
  % load polyglossia as late as possible as it *could* call bidi if RTL lang (e.g. Hebrew or Arabic)
%   \usepackage{polyglossia}
%   \setmainlanguage[]{french}
%   \usepackage[french]{babel} % cjns1989 - 1.43 version of polyglossia on this system does not allow disabling the autospacing feature
\fi

\title{SAINT-SIMON}
\author{Mémoires XIII}
\date{}

\begin{document}
\maketitle

\hypertarget{chapitre-premier.}{%
\chapter{CHAPITRE PREMIER.}\label{chapitre-premier.}}

~

{\textsc{Amours du roi.}} {\textsc{- Belle inconnue très connue.}}
{\textsc{- M\textsuperscript{me} Scarron\,; ses premiers temps.}}
{\textsc{- Extraction, famille et fortune du maréchal d'Albret.}}
{\textsc{- M\textsuperscript{me} Scarron élève en secret M. du Maine et
M\textsuperscript{me} la Duchesse, et {[}eux{]} reconnus et à la cour,
demeure leur gouvernante.}} {\textsc{- Le roi ne la peut souffrir et
s'en explique très fortement.}} {\textsc{- Elle prend le nom de
Maintenon en acquérant la terre.}} {\textsc{- Le roi rapproché de
M\textsuperscript{me} de Maintenon, qui enfin supplante
M\textsuperscript{me} de Montespan.}} {\textsc{- Le roi épouse
M\textsuperscript{me} de Maintenon.}} {\textsc{- M\textsuperscript{me}
de Maintenon toute-puissante quitte les armes de son premier mari, à
l'exemple de M\textsuperscript{me} de Montespan et de
M\textsuperscript{me} de Thianges.}}

~

De tels excès de puissance, et si mal entendus, faut-il passer à
d'autres plus conformes à la nature, mais qui, en leur genre, furent
bien plus funestes\,? ce sont les amours du roi. Leur scandale a rempli
l'Europe, a confondu la France, a ébranlé l'État, a sans doute attiré
les malédictions sous le poids desquelles il s'est vu si imminemment
près du dernier précipice, et a réduit sa postérité légitime à un filet
unique de son extinction en France. Ce sont des maux qui se sont tournés
en fléaux de tout genre, et qui se feront sentir longtemps. Louis XIV,
dans sa jeunesse, plus fait pour les amours qu'aucun de ses sujets,
lassé de voltiger et de cueillir des faveurs passagères, se fixa enfin à
La Vallière. On en sait les progrès et les fruits.

M\textsuperscript{me} de Montespan fut celle dont la rare beauté le
toucha ensuite, même pendant le règne de M\textsuperscript{me} de La
Vallière. Elle s'en aperçut bientôt, elle pressa vainement son mari de
l'emmener en Guyenne\,; une folle confiance ne voulut pas l'écouter.
Elle lui parlait alors de bonne foi. À la fin le roi en fut écouté, et
l'enleva à son mari, avec cet épouvantable fracas qui retentit avec
horreur chez toutes les nations, et qui donna au monde le spectacle
nouveau de deux maîtresses à la fois. Il les promena aux frontières, aux
camps, des moments aux armées, toutes deux dans le carrosse de la reine.
Les peuples accourant de toutes parts se montraient les trois reines, et
se demandaient avec simplicité les uns aux autres s'ils les avaient
vues.

À la fin M\textsuperscript{me} de Montespan triompha, et disposa seule
du maître et de sa cour, avec un éclat qui n'eut plus de voile\,; et
pour qu'il ne manquât rien à la licence publique de cette vie, M. de
Montespan, pour en avoir voulu prendre, fut mis à la Bastille, puis
relégué en Guyenne, et sa femme eut de la comtesse de
Soissons\footnote{Voy t. VIII, p.~450, note sur Olympe Mancini.}, forcée
par sa disgrâce, la démission de la charge créée pour elle de
surintendante de la maison de la reine, à laquelle on supposa le
tabouret attaché, parce qu'ayant un mari elle ne pouvait être faite
duchesse.

On vit après sortir de son cloître de Fontevrault la reine des abbesses,
qui, chargée de son voile et de ses vœux, avec plus d'esprit et de
beauté encore que M\textsuperscript{me} de Montespan sa soeur, vint
jouir de la gloire de cette Niquée\footnote{M\textsuperscript{me} de
  Sévigné (lettre du 11 juin 1677) a employé cette locution (\emph{la
  gloire de Niquée}) à l'occasion de M\textsuperscript{me} de Montespan
  et de sa sœur\,: \emph{se trouvant en elle-même la gloire de Niquée}.
  Voy. encore la lettre du 29 juillet 1676. --- Niquée est une héroïne
  du roman alors célèbre de l'\emph{Amadis des Gaules}. Les précédents
  édition de Saint-Simon ont supprimé ce nom et remplacé \emph{la gloire
  de cette Niquée} par \emph{sa gloire}.}, et être de tous les
particuliers du roi les plus charmants, par l'esprit et par les fêtes,
avec M\textsuperscript{me} de Thianges, son autre sœur, et l'élixir le
plus trayé de toutes les dames de la cour.

Les grossesses et les couches furent publiques. La cour de
M\textsuperscript{me} de Montespan devint le centre de la cour, des
plaisirs, de la fortune, de l'espérance et de la terreur des ministres
et des généraux d'armée, et l'humiliation de toute la France. Ce fut
aussi le centre de l'esprit, et d'un tour si particulier, si délicat, si
fin, mais toujours si naturel et si agréable, qu'il se faisait
distinguer à son caractère unique.

C'était celui de ces trois soeurs, qui toutes trois en avaient
infiniment, et avaient l'art d'en donner aux autres. On sent encore avec
plaisir ce tour charmant et simple dans ce qui reste de personnes
qu'elles ont élevées chez elles et qu'elles s'étaient attachées\,; entre
mille autres on les distinguerait dans les conversations les plus
communes.

M\textsuperscript{me} de Fontevrault était celle des trois qui en avait
le plus\,; c'était peut-être aussi la plus belle. Elle y joignait un
savoir rare et fort étendu\,: elle savait bien la théologie et les
Pères, elle était versée dans l'Écriture, elle possédait les langues
savantes, elle parlait à enlever quand elle traitait quelque matière.
Hors de cela l'esprit ne se pouvait cacher, mais on ne se doutait pas
qu'elle sût rien de plus que le commun de son sexe. Elle excellait en
tous genres d'écrire. Elle avait un don tout particulier pour le
gouvernement et pour se faire adorer de tout son ordre, en le tenant
toutefois dans la plus exacte régularité. Quoiqu'elle eût été faite
religieuse plus que très cavalièrement, la sienne était pareille dans
son abbaye. Ses séjours à la cour, où elle ne sortait point de chez ses
soeurs, ne donnèrent jamais d'atteinte à sa réputation que par l'étrange
singularité de voir un tel habit partager une faveur de cette nature\,;
et si la bienséance eût pu y être en soi, il se pouvait dire que, dans
cette cour même, elle ne s'en serait jamais écartée.

M\textsuperscript{me} de Thianges dominait ses deux sœurs, et le roi
même qu'elle amusait plus qu'elles. Tant qu'elle vécut, elle le domina,
et conserva, même après l'expulsion de M\textsuperscript{me} de
Montespan hors de la cour, les plus grandes privances et des
distinctions uniques.

Pour M\textsuperscript{me} de Montespan, elle était méchante,
capricieuse, avait beaucoup d'humeur, et une hauteur en tout dans les
nues dont personne n'était exempt, le roi aussi peu que tout autre. Les
courtisans évitaient de passer sous ses fenêtres, surtout quand le roi y
était avec elle. Ils disaient que c'était passer par les armes, et ce
mot passa en proverbe à la cour. Il est vrai qu'elle n'épargnait
personne, très souvent sans autre dessein que de divertir le roi\,; et
comme elle avait infiniment d'esprit, de tour et de plaisanterie fine,
rien n'était plus dangereux que les ridicules qu'elle donnait mieux que
personne. Avec cela elle aimait sa maison et ses parents, et ne laissait
pas de bien servir les gens pour qui elle avait pris de l'amitié. La
reine supportait avec peine sa hauteur avec elle, bien différente des
ménagements continuels et des respects de la duchesse de La Vallière
qu'elle aima toujours, au lieu que de celle-ci il lui échappait souvent
de dire\,: «\,Cette pute me fera mourir.\,» On a vu en son temps la
retraite, l'austère pénitence et la pieuse fin de M\textsuperscript{me}
de Montespan.

Pendant son règne elle ne laissa pas d'avoir des jalousies.
M\textsuperscript{lle} de Fontange plut assez au roi pour devenir
maîtresse en titre. Quelque étrange que fût ce doublet, il n'était pas
nouveau. On l'avait vu de M\textsuperscript{me} de La Vallière et de
M\textsuperscript{me} de Montespan, à qui celle-ci ne fit que rendre ce
qu'elle avait prêté à l'autre. Mais M\textsuperscript{me} de Fontange ne
fut pas si heureuse ni pour le vice, ni pour la fortune, ni pour la
pénitence. Sa beauté la soutint un temps, mais son esprit n'y répondit
en rien. Il en fallait au roi pour l'amuser et le tenir. Avec cela il
n'eut pas le loisir de s'en dégoûter tout a fait. Une mort prompte, qui
ne laissa pas de surprendre, finit en bref ces nouvelles amours. Presque
tous ne furent que passades.

Un seul subsista longtemps, et se convertit en affection jusqu'à la fin
de la vie de la belle qui sut en tirer les plus prodigieux avantages
jusqu'au tombeau, et en laisser à ses deux fils l'abominable et
magnifique héritage, qu'ils surent bien faire valoir. L'infâme politique
du mari, qui a un nom propre en Espagne qui veut dire cocu volontaire et
ne s'y pardonne jamais, souffrit volontiers cet amour, et en recueillit
des fruits immenses en se confinant à Paris, servant à l'armée, n'allant
presque point à la cour, faisant obscurément les fonds, et distribuant
tous les avantages que de concert avec lui sa belle moitié en tirait.
C'était la maréchale de Rochefort chez qui elle allait attendre l'heure
du berger, laquelle l'y conduisait, et qui me l'a conté plus d'une fois,
avec des contre-temps qui lui arrivèrent, mais qui ne firent obstacle à
rien, et ne venaient point du mari, qui était au fond de sa maison à
Paris, qui, sachant et conduisant tout, ignorait tout avec le plus grand
soin, et changea depuis son étroite maison de la place Royale pour le
palais des Guise, dont ils ne pourraient reconnaître l'étendue, ni la
somptuosité qu'il a prises depuis entre ses mains et en celles de ses
deux fils. La même politique continua le mystère de cet amour, qui ne le
demeura que de nom, et tout au plus en très fine écorce. Le mystère le
fit durer, l'art de s'y conduire gagna les plus intéressées, et en bâtit
la plus rapide et la plus prodigieuse fortune. Le même art le soutint
toujours croissant, et sut, quand il en fut encore temps, le tourner en
amitié et en considération la plus distinguée.

Il mit les enfants de cette belle, qui était pourtant rousse, en
situation de s'élever et de s'enrichir eux et les leurs de plus en plus,
même après elle, et de parvenir à un comble de tout, dont {[}après{]}
eux jouit avec éclat la troisième génération aujourd'hui dans toute son
étendue, et qui a mis les plus obscurs par eux-mêmes et les plus
ténébreux, mais de leur nom, en splendeur inhérente. C'est savoir tirer
plus que très grand parti\,: la femme de sa beauté\,; le mari de sa
politique et de son infamie\,; les enfants de tous les moyens mis en
main par de tels parents, mais toujours comme les fils de la belle.

Une autre tira beaucoup aussi toute sa vie de la même conduite, mais ni
la beauté, ni l'art, ni la position de cette belle, ni de son camard et
bouffon de mari, ne permit à celle-ci ni la durée, ni la continuité, ni
rien de l'éclat où l'autre parvint et se maintint, et qu'elle fit passer
à ses enfants, petits-enfants, et en gros à tout leur nom. Celle-ci
n'avait qu'à vouloir. Quoique le commerce fût fini depuis très
longtemps, et que les ménagements extérieurs fussent extrêmes, on
connaissait son pouvoir à la cour, tout y était en respect devant elle.
Ministres, princes du sang, rien ne résistait à ses volontés. Ses
billets allaient droit au roi, et les réponses toujours à l'instant du
roi à elle, sans que personne s'en aperçût. Si très rarement, par cette
commodité unique d'écriture, elle avait à parler au roi, ce qu'elle
évitait autant que cela était possible, elle était admise à l'instant
qu'elle le voulait. C'était toujours à des heures publiques, mais dans
le petit cabinet du roi, qui était et est encore celui du conseil, tous
deux assis au fond, mais les portes des deux côtés absolument ouvertes,
affectation qui ne se pratiquait jamais que lorsqu'elle était avec le
roi, et la pièce publique contiguë à ce cabinet pleine de tous les
courtisans. Si quelquefois elle ne voulait dire qu'un mot, c'était
debout à la porte, en dehors du même cabinet, et devant tout le monde
qui, aux manières du roi de l'aborder, de l'écouter, de la quitter,
n'avait pas peine à remarquer jusque dans les derniers temps de sa vie,
qui finit plusieurs années avant celle du roi, qu'elle ne lui était pas
indifférente. Elle fut belle jusqu'à la fin. Une fois en trois ans un
court voyage à Marly, jamais d'aucun particulier avec le roi, même avec
d'autres dames\,; l'unisson soigneusement gardé avec tout le reste de la
cour. Elle y était presque toujours, et souvent au souper du roi, où il
ne la distingua jamais en rien. Telle était la convention avec
M\textsuperscript{me} de Maintenon, qui de son côté contribua en
récompense à tout ce qu'elle put désirer. Le mari, qui l'a survécue de
quelques années, presque jamais à la cour, et des moments, vivait obscur
à Paris, enterré dans le soin de ses affaires domestiques qu'il
entendait parfaitement, s'applaudissant du bon sens qui, de concert avec
sa femme, l'avait porté à tant de richesses, d'établissements et de
grandeurs, sous les rideaux de gaze qui demeurèrent rideaux, mais qui ne
furent rien moins qu'impénétrables.

Il ne faut pas oublier la belle Ludre\footnote{Voy., sur cette dame, la
  lettre de M\textsuperscript{me} de Sévigné, du 11 juin 1677 et les
  lettres suivantes.}, demoiselle de Lorraine, fille d'honneur de
Madame, qui fut aimée un moment à découvert. Mais cet amour passa avec
la rapidité d'un éclair, et l'amour de M\textsuperscript{me} de
Montespan demeura le triomphant.

Il faut passer à un autre genre d'amour, qui n'étonna pas moins toutes
les nations que celui-ci les avait scandalisées, et que le roi emporta
tout entier au tombeau. À ce peu de mots qui ne reconnaîtrait la célèbre
Françoise d'Aubigné, marquise de Maintenon, dont le règne permanent n'a
pas duré moins de trente-deux ans. Née dans les îles de l'Amérique où
son père, peut-être gentilhomme, était allé avec sa mère chercher du
pain, et que l'obscurité y a étouffés, revenue seule et au hasard en
France, abordée à la Rochelle, recueillie au voisinage par pitié chez
M\textsuperscript{me} de Neuillant, mère de la maréchale-duchesse de
Navailles, réduite par sa pauvreté et par l'avarice de cette vieille
dame à garder les clefs de son grenier et à voir mesurer tous les jours
l'avoine à ses chevaux\,; venue à Paris à sa suite, jeune, adroite,
spirituelle et belle, sans pain et sans parents, d'heureux hasards la
firent connaître au fameux Scarron. Il la trouva aimable, ses amis
peut-être encore plus. Elle crut faire la plus grande fortune, et la
plus inespérable d'épouser ce joyeux et savant cul-de-jatte, et des gens
qui avaient peut-être plus besoin de femme que lui l'entêtèrent de faire
ce mariage, et vinrent à bout de lui persuader de tirer par là de la
misère cette charmante malheureuse.

Le mariage se fit, la nouvelle épouse plut à toutes les compagnies qui
allaient chez Scarron. Il la voyait fort bonne, et en tous genres\,;
c'était la mode d'aller chez lui, gens d'esprit, gens de la cour et de
la ville, et ce qu'il y avait de meilleur et de plus distingué, qu'il
n'était pas en état d'aller chercher hors de chez lui, et que les
charmes de son esprit, de son savoir, de son imagination, de cette
gaieté incomparable parmi ses maux, et toujours nouvelle, cette rare
fécondité, et la plaisanterie du meilleur goût qu'on admire encore dans
ses ouvrages, attiraient continuellement chez lui.

M\textsuperscript{me} Scarron fit donc là des connaissances de toutes
les sortes qui pourtant, à la mort de son mari, ne l'empêchèrent pas
d'être réduite à la charité de la paroisse de Saint-Eustache. Elle y
prit une chambre pour elle et pour une servante dans une montée, où elle
vécut très à l'étroit. Ses appas élargirent peu à peu ce mal-être.
Villars, père du maréchal\,; Beuvron, père d'Harcourt\,; les trois
Villarceaux qui demeurèrent les trois tenants\,; bien d'autres
l'entretinrent\footnote{Voy. l'\emph{Histoire du M\textsuperscript{me}
  de Maintenon}, par M. de Noailles, et les \emph{Œuvres de
  M\textsuperscript{me} de Maintenon}, publiées par M. Théoph. Lavallée.}.

Cela la remit à flot, et peu à peu l'introduisit à l'hôtel d'Albret, par
là à l'hôtel de Richelieu et ailleurs\,; ainsi de l'un à l'autre. Dans
ces maisons, M\textsuperscript{me} Scarron n'était rien moins que sur le
pied de compagnie. Elle y était à tout faire, tantôt à demander du bois,
tantôt si on servirait bientôt\,; une autre fois si le carrosse de
celui-ci, ou de celle-là était revenu\,; et ainsi de mille petites
commissions dont l'usage des sonnettes, introduit longtemps depuis, a
ôté l'importunité.

C'est dans ces maisons, principalement à l'hôtel de Richelieu, beaucoup
plus encore à l'hôtel d'Albret où le maréchal d'Albret tenait un fort
grand état, où M\textsuperscript{me} Scarron fit la plupart de ses
connaissances, dont les unes lui servirent tant, et les autres leur
devinrent si utiles\footnote{Ce membre de phrase veut dire que \emph{les
  autres amis de M\textsuperscript{me} de Maintenon profitèrent de leurs
  relations avec elle pour leur fortune}.}. Les maréchaux de Villars et
d'Harcourt par leurs pères, et avant eux, Villars, père du maréchal, en
firent leur fortune\,; la duchesse d'Arpajon, sœur de Beuvron, en fut,
sans l'avoir pu imaginer, dame d'honneur de M\textsuperscript{me} la
dauphine de Bavière, à la mort de la duchesse de Richelieu, que la même
raison avait faite aussi dame d'honneur de la reine, puis par confiance
de M\textsuperscript{me} la dauphine de Bavière, et le duc de Richelieu
chevalier d'honneur pour rien, qui en eut de Dangeau cinq cent mille
livres, à qui cette charge fit la fortune. La princesse d'Harcourt,
fille de Brancas, si connu par son esprit et par ses rares distractions,
qui avait été bien avec elle\,; Villarceaux et Montchevreuil, chevaliers
de l'ordre tous deux, au premier desquels son père fit passer à
trente-cinq ans le collier qui lui était destiné, et nombre d'autres se
sentirent grandement de ces premiers temps. Mais avant d'aller plus
loin, il faut éclaircir le maréchal d'Albret en peu de mots.

Charles II d'Albret, comte de Dreux, vicomte de Tartas, fils de Charles
Ier, connétable de France, eut d'Anne d'Armagnac, pour cinquième et
dernier fils, Gilles d'Albret, seigneur de Castelmoron, mort sans
enfants d'Anne d'Aiguillon en 1479, qui de Jean Le Tellier laissa un
bâtard nommé Étienne qui fut légitimé par François Ier en 1527 et
sénéchal du pays de Foix. De l'héritière de Miossens il laissa
Jean-Baptiste de Miossens, qui fut lieutenant général d'Henri d'Albret,
roi de Navarre, en ses pays et seigneuries, et qui de Suzanne, fille de
Pierre, seigneur de Busset, bâtard de Bourbon, évêque de Liège, laquelle
fut gouvernante de notre roi Henri IV, laissa Henri-Baptiste de
Miossens, chevalier du Saint-Esprit en 1595, et gouverneur et sénéchal
de Navarre et Béarn, qui d'Antoinette de Pons, fille du comte de
Marennes, chevalier du Saint-Esprit, et sœur de la fameuse marquise de
Guercheville, mère du duc de Liancourt, eut Henri, comte de Miossens,
qui d'Anne de Pardaillan, sœur du père de M. de Montespan, mari de la
maîtresse de Louis XIV, eut trois fils et plusieurs filles. L'aîné fut
le premier mari d'Anne Poussard, qui se remaria au duc de Richelieu, et
mourut dame d'honneur de M\textsuperscript{me} la dauphine de Bavière,
sans enfants du duc de Richelieu, mais elle avait eu un fils de son
premier mari. Le second fut le maréchal d'Albret\,; le troisième, aussi
comte de Miossens, tué en duel en 1672 par Saint-Léger-Corbon, sans
enfants.

Le maréchal d'Albret, fort dans le grand monde et les intrigues de la
cour, eut la compagnie des gens d'armes de la garde, et fut chargé par
le cardinal Mazarin de la conduite de M. le Prince, M. le prince de
Conti et M. de Longueville, du Palais-Royal, où ils furent arrêtés, à
Vincennes, moyennant la promesse d'un bâton de maréchal de France, qu'il
n'eut pourtant qu'à force de menaces en 1653. Il avait été fait
chevalier du Saint-Esprit en 1661, et il eut le gouvernement de Guyenne
à la fin de 1670. Sans avoir beaucoup servi, et jamais en chef, ce fut
un homme qui par son esprit, son adresse, sa hardiesse et sa
magnificence se fit toujours fort compter. Il n'avait qu'une fille
unique de la fille de Guénégaud, trésorier de l'épargne, frère du
secrétaire d'État, qu'il avait épousée. Il la maria au fils unique de
son frère aîné, et de la duchesse de Richelieu, lequel fut tué en
galanterie, et sans enfants, en 1678\,; et sa veuve, qui était dame du
palais de la reine, fut depuis la première femme du comte de Marsan,
dont elle s'amouracha, et qui lui donna tout son bien.

Le maréchal d'Albret et M. et M\textsuperscript{me} de Richelieu
vécurent toujours dans l'amitié la plus intime. Il vécut de même avec M.
de Montespan, son cousin germain, et M\textsuperscript{me} de Montespan.
Mais quand celle-ci fut maîtresse, il devint son conseil, et abandonna
pour elle M. de Montespan, par où il se maintint eu grand crédit jusqu'à
sa mort, qui arriva à Bordeaux le 3 septembre 1676, à soixante-deux ans,
où il n'y avait pas longtemps qu'il était allé.

Il avait, comme on l'a vu ailleurs, marié M\textsuperscript{lle}s de
Pons, ses nièces à la mode de Bretagne\,: l`une a son frère cadet, tué
en duel\,; l'autre fort belle à Heudicourt, à qui il fit acheter de
Saint-Herem la charge de grand louvetier pour le décrasser, et pour que
sa femme pût paraître à la cour où on l'a vue vivre longtemps, et mourir
dans la faveur et les privances de M\textsuperscript{me} de Maintenon et
du roi, et faire fort étrangement dame du palais M\textsuperscript{me}
de Montgon, sa fille, au mariage de M\textsuperscript{me} la duchesse de
Bourgogne, laquelle avait été toute petite élevée avec M. du Maine et
M\textsuperscript{me} la Duchesse, et logée avec eux, lorsqu'ils étaient
cachés à Paris sous M\textsuperscript{me} Scarron, leur gouvernante, qui
l'avait prise pour en soulager M\textsuperscript{me} d'Heudicourt, sa
bonne amie, qui, fille et mariée, ne bougeait de l'hôtel d'Albret où
M\textsuperscript{me} Scarron l'avait fort courtisée, et où leur liaison
intime s'était faite. Revenons à cette heure à M\textsuperscript{me}
Scarron.

Elle dut à la proche parenté du maréchal d`Albret et de M. de Montespan
l'introduction décisive à l'incroyable fortune qu'elle fit quatorze ou
quinze ans après. M. et M\textsuperscript{me} de Montespan ne bougeaient
de chez le maréchal d'Albret qui tenait à Paris la plus grande et la
meilleure maison, où abondait la compagnie de la cour et de la ville la
plus distinguée et la plus choisie. Les respects, les soins de plaire,
l'esprit et les agréments de M\textsuperscript{me} Scarron réussirent
fort auprès de M\textsuperscript{me} de Montespan. Elle prit de l'amitié
pour elle, et quand elle eut ses premiers enfants du roi, M. du Maine et
M\textsuperscript{me} la Duchesse qu'on voulut cacher, elle lui proposa
de les confier à M\textsuperscript{me} Scarron, à qui on donna une
maison au Marais pour y loger avec eux, et de quoi les entretenir et les
élever dans le dernier secret. Dans les suites, ces enfants furent
amenés à M\textsuperscript{me} de Montespan, puis montrés au roi, et de
là peu à peu tirés du secret, et avoués. Leur gouvernante, fixée avec
eux à la cour, y plut de plus en plus à M\textsuperscript{me} de
Montespan, qui lui fit donner par le roi à diverses reprises. Lui, au
contraire, ne la pouvait souffrir\,; ce qu'il lui donnait quelquefois,
et toujours peu, n'était que par excès de complaisance, et avec un
regret qu'il ne cachait pas.

La terre de Maintenon étant tombée en vente, la proximité de Versailles
en tenta si bien M\textsuperscript{me} de Montespan, pour
M\textsuperscript{me} Scarron, qu'elle ne laissa point de repos au roi
qu'elle n'en eût tiré de quoi la faire acheter à cette femme, qui prit
alors le nom de Maintenon, ou fort peu de temps après. Elle obtint aussi
de quoi en raccommoder le château, et attaqua le roi encore pour donner
de quoi rajuster le jardin, car MM. d'Angennes y avaient tout laissé
ruiner.

C'était à sa toilette où cela se passait, et où le seul capitaine des
gardes en quartier suivait le roi. C'était M. le maréchal de Lorges,
homme le plus vrai qui fut jamais, et qui m'a souvent conté la scène
dont il fut témoin ce jour-là. Le roi fit d'abord la sourde oreille,
puis refusa. Enfin impatienté de ce que M\textsuperscript{me} de
Montespan ne démordait point et insistait toujours, il se fâcha, lui dit
qu'il n'avait déjà que trop fait pour cette créature, qu'il ne
comprenait pas la fantaisie de M\textsuperscript{me} de Montespan pour
elle, et son opiniâtreté à la garder, après tant de fois qu'il l'avait
priée de s'en défaire\,; qu'il avouait pour lui qu'elle lui était
insupportable, et que pourvu qu'on lui promit qu'il ne la verrait plus,
et qu'on ne lui en parlerait jamais, il donnerait encore, quoique, pour
en dire la vérité, il n'eut déjà que beaucoup trop donné pour une
créature de cette espèce. Jamais M. le maréchal de Lorges n'a oublié ces
propres paroles\,; et à moi et à d'autres il les a toujours rapportées
précises et dans le même ordre, tant il en fut frappé alors, et bien
plus à tout ce qu'il vit depuis de si étonnant et de si contradictoire.
M\textsuperscript{me} de Montespan se tut bien court, et bien en peine
d'avoir trop pressé le roi.

M. du Maine était extrêmement boiteux. On disait que c'était d'être
tombé d'entre les bras d'une nourrice. Tout ce qu'on lui fit n'ayant pas
réussi, on prit le parti de l'envoyer chez divers artistes en Flandre et
ailleurs dans le royaume, puis aux eaux, entre autres à Barèges. Les
lettres que la gouvernante écrivait à M\textsuperscript{me} de
Montespan, pour lui rendre compte de ces voyages, étaient montrées au
roi. Il les trouva bien écrites, il les goûta, et les dernières
commencèrent à diminuer son éloignement.

Les humeurs de M\textsuperscript{me} de Montespan achevèrent l'ouvrage.
Elle en avait beaucoup, elle s'était accoutumée à ne s'en pas
contraindre. Le roi en était l'objet plus souvent que personne\,; il en
était encore amoureux, mais il en souffrait. M\textsuperscript{me} de
Maintenon le reprochait à M\textsuperscript{me} de Montespan, qui lui en
rendit de bons offices auprès du roi. Ces soins d'apaiser sa maîtresse
lui revinrent aussi d'ailleurs, et l'accoutumèrent à parler quelquefois
à M\textsuperscript{me} de Maintenon, à s'ouvrir à elle de ce qu'il
désirait qu'elle fît auprès de M\textsuperscript{me} de Montespan, enfin
à lui conter ses chagrins contre elle, et à la consulter là-dessus.

Admise ainsi peu à peu dans l'intime confidence, et sans milieu, de
l'amant et de la maîtresse, et par le roi même, l'adroite suivante sut
la cultiver, et fit si bien par son industrie, que peu à peu elle
supplanta M\textsuperscript{me} de Montespan, qui s'aperçut trop tard
qu'elle lui était devenue nécessaire. Parvenue à ce point,
M\textsuperscript{me} de Maintenon fit à son tour ses plaintes au roi de
tout ce qu'elle avait à souffrir d'une maîtresse qui l'épargnait si peu
lui-même, et à force de se plaindre l'un à l'autre de
M\textsuperscript{me} de Montespan, celle-ci en prit tout à fait la
place, et se la sut bien assurer.

La fortune, pour n'oser nommer ici la Providence, qui préparait au plus
superbe des rois l'humiliation la plus profonde, la plus publique, la
plus durable, la plus inouïe, fortifia de plus en plus son goût pour
cette femme adroite et experte au métier, que les jalousies continuelles
de M\textsuperscript{me} de Montespan rendaient encore plus solide, par
les sorties fréquentes que son humeur aigrie lui faisait faire sans
ménagement sur le roi et sur elle, et c'est ce que M\textsuperscript{me}
de Sévigné sait peindre si joliment en énigmes, dans ses lettres à
M\textsuperscript{me} de Grignan, où elle l'entretient quelquefois de
ces mouvements de cour, parce que M\textsuperscript{me} de Maintenon
avait été à Paris assez de la société de M\textsuperscript{me} de
Sévigné, de M\textsuperscript{me} de Coulange, de M\textsuperscript{me}
de La Fayette, et qu'elle commençait à leur faire sentir son importance.
On y voit aussi dans le même goût des traits charmants sur la faveur
voilée, mais brillante, de M\textsuperscript{me} de Soubise.

Cette même Providence, maîtresse absolue des temps et des événements,
les disposa encore, en sorte que la reine vécut assez pour laisser
porter ce goût à son comble, et point assez pour le laisser refroidir.
Le plus grand malheur qui soit donc arrivé au roi, et les suites doivent
faire ajouter à l'État, fut la perte si brusque de la reine, par
l'ignorance profonde et l'opiniâtreté du premier médecin Daquin, au plus
fort de ce nouvel attachement enté sur le dégoût de la maîtresse, dont
les humeurs étaient devenues insupportables, et que nulle politique
n'avait pu arrêter. Cette beauté impérieuse, accoutumée à dominer et à
être adorée, ne pouvait résister au désespoir toujours présent de la
décadence de son pouvoir\,; et ce qui la jetait hors de toute mesure,
c'était de ne pouvoir se dissimuler une rivale abjecte à qui elle avait
donné du pain, qui n'en avait encore que par elle, qui de plus, lui
devait cette affection qui devenait son bourreau, par l'avoir assez
aimée pour n'avoir pu se résoudre à la chasser tant de fois que le roi
l'en avait pressée, une rivale encore si au-dessous d'elle en beauté, et
plus âgée qu'elle de plusieurs années\,; sentir que c'était pour cette
suivante, pour ne pas dire servante, que le roi venait le plus chez
elle, qu'il n'y cherchait qu'elle, qu'il ne pouvait dissimuler son
malaise lorsqu'il ne l'y trouvait pas\,; et le plus souvent la quitter
elle, pour entretenir l'autre tête à tête\,; enfin avoir à tous moments
besoin d'elle pour attirer le roi, pour se raccommoder avec lui de leurs
querelles, pour en obtenir des grâces qu'elle lui demandait. Ce fut donc
dans des temps si propices à cette enchanteresse que le roi devint
libre.

Il passa les premiers jours à Saint-Cloud, chez Monsieur, d'où il alla à
Fontainebleau, où il passa tout l'automne. Ce fut là ou son goût, piqué
par l'absence, la lui fit trouver insupportable. À son retour on
prétend, car il faut distinguer le certain de ce qui ne l'est pas, on
prétend, dis-je, que le roi parla plus librement à M\textsuperscript{me}
de Maintenon, et qu'elle, osant essayer ses forces, se retrancha
habilement sur la dévotion, et sur la pruderie de son dernier état\,;
que le roi ne se rebuta point\,; qu'elle le prêcha et lui fit peur du
diable, et qu'elle ménagea son amour et sa conscience l'un par l'autre
avec un si grand art, qu'elle parvint à ce que nos yeux ont vu, et que
la postérité refusera de croire.

Mais ce qui est très certain, et bien vrai, c'est que quelque temps
après le retour du roi de Fontainebleau, et au milieu de l'hiver qui
suivit la mort de la reine, chose que la postérité aura peine à croire,
quoique parfaitement vraie et avérée, le P. de La Chaise, confesseur du
roi, dit la messe en pleine nuit dans un des cabinets du roi à
Versailles. Bontems, gouverneur de Versailles, premier valet de chambre
en quartier, et le plus confident des quatre, servit cette messe où ce
monarque et la Maintenon furent mariés, en présence d'Harlay, archevêque
de Paris, comme diocésain, de Louvois, qui tous deux avaient, comme on
l'a dit, tiré parole du roi qu'il ne déclarerait jamais ce mariage, et
de Montchevreuil, uniquement en troisième, parent, ami, et du même nom
de Mornay que Villarceaux, à qui autrefois il prêtait sa maison de
Montchevreuil tous les étés, sans en bouger lui-même avec sa femme, où
Villarceaux entretenait cette reine comme à Paris, et où il payait toute
la dépense, parce que son cousin était fort pauvre, et qu'il avait honte
de ce concubinage chez lui à Villarceaux, en présence de sa femme, dont
il respectait la patience et la vertu.

M\textsuperscript{me} de Maintenon, n'osant porter les armes d'un tel
époux, supprima celles de son premier mari, et ne porta plus que les
siennes seules, et sans cordelière, imitant à meilleur titre
M\textsuperscript{me} de Montespan depuis ses amours, et même
M\textsuperscript{me} de Thianges, qui du vivant de leurs maris
quittèrent leurs armes et leur livrée qu'elles ne reprirent jamais, et
portèrent toujours depuis celles de Rochechouart seules. On a vu, à
l'occasion de la mort du duc de Créqui, les prédictions étonnantes de
cette épouvantable fortune.

La satiété des noces ordinairement si fatale, et des noces de cette
espèce, ne fit que consolider la faveur de M\textsuperscript{me} de
Maintenon. Bientôt après elle éclata par l'appartement qui lui fut donné
à Versailles au haut du grand escalier, vis-à-vis de celui du roi, et de
plain-pied. Depuis ce moment, le roi y alla tous les jours de sa vie
passer plusieurs heures à Versailles, et en quelque lieu qu'il fût, où
elle fut toujours logée aussi proche de lui, et de plain-pied autant
qu'il fut possible.

Les suites, les succès, l'entière confiance, la rare dépendance, la
toute-puissance, l'adoration publique, universelle, les ministres, les
généraux d'armée, la famille royale la plus proche, tout en un mot à ses
pieds\,; tout bon et tout bien par elle, tout réprouvé sans elle\,; les
hommes, les affaires, les choses, les choix, les justices, les grâces,
la religion, tout sans exception en sa main, et le roi et l'État ses
victimes\,; quelle elle fut, cette fée incroyable, et comment elle
gouverna sans lacune, sans obstacle, sans nuage le plus léger, plus de
trente ans entiers, et même trente-deux\,; c'est l'incomparable
spectacle qu'il s'agit de se retracer, et qui a été celui de toute
l'Europe.

\hypertarget{chapitre-ii.}{%
\chapter{CHAPITRE II.}\label{chapitre-ii.}}

~

{\textsc{Caractère de M\textsuperscript{me} de Maintenon.}} {\textsc{-
Goût de direction.}} {\textsc{- Persécution du jansénisme.}} {\textsc{-
Antérieures dissipations des saints et savants solitaires de
Port-Royal.}} {\textsc{- Révocation de l'édit de Nantes.}} {\textsc{-
Établissement de Saint-Cyr.}} {\textsc{- Vues de M\textsuperscript{me}
de Maintenon, qui manque une seconde fois la déclaration de son
mariage.}} {\textsc{- M\textsuperscript{me} de Maintenon seconde dame
d'atours de la Dauphine de Bavière, qu'elle environne de personnes
toutes à elle, inutilement.}} {\textsc{- Malheurs et mort de cette
Dauphine.}} {\textsc{- Fénelon, archevêque de Cambrai, et Bossuet,
évêque de Meaux, consultés et contraires à la déclaration du mariage.}}
{\textsc{- Le premier achève d'être perdu.}} {\textsc{- Raisons qui
sauvent l'autre.}} {\textsc{- M\textsuperscript{me} de Montespan chassée
pour toujours de la cour.}} {\textsc{- Époque de l'union la plus intime
entre M\textsuperscript{me} de Maintenon et le duc du Maine.}}
{\textsc{- Crayon léger de celui-ci.}}

~

C'était une femme de beaucoup d'esprit, que les meilleures compagnies,
où elle avait d'abord été soufferte, et dont bientôt elle fit le
plaisir, avaient fort polie et ornée de la science du monde, et que la
galanterie avait achevé de tourner au plus agréable. Ses divers états
l'avaient rendue flatteuse, insinuante, complaisante, cherchant toujours
à plaire. Le besoin de l'intrigue, toutes celles qu'elle avait vues, en
plus d'un genre, et de beaucoup desquelles elle avait été, tant pour
elle-même que pour en servir d'autres, l'y avaient formée, et lui en
avaient donné le goût, l'habitude et toutes les adresses. Une grâce
incomparable à tout, un air d'aisance, et toutefois de retenue et de
respect, qui par sa longue bassesse lui était devenu naturel, aidaient
merveilleusement ses talents, avec un langage doux, juste, en bons
termes, et naturellement éloquent et court. Son beau temps, car elle
avait trois ou quatre ans plus que le roi, avait été celui des belles
conversations, de la belle galanterie, en un mot de ce qu'on appelait
les ruelles\,; lui en avait tellement donné l'esprit, qu'elle en retint
toujours le goût et la plus forte teinture. Le précieux et le guindé
ajouté à l'air de ce temps-là, qui en tenait un peu, s'était augmenté
par le vernis de l'importance, et s'accrut depuis par celui de la
dévotion, qui devint le caractère principal, et qui fit semblant
d'absorber tout le reste. Il lui était capital pour se maintenir où il
l'avait portée, et ne le fut pas moins pour gouverner. Ce dernier point
était son être\,; tout le reste y fut sacrifié sans réserve. La droiture
et la franchise étaient trop difficiles à accorder avec une telle vue,
et avec une telle fortune ensuite, pour imaginer qu'elle en retînt plus
que la parure. Elle n'était pas aussi tellement fausse que ce fût son
véritable goût, mais la nécessité lui en avait de longue main donné
l'habitude, et sa légèreté naturelle la faisait paraître au double de
fausseté plus qu'elle n'en avait.

Elle n'avait de suite en rien que par contrainte et par force. Son goût
était de voltiger en connaissances et en amis comme en amusements,
excepté quelques amis fidèles de l'ancien temps dont on a parlé, sur qui
elle ne varia point, et quelques nouveaux des derniers temps qui lui
étaient devenus nécessaires. À l'égard des amusements, elle ne les put
guère varier depuis qu'elle se vit reine. Son inégalité tomba en plein
sur le solide, et fit par là de grands maux. Aisément engouée, elle
l'était à l'excès\,; aussi facilement déprise, elle se dégoûtait de
même, et l'un et l'autre très souvent sans cause ni raison.

L'abjection et la détresse où elle avait si longtemps vécu lui avait
rétréci l'esprit, et avili le coeur et les sentiments. Elle pensait et
sentait si fort en petit, en toutes choses, qu'elle était toujours en
effet moins que M\textsuperscript{me} Scarron, et qu'en tout et partout
elle se retrouvait telle. Rien n'était si rebutant que cette bassesse
jointe à une situation si radieuse\,; rien aussi n'était à tout bien
empêchement si dirimant, comme rien de si dangereux que cette facilité à
changer d'amitié et de confiance.

Elle avait encore un autre appât trompeur. Pour peu qu'on pût être admis
à son audience, et qu'elle y trouvât quelque chose à son goût, elle se
répandait avec une ouverture qui surprenait, et qui ouvrait les plus
grandes espérances\,; dès la seconde, elle s'importunait, et devenait
sèche et laconique. On se creusait la tête pour démêler et la grâce et
la disgrâce, si subites toutes les deux\,; on y perdait son temps. La
légèreté en était la seule cause, et cette légèreté était telle qu'on ne
se la pouvait imaginer. Ce n'est pas que quelques-uns n'aient échappé à
cette vacillité si ordinaire, mais ces personnes n'ont été que des
exceptions, qui ont d'autant plus confirmé la règle qu'elles-mêmes ont
éprouvé force nuages dans leur faveur, et que, quelle qu'elle ait été,
c'est-à-dire depuis son dernier mariage, aucune ne l'a approchée qu'avec
précaution, et dans l'incertitude.

On peut juger des épines de sa cour, qui d'ailleurs était presque
inaccessible et par sa volonté et par le goût du roi, et encore par la
mécanique des temps et des heures, d'une cour qui toutefois opérait une
grande et intime partie de toutes choses, et qui presque toujours
influait sur tout le reste.

Elle eut la faiblesse d'être gouvernée par la confiance, plus encore par
les espèces de confessions, et d'en être la dupe par la clôture où elle
s'était renfermée. Elle eut aussi la maladie des directions, qui lui
emporta le peu de liberté dont elle pouvait jouir. Ce que Saint-Cyr lui
fit perdre de temps en ce genre est incroyable\,; ce que mille autres
couvents lui en coûtèrent ne l'est pas moins. Elle se croyait l'abbesse
universelle, surtout pour le spirituel, et de là entreprit des détails
de diocèses. C'étaient là ses occupations favorites. Elle se figurait
être une mère de l'Église. Elle en pesait les pasteurs du premier ordre,
les supérieurs de séminaires et de communautés, les monastères et les
filles qui les conduisaient, ou qui y étaient les principales. De là une
mer d'occupations frivoles, illusoires, pénibles, toujours trompeuses,
des lettres et des réponses à l'infini, des directions d'âmes choisies,
et toutes sortes de puérilités qui aboutissaient d'ordinaire à des
riens, quelquefois aussi à des choses importantes, et à de déplorables
méprises en décisions, en événements d'affaires, et en choix.

La dévotion qui l'avait couronnée, et par laquelle elle sut se
conserver, la jeta par art et par goût de régenter, qui se joignit à
celui de dominer, dans ces sortes d'occupations\,; et l'amour-propre,
qui n'y rencontrait jamais que des adulateurs, s'en nourrissait. Elle
trouva le roi qui se croyait apôtre, pour avoir toute sa vie persécuté
le jansénisme, ou ce qui lui était présenté comme tel. Ce champ parut
propre à M\textsuperscript{me} de Maintenon à repaître ce prince de son
zèle, et à s'introduire dans tout.

L'ignorance la plus grossière en tous genres dans laquelle on avait eu
grand soin d'élever le roi, et par divers intérêts de l'entretenir
ensuite, et de lui inculquer de bonne heure la défiance générale et
l'exacte clôture dans lesquelles il s'est barricadé sous la clef de ses
ministres, et, à d'autres égards, sous celle de son confesseur et de
ceux qu'il a eu intérêt de lui produire, lui avait fait prendre de bonne
heure la pernicieuse habitude de prendre parti sur parole dans les
questions de théologie, et entre les différentes écoles catholiques,
jusqu'à en faire sa propre affaire à Rome.

La reine mère, et le roi bien plus qu'elle dans les suites, séduits par
les jésuites, s'étaient laissé persuader par eux le contradictoire exact
et précis de la vérité\,: savoir que toute autre école que la leur en
voulait à l'autorité royale, et n'avait qu'un esprit d'indépendance et
républicain. Le roi là-dessus, ni sur bien d'autres choses, n'en savait
pas plus qu'un enfant. Les jésuites n'ignoraient pas à qui ils avaient
affaire. Ils étaient en possession d'être les confesseurs du roi, et les
distributeurs des bénéfices dont ils avaient la feuille\,; l'ambition
des courtisans et la crainte que ces religieux inspiraient aux ministres
leur donnait une entière liberté. L'attention si vigilante du roi à se
tenir toute sa vie barricadé contre tout le monde, en affaires, leur
était un rempart assuré, et leur donnait la facilité de lui parler, et
la sécurité d'y être seuls reçus sur les choses qui regardaient la
religion, et d'être seuls écoutés. Il leur fut donc aisé de le
préoccuper, jusqu'à l'infatuation la plus complète, que quiconque
parlait autrement qu'eux, était janséniste, et que janséniste était être
ennemi du roi et de son autorité, laquelle était la partie faible et
sensible du roi jusqu'à l'incroyable. Ils parvinrent donc à disposer en
plein de lui à leur gré, et par conscience et par jalousie de son
autorité sur tout ce qui regardait cette affaire, et encore sur tout ce
qui y avait le moindre trait, c'est-à-dire sur toutes choses et gens
qu'il leur convenait de lui montrer par ce côté.

C'est par où ils dissipèrent ces saints solitaires illustres, que
l'étude et la pénitence avaient assemblés à Port-Royal, qui firent de si
grands disciples, et à qui les chrétiens seront à jamais redevables de
ces ouvrages fameux qui ont répandu une si vive et si solide lumière
pour discerner la vérité des apparences, le nécessaire de l'écorce, en
faire toucher au doigt l'étendue si peu connue, si obscurcie, et
d'ailleurs si déguisée, éclairer la foi, allumer la charité, développer
le cœur de l'homme, régler ses mœurs, lui présenter un miroir fidèle, et
le guider entre la juste crainte et l'espérance raisonnable. C'était
donc à en poursuivre jusqu'aux derniers restes, et partout\,; que la
dévotion du roi s'exerçait, et celle de M\textsuperscript{me} de
Maintenon conformée sur la sienne, lorsqu'un autre champ parut plus
propre à présenter à ce prince.

Le jansénisme commençait à paraître usé\,; il ne semblait plus bon aux
jésuites qu'à faute de mieux, et au besoin ils étaient bien sûrs d'y
retrouver longtemps de quoi glaner, lorsque après quelque intervalle ils
lui pourraient rendre quelques grâces de nouveauté. Avec de telles
avances pour se croire en droit de commander aux consciences, il restait
peu à faire pour exciter le zèle du roi contre une religion
solennellement frappée des plus éclatants anathèmes par l'Église
universelle, et qui s'en était elle-même frappée la première en se
séparant de toute l'antiquité sur des points de foi fondamentaux.

Le roi était devenu dévot, et dévot dans la dernière ignorance. À la
dévotion se joignit la politique. On voulut lui plaire par les endroits
qui le touchaient le plus sensiblement, la dévotion et l'autorité. On
lui peignit les huguenots avec les plus noires couleurs\,: un État dans
un État, parvenu à ce point de licence à force de désordres, de
révoltes, de guerres civiles, d'alliances étrangères, de résistances à
force ouverte contre les rois ses prédécesseurs, et jusqu'à lui-même
réduit à vivre en traités avec eux. Mais on se garda bien de lui
apprendre la source de tant de maux, les origines de leurs divers degrés
et de leurs progrès, pourquoi et par qui les huguenots furent
premièrement armés, puis soutenus, et surtout de lui dire un seul mot
des projets de si longue main pourpensés, des horreurs et des attentats
de la Ligue contre sa couronne, contre sa maison, contre son père, son
aïeul et tous les siens.

On lui voila avec autant de soin ce que l'Évangile, et, d'après cette
divine loi, les apôtres et tous les Pères à leur suite enseignent sur la
manière de prêcher Jésus-Christ, de convertir les infidèles et les
hérétiques, et de se conduire en ce qui regarde la religion. On toucha
un dévot de la douceur de faire aux dépens d'autrui une pénitence
facile, qu'on lui persuada sûre pour l'autre monde. On saisit l'orgueil
d'un roi en lui montrant une action qui passait le pouvoir de tous ses
prédécesseurs, en lui détournant les yeux de tant de grands exploits
personnels et de tant de hauts faits d'armes pensés et résolus par son
héroïque père, et par lui-même exécutés à la tête de ses troupes avec
une vaillance qui leur en donnait et qui les fit vaincre souvent contre
toute apparence dans les plus grands périls, en l'y voyant à leur tête
aussi exposé qu'eux, et de toute la conduite de ce grand roi, qui
abattit sans ressource ce grand parti huguenot, lequel avait soutenu sa
lutte depuis François Ier avec tant d'avantages, et qui, sans la tête et
le bras de Louis le Juste, ne serait pas tombé sous les volontés de
Louis XIV. Ce prince était bien éloigné d'arrêter sa vue sur un si
solide emprunt.

On le détermina, lui qui se piquait si principalement de gouverner par
lui-même, d'un chef-d'œuvre tout à la fois de religion et de politique,
qui faisait triompher la véritable par la ruine de toute autre, et qui
rendait le roi absolu en brisant toutes ses chaînes avec les huguenots,
et en détruisant à jamais ces rebelles, toujours prêts à profiter de
tout pour relever leur parti et donner la loi à ses rois.

Les grands ministres n'étaient plus alors. Le Tellier au lit de la mort,
son funeste fils était le seul qui restât\,; car Seignelay ne faisait
guère que poindre. Louvois, avide de guerre, atterré sous le poids d'une
trêve de vingt ans, qui ne faisait presque que d'être signée, espéra
qu'un si grand coup porté aux huguenots remuerait tout le protestantisme
de l'Europe, et s'applaudit en attendant de ce que, le roi ne pouvant
frapper sur les huguenots que par ses troupes, il en serait le principal
exécuteur, et par là de plus en plus en crédit. L'esprit et le génie de
M\textsuperscript{me} de Maintenon, tel qu'il vient d'être représenté
avec exactitude, n'était rien moins que propre ni capable d'aucune
affaire au delà de l'intrigue. Elle n'était pas née ni nourrie à voir
sur celles-ci au delà de ce qui lui en était présenté, moins encore pour
ne pas saisir avec ardeur une occasion si naturelle de plaire,
d'admirer, de s'affermir de plus en plus par la dévotion. Qui d'ailleurs
eût su un mot de ce qui ne se délibérait qu'entre le confesseur, le
ministre alors comme unique, et l'épouse nouvelle et chérie\,; et qui de
plus eût osé contredire\,? C'est ainsi que sont menés à tout, par une
vole ou par une autre, les rois qui, par grandeur, par défiance, par
abandon à ceux qui les tiennent, par paresse ou par orgueil, ne se
communiquent qu'à deux ou trois personnes, et bien souvent à moins, et
qui mettent entre eux et tout le reste de leurs sujets une barrière
insurmontable.

La révocation de l'édit de Nantes sans le moindre prétexte et sans aucun
besoin, et les diverses proscriptions plutôt que déclarations qui la
suivirent, furent les fruits de ce complot affreux qui dépeupla un quart
du royaume, qui ruina son commerce, qui l'affaiblit dans toutes ses
parties, qui le mit si longtemps au pillage public et avoué des dragons,
qui autorisa les tourments et les supplices dans lesquels ils firent
réellement mourir tant d'innocents de tout sexe par milliers, qui ruina
un peuple si nombreux, qui déchira un monde de familles, qui arma les
parents contre les parents pour avoir leur bien et les laisser mourir de
faim\,; qui fit passer nos manufactures aux étrangers, fit fleurir et
regorger leurs États aux dépens du nôtre et leur fit bâtir de nouvelles
villes, qui leur donna le spectacle d'un si prodigieux peuple proscrit,
nu, fugitif, errant sans crime, cherchant asile loin de sa patrie\,; qui
mit nobles, riches, vieillards, gens souvent très estimés pour leur
piété, leur savoir, leur vertu, des gens aisés, faibles, délicats, à la
rame, et sous le nerf très effectif du comité, pour cause unique de
religion\,; enfin qui, pour comble de toutes horreurs, remplit toutes
les provinces du royaume de parjures et de sacrilèges, où tout
retentissait de hurlements de ces infortunées victimes de l'erreur,
pendant que tant d'autres sacrifiaient leur conscience à leurs biens et
à leur repos, et achetaient l'un et l'autre par des abjurations simulées
d'où sans intervalle on les traînait à adorer ce qu'ils ne croyaient
point, et à recevoir réellement le divin corps du Saint des saints,
tandis qu'ils demeuraient persuadés qu'ils ne mangeaient que du pain
qu'ils devaient encore abhorrer. Telle fut l'abomination générale
enfantée par la flatterie et par la cruauté. De la torture à
l'abjuration, et de celle-ci à la communion, il n'y avait pas souvent
vingt-quatre heures de distance, et leurs bourreaux étaient leurs
conducteurs et leurs témoins. Ceux qui, par la suite, eurent l'air
d'être changés avec plus de loisir, ne tardèrent pas, par leur fuite ou
par leur conduite, à démentir leur prétendu retour.

Presque tous les évêques se prêtèrent à cette pratique subite et impie.
Beaucoup y forcèrent\,; la plupart animèrent les bourreaux, forcèrent
les conversions, et ces étranges convertis à la participation des divins
mystères, pour grossir le nombre de leurs conquêtes, dont ils envoyaient
les états à la cour pour en être d'autant plus considérés et approchés
des récompenses.

Les intendants des provinces se distinguèrent à l'envi à les seconder,
eux et les dragons, et à se faire valoir aussi à la cour par leurs
listes. Le très peu de gouverneurs et de lieutenants généraux de
province qui s'y trouvaient, et le petit nombre de seigneurs résidant
chez eux, et qui purent trouver moyen de se faire valoir à travers les
évêques et les intendants, n'y manquèrent pas.

Le roi recevait de tous les côtés des nouvelles et des détails de ces
persécutions et de toutes ces conversions. C'était par milliers qu'on
comptait ceux qui avaient abjuré et communié\,: deux mille dans un lieu,
six mille dans un autre, tout à la fois, et dans un instant. Le roi
s'applaudissait de sa puissance et de sa piété. Il se croyait au temps
de la prédication des apôtres, et il s'en attribuait tout l'honneur. Les
évêques lui écrivaient des panégyriques\,; les jésuites en faisaient
retentir les chaires et les missions. Toute la France était remplie
d'horreur et de confusion, et jamais tant de triomphes et de joie,
jamais tant de profusion de louanges. Le monarque ne doutait pas de la
sincérité de cette foule de conversions\,; les convertisseurs avaient
grand soin de l'en persuader et de le béatifier par avance. Il avalait
ce poison à longs traits. Il ne s'était jamais cru si grand devant les
hommes, ni si avancé devant Dieu dans la réparation de ses péchés et du
scandale de sa vie. Il n'entendait que des éloges, tandis que les bons
et vrais catholiques et les saints évêques gémissaient de tout leur cœur
de voir des orthodoxes imiter, contre les erreurs et les hérétiques, ce
que les tyrans hérétiques et païens avaient fait contre la vérité,
contre les confesseurs et contre les martyrs. Ils ne se pouvaient
surtout consoler de cette immensité de parjures et de sacrilèges. Ils
pleuraient amèrement l'odieux durable et irrémédiable que de détestables
moyens répandaient sur la véritable religion, tandis que nos voisins
exultaient de nous voir ainsi nous affaiblir et nous détruire
nous-mêmes, profitaient de notre folie, et bâtissaient des desseins sur
la haine que nous nous attirions de toutes les puissances protestantes.

Mais à ces parlantes vérités le roi était inaccessible. La conduite même
de Rome à son égard ne put lui ouvrir les yeux\,; de cette cour qui
n'avait pas eu honte autrefois d'exalter la Saint-Barthélemy, jusqu'à en
faire des processions publiques pour en remercier Dieu, et jusqu'à avoir
employé les plus grands maîtres à peindre dans le Vatican cette action
exécrable.

Odescalchi occupait le pontificat, sous le nom d'Innocent XI. C'était un
bon évêque, mais un prince très incapable, entièrement autrichien, et
ses ministres de même génie. La grande affaire de la régale l'avait
brouillé avec le roi dès l'entrée de son pontificat. Les quatre
propositions de l'assemblée du clergé de 1682\footnote{Les quatre
  propositions de l'assemblée de 1682 contiennent en substance les
  articles suivants\,: 1° Les rois ne sont point soumis pour le temporel
  à la puissance ecclésiastique\,: ils ne peuvent être déposés par les
  papes ni leurs sujets déliés du serment de fidélité\,; 2° les décrets
  du concile de Constance sur l'autorité des conciles généraux doivent
  être admis dans leur plénitude\,; 3° l'exercice de la puissance
  ecclésiastique doit être réglé d'après les canons\,; les lois et
  coutumes de l'Église gallicane doivent être observées\,; 4° le
  jugement du pape, même en matière de foi, n'est infaillible que
  lorsqu'il est approuvé par le consentement de toute l'Église. Ces
  propositions célèbres furent défendues par Bossuet.}, l'irritèrent
bien davantage. Cette main basse sur les huguenots ne put tirer de lui
la moindre approbation. Il s'en tint toujours à l'attribuer à politique
pour détruire un parti qui avait tant et si longtemps agité la France,
et l'affaire des franchises étant survenue après, les deux cours se
portèrent à de grandes extrémités. Par l'événement, et sur le point
d'honneur des franchises, et sur le point si capital des propositions de
1682, on ne s'aperçut que trop que M. de Lyonne n'était plus, et que
nous étions bien éloignés du temps de la fameuse affaire des Corses et
du traité de Pise.

Le magnifique établissement de Saint-Cyr suivit de près la révocation de
l'édit de Nantes. M\textsuperscript{me} de Montespan avait bâti à Paris
une belle maison de Filles de Saint-Joseph qu'elle avait fondée pour
l'instruction des jeunes filles, et leur apprendre toutes sortes
d'ouvrages, dont il en est sorti de parfaitement beaux en toutes sortes
d'ornements d'église, et d'autres meubles superbes pour le roi, et pour
qui en a voulu faire faire\,; et c'est dans cette maison que
M\textsuperscript{me} de Montespan se retira lorsqu'elle fut obligée de
quitter tout à fait la cour. L'émulation porta M\textsuperscript{me} de
Maintenon à des vues plus hautes et plus vastes, qui, en gratifiant la
pauvre noblesse, l'en pût faire renarder comme une protectrice en qui
toute la nobles se devait s'intéresser. Elle espéra s'aplanir un chemin
à faire déclarer son mariage, en s'illustrant par un monument dont elle
pût entretenir et amuser le roi, qui l'amusât elle-même, et qui pût lui
servir de retraite si elle avait le malheur de perdre le roi, comme il
arriva en effet. La riche mense abbatiale\footnote{On appelait mense
  abbatiale la partie des revenus d'un monastère qui était spécialement
  affectée aux dépenses de l'abbé.} de Saint-Denis, qu'elle fit unir à
Saint-Cyr, diminua d'autant la dépense d'une si grande fondation aux
yeux du roi et du public, et l'objet en était en soi si utile qu'il ne
reçut que de justes applaudissements.

Sa déclaration était toujours son plus ardent désir. L'opposition que
Louvois y avait si héroïquement mise sur le point d'éclater le perdit
bientôt après, comme on l'a vu, et l'archevêque de Paris avec lui, qui
s'y était associé. Elle n'éteignit pas pour cela toute son espérance.
Elle s'était flattée d'en avoir jeté les fondements sans y avoir pu
penser alors\,; car ce fut du vivant de la reine que, pour se recrépir
et passer l'éponge sur sa première vie, elle fit entendre au roi
modestement sa noblesse, puis au mariage de Monseigneur l'importance
d'environner la Dauphine de personnes sûres, et de lui donner à
elle-même un titre auprès d'elle, qui lui donnât droit et moyen d'y
veiller.

C'est ce qui, comme on l'a vu, y fit passer M\textsuperscript{me} de
Richelieu dame d'honneur de la reine, moyennant la charge de chevalier
d'honneur à son mari, pour l'exercer et la vendre après tant qu'il
pourrait sans en avoir rien payé, qui étaient, comme on l'a vu, les
anciens et intimes amis de M\textsuperscript{me} de Maintenon, laquelle
fut faite seconde dame d'atours avec la maréchale de Rochefort. La
distance était étrange entre les deux dames d'atours\,; il n'en fallait
qu'une\,; le choix de la seconde indigna tout le monde. La première
était de longue main accoutumée au servage des ministres et des
maîtresses, et ne songea qu'à plaire à ce soleil levant dans son
automne. Elle se flatta aussi de succéder à la duchesse de Richelieu,
beaucoup plus âgée qu'elle et infirme\,; elle y fut trompée, le roi
voulut une duchesse. On a vu comment et pourquoi M\textsuperscript{me}
de Maintenon y bombarda M\textsuperscript{me} d'Arpajon, à l'étonnement
de toute la cour, et plus de la duchesse d'Arpajon que de personne.

Malgré tous ces entours, la fierté allemande séduisit l'esprit et le
plus cher intérêt de la Dauphine. Monseigneur qui n'aimait point
M\textsuperscript{me} de Maintenon ne contraignit point son épouse. Il
était toujours alors avec la princesse de Conti qui le gouvernait, et
qui, fille de M\textsuperscript{me} de La Vallière, n'avait rien de
commun avec les enfants de M\textsuperscript{me} de Montespan, ni avec
leur gouvernante, desquels tous elle était fort éloignée. Elle n'aimait
pas mieux la Dauphine, dont elle craignait la concurrence et pis dans la
confiance de Monseigneur. Elle ne fut donc pas fâchée de la voir prendre
si mal avec M\textsuperscript{me} de Maintenon, et se mettre par ses
manières à cet égard de travers avec le roi, et perdre toute
considération, comme il arriva. Elle fut peu comptée. On prétendit que
la princesse de Conti excessivement parfumée la vit de fort près et
longtemps, comme elle venait d'accoucher de M. le duc de Berry. Quoi
qu'il en soit, sa courte vie depuis ne fut plus qu'une maladie
continuelle, plus ou moins forte\,; et sa mort soulagea mari, beau-père,
et plus que tous, belle-mère, qui, quatorze mois après, se vit aussi
délivrée de Louvois.

Ce fut pour lors que l'espérance d'être déclarée reprit toutes ses
forces. Monseigneur et Monsieur y auraient été des obstacles\,; mais ils
vivaient dans une telle dépendance du roi que leur considération n'était
comptée pour rien à cet égard. On a vu combien le bruit fut grand que la
déclaration du mariage était imminente lors de l'ouverture de
l'appartement de la reine, demeuré jusque-là fermé, depuis que la
Dauphine y était morte\,; que ce fut sous prétexte d'y exposer à
l'admiration de la cour les superbes ornements des quatre couleurs que
le roi envoyait à l'église de Strasbourg, et le mot étrange à bout
portant que Tonnerre, évêque-comte de Noyon, lâcha au roi en plein petit
couvert sur cette déclaration.

Ce fut en effet alors qu'elle fut sur le point d'être faite. Mais le
roi, plein encore de ce qui lui était arrivé là-dessus, consulta le
célèbre Bossuet, évêque de Meaux, et Fénelon, archevêque de Cambrai, qui
l'en dissuadèrent l'un et l'autre, et qui, cette seconde fois, firent
manquer le coup pour toujours. L'archevêque était déjà mal avec
M\textsuperscript{me} de Maintenon sur l'affaire de
M\textsuperscript{me} Guyon, sans espérance de retour, à cause de Godet,
évêque de Chartres, comme on l'a vu en son temps, mais encore alors
assez entier auprès du roi, où il ne tarda pas d'être perdu sans
ressource. Bossuet échappa à la disgrâce que M\textsuperscript{me} de
Maintenon n'entreprit même pas, par plusieurs raisons. Godet, qui la
possédait absolument, comme on l'a vu ailleurs, avait besoin de la plume
et du grand nom de Bossuet pour pousser Fénelon à bout. Bossuet tenait
au roi par l'habitude et l'estime, et par être entré en évêque des
premiers temps dans la confiance la plus intime du roi et la plus
secrète dans les temps de ses désordres\,; enfin il avait rendu à
M\textsuperscript{me} de Maintenon, sans que ce fût son objet, le
service le plus sensible.

C'était un homme dont l'honneur, la vertu, la droiture était aussi
inséparable que la science et la vaste érudition. Sa place de précepteur
de Monseigneur l'avait familiarisé avec le roi, qui s'était adressé plus
d'une fois à lui dans les scrupules de sa vie. Bossuet lui avait souvent
parlé là-dessus avec une liberté digne des premiers siècles et des
premiers évêques de l'Église. Il avait interrompu le cours du désordre
plus d'une fois\,; il avait osé poursuivre le roi, qui lui avait
échappé. Il fit à la fin cesser tout mauvais commerce, et il acheva de
couronner cette grande œuvre par les derniers coups qui chassèrent pour
jamais M\textsuperscript{me} de Montespan de la cour.
M\textsuperscript{me} de Maintenon, au centre de la gloire, ne pouvait
goûter de repos tant qu'elle y voyait son ancienne maîtresse demeurante,
et tous les jours visitée par le roi. C'était, ce lui semblait, autant
de temps et de reste d'autorité pris sur elle. De plus, elle ne pouvait
éviter de lui rendre, sinon d'anciens respects, au moins de grands
égards, et des devoirs apparents. Outre qu'ils la faisaient trop
souvenir de son ancienne bassesse, elle en éprouvait souvent de
M\textsuperscript{me} de Montespan d'amères et de bien expresses
commémoraisons, sans ménagements. Les visites journelles en demi-public
du roi à son ancienne maîtresse, toujours entre la messe et le dîner,
pour les rendre plus nécessairement courtes, et par bienséance,
faisaient un contraste fort ridicule avec son assiduité longue de tous
les jours chez celle qui l'avait servie, et chez qui, sans nom de
maîtresse ni d'épouse, était le creuset de la cour et de l'État. Cette
sortie de la cour de M\textsuperscript{me} de Montespan, pour n'y plus
revenir, fut donc une grande délivrance pour M\textsuperscript{me} de
Maintenon, et elle n'ignora pas qu'elle la dut à M. de Meaux tout
entière, qui à la fin lui en attira les ordres réitérés.

Ce fut l'époque de l'union si parfaite et si intime de M. du Maine et de
M\textsuperscript{me} de Maintenon, et de l'adoption qu'elle en fit, qui
s'approfondit et se consolida toujours depuis de plus en plus, qui lui
fraya le chemin à toutes les incroyables grandeurs où de l'une à l'autre
il parvint, et qui enfin l'aurait mis sur le trône, si telle avait pu
être la puissance de son ancienne mie.

le duc du Maine était trop continuellement dans l'intérieur du roi, pour
ne s'être pas aperçu de bonne heure de la faveur naissante de
M\textsuperscript{me} de Maintenon, de ses progrès rapides, et que les
premiers effets n'en pouvaient être que la disgrâce de
M\textsuperscript{me} de Montespan. Personne n'avait plus d'esprit que
le duc du Maine, ni d'art caché sous toutes les sortes de grâces qui
peuvent charmer, avec l'air, le plus naturel, le plus simple,
quelquefois le plus naïf\,; personne ne prenait plus aisément toutes
sortes de formes\,; personne ne connaissait mieux les gens qu'il avait
intérêt de connaître\,; personne n'avait plus de tour, de manège,
d'adresse pour s'insinuer auprès d'eux\,; personne encore, sous un
extérieur dévot, solitaire, philosophe, sauvage, ne cachait des vues
plus ambitieuses ni plus vastes, que son extrême timidité de plus d'un
genre servait encore à couvrir. On a vu ailleurs son caractère\,; on
n'en rappelle ici que ce qui sert à la matière que l'on traite, sans
vouloir s'en écarter.

Le duc du Maine s'aperçut donc de bonne heure des épines de sa position
entre sa mère et sa gouvernante, que l'enlèvement du cœur du roi rendait
irréconciliables. Il sentit en même temps que sa mère ne lui serait
qu'un poids fort entravant, tandis qu'il pouvait tout espérer de sa
gouvernante. Le sacrifice lui en fut donc bientôt fait. Il entra dans
tout avec M. de Meaux pour hâter la retraite de sa mère\,; il se fit un
mérite auprès de M\textsuperscript{me} de Maintenon de presser lui-même
M\textsuperscript{me} de Montespan de s'en aller à Paris pour ne plus
revenir à la cour\,; il se chargea de lui en porter l'ordre du roi, et à
la fin l'ordre très positif\,; il s'en acquitta sans ménagement\,; il la
fit obéir, et se dévoua par là M\textsuperscript{me} de Maintenon sans
réserve. Il fut longtemps très mal avec sa mère, qui ne le voulait point
voir, et jamais depuis il n'y fut véritablement bien. Ce fut aussi la
moindre de ses peines. Il eut à lui celle qui régnait, et qui régna
toujours, et il l'eut au point d'en disposer toute sa vie, et que toute
la sienne elle ne mit point de bornes à son affection pour lui.

\hypertarget{chapitre-iii.}{%
\chapter{CHAPITRE III.}\label{chapitre-iii.}}

~

{\textsc{Mécanique, vie particulière et conduite de
M\textsuperscript{me} de Maintenon.}} {\textsc{- Adresse et conduite de
M\textsuperscript{me} de Maintenon pour gouverner.}} {\textsc{- Coups de
caveçon du roi pour gouverner, qui ne l'empêchent pas de l'être en
plein.}} {\textsc{- Dureté du roi\,; excès de contrainte avec lui.}}
{\textsc{- Voyages du roi.}} {\textsc{- Sa manière d'aller.}} {\textsc{-
Aventure de la duchesse de Chevreuse.}} {\textsc{- M\textsuperscript{me}
de Maintenon voyage à part, n'en est guère moins contrainte.}}
{\textsc{- Domestique de M\textsuperscript{me} de Maintenon.}}
{\textsc{- Nécessité des détails sur M\textsuperscript{me} de
Maintenon.}} {\textsc{- Grandeur particulière de M\textsuperscript{me}
de Maintenon.}} {\textsc{- Autorité particulière de
M\textsuperscript{me} de Maintenon.}}

~

Ce grand pas fait de l'expulsion sans retour de M\textsuperscript{me} de
Montespan, M\textsuperscript{me} de Maintenon prit un nouvel éclat.
Ayant manqué pour la seconde fois la déclaration de son mariage, elle
comprit qu'il n'y avait plus à y revenir, et eut assez de force sur
elle-même pour couler doucement par-dessus, et ne se pas creuser une
disgrâce pour n'avoir pas été déclarée reine. Le roi, qui se sentit
affranchi, lui sut un gré de cette conduite qui redoubla pour elle son
affection, sa considération, sa confiance. Elle eût peut-être succombé
sous le poids de l'éclat de ce qu'elle avait voulu paraître, elle
s'établit de plus en plus par la confirmation de sa transparente énigme.

Mais il ne faut pas s'imaginer que, pour en user et s'y soutenir, elle
n'eût besoin d'aucune adresse. Son règne, au contraire, ne fut qu'un
continuel manège, et celui du roi une perpétuelle duperie. Elle ne
voyait personne chez elle en visite, et n'en rendait jamais aucune. Cela
n'avait que fort peu d'exceptions. Elle allait voir la reine
d'Angleterre et la recevait chez elle, quelquefois chez
M\textsuperscript{me} de Montchevreuil, sa plus intime amie, qui allait
très ordinairement chez elle. Depuis sa mort elle alla voir quelquefois
M. de Montchevreuil, mais rarement, qui entrait chez elle toutes les
fois qu'il voulait, mais des instants. Le duc de Richelieu eut toute sa
vie le même privilège. Elle allait quelquefois encore chez
M\textsuperscript{me} de Caylus, sa bonne nièce, qui était souvent chez
elle. Si, en deux ans une fois, elle allait chez la duchesse du Lude, ou
quelque femme aussi marquée, entre trois ou quatre au plus, c'était une
distinction et une nouvelle, quoiqu'il ne s'agît que d'une simple
visite. M\textsuperscript{me} d'Heudicourt, son ancienne amie, allait
aussi chez elle à peu près quand elle voulait, et sur les fins le
maréchal de Villeroy, quelquefois Harcourt, jamais d'autres. On a vu,
lors du brillant voyage de M\textsuperscript{me} des Ursins, qu'elle
allait aussi très souvent chez elle en particulier à Marly\,; et
M\textsuperscript{me} de Maintenon la fut voir une fois. Jamais elle
n'allait chez aucune princesse du sang, même chez Madame. Aucune d'elles
aussi n'allait chez elle, à moins que ce ne fût par audiences\,; ce qui
était extrêmement rare et qui faisait nouvelle. Mais si elle avait à
parler aux filles du roi, ce qui n'arrivait pas souvent, et presque
jamais que pour leur laver la tête, elle les envoyait chercher. Elles y
arrivaient tremblantes, et en sortaient en pleurs. Pour le duc du Maine,
les portes tombèrent toujours devant lui en quelque lieu qu'il fût\,; et
depuis le mariage du duc de Noailles, il la voyait aussi quand il
voulait, son père avec ménagement, sa mère fort à lèche-doigt\,; le roi
et elle la craignaient et ne l'aimaient point.

Le cardinal de Noailles, jusqu'à l'affaire de la constitution, la voyait
règlement en particulier le jour qu'il avait son audience du roi, une
fois la semaine\,; et après, le cardinal de Bissy à peu près tant qu'il
voulut, et le cardinal de Rohan avec mesure. Son frère tant qu'il vécut
la désola. Il entrait chez elle à toute heure, lui tenait des propos de
l'autre monde, et lui faisait souvent des sorties. De crédit avec elle,
pas le moins du monde. Sa belle-soeur ne parut jamais à la cour ni dans
le monde\,; M\textsuperscript{me} de Maintenon la traitait bien par
pitié, sans que cela allât au plus petit crédit\,; mais elle dînait
quelquefois avec elle, et ne la laissait venir à Versailles que le moins
qu'elle pouvait, peut-être deux ou trois fois l'an au plus, et coucher
une nuit. Godet, évêque de Chartres, et Aubigny, archevêque de Rouen,
elle ne les voyait qu'à Saint-Cyr.

Ses audiences étaient pour le moins aussi difficiles à obtenir que
celles du roi\,; et le peu qu'elle en accordait, presque toutes à
Saint-Cyr où on allait la trouver au jour et heure donnés. On
l'attendait à Versailles à sortir de chez elle ou à y rentrer, quand on
avait un mot à lui dire, gens de peu et même pauvres gens, et personnes
considérables. On n'avait là qu'un instant, et c'était à qui le
saisirait. Les maréchaux de Villeroy, Harcourt, souvent Tessé,
quelquefois dans les derniers temps M. de Vaudemont, lui ont parlé de la
sorte, et si c'était en rentrant chez elle, ils ne la suivaient pas au
delà de son antichambre, où elle coupait très court et les laissait.
Bien d'autres lui ont parlé de la sorte. Moi jamais en pas un lieu que
ce que j'ai rapporté. Un très petit nombre de dames, à qui le roi était
accoutumé et qui étaient de ses particuliers, la voyaient quelquefois
aux heures où le roi n'était pas, et rarement quelques-unes dînaient
avec elle.

Ses matinées, qu'elle commençait de fort bonne heure, étaient remplies
par des audiences obscures de charité ou de gouvernement spirituel\,;
quelquefois par quelques ministres, très rarement par quelques généraux
d'armée, encore ces derniers, quand ils avaient un rapport particulier à
elle, comme les maréchaux de Villars, de Villeroy, d'Harcourt et
quelquefois Tessé. Assez souvent, dès huit heures du matin et plus tôt,
elle allait chez quelque ministre. Rarement elle dînait chez eux avec
leurs femmes et une compagnie fort trayée. C'étaient là les grandes
faveurs, et une nouvelle, mais qui ne menaient à rien qu'à de l'envie et
à quelque considération. M. de Beauvilliers fut des premiers et des plus
longtemps favorisé de ces dîners, et fréquents, comme on l'a remarqué
ailleurs, jusqu'à ce que Godet, évêque de Chartres, en renversa les
escabelles, et arrêta tout court les progrès de Fénelon qui s'était fait
leur docteur. Les ministres chargés de la guerre, surtout des finances,
furent toujours ceux à qui M\textsuperscript{me} de Maintenon avait le
plus affaire, et qu'elle cultiva. Rarement, et plus que rarement,
alla-t-elle chez les autres, mais pour affaires, et souvent d'État, et
dès le matin, sans jamais dîner chez ces derniers.

L'ordinaire, dès qu'elle était levée, c'était de s'en aller à Saint-Cyr,
et d'y dîner dans son appartement seule, ou avec quelque favorite de la
maison, d'y donner des audiences le moins qu'elle pouvait, d'y régenter
au dedans, d'y gouverner l'Église au dehors, d'y lire et d'y répondre
des lettres, d'y gouverner des monastères de filles de toutes parts, d'y
recevoir des avis et des lettres d'espionnages, et de revenir à peu près
justement au temps que le roi passait chez elle. Devenue plus vieille et
plus infirme, en arrivant entre sept et huit heures du matin à
Saint-Cyr, elle s'y mettait au lit pour se reposer, ou faire quelque
remède.

À Fontainebleau, elle avait une maison à la ville, où elle allait
souvent pour y faire les mêmes choses qu'à Saint-Cyr. À Marly, elle
s'était fait accommoder un petit appartement qui avait une fenêtre dans
la chapelle. Elle en faisait souvent le même usage que de Saint-Cyr\,;
mais cela s'appelait le repos, et ce repos était inaccessible, sans
exception que de M\textsuperscript{me} la duchesse de Bourgogne.

À Marly, à Trianon, à Fontainebleau, le roi allait chez elle les matins
des jours qu'il n'y avait point de conseil, et qu'elle n'était pas à
Saint-Cyr\,; à Fontainebleau, depuis la messe jusqu'au dîner, quand le
dîner n'était pas quelquefois au sortir de la messe pour aller courre le
cerf\,; et il y était une heure et demie, et quelquefois davantage. À
Trianon et à Marly, la visite durait beaucoup moins, parce qu'en sortant
de chez elle il s'allait promener dans ses jardins. Ces visites étaient
presque toujours tête à tête, sans préjudice de celles de toutes les
après-dînées, qui étaient rarement tête-à-tête que fort peu de temps,
parce que les ministres y venaient chacun à son tour travailler avec le
roi. Le vendredi, qu'il arrivait souvent qu'il n'y en avait point,
c'étaient les dames familières avec qui il jouait, ou une musique\,; ce
qui se doubla et tripla de jours tout à la fin de sa vie.

Vers les neuf heures du soir, deux femmes de chambre venaient
déshabiller M\textsuperscript{me} de Maintenon. Aussitôt après, son
maître d'hôtel et un valet de chambre apportaient son couvert, un potage
et quelque chose de léger. Dès qu'elle avait achevé de souper, ses
femmes la mettaient dans son lit, et tout cela en présence du roi et du
ministre, qui n'en discontinuait pas son travail, et qui n'en parlait
pas plus bas, ou, s'il n'y en avait point, des dames familières. Tout
cela gagnait dix heures, que le roi allait souper, et en même temps on
tirait les rideaux de M\textsuperscript{me} de Maintenon.

Dans les voyages, c'était la même chose. Elle partait de bonne heure
avec quelque favorite, comme M\textsuperscript{me} de Montchevreuil
toujours tant qu'elle vécut, M\textsuperscript{me} d'Heudicourt,
M\textsuperscript{me} de Dangeau, M\textsuperscript{me} de Caylus. Un
carrosse du roi la menait, toujours affecté pour elle, même pour aller
de Versailles, etc., à Saint-Cyr\,; et des Épinays, écuyer de la petite
écurie, la mettait dans le carrosse et l'accompagnait à cheval\,;
c'était sa tâche de tous les jours. Dans les voyages, le carrosse de
M\textsuperscript{me} de Maintenon menait ses femmes de chambre, et
suivait celui du roi où elle était. Elle s'arrangeait de façon que le
roi, en arrivant, la trouvait tout établie lorsqu'il passait chez elle.
Partie autorité, partie invention de seconde dame d'atours de la
Dauphine de Bavière, son carrosse et sa chaise, avec ses porteurs ayant
sa livrée, entraient partout comme ceux des gens titrés.

Reine en particulier, à l'extérieur pour le ton, le siège et la place en
présence du roi, de Monseigneur, de Monsieur, de la cour d'Angleterre et
de qui que ce fût, elle était très simple particulière au dehors, et
toujours aux dernières places. J'en ai vu les fins aux dîners du roi à
Marly, mangeant avec lui et les dames, et à Fontainebleau en grand habit
chez la reine d'Angleterre, comme je l'ai remarqué ailleurs, cédant
absolument sa place, et se reculant partout pour les femmes titrées,
même pour des femmes de qualité distinguées, ne se laissant jamais
forcer par les titrées, mais par celles de qualité ordinaire, avec un
air de peine et de civilité, et par tous ses endroits polie, affable,
parlante, comme une personne qui ne prétend rien et qui ne montre rien,
mais qui imposait tort, à ne considérer que ce qui était autour d'elle.

Toujours très bien mise, noblement, proprement, de bon goût, mais très
modestement et plus vieillement alors que son âge. Depuis qu'elle ne
parut plus en public, on ne voyait que coiffes et écharpe noire quand
par hasard on l'apercevait.

Elle n'allait jamais chez le roi qu'il ne fût malade, ou que les matins
des jours qu'il avait pris médecine, et à peu près de même chez
M\textsuperscript{me} la duchesse de Bourgogne, jamais ailleurs pour
aucun devoir.

Chez elle, avec le roi, ils étaient chacun dans leur fauteuil, une table
devant chacun d'eux, aux deux coins de la cheminée, elle du côté du lit,
le roi le dos à la muraille du côté de la porte de l'antichambre, et
deux tabourets devant sa table, un pour le ministre qui venait
travailler, l'autre pour son sac. Les jours de travail, ils n'étaient
seuls ensemble que fort peu de temps avant que le ministre entrât, et
moins encore fort souvent après qu'il était sorti. Le roi passait à une
chaise percée, revenait au lit de M\textsuperscript{me} de Maintenon, où
il se tenait debout fort peu, lui donnait le bonsoir, et s'en allait se
mettre à table. Telle était la mécanique de chez M\textsuperscript{me}
de Maintenon. On a vu sur M\textsuperscript{me} la duchesse de Bourgogne
ce qui l'y regardait, tant qu'elle a vécu.

Pendant le travail, M\textsuperscript{me} de Maintenon lisait ou
travaillait en tapisserie. Elle entendait tout ce qui se passait entre
le roi et le ministre, qui parlaient tout haut. Rarement elle y mêlait
son mot, plus rarement ce mot était de quelque conséquence. Souvent le
roi lui demandait son avis. Alors elle répondait avec de grandes
mesures. Jamais, ou comme jamais, elle ne paraissait affectionner rien,
et moins encore s'intéresser pour personne\,; mais elle était d'accord
avec le ministre qui n'osait en particulier ne pas convenir de ce
qu'elle voulait, ni encore moins broncher en sa présence. Dès qu'il
s'agissait donc de quelque grâce ou de quelque emploi, la chose était
arrêtée entre eux avant le travail où la décision s'en devait faire, et
c'est ce qui la retardait quelquefois, sans que le roi ni personne en
sût la cause.

Elle mandait au ministre qu'elle voulait lui parler auparavant. Il
n'osait mettre la chose sur le tapis qu'il n'eût reçu ses ordres, et que
la mécanique roulante des jours et des temps leur eût donné le loisir de
s'entendre. Cela fait, le ministre proposait et montrait une liste. Si
de hasard le roi s'arrêtait à celui que M\textsuperscript{me} de
Maintenon voulait, le ministre s'en tenait là, et faisait en sorte de
n'aller pas plus loin. Si le roi s'arrêtait à quelque autre, le ministre
proposait de voir ceux qui étaient aussi à portée, laissait après dire
le roi, et en profitait pour exclure. Rarement proposait-il expressément
celui à qui il en voulait venir, mais toujours plusieurs qu'il tâchait
de balancer également pour embarrasser le roi sur le choix. Alors le roi
lui demandait son avis, il parcourait encore les raisons de
quelques-uns, et appuyait enfin sur celui qu'il voulait. Le roi presque
toujours balançait, et demandait à M\textsuperscript{me} de Maintenon ce
qu'il lui en semblait. Elle souriait, faisait l'incapable, disait
quelquefois un mot de quelque autre, puis revenait, si elle ne s'y était
pas tenue d'abord, sur celui que le ministre avait appuyé, et
déterminait\,; tellement que les trois quarts des grâces et des choix,
et les trois quarts encore du quatrième quart de ce qui passait par le
travail des ministres chez elle, c'était elle qui en disposait.
Quelquefois aussi, quand elle n'affectionnait personne, c'était le
ministre même, avec son agrément et son concours, sans que le roi en eût
aucun soupçon. Il croyait disposer de tout et seul, tandis qu'il ne
disposait, en effet, que de la plus petite partie, et toujours encore
par quelque hasard, excepté des occasions rares de quelqu'un qu'il
s'était mis dans la fantaisie, ou si quelqu'un qu'il voulait favoriser
lui avait parlé pour quelqu'un.

En affaires, si M\textsuperscript{me} de Maintenon les voulait faire
réussir, manquer, ou tourner d'une autre façon, ce qui était beaucoup
moins ordinaire que ce qui regardait les emplois et les grâces, c'était
la même intelligence entre elle et le ministre, et le même manège à peu
près. Par ce détail, on voit que cette femme habile faisait presque tout
ce qu'elle voulait, mais non pas tout, ni quand et comme elle voulait.

Il y avait une autre ruse si le roi s'opiniâtrait\,: c'était alors
d'éviter la décision en brouillant et allongeant la matière, en en
substituant une autre comme venant à propos de celle-là, et qui la
détournât, ou en proposant quelque éclaircissement à prendre. On
laissait ainsi émousser les premières idées, et on revenait une autre
fois à la charge avec la même adresse, qui très souvent réussissait.
C'était encore presque la même chose pour charger ou diminuer les
fautes, faire valoir les lettres et les services, ou y glisser
légèrement, et préparer ainsi la perte ou la fortune.

C'est là ce qui rendait ce travail chez M\textsuperscript{me} de
Maintenon si important pour les particuliers, et c'est ce qui rendait
les ministres si nécessaires à M\textsuperscript{me} de Maintenon à
avoir dans sa dépendance. C'est aussi ce qui les aida puissamment à
s'élever à tout, et à augmenter sans cesse leur crédit et leur pouvoir,
et pour eux et pour les leurs, parce que M\textsuperscript{me} de
Maintenon leur faisait litière de toutes ces choses pour se les attacher
entièrement.

Quand ils étaient près de venir travailler, ou qu'ils sortaient de chez
elle, elle prenait son temps de sonder le roi sur eux, de les excuser ou
de les vanter, de les plaindre de leur grand travail, d'en exalter le
mérite, et s'il s'agissait de quelque chose pour eux, d'en préparer les
voies, quelquefois d'en rompre la glace, sous prétexte de leur modestie
et du service du roi qui demandait qu'ils fussent excités à le soulager
et à faire de bien en mieux. Ainsi c'était entre eux un cercle de
besoins et de services réciproques, dont le roi ne se doutait pas le
moins du monde. Aussi les ménagements entre eux étaient-ils infinis et
continuels.

Mais si M\textsuperscript{me} de Maintenon ne pouvait rien, ou presque
rien, sans eux, de ce qui passait par eux, eux aussi ne pouvaient se
maintenir sans elle, beaucoup moins malgré elle. Dès qu'elle se voyait à
bout de les pouvoir ramener à son point quand ils s'en étaient écartés,
ou qu'ils étaient tombés en disgrâce auprès d'elle, leur perte était
jurée\,; elle ne les manquait pas. Il lui fallait du temps, des
couleurs, des souplesses, quelquefois beaucoup, comme lorsqu'elle perdit
Chamillart. Louvois y avait succombé avant lui. Pontchartrain ne s'en
sauva qu'à l'aide de son esprit qui plaisait au roi, et des épines des
finances pendant la guerre, et du sens et de l'adresse de sa femme
demeurée longtemps bien avec M\textsuperscript{me} de Maintenon, depuis
même qu'il y fut mal, enfin par la porte dorée de la chancellerie qui
s'ouvrit bien à propos pour lui. Le duc de Beauvilliers y pensa faire
naufrage par deux fois à longue distance l'une de l'autre, et n'en
aurait pas échappé sans deux espèces de miracles, comme on l'a vu ici en
son temps.

Si les ministres, et les plus accrédités, en étaient là avec
M\textsuperscript{me} de Maintenon, on peut juger de ce qu'elle pouvait
à l'égard de toutes les autres sortes de personnes bien moins à portée
de se défendre, et même de s'apercevoir. Bien des gens eurent donc le
cou rompu sans en avoir pu imaginer la cause, et se donnèrent bien des
sortes de mouvements pour la découvrir, et pour y remédier, et très
inutilement.

Le court et rare travail des généraux d'armée se passait ordinairement
les soirs en sa présence et du secrétaire d'État de la guerre. Par celui
de Pontchartrain, rempli du rapport des espionnages et des histoires de
toute espèce de Paris et de la cour, elle était à portée de faire
beaucoup de bien et de mal. Torcy ne travaillait point chez elle, et ne
la voyait comme jamais. Aussi ne l'aimait-elle point, et moins encore sa
femme, dont le nom d'Arnauld gâtait tout leur mérite. Torcy avait les
postes. C'était par lui que le secret en passait au roi tête à tête, et
le roi souvent en portait des morceaux à lire à M\textsuperscript{me} de
Maintenon\,; mais cela n'avait point de suite\,; elle n'en savait que
par lambeaux, selon ce que le roi s'avisait de lui en dire ou de lui en
porter.

Toutes les affaires étrangères passaient au conseil d'État, ou, si
c'était quelque chose de pressé, Torcy le portait surle-champ au roi,
ainsi à des heures rompues, et point de travail réglé et particulier
avec lui. M\textsuperscript{me} de Maintenon eût fort désiré ce genre de
travail réglé chez elle, pour avoir la même influence sur les affaires
d'État, et sur ceux qui s'en mêlaient, comme elle l'avait sur les autres
parties. Mais Torcy sut bien sagement se préserver de ce dangereux
piège. Il s'en défendit toujours, en disant modestement qu'il n'avait
point d'affaires pour entretenir ce travail. Ce n'était pas que le roi
ne lui dît tout là-dessus\,; mais elle sentait toute la différence
d'assister à un travail réglé où elle agissait avec loisir, adresse et
mesures prises, ou d'être obligée de prendre son parti entre le roi et
elle sur ce qu'il lui apprenait de cette matière, et de n'avoir d'autre
ressource qu'en elle-même, et d'aller de front avec lui, si elle voulait
une chose plutôt qu'une autre, nuire aux gens à découvert, ou les servir
de même.

Le roi y était même fort en garde. Il lui est arrivé plusieurs fois que,
lorsqu'on ne s'y prenait pas avec assez de tout et de délicatesse, et
qu'il apercevait que le ministre ou le général d'armée favorisait un
parent ou un protégé de M\textsuperscript{me} de Maintenon, il tenait
ferme contre, pour cela même\,; puis disait, partie fâché, partie se
moquant d'eux\,: «\,Un tel a bien fait sa cour\,; car il n'a pas tenu à
lui de bien servir un tel, parce qu'il est parent ou protégé de
M\textsuperscript{me} de Maintenon.\,» Et ces coups de caveçon la
rendaient très timide et très mesurée, quand il était question de se
montrer au roi à découvert sur quelque chose ou sur quelqu'un. Aussi
répondait-elle toujours à quiconque s'adressait à elle, même pour les
moindres choses, qu'elle ne se mêlait de rien\,; et si bien rarement
elle s'ouvrait davantage et que la chose regardât le département d'un
ministre sur lequel elle comptât, elle renvoyait à lui et promettait de
lui en parler. Mais encore une fois, rien n'était plus rare. On ne
laissait pas cependant d'aller à elle, pour, par ce devoir, ne l'avoir
pas contraire, et par l'espérance aussi que, nonobstant cette réponse
banale, elle ferait peut-être ce qu'on désirait, comme cela arrivait
quelquefois.

Il y avait peut-être cinq ou six personnes au plus de tous états,
desquelles la plupart étaient de ces amis de son ancien temps, à qui
elle répondait plus franchement, quoique toujours faiblement et
mesurément, et pour qui en effet elle agissait au mieux qu'il lui était
possible\,; ce néanmoins réussissant très ordinairement pour eux, elle
n'y réussissait pas toujours.

Ce fut par le désir extrême de se mêler des affaires étrangères, comme
elle se mêlait de toutes les autres, et l'impossibilité d'en attirer le
travail chez elle, qu'elle prit le parti, qu'on a détaillé en son temps,
de tous les manèges par lesquels elle rendit la princesse des Ursins
maîtresse de tout en Espagne, et l'y maintint jusqu'à la paix d'Utrecht,
aux dépens de Torcy et des ambassadeurs de France en Espagne,
c'est-à-dire, comme on l'a vu, aux dépens de l'Espagne et de la France,
parce que M\textsuperscript{me} des Ursins eut l'adresse de lui faire
tout passer par les mains, et de lui persuader qu'elle ne gouvernait la
cour et l'État en Espagne que sous ses ordres et par ses volontés.
Revenons un moment à ces coups de caveçon du roi dont on vient de
parler.

Le Tellier, dans des temps bien antérieurs, et longtemps avant d'être
chancelier de France, connaissait bien le roi là-dessus. Un de ses
meilleurs amis, car il en avait parce qu'il savait en avoir, l'avait
prié de quelque chose qu'il désirait fort et qui devait être proposé
dans le travail particulier de ce ministre avec le roi. Le Tellier
l'assura qu'il y ferait tout son possible. Son ami ne goûta point sa
réponse, et lui dit franchement que dans la place et le crédit où il
était, ce n'était pas de celles-là qu'il lui fallait donner. «\,Vous ne
connaissez pas le terrain, lui répliqua Le Tellier. De vingt affaires
que nous portons ainsi au roi, nous sommes sûrs qu'il en passera
dix-neuf à notre gré\,; nous le sommes également que la vingtième sera
décidée au contraire. Laquelle des vingt sera décidée contre notre avis
et notre désir, c'est ce que nous ignorons toujours, et très souvent
c'est celle où nous nous intéressons le plus. Le roi se réserve cette
bisque pour nous faire sentir qu'il est le maître et qu'il gouverne\,;
et si par hasard il se présente quelque chose sur quoi il s'opiniâtre\,;
et qui soit assez importante pour que nous nous opiniâtrions aussi, ou
pour la chose même, ou pour l'envie que nous avons qu'elle réussisse
comme nous le désirons, c'est très souvent alors, dans le rare que cela
arrive, une sortie sûre\,; mais, à la vérité, la sortie essuyée et
l'affaire manquée, le roi, content d'avoir montré que nous ne pouvons
rien et peiné de nous avoir fâchés, devient après souple et flexible, en
sorte que c'est alors le temps où nous faisons tout ce que nous
voulons.\,»

C'est, en effet, comme le roi se conduisit avec ses ministres toute sa
vie, toujours parfaitement gouverné par eux, même par les plus jeunes et
les plus médiocres\,; même par les moins accrédités et considérés, et
toujours en garde pour ne l'être point, et toujours persuadé qu'il
réussissait pleinement à ne le point être.

Il avait la même conduite avec M\textsuperscript{me} de Maintenon, à qui
de fois à autres il faisait des sorties terribles, et dont il
s'applaudissait. Quelquefois elle se mettait à pleurer devant lui, et
elle était plusieurs jours sur de véritables épines. Quand elle eut mis
Fagon auprès du roi, au lieu de Daquin qu'elle fit chasser, parce qu'il
était de la main de M\textsuperscript{me} de Montespan, et pour avoir un
homme tout à elle et de beaucoup d'esprit, qu'elle s'était attaché dans
les voyages aux eaux où il avait suivi le duc du Maine, et un homme dont
elle pût tirer un continuel parti dans cette place intime de premier
médecin qu'elle voyait tous les matins, elle faisait la malade quand il
lui arrivait de ces scènes, et c'était d'ordinaire par où elle les
faisait finir avec plus d'avantage.

Ce n'est pas que cet artifice, ni même la réalité la plus effective, eût
aucun pouvoir d'ailleurs de contraindre le roi en quoi que ce pût être.
C'était un homme uniquement personnel, et qui ne comptait tous les
autres, quels qu'ils fussent, que par rapport à sol. Sa dureté là-dessus
était extrême. Dans les temps les plus vifs de sa vie pour ses
maîtresses, leurs incommodités les plus opposées aux voyages et au grand
habit de cour, car les dames les plus privilégiées ne paraissaient
jamais autrement dans les carrosses ni en aucun lieu de cour, avant que
Marly eût adouci cette étiquette, rien, dis-je, ne les en pouvait
dispenser. Grosses, malades, moins de six semaines après leurs couches,
dans d'autres temps fâcheux, il fallait être en grand habit, parées et
serrées dans leurs corps, aller en Flandre et plus loin encore, danser,
veiller, être des fêtes, manger, être gaies et de bonne compagnie,
changer de lieu, ne paraître craindre, ni être incommodées du chaud, du
froid, de l'air, de la poussière, et tout cela précisément aux jours et
aux heures marqués, sans déranger rien d'une minute.

Ses filles, il les a traitées toutes pareillement. On a vu en son temps
qu'il n'eut pas plus de ménagement pour M\textsuperscript{me} la
duchesse de Berry, ni même pour M\textsuperscript{me} la duchesse de
Bourgogne, quoi que Fagon, M\textsuperscript{me} de Maintenon, etc.,
pussent dire et faire (quoiqu'il aimât M\textsuperscript{me} la duchesse
de Bourgogne aussi tendrement qu'il en était capable) qui toutes les
deux s'en blessèrent, et ce qu'il en dit avec soulagement, quoiqu'il n'y
eût point encore d'enfants.

Il voyageait toujours son carrosse plein de femmes\,: ses maîtresses,
après ses bâtardes, ses belles-filles, quelquefois Madame, et des dames
quand il y avait place. Ce n'était que pour les rendez-vous de chasse,
les voyages de Fontainebleau, de Chantilly, de Compiègne, et les vrais
voyages, que cela était ainsi. Pour aller tirer, se promener, ou pour
aller coucher à Marly ou à Meudon, il allait seul dans une calèche. Il
se déliait des conversations que ses grands officiers auraient pu tenir
devant lui dans son carrosse\,; et on prétendait que le vieux Charost,
qui prenait volontiers ces temps-là pour dire bien des choses, lui avait
fait prendre ce parti, il y avait plus de quarante ans. Il convenait
aussi aux ministres qui, sans cela, auraient eu de quoi être inquiets
tous les jours, et à la clôture exacte qu'en leur faveur lui-même
s'était prescrite, et à laquelle il fut si exactement fidèle. Pour les
femmes, ou maîtresses d'abord, ou filles ensuite, et le peu de dames qui
pouvaient y trouver place, outre que cela ne se pouvait empêcher, les
occasions en étaient restreintes à une grande rareté, et le babil fort
peu à craindre.

Dans ce carrosse, lors des voyages, il y avait toujours beaucoup de
toutes sortes de choses à manger\,: viandes, pâtisseries, fruits. On
n'avait pas sitôt fait un quart de lieue que le roi demandait si on ne
voulait pas manger. Lui jamais ne goûtait à rien entre ses repas, non
pas même à aucun fruit, mais il s'amusait à voir manger, et manger à
crever. Il fallait avoir faim, être gaies, et manger avec appétit et de
bonne grâce, autrement il ne le trouvait pas bon, et le montrait même
aigrement. On faisait la mignonne, on voulait faire la délicate, être du
bel air, et cela n'empêchait pas que les mêmes dames ou princesses qui
soupaient avec d'autres à sa table le même jour, ne fussent obligées,
sous les mêmes peines, d'y faire aussi bonne contenance que si elles
n'avaient mangé de la journée. Avec cela, d'aucuns besoins il n'en
fallait point parler, outre que pour des femmes ils auraient été très
embarrassants avec les détachements de la maison du roi, et les gardes
du corps devant et derrière le carrosse, et les officiers et les écuyers
aux portières, qui faisaient une poussière qui dévorait tout ce qui
était dans le carrosse. Le roi, qui aimait l'air, en voulait toutes les
glaces baissées, et aurait trouvé fort mauvais que quelque dame eût tiré
le rideau contre le soleil, le vent ou le froid. Il ne fallait seulement
pas s'en apercevoir, ni d'aucune autre sorte d'incommodité, et {[}le
roi{]} allait toujours extrêmement vite, avec des relais le plus
ordinairement. Se trouver mal était un démérite à n'y plus revenir.

J'ai ouï conter à la duchesse de Chevreuse, que le roi a toujours fort
aimée et distinguée, et qu'il a, tant qu'elle l'a pu, voulu avoir
toujours dans ses voyages et dans ses particuliers, qu'allant dans son
carrosse avec lui de Versailles à Fontainebleau, il lui prit au bout de
deux lieues un de ces besoins pressants auxquels on ne croit pas pouvoir
résister. Le voyage était tout de suite, et le roi arrêta en chemin,
pour dîner sans sortir de son carrosse. Ces besoins, qui redoublaient à
tous moments, ne se faisaient pas sentir à propos, comme à cette dînée,
où elle eût pu descendre un moment dans la maison vis-à-vis. Mais le
repas, si ménagé qu'elle le pût faire, redoubla l'extrémité de son état.
Prête par moments à être forcée de l'avouer et de mettre pied à terre,
prête aussi très souvent à perdre connaissance, son courage la soutint
jusqu'à Fontainebleau où elle se trouva à bout. En mettant pied à terre,
elle vit le duc de Beauvilliers, arrivé de la veille avec les enfants de
France, à la portière du roi. Au lieu de monter à sa suite, elle prit le
duc par le bras, et lui dit qu'elle allait mourir si elle ne se
soulageait. Ils traversèrent un bout de la cour Ovale, et entrèrent dans
la chapelle de cette cour, qui heureusement se trouva ouverte, et où on
disait des messes tous les matins. La nécessité n'a point de loi\,;
M\textsuperscript{me} de Chevreuse se soulagea pleinement dans cette
chapelle, derrière le duc de Beauvilliers qui en tenait la porte. Je
rapporte cette misère pour montrer quelle était la gêne qu'éprouvait
journellement ce qui approchait le roi avec le plus de faveur et de
privance, car c'était alors l'apogée de celle de la duchesse de
Chevreuse. Ces choses qui semblent des riens, et qui sont des riens en
effet, caractérisent trop pour les omettre. Le roi avait quelquefois des
besoins, et ne se contraignait pas de mettre pied à terre. Alors les
dames ne bougeaient de carrosse.

M\textsuperscript{me} de Maintenon, qui craignait fort l'air et bien
d'autres incommodités, ne put gagner là-dessus aucun privilège. Tout ce
qu'elle obtint, sous prétexte de modestie et d'autres raisons, fut de
voyager à part, de la manière que je l'ai rapporté\,; mais, en quelque
état qu'elle fût, il fallait marcher, et suivre à point nommé, et se
trouver arrivée et rangée avant que le roi entrât chez elle. Elle fit
bien des voyages à Marly dans un état à ne pas faire marcher une
servante. Elle en fit un à Fontainebleau qu'on ne savait pas
véritablement si elle ne mourrait pas en chemin. En quelque état qu'elle
fût, le roi allait chez elle à son heure ordinaire, et y faisait ce
qu'il avait projeté\,; tout au plus elle était dans son lit, plusieurs
fois y suant la fièvre à grosses gouttes. Le roi qui, comme on l'a dit,
aimait l'air, et qui craignait le chaud dans les chambres, s'étonnait en
arrivant de trouver tout fermé, et faisait ouvrir les fenêtres, et n'en
rabattait rien, quoiqu'il la vît dans cet état, et jusqu'à dix heures
qu'il s'en allait souper, et sans considération pour la fraîcheur de la
nuit. S'il devait y avoir musique, la fièvre, le mal de tête n'empêchait
rien\,; et cent bougies dans les yeux. Ainsi le roi allait toujours son
train, sans lui demander jamais si elle n'en était point incommodée.

Les gens de M\textsuperscript{me} de Maintenon, car tout en est curieux,
étaient en très petit nombre, peu répandus, modestes, respectueux,
humbles, silencieux, et ne s'en firent jamais accroire. C'était l'air de
la maison, et ils n'y seraient pas demeurés sans cela. Ils y faisaient
avec le temps une fortune modérée, suivant leur état, et qui ne pouvait
donner d'envie ni occasion de parler\,; tous demeuraient dans une
obscurité plus ou moins aisée. Ses femmes passaient leur vie enfermées
chez elles. Non seulement elle ne voulait point qu'elles sortissent,
mais elle les empêchait de recevoir personne, et la fortune qu'elle leur
faisait était courte et rare. Le roi les connaissait toutes et tous\,;
il était familier avec eux, et y causait souvent, lorsqu'il passait
quelquefois chez elle avant qu'elle y fût rentrée.

Il n'y avait d'un peu distingué que cette ancienne servante du temps
qu'après la mort de Scarron elle était à la charité de Saint-Eustache,
logée dans cette montée où cette servante faisait sa chambre et son
petit pot-au-feu dans la même chambre. Nanon de ce temps-là, et que
M\textsuperscript{me} de Maintenon a toujours appelée ainsi, qui d'abord
avait été son unique domestique, et qui l'avait constamment suivie et
servie dans tous ses divers états, était devenue M\textsuperscript{lle}
Balbien, dévote comme elle, et vieille. Elle était d'autant plus
importante qu'elle avait toute la confiance domestique de
M\textsuperscript{me} de Maintenon, et l'oeil sur ces demoiselles qu'on
a vu ailleurs qui se succédaient de Saint-Cyr auprès d'elle, sur ses
nièces, et sur M\textsuperscript{me} la duchesse de Bourgogne même, qui
ne l'ignorait pas, et qui habilement, sans la gâter, en avait fait sa
bonne amie. Elle se coiffait et s'habillait comme sa maîtresse\,; elle
affectait d'en tout imiter. À commencer par les enfants légitimes et les
bâtards, à continuer par les princes du sang et par les ministres, il
n'y avait celui ni celle qui ne la ménageât, et qui ne fût en
contrainte, et, le dirai-je, en respect devant elle. S'en servait qui
pouvait pour de l'argent, quoique au fond elle se mêlât de fort peu de
chose. Elle était très raisonnablement sotte, et n'était méchante que
rarement, et encore par bêtise, quoique ce fût une personne toute
composée, toute sur le merveilleux, et qui ne se montrait presque
jamais. On en a pourtant vu un échantillon à propos de la place qu'eut
la duchesse du Lude, que quatre heures devant le roi avait paru si
éloigné de lui donner. Sa protection pour aller à Marly ne lui fut pas
infructueuse. Elle avait l'air doux, humble, empesé, important, et
toutefois respectueux.

On l'a dit, M\textsuperscript{me} de Maintenon était particulière en
public\,; hors de ses yeux, reine\,; quelquefois même sous ses yeux,
comme à l'attaque de Compiègne dont il a été parlé ici en son temps, et
aux promenades de Marly, quand par complaisance elle en faisait
quelqu'une où le roi voulait lui montrer quelque chose de nouvellement
achevé. Je me trouve, je l'avoue, entre la crainte de quelques redites
et celle de ne pas expliquer assez en détail des curiosités que nous
regrettons dans toutes les histoires, et dans presque tous les Mémoires
des divers temps. On voudrait y voir les princes, avec leurs maîtresses
et leurs ministres, dans leur vie journalière. Outre une curiosité si
raisonnable, on en connaîtrait bien mieux les mœurs du temps et le génie
des monarques, celui de leurs maîtresses et de leurs ministres, de leurs
favoris, de ceux qui les ont le plus approchés, et les adresses qui ont
été employées pour les gouverner ou pour arriver aux divers buts qu'on
s'est proposés. Si ces choses doivent passer pour curieuses, et même
pour instructives dans tous les règnes, à plus forte raison d'un règne
aussi long et aussi rempli que l'a été celui de Louis XIV, et d'un
personnage unique dans la monarchie depuis qu'elle est connue, qui a,
trente-deux ans durant, revêtu ceux de confidente, de maîtresse,
d'épouse, de ministre, et de toute-puissante, après avoir été si
longuement néant, et comme on dit, avoir si longtemps et si publiquement
rôti le balai. C'est ce qui m'enhardit sur l'inconvénient des redites.
Tout bien considéré, j'estime qu'il vaut mieux hasarder qu'il m'en
échappe quelqu'une que de ne pas mettre sous les yeux un tout ensemble
si intéressant. Revenons donc un moment sur nos pas.

Reine dans le particulier, M\textsuperscript{me} de Maintenon n'était
jamais que dans un fauteuil, et dans le lieu le plus commode de sa
chambre, devant le roi, devant toute la famille royale, même devant la
reine d'Angleterre. Elle se levait tout au plus pour Monseigneur et pour
Monsieur, parce qu'ils allaient rarement chez elle\,; M. le duc
d'Orléans, ni aucun prince du sang, jamais que par audiences, et comme
jamais\,; mais Monseigneur, Mgrs ses fils, Monsieur et M. le duc de
Chartres, toujours en partant pour l'armée, et le soir même qu'ils en
arrivaient, ou, s'il était trop tard, de bonne heure le lendemain. Pour
aucun autre fils de France, leurs épouses, ou les bâtards du roi, elle
ne se levait point, ni pour personne, sinon un peu pour les personnes
ordinaires avec qui elle n'avait point de familiarité, et qui en
obtenaient des audiences\,; car modeste et polie, elle l'a toujours
affecté à ces égards-là.

Presque jamais elle n'appelait M\textsuperscript{me} la Dauphine que
mignonne, même en présence du roi et des dames familières et des dames
du palais, et cela jusqu'à sa mort, et quand elle parlait d'elle ou de
M\textsuperscript{me} la duchesse de Berry, et devant les mêmes, jamais
elle ne disait que la duchesse de Bourgogne et la duchesse de Berry, ou
la Dauphine, très rarement M\textsuperscript{me} la Dauphine, et de même
le duc de Bourgogne, le duc de Berry, le Dauphin, presque jamais M. le
Dauphin\,; on peut juger des autres.

On a vu comment elle mandait les princesses, légitimes et bâtardes,
comme elle leur lavait la tête, les transes avec quoi elles venaient à
ses ordres, les pleurs avec lesquels elles s'en retournaient, et leurs
inquiétudes tant que la disgrâce durait, et qu'il n'y avait que
M\textsuperscript{me} la duchesse de Bourgogne qui eût pris le dessus
avec les grâces nonpareilles et ce soin attentif qu'on en a vu en
parlant d'elle. Elle ne l'appelait jamais que ma tante.

Ce qui étonnait toujours, c'étaient les promenades qu'on vient de dire
qu'elle faisait avec le roi par excès de complaisance dans les jardins
de Marly. Il aurait été cent fois plus librement avec la reine, et avec
moins de galanterie. C'était un respect le plus marqué, quoique au
milieu de la cour et en présence de tout ce qui s'y voulait trouver des
habitants de Marly. Le roi s'y croyait en particulier, par ce qu'il
était à Marly. Leurs voitures allaient joignant à côté l'une de l'autre,
car presque jamais elle ne montait en chariot\,: le roi seul dans le
sien, elle dans une chaise à porteurs. S'il y avait à leur suite
M\textsuperscript{me} la Dauphine ou M\textsuperscript{me} la duchesse
de Berry, ou des filles du roi, elles suivaient ou environnaient à pied,
ou si elles montaient en chariot avec des dames, c'était pour suivre, et
à distance, sans jamais doubler. Souvent le roi marchait à pied à côté
de la chaise. À tous moments il ôtait son chapeau et se baissait pour
parler à M\textsuperscript{me} de Maintenon, ou pour lui répondre, si
elle lui parlait, ce qu'elle faisait bien moins souvent que lui, qui
avait toujours quelque chose à lui dire ou à lui faire remarquer. Comme
elle craignait l'air dans les temps même les plus beaux et les plus
calmes, elle poussait à chaque fois la glace de côté de trois doigts, et
la refermait incontinent. Posée à terre à considérer la fontaine
nouvelle, c'était le même manège. Souvent alors la Dauphine se venait
percher sur un des bâtons de devant, et se mettait de la conversation,
mais la glace de devant demeurait toujours fermée. À la fin de la
promenade, le roi conduisait M\textsuperscript{me} de Maintenon
jusqu'auprès du château, prenait congé d'elle, et continuait sa
promenade. C'était un spectacle auquel on ne pouvait s'accoutumer. Ces
bagatelles échappent presque toujours aux Mémoires. Elles donnent
cependant plus que tout l'idée juste de ce que l'on y recherche, qui est
le caractère de ce qui a été, qui se présente ainsi naturellement par
les faits.

La conduite des belles-petites-filles du roi et de ses bâtardes, les
ordres à y mettre et à y donner, les galanteries et la dévotion, ou la
régularité des dames de la cour, les aventures diverses, le maintien des
femmes des ministres, et celui des ministres mêmes, les espionnages de
toutes les sortes dont la cour était pleine, les parties qui se
faisaient de ces princesses avec les jeunes dames, ou celles de leur
âge, et tout ce qui s'y passait, les punitions qui allaient quelquefois
à être en pénitence, et même chassé\,; les récompenses, qui étaient la
distribution arrêtée tout à fait, ou plus ou moins fréquente des
distinctions, d'être des voyages de Marly, ou des amusements de la
Dauphine, toutes ces choses entraient dans les occupations de
M\textsuperscript{me} de Maintenon. Elle en amusait le roi, enclin à les
prendre sérieusement\,; elles étaient utiles à entretenir la
conversation, à servir ou à nuire, et à prendre de loin des tournants
auprès du roi sur bien des choses qu'elle y savait habilement faire
entrer de droite et de gauche.

On a déjà vu qu'elle répondait à tout ce qui avait recours à elle\,:
qu'elle ne se mêlait de rien\,; et que ce qui l'approchait de bien près
n'avait pas peu à essuyer de cette prodigieuse inconstance naturelle,
qui, sans autre cause, changeait si souvent ses goûts, ses inclinations,
ses volontés. Les remèdes qu'on y cherchait y étaient des poisons.
L'unique parti à prendre était de glisser, de se tenir plus réservé,
plus à l'écart, comme on se met à couvert de la pluie en se détournant
un peu de son chemin. Quelquefois elle se rapprochait et se rouvrait
d'elle-même, comme d'elle-même elle s'était fermée et éloignée, sinon il
n'y avait point de ressource à espérer. Ces mutations qui étaient
également en gens et en choses, étaient accablantes pour les ministres,
pour les personnes qui se trouvaient en quelque commerce d'affaires avec
elle, et pour les femmes dont en très petit nombre et très rare elle
s'était imaginée de vouloir régler la conduite. Ce qui lui plaisait
hier, pas plus loin que cela, était un démérite aujourd'hui. Ce qu'elle
avait approuvé, même suggéré, elle le blâmait ensuite, tellement qu'on
ne savait jamais si on était digne d'amour ou de haine. C'eût été se
perdre de lui montrer en excuse cette variation, qui s'étendait sur ces
personnes choisies, jusqu'à leur manière de s'habiller et de se coiffer,
et personne de tout ce qui à divers titres l'a approchée de près n'a été
exempt, plus ou moins, de ces hauts et bas insupportables. La domination
et le gouvernement furent les seules choses sur lesquelles elle n'en eut
jamais.

\hypertarget{chapitre-iv.}{%
\chapter{CHAPITRE IV.}\label{chapitre-iv.}}

~

{\textsc{Adresse de M\textsuperscript{me} de Maintenon à se saisir des
affaires ecclésiastiques.}} {\textsc{- Innocence éminente de la vie et
de la fortune du cardinal de Noailles.}} {\textsc{- Cabales dévotes.}}
{\textsc{- Utilité de la constitution à M\textsuperscript{me} de
Maintenon.}} {\textsc{- Malheurs des dernières années du roi le rendent
plus dur et non moins dupe.}} {\textsc{- Adresse de Mansart.}}
{\textsc{- Malheurs du roi dans sa famille et dans son intime
domestique, et sa grandeur dans les revers de la fortune.}} {\textsc{-
Le roi considéré à l'égard de ses bâtards.}} {\textsc{- Piété et fermeté
du roi jusqu'à sa mort.}} {\textsc{- Réflexions.}} {\textsc{- Jésuites
laïques.}} {\textsc{- Autres réflexions.}} {\textsc{- Abandon du roi aux
derniers jours de sa vie.}} {\textsc{- Horreur du duc du Maine.}}

~

On a vu avec quelle adresse elle {[}M\textsuperscript{me} de
Maintenon{]} se servit de la princesse des Ursins pour se mêler de tout
ce qui regarda la cour et les affaires d'Espagne, et les ôter de la main
de Torcy autant qu'elle le put pour avoir échoué à faire venir
travailler chez elle ce ministre, comme faisaient les autres, et jusqu'à
quel point M\textsuperscript{me} des Ursins en sut profiter. Les
affaires ecclésiastiques furent de même bien longtemps l'objet de son
envie. Elle leur donna quelques légères atteintes à l'occasion du
jansénisme et de la révocation de l'édit de Nantes, comme on l'a vu,
mais passagèrement, et on n'a fait qu'effleurer ce grand objet, qui fut
la cause de sa préférence pour le duc de Noailles, en parlant de ce
mariage en son temps. Il faut maintenant expliquer mieux comment elle
réussit enfin à entrer aussi dans les matières ecclésiastiques, et à
prendre aussi une part principale dans cette partie du gouvernement.

Elle vit longtemps avec grande amertume le P. de La Chaise en possession
de tout ce ministère, non seulement avec une entière indépendance
d'elle, mais sans aucuns devoirs de sa part, et elle dans une entière
ignorance à cet égard. L'éloignement du roi marqué pour Harlay,
archevêque de Paris, après une faveur si entière et si longue, avait
satisfait sa vengeance\,: on en a vu la cause, mais non ses désirs. Le
confesseur du roi n'en était devenu que plus maître des bénéfices, et de
tout ce qui regardait les affaires dont l'archevêque avait été tout à
fait écarté. C'est ce qui donna si peu de goût à M\textsuperscript{me}
de Maintenon pour le mariage de sa nièce avec le petit-fils du duc de La
Rochefoucauld, qu'on a vu que le roi voulait faire, et qui en valut la
préférence aux Noailles. Je n'assurerai pas que ce fut dans cette vue
éloignée qu'elle leur aida à faire nommer le frère du maréchal-duc de
Noailles à l'archevêché de Paris, à la mort d'Harlay, en août 1695,
chose d'autant plus difficile que les jésuites ne l'aimaient pas, que le
roi ne le connaissait comme point, parce qu'il ne venait presque jamais
à Paris, et encore pour des moments, et qu'il fallut le porter à Paris
sans aucune participation du P. de La Chaise.

On ne put même l'y bombarder à l'insu du confesseur, parce qu'il fallut
forcer ce prélat, qui non seulement fit toute la résistance qui lui fut
possible, mais qui affecta de se rendre suspect du côté de la doctrine.
Il avait d'abord été nommé à l'évêché de Cahors. Quelques mois après il
fut transféré à Châlons. La proximité ni la dignité de ce siège, dont
l'évêque est comte et pair de France, ne purent le résoudre à quitter
l'épouse à laquelle il avait été destiné par son sacre, quoiqu'il ne pût
encore l'avoir connue\,; il fallut un commandement exprès du pape pour
l'y obliger.

Il brilla à Châlons avec les mœurs d'un ange, par une résidence
continuelle, une sollicitude pastorale, douce, appliquée, instructive,
pleine des plus grands exemples, et une désoccupation totale de tout ce
qui n'était point de son ministère. Le crédit de sa famille armée d'une
si grande réputation l'emporta sur les voies ordinaires. Il réussit à
Paris comme il avait fait à Châlons, sans être ébloui d'un si grand
théâtre\,; il plut extrêmement au roi et à M\textsuperscript{me} de
Maintenon, et pour achever ce qui le regarde ici personnellement, il ne
parut ni neuf ni embarrassé aux affaires, et il fit admirer ses
lumières, son savoir, et ce qui est fort rare en même temps sa modestie
et une magnificence convenable, aux assemblées du clergé où il présida
au gré du clergé et de la cour. Enfin il fut cardinal en 1700 avec la
même répugnance qu'il avait eue à changer de siège.

Tant de vertus reçurent à la fin la récompense que le monde leur donne,
beaucoup de croix et de tribulations qu'il porta avec courage, et pour
le bien de l'Église avec trop de douceur, d'équanimité, de crainte de se
retrouver soi-même, de ménagement et de charité pour ceux qui en surent
étrangement profiter, et qui ont achevé de l'épurer et de le sanctifier,
sans avoir pu ébranler son âme, ni la pureté de ses intentions et de sa
doctrine. Car pour ses dernières années, la tête n'y était plus\,; elle
avait succombé sous le poids des années, des travaux, de la persécution.
J'en ai été le témoin oculaire, et si Dieu m'en accorde le temps, je ne
le laisserai pas ignorer à la fin de ces Mémoires, quoique cet événement
outrepasse les bornes que je m'y suis proposées.

On ne répétera pas ce qu'on a vu sur Godet, évêque de Chartres, ni même
sur Bissy, depuis cardinal. On se contentera de faire souvenir ici que
La Chétardie dont on a parlé au long aux mêmes dernières pages, et Bissy
alors, n'étaient pas à portée du roi, et que Godet, qui n'avait point
d'occasion ordinaire d'approcher du roi, ne pouvait que s'y présenter de
front et à découvert bien rarement, sur chose préparée par
M\textsuperscript{me} de Maintenon. Mais il n'y pouvait revenir souvent,
ni être à portée de ces puissants moyens d'insinuation qui opèrent tout
avec de la suite par des conversations fréquentes sans objet apparent.
Le P. de La Chaise les avait tous, et se gardait fort d'être emblé, ni
même écorné par l'évêque de Chartres, qui lui en donnait pourtant
quelquefois, et dont chaque écorne le réveillait et le rendait plus
attentif.

Un archevêque de Paris, avec la grâce du choix tout frais et de la
nouveauté, porté par sa réputation, par une famille si établie, et par
tout l'art de M\textsuperscript{me} de Maintenon qui tout d'abord comme
son ouvrage l'avait pris en grand goût, était un instrument bien plus à
la main avec un jour d'audience du roi réglé par semaine, et toujours
matière à la fournir, et même à la redoubler quand il en avait envie.
C'est ce qui forma cette grande faveur, dont sa droiture et ses
ménagements de conscience, si fort en garde contre soi-même, et si peu
contre les autres, perdirent tous les avantages dans les suites, mais
dont M\textsuperscript{me} de Maintenon sut tirer tous les siens pour
entrer enfin dans les matières ecclésiastiques.

Elle s'y initia par l'affaire de M. de Cambrai qui lia si étroitement
l'archevêque de Paris avec elle, et avec M. de Chartres. Par ce moyen
elle saisit auprès du roi la clef de la seule espèce d'affaires et de
grâces où jusqu'alors elle n'avait pu donner que de légères atteintes,
et c'est ce qui lui fit préférer le neveu de l'archevêque de Paris à
tout autre mariage, en mars 1698. Elle fit, comme on l'a vu, épouser au
roi la querelle contre M. de Cambrai à Rome, jusqu'à en faire sa propre
affaire à découvert, et par là, s'établir de plus en plus dans la
confiance des matières de religion qui entraînaient si nécessairement
celles des bénéfices, et les moyens d'avancer et de reculer qui bon lui
semblait.

On a vu que M. de Chartres était passionné sulpicien, qu'il logeait
toujours à Paris dans ce séminaire, qu'il l'éleva sur les ruines de
celui des Missions étrangères de Saint-Magloire, et des pères de
l'Oratoire\,; enfin qu'il se substitua, en mourant, La Chétardie, curé
de Saint-Sulpice, auprès de M\textsuperscript{me} de Maintenon, qu'il
dirigea, et dont il eut toute la confiance.

Il faut le dire encore, la crasse ignorance des sulpiciens, leur
platitude suprême, leurs sentiments follement ultramontains, ne
pouvaient barrer les vastes desseins des jésuites, et ils étaient tout
ce qu'il leur fallait pour ruiner l'élévation, l'excellente morale, le
goût de l'antiquité, le savoir juste et exact qu'on puisait chez les
pères de l'Oratoire, si éloignés en tout des sentiments de la compagnie,
et si conformes pour le gros avec l'Université, et les restes précieux
du fameux Port-Royal, dont les jésuites étaient les ennemis et les
persécuteurs. Ils en achevaient ainsi la ruine par des gens dévoués à
Rome par une conscience stupide, qui mettaient tout le mérite en des
pratiques basses, vaines, ridicules, sous le poids desquelles ils
abrutissaient les jeunes gens qui leur étaient confiés, à qui ils ne
pouvaient rien apprendre, parce qu'eux-mêmes ne savaient rien du tout,
pas même vivre, marcher, ni dire quoi que ce soit à propos. Aussi la
vogue des prêtres de la Mission, dont l'institut n'était que faire le
catéchisme dans les villages, et qui ne s'étaient pas rendus capables de
mieux, et de ceux de Saint-Sulpice aussi grossiers, aussi ignorants, et
aussi ultramontains les uns que les autres, prit le grand vol, parce que
la porte des bénéfices fut fermée à la fin à tout ce qui n'était pas
élevé chez eux.

M\textsuperscript{me} de Maintenon, séduite par La Chétardie et par
Bissy, sur les mêmes voies dont le feu évêque de Chartres l'avait de
longue main entêtée, régnait sur ces nouveaux séminaires de mode. Elle
en était devenue la protectrice déclarée depuis que l'art des jésuites
l'avait brouillée sans y paraître avec les directeurs des Missions
étrangères qui avaient été longtemps ses directeurs à elle-même,
auxquels M. de Chartres succéda auprès d'elle, lorsque la fameuse
affaire des cérémonies chinoises et indiennes brouilla les Missions
étrangères avec les jésuites de la manière la plus éclatante et la plus
irréconciliable. Ce n'est pas que les jésuites n'eussent de la jalousie
de cette basse prêtraille qui usurpait trop de crédit à leur gré, et
réciproquement ceux-ci des jésuites, mais ils se souffraient et vivaient
bien ensemble par le besoin qu'ils avaient les uns des autres dans leur
haine commune des pères de l'Oratoire, et du clergé éclairé qu'ils
taxaient à tout hasard de jansénisme.

À la tête de ceux-ci était le cardinal de Noailles qui avait bien la
science des saints mais non assez de celle des hommes pour les soutenir,
ni pour se soutenir lui-même\,; trop de droiture, de conscience, de
piété pour prévoir, ni pour remédier après avoir éprouvé.

Bissy, qui de loin, et dès Toul, avait su prendre ses contours secrets
par les jésuites, par Saint-Sulpice, par M. de Chartres qui s'en était
entêté, et qui le laissa à M\textsuperscript{me} de Maintenon comme son
Élisée, allait au grand, et sentit le besoin qu'il avait de quelque
grande affaire par le cours et les intrigues de laquelle il pût se
rendre le maître de M\textsuperscript{me} de Maintenon, du roi par elle,
et par un concert étroit et secret, ne faire qu'un avec les jésuites par
leur besoin réciproque, eux de lui auprès de M\textsuperscript{me} de
Maintenon, lui d'eux à Rome, et gouverner ainsi toutes les affaires
ecclésiastiques.

La frayeur que les jésuites avaient conçue de l'élévation du cardinal de
Noailles, sans eux, de sa faveur, de l'appui qu'il trouvait dans sa
famille, s'était tournée en fureur. Leur P. Tellier, que Saint-Sulpice
avait, comme on l'a vu, fait succéder au P. de La Chaise, était un homme
bien différent de lui. Il ne tarda pas à sentir ses forces, à
embarrasser dans ses toiles le cardinal de Noailles, comme une araignée
fait une mouche, à lui susciter mille défensives, à profiter de sa
vertu, de sa candeur, de sa modération, enfin, à le pousser jusqu'à
donner fatalement les mains à la destruction radicale de ce fameux reste
de Port-Royal des Champs, qui palpitait encore, dont la barbare
dispersion de ce qui y restait de religieuses, le rasement des bâtiments
à n'y pas laisser pierre sur pierre, le violement des sépulcres, la
profanation de ce lieu saint réduit en guéret, excita l'indignation
publique, et fit une brèche irréparable au cardinal de Noailles.

De l'un à l'autre, à force des plus profondes menées, se noua la
terrible affaire de la constitution, qui perdit ce cardinal avec
M\textsuperscript{me} de Maintenon, plus encore qu'avec le roi. Les
mêmes intrigues firent déclarer le roi et M\textsuperscript{me} de
Maintenon parties, avec une violence qui fit la fortune de Bissy, et lui
donna toute la confiance de M\textsuperscript{me} de Maintenon qui
n'aimait pas les jésuites ni le P. Tellier.

Ainsi Bissy au comble de ses voeux, après tant d'années de soupirs et
d'intrigues, devint le premier personnage\,; et jusqu'à quel point n'en
abusa-t-il pas, tandis que M\textsuperscript{me} de Maintenon était la
dupe de son hypocrisie\,! Trompée qu'elle fut par ses souplesses, ses
bassesses, et par les éloges qu'il lui donnait avec sa fausse
simplicité, et son apparence grossière, elle se crut la prophétesse qui
sauvait le peuple de Dieu de l'erreur, de la révolte et de l'impiété.
Dans cette idée, excitée par Bissy, et pour se mêler de plus en plus des
choses ecclésiastiques, elle anima le roi à toutes les horreurs, à
toutes les violences, à toute la tyrannie qui furent alors exercées sur
les consciences, les fortunes, et les personnes, dont les prisons et les
cachots furent remplis. Bissy lui suggérait tout, et obtenait tout.

Ce fut alors qu'elle nagea en plein dans la direction des affaires de
l'Église, et il fallut que le P. Tellier, malgré toutes ses profondeurs,
vînt par Bissy compter avec elle jusque sur la distribution des
bénéfices. Cela lui pesait cruellement, mais la persécution qu'il avait
entreprise, la perte surtout du cardinal de Noailles qu'il ne prétendait
pas dépouiller de moins que de la pourpre, de son siège et de la
liberté, enfin le triomphe de leur moderne école sur la ruine de toutes
les autres, étaient pour lui des objets si intéressants et si vifs,
qu'il n'y avait chose qu'il ne leur sacrifiât.

On a vu qu'il n'y en eut qu'une qu'il ne put digérer. Ce fut le choix de
Fleury pour précepteur. Lui était nommé confesseur et sous-précepteur.
Il lui était donc capital pour être le maître, et il le voulait être
partout, de faire un précepteur à son gré. Il s'y opposa en face entre
le roi et M\textsuperscript{me} de Maintenon dans la chambre de
celle-ci, et si ses efforts ne réussirent pas, ce ne fut pas sans lui en
avoir donné toute la peur, et Fleury ne l'a oublié de sa vie. Il ne lui
en fallait pas tant pour ne jamais pardonner.

Tellier n'a pas assez vécu pour voir, ni même pour se douter du succès
inouï de ce premier degré de fortune. S'il l'avait vu d'où il est, et
que de là on fût aussi sensible aux mêmes passions qui ont occupé tout
entières nos âmes pendant leur union avec leurs corps, il aurait su bien
bon gré aux jésuites de l'art infini avec lequel ils parvinrent à manier
ce maître du royaume malgré tout son éloignement d'eux, et se servir de
lui, sans qu'il s'en soit jamais douté, à tout ce qui leur fut utile,
pour ruiner tout ce qu'ils haïssaient et craignaient, et pour y
substituer tout ce qui leur fut avantageux. Mais ce n'est pas ici le
lieu ni le temps de s'étendre sur cette matière.

Celle de la constitution, poursuivie avec tant de suite, d'artifices,
d'acharnement, de violence et de tyrannie, fut donc, comme on l'a vu, le
fruit amer de la nécessité pressante où les affaires indiennes et
chinoises réduisirent les jésuites, de l'ambition démesurée de Bissy
pour sa fortune, de celle de Rohan pour augmenter la sienne du moment
que Tallard pour ses vues personnelles l'y eut déterminé, et tous deux
pour être chefs du parti tout-puissant\,; enfin de l'intérêt de
M\textsuperscript{me} de Maintenon de gouverner l'Église comme elle
faisait l'État depuis si longtemps, et que cette partie principale
n'échappât plus à sa domination. Ce champ une fois ouvert, il n'y eut
plus de bornes.

Le goût changeant de M\textsuperscript{me} de Maintenon s'était dépris
du cardinal de Noailles à force d'artifices de Bissy, et des sulpiciens
et missionnaires, aiguisés et soufflés par les jésuites. Elle n'avait
plus besoin de lui pour s'initier dans les affaires ecclésiastiques. Ce
pont dont elle s'était pour cela si utilement servie n'avait plus
d'usage. Engouée de la nouveauté de Bissy, l'Élisée du feu évêque de
Chartres auprès d'elle, et l'admiration de l'idiot La Chétardie divinisa
toute sa conduite à ses propres yeux. Son alliance avec les Noailles,
son ancienne amitié pour le cardinal de Noailles, qui se tournèrent en
fureur contre lui, l'enfla comme d'un sacrifice fait à la vérité et à la
soumission à l'Église.

La conduite barbare qu'on avait tenue avec les huguenots après la
révocation de l'édit de Nantes devint en gros le modèle de celle qu'on
tint\,; et souvent toute la même, à l'égard de tout ce qui ne put goûter
la constitution. De là les artifices sans nombre pour intimider et
gagner les évêques, les écoles, le second ordre et le bas clergé, de là
cette grêle immense et infatigable de lettres de cachet\,; de là cette
butte avec les parlements\,; de là ces évocations sans nombre ni mesure,
cette interdiction de tous les tribunaux\,; enfin, ce déni total et
public de justice, et de tous moyens d'en pouvoir être protégé pour
quiconque ne ployait pas sa conscience sous le joug nouveau, et même
encore sous la manière dont il était présenté\,; de là cette inquisition
ouverte jusque sur les simples laïques, et la persécution ouverte\,; ce
peuple entier d'exilés et d'enfermés dans les prisons, et beaucoup dans
les cachots, et le trouble et la subversion dans les monastères\,; de
là, enfin, cet inépuisable pot au noir pour barbouiller qui on voulait,
qui ne s'en pouvait douter, pour estropier auprès du roi qui on jugeait
à propos des gens de la cour et du monde, pour écarter et pour proscrire
toutes sortes de personnes, et disposer de leurs places à la volonté des
chefs du parti régnant\,; des jésuites et de Saint-Sulpice, qui
pouvaient tout en ce genre, et qui obtenaient tout sans le plus léger
examen\,; de là ce monde innombrable de personnes de tout état et de
tout sexe dans les mêmes épreuves que les chrétiens soutinrent sous les
empereurs ariens, surtout sous Julien l'Apostat, duquel on sembla
adopter la politique et imiter les violences\,; et s'il n'y eut point de
sang précisément répandu, je dis précisément, parce qu'il en coûta la
vie d'une autre sorte à bien de ces victimes, ce ne fut pas la faute des
jésuites, dont l'emportement surmonta cette fois la prudence, jusqu'à ne
se pas cacher de dire qu'il fallait répandre du sang.

On a vu ailleurs combien le crédit de Godet, évêque de Chartres, avait
perdu l'épiscopat en France en le remplissant de cuistres de séminaires
et de leurs élèves sans science, sans naissance, dont l'obscurité et la
grossièreté faisaient tout le mérite, et que Tellier acheva de
l'anéantir en le vendant à découvert, non pour de l'argent, mais pour
ses desseins, et sous des conventions sur lesquelles son esprit emporté,
violent à l'excès, sa sagacité et ses artificieuses précautions, le
gardèrent de se laisser tromper, dont le secret ne put demeurer
longtemps caché, et dont la découverte ne l'arrêta pas dans la posture
où il était parvenu à se mettre. On peut comprendre et mieux voir
encore, par tout ce qui est arrivé, ce qui se pouvait attendre de tous
ces choix. Bissy, dans les mêmes errements, le soutenait de toutes ses
forces naissantes, et a bien profité depuis de ses leçons. Tels ont été
les funestes ressorts qui ont perdu l'Église de France, et qui, la
dernière de toutes les nationales, l'ont enfin abattue sous le joug de
l'empire romain, lequel par différentes routes avait déjà écrasé toutes
les autres. C'est à quoi la faveur personnelle du cardinal Fleury contre
le P. Quesnel, dont on a vu la cause, a eu l'honneur de mettre le
comble, d'inonder la France non seulement de proscriptions, mais
d'expatriations, de l'accabler de {[}trente mille{]}\footnote{Le nombre
  des lettres de cachet est resté en blanc dans le manuscrit. On a
  supplée cette lacune d'après les histoires du temps.} lettres de
cachet, de compte fait après sa mort dans les bureaux des secrétaires
d'État, et de pourvoir dignement et sûrement après sa mort à la
continuité de sa vengeance.

Telles fuient les dernières années de ce long règne de Louis XIV, si peu
le sien, si continuellement et successivement celui de quelques autres.
Dans ces derniers temps, abattu sous le poids d'une guerre fatale,
soulagé de personne par l'incapacité de ses ministres et de ses
généraux, en proie tout entier à un obscur et artificieux domestique,
pénétré de douleur, non de ses fautes qu'il ne connaissait ni ne voulait
connaître, mais de son impuissance contre toute l'Europe réunie contre
lui, réduit aux plus tristes extrémités pour ses finances et pour ses
frontières, il n'eut de ressource qu'à se reployer sur lui-même, et à
appesantir sur sa famille, sur sa cour, sur les consciences, sur tout
son malheureux royaume cette dure domination, {[}de sorte{]} que pour
avoir voulu trop l'étendre, et par des voies trop peu concertées, il en
avait manifesté la faiblesse, dont ses ennemis abusaient avec mépris.

Retranché jusque dans ses tables à Marly, et dans ses bâtiments, il
éprouvait, jusque dans la bagatelle de ces derniers, les mêmes artifices
par lesquels il était gouverné en grand. Mansart, qui en était le
surintendant peu capable, mais pourtant avec un peu plus de goût que son
maître, l'obsédait avec des projets, qui de l'un à l'autre le
conduisaient aux plus fortes dépenses. C'étaient autant d'occasions de
s'enrichir, où il réussit merveilleusement, et de se perpétuer les
privances qui le rendaient une sorte de personnage que les ministres
mêmes ménageaient, et à qui toute la cour faisait la sienne. Il avait
l'art d'apporter au roi des plans informes, mais qui lui mettaient le
doigt sur la lettre, à quoi ce délié maçon aidait imperceptiblement. Le
roi voyait ainsi, ou le défaut à corriger, ou le mieux à faire. Mansart,
toujours étonné de la justesse du roi, se pâmait d'admiration, et lui
faisait accroire qu'il n'était lui-même qu'un écolier auprès de lui, et
qu'il possédait les délicatesses de l'architecture et des beautés des
jardins aussi excellemment que l'art de gouverner. Le roi l'en croyait
volontiers sur sa parole, et si, comme il arrivait souvent, il
s'opiniâtrait sur quelque chose de mauvais goût, Mansart admirait
également et l'exécutait jusqu'à ce que le goût du changement donnât
ouverture pour y en faire. Avec tout cela Mansart, devenu insolent, se
mit à fatiguer le roi de demandes pour soi et pour les siens, souvent
étranges, et fit si bien, qu'il fut aussi de ceux dont le roi se sentit
fort soulagé quand il mourut. Sa brusque fin fut, comme on l'a vu, le
commencement de la fortune de d'Antin, qui eut sa charge, à la vérité
fort rognée de nom et d'autorité, par le démérite de n'être pas, comme
Mansart, de race et de condition servile. Tant que M\textsuperscript{me}
de Montespan vécut, jamais M\textsuperscript{me} de Maintenon n'avait
souffert qu'il parvînt à mieux qu'à des bagatelles, mais délivré de son
ancienne maîtresse, elle s'adoucit pour son fils qui en sut bien
profiter, et qui marcha depuis à pas de géant dans la privance, et
jusque dans une sorte de confiance du roi, comme il marcha du même pas à
la fortune.

À ces malheurs d'État, il s'en joignit de famille, et les plus sensibles
pour le roi. Il avait tenu avec grand soin les princes du sang fort bas,
instruit par l'expérience de son jeune âge. Leur rang n'était monté que
pour élever les bâtards, encore avec des préférences de ceux-ci pour
leurs principaux domestiques, qu'on a vues en leur lieu infiniment
dégoûtantes pour les princes du sang. De gouvernements ni de charges,
ils n'en avaient que ce qui avait été rendu au grand prince de Condé par
la paix des Pyrénées, non à lui, mais au dernier M. le Prince, son fils,
et continués au fils de ce dernier en épousant une bâtarde, puis au fils
de ce mariage, à la mort de son père. De privances ni d'entrées,
aucunes, sinon par ce mariage, qui n'avait rien communiqué au prince de
Conti\,; et pour le commandement des armées, on a vu avec quel soin ils
en furent tous écartés. Il fallut les derniers malheurs et toute la
faveur personnelle de Chamillart pour oser proposer d'en donner une au
prince de Conti, et par capitulation à M. le duc d'Orléans, pour qui le
roi eut encore moins de répugnance, non comme neveu, mais comme gendre
bâtardement, et quand l'excès de la décadence força enfin le roi de
donner l'armée de Flandre au prince de Conti, il n'était plus temps, et
ce prince, dont toute la vie s'était écoulée dans la disgrâce, mourut
avec le regret de ne jouir pas d'une destination qu'il avait tant et si
inutilement souhaitée, et qu'il avait eu la satisfaction de voir
également désirée par la cour, par les troupes et par toute la France,
desquels tous il était les délices et l'espérance.

On a vu en leur lieu les malheurs de M. le duc d'Orléans en Italie et
l'éclat contre lui en Espagne de la princesse des Ursins, si cruellement
appuyée en France de M\textsuperscript{me} de Maintenon.

Depuis l'année 1709, les plaies domestiques redoublèrent chaque année,
et ne se retirèrent plus de dessus la famille royale. Celle qui causa
trop tard la disgrâce du duc de Vendôme fut d'autant plus cruelle
qu'elle ouvrit peu les yeux. M. le prince de Conti et M. le Prince
furent emportés peu après, à six semaines l'un de l'autre. M. le Duc les
suivit dans l'année, c'est-à-dire dans les douze mois, et le plus vieux
des princes du sang qui restèrent n'avait alors au plus que dix-sept
ans. Monseigneur mourut ensuite. Mais bientôt après le roi fut attaqué
par des coups bien plus sensibles\,; son coeur, que lui-même avait comme
ignoré jusqu'alors par la perte de cette charmante Dauphine\,; son
repos, par celle de l'incomparable Dauphin\,; sa tranquillité sur la
succession à la couronne, par la mort de l'héritier huit jours après, et
par l'âge et le dangereux état de l'unique rejeton de cette précieuse
race, qui n'avait que cinq ans et demi\,: tous ces coups frappés
rapidement, tous avant la paix, presque tous durant les plus terribles
périls du royaume.

Mais qui pourrait expliquer les horreurs qui furent l'accompagnement des
trois derniers, leurs causes et leurs soupçons si diamétralement
opposés, si artificieusement semés et inculqués, et les effets cruels de
ces soupçons jusque dans leur faiblesse\,? La plume se refuse à ce
mystère d'abomination. Pleurons-en le succès funeste, comme la source
d'autres succès horribles dignes d'en être sortie\,; pleurons-les comme
le chef-d'œuvre des ténèbres, de la privation la plus sensible et qui
réfléchira sui la France dans toute la suite des générations, comme le
comble de tous les crimes, comme le dernier sceau des malheurs du
royaume\,; et que toute bouche française en crie sans cesse vengeance à
Dieu\,!

Telles furent les longues et cruelles circonstances des plus douloureux
malheurs qui éprouvèrent la constance du roi, et qui rendirent toutefois
un service à sa renommée plus solide que n'avait pu faire tout l'éclat
de ses conquêtes, ni la longue suite de ses prospérités\,; {[}telle
fut{]} la grandeur d'âme que montra constamment dans de tels et si longs
revers, parmi de si sensibles secousses domestiques, ce roi si accoutumé
au plus grand et au plus satisfaisant empire domestique, aux plus grands
succès au dehors, {[}qui{]} se vit enfin abandonné de toutes parts par
la fortune. Accablé au dehors par des ennemis irrités qui se jouaient de
son impuissance qu'ils voyaient sans ressource, et qui insultaient à sa
gloire passée, il se trouvait sans secours, sans ministres, sans
généraux, pour les avoir faits et soutenus par goût et par fantaisie, et
par le fatal orgueil de les avoir voulu et cru former lui-même. Déchiré
au dedans par les catastrophes les plus intimes et les plus poignantes,
sans consolation de personne, en proie à sa propre faiblesse\,; réduit à
lutter seul contre les horreurs mille fois plus affreuses que ses plus
sensibles malheurs, qui lui étaient sans cesse présentées par ce qui lui
restait de plus cher et de plus intime, et qui abusait ouvertement, et
sans aucun frein, de la dépendance où il s'était laissé tomber, et dont
il ne pouvait et ne voulait pas même se relever quoiqu'il en sentît tout
le poids\,; incapable d'ailleurs et par un goût invinciblement dominant,
et par une habitude tournée en nature, de faire aucune réflexion sur
l'intérêt et la conduite de ses geôliers\,; au milieu de ces fers
domestiques, cette constance, cette fermeté d'âme, cette égalité
extérieure, ce soin toujours le même de tenir tant qu'il pouvait le
timon, cette espérance contre toute espérance, par courage, par sagesse,
non par aveuglement, ces dehors du même roi en toutes choses, c'est ce
dont peu d'hommes auraient été capables, c'est ce qui aurait pu lui
mériter le nom de grand, qui lui avait été si prématuré. Ce fut aussi ce
qui lui acquit la véritable admiration de toute l'Europe, celle de ceux
de ses sujets qui en furent témoins, et ce qui lui ramena tant de coeurs
qu'un règne si long et si dur lui avait aliénés.

Il sut s'humilier en secret sous la main de Dieu, en reconnaître la
justice, en implorer la miséricorde, sans avilir aux yeux des hommes sa
personne ni sa couronne\,; il les toucha au contraire parle sentiment de
sa magnanimité, heureux si, en adorant la main qui le frappait, en
recevant ses coups avec une dignité qui honorait sa soumission d'une
manière si singulièrement illustre, il eût porté les yeux sur des motifs
et palpables et encore réparables, et qui frappaient tous autres que les
siens, au lieu qu'il ne considéra que ceux qui n'avaient plus de remèdes
que l'aveu, la douleur, l'inutile repentir\,!

Quel surprenant alliage de la lumière avec les plus épaisses ténèbres\,!
une soif de savoir tout, une attention à se tenir en garde contre tout,
un sentiment de ses liens, plein même de dépit jusqu'à l'aveu que lui en
entendirent faire les gens du parlement sur son testament, et tôt après
eux la reine d'Angleterre\,; une conviction entière de son injustice et
de son impuissance, témoignée de sa bouche, c'est trop peu dire,
décochée par ses propos à ses bâtards, et toutefois un abandon à eux et
à leur gouvernante devenue la sienne et celle de l'État, et abandon si
entier qu'il ne lui permit pas de s'écarter d'un seul point de toutes
leurs volontés\,; que, presque content de s'être défendu en leur faisant
sentir ses doutes et ses répugnances, {[}il{]} leur immola tout son
état, sa famille, son unique rejeton, sa gloire, son honneur, sa raison,
le mouvement intime de sa conscience, enfin sa personne, sa volonté, sa
liberté, et tout cela dans leur totalité entière, sacrifice digne par
son universalité d'être offert à Dieu seul, si par soi-même il n'eût pas
été abominable. Il le leur fit en leur en faisant sentir tout le vide,
en même temps tout le poids, et tout ce qu'il lui coûtait, pour en
recueillir au moins quelque gré, et soulager sa servitude, sans en avoir
pu rendre son joug plus léger à porter, tant ils sentirent leurs forces,
le besoin pressant et continuel de s'en servir, d'étreindre les chaînes
dont ils avaient su le garrotter, dans la continuelle crainte qu'il ne
leur échappât pour peu qu'ils lui laissassent de liberté.

Ce monarque si altier gémissait dans ses fers, lui qui y avait tenu
toute l'Europe, qui avait si fort appesanti les siens sur ses sujets de
tous états, sur sa famille de tout âge, qui avait proscrit toute liberté
jusqu'à la ravir aux consciences et les plus saintes et les plus
orthodoxes.

Ce gémissement plus fort que lui-même sortit violemment au dehors. Il ne
put être méconnu par ce qu'il dit et à la reine d'Angleterre et aux gens
du parlement\,: \emph{qu'il avait acheté son repos\,;} et qu'en leur
remettant son testament, lui si maître de soi et de ne dire que ce qu'il
voulait et comme il le voulait dire et témoigner, il ne put s'empêcher
de leur dire comme on a vu en son lieu\,: \emph{qu'il lui avait été
extorqué, et qu'on lui avait fait faire ce qu'il ne voulait pas, et ce
qu'il croyait ne pas devoir faire}. Étrange violence, étrange misère,
étrange aveu arraché par la force du sentiment et de la douleur\,!
Sentir en plein cet état et y succomber en plein, quel spectacle\,! Quel
contraste de force et de grandeur supérieure à tous les désastres, et de
petitesse et de faiblesse sous un domestique honteux, ténébreux,
tyrannique\,! et quelle vérification puissante de ce que le Saint-Esprit
a déclaré, dans les livres sapientiaux de l'Ancien Testament, du sort de
ceux qui se sont livrés à l'amour et à l'empire des femmes\,! Quelle fin
d'un règne si longuement admiré, et jusque dans ses derniers revers si
étincelant de grandeur, de générosité, de courage et de force\,! et quel
abîme de faiblesse, de misère, de honte, d'anéantissement, sentie,
goûtée, savourée, abhorrée, et toutefois subie dans toute son étendue,
et sans en avoir pu élargir ni soulager les liens\,! O Nabuchodonosor\,!
qui pourra sonder les jugements de Dieu, et qui osera ne pas s'anéantir
en leur présence\,?

On a vu en son lieu les divers degrés par lesquels les enfants du roi et
de M\textsuperscript{me} de Montespan ont été successivement tirés du
profond et ténébreux néant du double adultère, et portés plus qu'au
juste et parfait niveau des princes du sang, et jusqu'au sommet de
l'habilité de succéder à la couronne, ou en simple usage par adresse, ou
à force ouverte, ou en loi par des brevets, des déclarations, des édits
enregistrés. Le récit de ce nombreux amas de faits formerait seul un
volume, et le recueil de ces monstrueuses pièces en composerait un autre
fort gros. Ce qui est étrange, c'est que dans tous les temps, le roi, à
chaque fois, ne les voulut point accorder au point qu'à chaque fois il
le fit, et qu'il ne les voulut point marier, je dis ses fils, dans
l'intime conviction où il fut toujours de leur néant et de leur bassesse
innée, qui n'était relevée que par l'effort de son pouvoir sans bornes,
et qui après lui ne pouvait que retomber. C'est ce qu'il leur dit plus
d'une fois quand l'un et l'autre lui parlèrent de se marier. C'est ce
qu'il leur répéta au comble de leur grandeur, et à six semaines près de
la fin de sa vie, lorsque, malgré lui, il eut tout violé en leur faveur,
jusqu'à sa propre volonté, qui fléchit sous sa faiblesse. On a vu ce
qu'il leur en dit, on ne peut trop le répéter, et ce qui lui en échappa
aux gens du parlement et à la reine d'Angleterre.

On peut se souvenir aussi de l'ordre qu'on a vu qu'il donna si précis au
maréchal de Tessé, qui me l'a conté et à d'autres, sur M. de Vendôme, de
ne point éviter de le commander en Italie où on l'envoyait, et où
Vendôme était à la tête de l'armée, et {[}de{]} ce qu'il ajouta avec un
air chagrin\,: \emph{qu'il ne fallait pas accoutumer ces messieurs-là},
à ces ménagements, lequel duc de Vendôme bientôt après, parvint, et sans
patente, à commander les maréchaux de France, et ceux-là encore qui
longtemps avant lui avaient commandé des armées.

C'est un malheur dans la vie du roi et une plaie à la France, qui a
continuellement été en augmentant, que la grandeur de ses bâtards, qu'il
a enfin portée au comble inouï à la fin de sa vie, dont les derniers
temps n'ont été principalement occupés qu'à la consolider, en les
rendant puissants et redoutables. L'amirauté, l'artillerie, les
carabiniers, tant de troupes et de régiments particuliers, les Suisses,
les Grisons, la Guyenne, le Languedoc, la Bretagne en leurs mains les
rendaient déjà assez considérables, jusqu'à la charge de grand veneur,
pour leur donner de quoi plaire, et amuser un jeune roi. Leur rang
égalé, à celui des princes du sang avait coûté au roi le renversement de
toutes les règles et les droits, et celui des lois du royaume les plus
anciennes, les plus saintes, les plus fondamentales, les plus intactes.
Il lui en coûta encore des démêlés avec les puissances étrangères, avec
Rome surtout, à qui il fallut complaire en choses solides, et après
avoir lutté longtemps pour obtenir que les ambassadeurs et les nonces
rendissent aux bâtards les mêmes honneurs et les mêmes devoirs qu'aux
princes du sang, et avec les mêmes traitements réciproques.

Ce même intérêt, comme on l'a vu dès le commencement de ces Mémoires,
éleva les Lorrains sur les ducs en la promotion du Saint-Esprit de 1688,
contre le goût du roi et la justice par lui-même reconnue et avouée au
duc de Chevreuse, et a soutenu les mêmes en mille occasions pour les
ployer aux bâtards. Cette même considération, comme on l'a vu en son
temps, valut l'incognito si nouveau et si étrange au duc de Lorraine,
lors de son hommage, dont si étrangement aussi il essaya d'abuser. Cet
exemple acquit le même avantage aux électeurs de Cologne et de Bavière,
à la honte de la majesté de la couronne.

Le mariage monstrueux de M. le duc de Chartres, depuis d'Orléans et
régent, celui de M. le Duc, ceux des filles de ces mariages avec M. le
duc de Berry et avec M. le prince de Conti, ont opéré ce que le roi a vu
de ses yeux, et vu avec complaisance, qu'excepté son successeur unique
et la branche d'Espagne (mais exclue de la succession à la couronne par
les renonciations et les traités) et la seule M\textsuperscript{lle} de
La Roche-sur-Yon, fille de M. le prince de Conti et de la fille aînée de
M. le Prince, il n'y a plus qui que ce soit, ni mâle, ni femelle de la
maison royale, qui ne sorte directement des amours du roi et de
M\textsuperscript{me} de Montespan, et dont elle ne soit la mère ou la
grand'mère\,; et si la duchesse du Maine n'en vient pas par elle-même,
elle a épousé le fils du roi et de M\textsuperscript{me} de Montespan.
La fille unique du roi et de M\textsuperscript{me} de La Vallière épousa
l'aîné des deux princes de Conti, dont elle n'a point eu d'enfants, mais
ce n'a pas été la faute du roi si cette branche seule de princes du sang
a échappé à la bâtardise, jusqu'à ce qu'il l'en ait aussi entachée à la
fin dans la seconde génération.

N'oublions pas que c'est le refus que le prince d'Orange fit de cette
princesse, que nuls respects, désirs, soins, soumissions les plus
prolongées n'ont pu effacer du cœur du roi, qui a rendu ce fameux
prince, malgré lui, l'ennemi du roi et de la France\,; et que cette
haine a été la source et la cause fatale de ces ligues et de ces
guerres, sous le poids desquelles le roi a été si près de succomber,
fruit de cette même bâtardise qui, à trop juste titre, se peut appeler
un fruit de perdition.

Ce mélange du plus pur sang de nos rois, et il se peut dire hardiment de
tout l'univers, avec la boue infecte du double adultère, a donc été le
constant ouvrage de toute la vie du roi. Il a eu l'horrible satisfaction
de les épuiser ensemble, et de porter au comble un mélange inouï dans
tous les siècles, après avoir été le premier de tous les hommes, de
toutes les nations, qui ait tiré du néant les fruits du double adultère,
et qui leur ait donné l'être, dont le monde entier, et policé et
barbare, frémit d'abord, et qu'il a su accoutumer.

Tandis que le chemin de la fortune fut toujours l'attachement et la
protection des bâtards, celle des princes du sang, à commencer par
Monsieur, y fut toujours un obstacle invincible. Tels lurent les fruits
d'un orgueil sans bornes qui fit toujours regarder au roi avec des yeux
si différents ses bâtards et les princes de son sang, les enfants issus
du trône par des générations légitimes, et qui les rappelaient à leur
tour, et les enfants sortis de ses amours. Il considéra les premiers
comme les enfants de l'État et de la couronne, grands par là et par
eux-mêmes sans lui, tandis qu'il chérit les autres comme les enfants de
sa personne qui ne pouvaient devenir, faute d'être par eux-mêmes, par
toutes les lois, que les ouvrages de sa puissance et de ses mains.
L'orgueil et la tendresse se réunirent en leur faveur, le plaisir
superbe de la création l'augmenta sans cesse, et fut sans cesse
aiguillonné d'un regard de jalousie sur la naturelle indépendance de la
grandeur des autres sans son concours.

Piqué de n'oser égaler la nature, il approcha du moins ses bâtards des
princes du sang par tout ce qu'il leur donna d'abord d'établissements et
de rangs. Il tâcha ensuite de les confondre ensemble par des mariages
inouïs, monstrueux, multipliés, pour n'en faire qu'une seule et même
famille. Le fils unique de son unique frère y fut enfin immolé aussi
avec la plus ouverte violence. Après, devenu plus hardi à force de crans
redoublés, il mit une égalité parfaite entre ses bâtards et les princes
du sang. Enfin, près de mourir, il s'abandonna à leur en donner le nom
et le droit de succéder à la couronne, comme s'il eût pu en disposer, et
faire les hommes ce qu'ils ne sont pas de naissance.

Ce ne fut pas tout. Ses soins et ses dernières dispositions pour après
lui ne furent toutes qu'en leur faveur. Aliéné avec art de son neveu, et
soigneusement entretenu dans cette disposition par le duc du Maine et
par M\textsuperscript{me} de Maintenon, il subit le joug qu'il s'était
laissé imposer par eux, il en but le calice qu'il s'était à lui-même
préparé. On a vu les élans de sa résistance et de ses dépiteux
regrets\,; il ne put résister à ce qu'ils en extorquèrent. Son
successeur y fut pleinement sacrifié, et autant qu'il fut en lui, son
royaume.

Tout ce qui fut nommé par anticipation pour l'éducation du roi futur
n'eut d'autre motif que l'intérêt des bâtards, et rien moins que nul
autre. Le duc du Maine fut mis à la tête, et sous lui le maréchal de
Villeroy, l'homme le plus inepte à cet emploi qu'il y eût peut-être dans
toute la France\,; ajoutons que lors de ce choix il avait soixante et
onze ans, et que le prince dont il était destiné gouverneur en avait
cinq et demi. Saumery, très indigne sous-gouverneur de Mgr le duc de
Bourgogne, et qui, sous prétexte des eaux, s'était bien gardé de le
suivre à la campagne de Lille, avait fait ses infâmes preuves à son
retour en faveur de Vendôme, à la cabale duquel il s'était joint
hautement. C'en fut assez pour le faire choisir au duc du Maine pour
sous-gouverneur du roi futur, comme un homme vendu et à tout faire.

Je n'ai point su qui avait fait nommer Joffreville pour l'autre
sous-gouverneur, mais il était trop homme d'honneur pour accepter un
emploi où il fallait se vendre. Il s'en excusa. Ruffé lui fut substitué.
Il se disait Damas sans l'être\,; mais pauvre, court d'esprit, qui
n'envisagea que fortune, et subsistance en attendant, qui ne sentit pas
les dangers de la place, qui avait tout son bien dans le pays de Dombes,
et par là de tout temps sous la protection du duc du Maine, n'en vit
jamais que l'écorce, et qui l'accepta malgré sa prétendue naissance.
Tout le reste fut choisi de même, et M\textsuperscript{me} de Maintenon
qui fit son affaire de Fleury, qui pour cela venait de quitter Fréjus,
et qui en répondit.

Avec de tels entours, le duc du Maine ne se crut pas encore suffisamment
assuré. Ce fut à quoi le codicille pourvut, qui ne précéda la mort du
roi que de si peu de jours qui fut le dernier travail de ce monarque, et
son dernier sacrifice à la divinité qu'il s'était faite de ses bâtards.
Il faut le répéter\,: par ce dernier acte toute la maison civile et
militaire du roi était totalement et uniquement soumise au duc du Maine,
et sous lui au maréchal de Villeroy, indépendamment et privativement à
M. le duc d'Orléans, de façon qu'il n'en pouvait être reconnu ni obéi en
rien, mais les deux chefs de l'éducation en toutes choses qui devenaient
par là les maîtres de Paris et de la cour, et le régent livré entre
leurs mains sans aucune sûreté.

Ces énormes précautions parurent encore insuffisantes, si on ne
pourvoyait à ce qui pouvait arriver. Ainsi, en cas de mort du duc du
Maine ou du maréchal de Villeroy, le comte de Toulouse et le maréchal
d'Harcourt, duquel M\textsuperscript{me} de Maintenon répondit, leur
furent substitués en tout et partout, lequel Harcourt par son état
apoplectique était, si faire se pouvait, devenu encore plus inepte à ce
grand emploi que le maréchal de Villeroy.

Le testament avait nommé et réglé le conseil de régence, en telle sorte
que toute l'autorité de la régence fut ôtée à M. le duc d'Orléans, que
ce conseil ne fut composé presque que de tous gens à la dévotion du duc
du Maine, et desquels tous en particulier M. le duc d'Orléans avait de
grands sujets d'être aliéné.

Tels furent les derniers soins du roi, telles les dernières actions de
sa prévoyance, tels les derniers coups de sa puissance, ou plutôt de sa
déplorable faiblesse, et des suites honteuses de sa vie\,: état bien
misérable, qui abandonnait son successeur et son royaume à l'ambition à
découvert et sans bornes de qui n'aurait jamais dû y être seulement
connu, et qui exposait l'État aux divisions les plus funestes, en armant
contre le régent ceux qui devaient lui être les plus soumis, et le
jetant dans la plus indispensable nécessité de revendiquer son droit et
son autorité, dont on ne lui laissait que le vain nom avec l'ignominie
d'une impuissance et d'une nudité entière, et la réalité des plus
instants, des plus continuels, et des plus réels périls que l'âge auquel
se trouvait alors tout ce qu'il y avait de princes du sang portait au
comble.

Voilà au moins de quoi la mémoire du roi ne peut être lavée devant Dieu
ni devant les hommes. Voilà le dernier abîme où le conduisirent la
superbe et la faiblesse, une femme plus qu'obscure et des doubles
adultérins, à qui il s'abandonna, dont il fit ses tyrans, après l'avoir
été pour eux et pour tant d'autres, qui en abusèrent sans aucune pudeur
ni réserve, et un détestable confesseur du caractère du P. Tellier. Tel
fut le repentir, la pénitence, la réparation publique d'un double
adultère si criant, si long, si scandaleux à la face de toute l'Europe,
et les derniers sentiments d'une âme si hautement pécheresse, prête à
paraître devant Dieu, et de plus, chargée d'un règne de cinquante-six
ans, le sien, dont l'orgueil, le luxe, les bâtiments, les profusions en
tout genre et les guerres continuelles, et la superbe qui en fut la
source et la nourriture, avait répandu tant de sang, consumé tant de
milliards au dedans et au dehors, mis sans cesse le feu par toute
l'Europe, confondu et anéanti tous les ordres, les règles, les lois les
plus anciennes et les plus sacrées de l'État, réduit le royaume à une
misère irrémédiable, et si imminemment près de sa totale perte qu'il
n'en fut préservé que par un miracle du Tout-Puissant.

Que dire après cela de la fermeté constante et tranquille qui se fit
admirer dans le roi en cette extrémité de sa vie\,? car il est vrai
qu'en la quittant il n'en regretta rien, et que l'égalité de son âme fut
toujours à l'épreuve de la plus légère impatience, qu'il ne s'importuna
d'aucun ordre à donner, qu'il vit, qu'il parla, qu'il régla, qu'il
prévit tout pour après lui, dans la même assiette que tout homme en
bonne santé et très libre d'esprit aurait pu faire\,; que tout se passa
jusqu'au bout avec cette décence extérieure, cette gravité, cette
majesté qui avait accompagné toutes les actions de sa vie\,; qu'il y
surnagea un naturel, un air de vérité et de simplicité qui bannit
jusqu'aux plus légers soupçons de représentation et de comédie.

De temps en temps, dès qu'il était libre, et dans les derniers qu'il
avait banni toute affaire et tous autres soins, il était uniquement
occupé de Dieu, de son salut, de son néant, jusqu'à lui être échappé
quelquefois de dire\,: \emph{Du temps que j'étais roi}. Absorbé d'avance
en ce grand avenir où il se voyait si près d'entrer, avec un détachement
sans regret, avec une humilité sans bassesse, avec un mépris de tout ce
qui n'était plus pour lui, avec une bonté et une possession de son âme
qui consolait ses valets intérieurs qu'il voyait pleurer, il forma le
spectacle le plus touchant\,; et ce qui le rendit admirable, c'est qu'il
se soutint toujours tout entier et toujours le même\,: sentiment de ses
péchés sans la moindre terreur, confiance en Dieu, le dira-t-on\,? tout
entière, sans doute, sans inquiétude, mais fondée sur sa miséricorde et
sur le sang de Jésus-Christ, résignation pareille sur son état
personnel, sur sa durée, et regrettant de ne pas souffrir. Qui
n'admirera une fin si supérieure, et en même temps si chrétienne\,? mais
qui n'en frémira\,?

Rien de plus simple ni de plus court que son adieu à sa famille, ni de
plus humble, sans rien perdre de la majesté, que son adieu aux
courtisans, plus tendre encore que l'autre. Ce qu'il dit au roi futur a
mérité d'être recueilli, mais affiché depuis avec trop de restes de
flatterie, dont le maréchal de Villeroy donna l'exemple en le mettant à
la ruelle de son lit, comme il avait toujours dans sa chambre à l'armée
un portrait du roi tendu sous un dais, et comme il pleurait toujours
vis-à-vis du roi aux compliments que les prédicateurs lui faisaient en
chaire. Le roi, parlant à son successeur de ses bâtiments et de ses
guerres, omit son luxe et ses profusions. Il se garda bien de lui rien
toucher de ses funestes amours, article plus en sa place alors que tous
les autres\,; mais comment en parler devant ses bâtards, et en
consommant leur épouvantable grandeur par les derniers actes de sa
vie\,? Jusque-là, si on excepte cette étrange omission et sa cause plus
terrible encore, rien que de digne d'admiration, et d'une élévation
véritablement chrétienne et royale.

Mais que dire de ses derniers discours à son neveu, après son testament,
et depuis encore venant de faire son codicille, après avoir reçu les
derniers sacrements\,; de ses assurances positives, nettes, précises,
toutes les deux fois, qu'il ne trouverait rien dans ses dispositions qui
pût lui faire de peine, tandis qu'elles n'ont été faites, et à deux
reprises, que pour le déshonorer, le dépouiller, disons tout, pour
l'égorger\,? Cependant il le rassure, il le loue, il le caresse, il lui
recommande son successeur, qu'il lui a totalement soustrait, et son
royaume qu'il va, dit-il, seul gouverner, sur lequel il lui a ôté toute
autorité\,; et tandis qu'il vient d'achever de la livrer à ses ennemis
tout entière\,; et avec les plus formidables précautions, c'est à lui
qu'il envoie pour des ordres, comme à celui à qui désormais il
appartient seul d'en donner pour tout et sur tout. Est-ce artifice\,?
est-ce tromperie\,? est-ce dérision jusqu'en mourant\,? Quelle énigme à
expliquer\,! Tâchons plutôt de nous persuader que le roi se répondait à
soi-même.

Il répondait à ce qu'il avait toujours paru croire de l'impuissance de
l'effet de ce qui lui avait été extorqué, et que la faiblesse lui avait
arraché malgré lui. Disons plus, il ne douta point, il espéra peut-être
qu'un testament inique et scandaleux, propre à mettre le feu dans sa
famille et dans le royaume, tel enfin qu'il était réduit à en cacher
profondément le secret, ne trouverait pas plus d'appui que n'en avait
reçu le testament du roi son père, si sage, si sensé, si pesé, si juste,
et par lui-même rendu public avec un véritable et général
applaudissement. Tout ce que le roi avait senti de violence en faisant
le sien, tout ce qu'il en avait dit si amèrement à ses bâtards après
l'avoir fait, aux gens du parlement en le leur remettant, à la reine
d'Angleterre du moment qu'il la vit, et toujours leur en parlant le
premier comme plein d'amertume, on peut ajouter de dépit, de sa
faiblesse, et de l'abus énorme que lui en fait ce qu'il a de seul intime
et dont il ne se peut détacher\,; ce codicille monstrueux arraché après
avoir reçu ses sacrements, dans un état de mourant qui lui en laissait
sentir les horreurs sans lui permettre d'y résister\,; ce tout ensemble,
ce groupe effroyable d'iniquité et de renversement de toutes choses pour
faire de ses bâtards, et du duc du Maine en particulier, un colosse
immense de puissance et de grandeur, et la destruction de toutes les
lois, de son neveu, et peut-être de son royaume et de son successeur,
livrés à de si étranges mains, serait-ce trop dire\,? si cruelles et si
fort approchées du trône\,; cet amas prodigieux d'iniquités si
concertées, mais si mal colorées, quelques soins qu'on s'en fût donnés,
qu'elles sautaient aux yeux, tout cela le rassura peut-être contre ce
qu'on en avait prétendu. Il n'avait jamais cru, comme il s'en était
expliqué plusieurs fois, qu'aucune des choses qu'il venait de faire ou
de confirmer pût subsister un moment après lui. En ce moment qu'il parla
à M. le duc d'Orléans, il s'en flatta peut-être plus que jamais, pour
s'apaiser soi-même, tout rempli qu'il devait être de son codicille,
qu'il avait fait il n'y avait pas plus d'une heure. Il parla peut-être à
son neveu avant et après le codicille tout plein de cette pensée\,; il
put donc ainsi le regarder, en effet, comme l'administrateur du royaume,
et lui parler en ce sens. C'est du moins ce qu'il peut être permis de
présumer.

Mais qui pourra ne pas s'étonner au dernier point, on ne peut s'empêcher
de le répéter, de la paisible et constante tranquillité de ce roi
mourant, et de cette inaltérable paix sans la plus légère inquiétude,
parmi tant de piété et une application si fervente à profiter de tous
les moments\,? Les médecins prétendirent que la même cause qui amortit
et qui ôte même toutes les douleurs du corps, qui est un sang
entièrement gangrené, calme aussi et anéantit toutes celles du coeur et
les agitations de l'esprit\,; et il est vrai que le roi mourut de cette
maladie.

D'autres en ont donné une autre raison, et ceux-là étaient dans
l'intrinsèque de la chambre pendant cette dernière maladie, et y furent
seuls les derniers jours. Les jésuites ont constamment des laïques de
tous états, même mariés, qui sont de leur compagnie. Ce fait est
certain\,; il n'est pas douteux que des Noyers, secrétaire d'État sous
Louis XIII, n'ait été de ce nombre, et bien d'autres. Ces agrégés font
les mêmes voeux des jésuites en tout ce que leur état peut permettre,
c'est-à-dire d'obéissance sans restriction aucune au père général et aux
supérieurs de la compagnie. Ils sont obligés de suppléer à ceux de
pauvreté et de chasteté par tous les services et par toute la protection
qu'ils doivent aveuglément à la compagnie, surtout par une soumission
sans bornes aux supérieurs et à leur confesseur. Ils doivent être exacts
à de légers exercices de piété que leur confesseur ajuste à leur temps
et à leur esprit, et qu'il simplifie tant qu'il veut. La politique a son
compte par le secours assuré de ces auxiliaires cachés à qui ils font
bon marché du reste. Mais il ne se doit rien passer dans leur âme, ni
quoi que ce soit qui vienne à leur connaissance, qu'ils ne le révèlent à
leur confesseur, et, pour ce qui n'est pas du secret de la conscience,
aux supérieurs, si le confesseur le juge à propos. Ils se doivent aussi
conduire en tout suivant les ordres des supérieurs et du confesseur avec
une soumission sans réplique.

On a prétendu que le P. Tellier avait inspiré au roi longtemps avant sa
mort de se faire agréger ainsi dans la compagnie\,; qu'il lui en avait
vanté les privilèges certains pour le salut, les indulgences plénières
qui y sont attachées\,; qu'il l'avait persuadé que quelques crimes qu'on
eût commis, et dans quelque difficulté qu'on se trouvât de les réparer,
cette profession secrète lavait tout, et assurait infailliblement le
salut, pourvu qu'on fût fidèle à ses vœux\,; que le général de la
compagnie fut admis du consentement du roi dans le secret\,; que le roi
en fit les vœux entre les mains du P. Tellier\,; que dans les derniers
jours de sa vie on les entendit tous deux, l'un fortifier, l'autre
s'appuyer sur ces promesses\,; qu'enfin le roi reçut de lui la dernière
bénédiction de la compagnie comme un des religieux\,; qu'il lui fit
prononcer des formules de prières qui n'en laissaient point douter, et
qu'on entendit en partie, et qu'il lui en avait donné l'habit ou le
signe presque imperceptible, comme une autre sorte de scapulaire, qui
fut trouvé sur lui. Enfin la plupart de ce qui approcha de plus près
demeurèrent persuadés que cette pénitence faite aux dépens d'autrui, des
huguenots, des jansénistes, des ennemis des jésuites, ou de ceux qui ne
leur furent pas abandonnés, des défenseurs des droits des rois et des
nations, des canons et de la hiérarchie contre la tyrannie et les
prétentions ultramontaines, cet attachement pharisaïque à l'extérieur de
la loi et à l'écorce de la religion, ont formé cette sécurité si
surprenante dans ces terribles moments où disparaît si ordinairement
celle qui, fondée sur l'innocence et la pénitence fidèle, semble le plus
solidement devoir rassurer\,: droits terribles de l'art de tromper qui
remplissent toutes les conditions de jésuites inconnus, dont l'ignorance
les sert à tous les usages importants qu'ils en savent tirer dans la
persuasion d'un salut certain sans repentir, sans réparation, sans
pénitence de quelque vie qu'on ait menée, et d'une abominable doctrine,
qui pour des intérêts temporels abuse les pécheurs jusqu'au tombeau, et
les y conduit dans une paix profonde par un chemin semé de fleurs.

Ainsi mourut un des plus grands rois de la terre entre les bras d'une
indigne et ténébreuse épouse, et de ses doubles bâtards, maîtres de lui
jusqu'à sa consommation pour eux, muni des sacrements de l'Église de la
main du fils de son autre bien-aimée, plus que comblé des faveurs que
celles de sa mère avaient valu à sa famille, et assisté uniquement par
un confesseur tel qu'on a vu qu'était le P. Tellier. Si telle peut être
la mort des saints, ce n'est pas là au moins leur assistance.

Aussi cette assistance ne fut-elle pas poussée jusqu'au bout. Maîtres du
roi et de sa chambre, et n'y admettant qu'eux et ce peu de dévoués qui
leur étaient nécessaires, leur assiduité ne se démentit point tant
qu'ils en eurent besoin. Mais, le codicille fait et remis à Voysin, ils
n'eurent plus rien à faire, et tout aussitôt n'eurent pas honte de se
retirer. Les devoirs, désormais infructueux auprès d'un mourant dont ils
avaient arraché jusqu'à l'impossible, leur devinrent en un moment trop à
charge et trop fatigants pour continuer à voir un spectacle si triste et
si peu utile.

On a vu combien le tendre compliment du roi à M\textsuperscript{me} de
Maintenon sur l'espérance d'en être bientôt rejoint déplut à cette
vieille fée, qui, non contente d'être reine, voulait apparemment être
encore immortelle. On a vu que, dès le mercredi, c'est-à-dire quatre
jours avant la mort du roi, elle l'abandonna pour toujours, que le roi
s'en aperçut avec tant de peine qu'il la redemanda sans cesse, ce qui la
força de revenir de Saint-Cyr, et qu'elle n'eut pas la patience
d'attendre sa fin pour y retourner, et n'en plus revenir.

Bissy et Rohan, contents d'avoir paré ce grand coup du retour du
cardinal de Noailles, ne s'incommodèrent plus d'aucune assiduité,
jusque-là que Rohan laissa le roi sans messe, et que, sans Charost,
comme on l'a vu, il n'en eût plus été question, quoique le roi fût en
pleine connaissance et qu'il dît qu'il désirait l'entendre quand on le
lui proposa, et qu'à l'égard de la tête et de la parole il fût comme en
pleine santé.

Le duc du Maine marqua aussi toute la bonté de son cœur, et toute sa
reconnaissance pour un père qui lui avait tout sacrifié. Il se trouva à
la consultation de cet homme arrivant de Provence, dont on a parlé, qui
donna de son élixir au roi. Fagon, accoutumé à régner sur la médecine
avec despotisme, trouva une manière de paysan très grossier, qui le
malmena fort brutalement. M. du Maine, qui n'avait plus lieu de rien
arracher, et qui se comptait déjà le maître du royaume, raconta le soir
chez lui, parmi ses confidents, avec ce facétieux et cet art de fine
plaisanterie qu'il possédait si bien, l'empire que ce malotru avait pris
sur la médecine, l'étonnement, le scandale, l'humiliation de Fagon pour
la première fois de sa vie, qui, à bout de son art et de ses espérances,
s'était limaçonné en grommelant sur son bâton, sans oser répliquer, de
peur d'essuyer pis. Ce bon et tendre fils leur fit de cette aventure le
conte si plaisamment, que les voilà tous aux grands éclats de rire, et
lui aussi, qui durèrent fort longtemps. L'excès de la joie de toucher à
la toute-puissance, à la délivrance, au comble presque de ses voeux, lui
avait fait oublier une indécence que les antichambres surent bien
remarquer, et la galerie encore sur laquelle cet appartement donnait,
proche et de plain-pied de la chapelle, où des passants de distinction
entendirent ces éclats.

le duc du Maine retrancha des assiduités inutiles. C'était pour lui un
spectacle trop attendrissant\,; il aima mieux n'y plus paraître que de
rares instants, et renfermer sa douleur dans son cabinet, au pied de son
crucifix, ou s'y appliquer à tous les ordres futurs pour l'exécution de
ce qu'il s'était fait attribuer.

Le P. Tellier se lassait depuis longtemps d'assister un mourant. II
n'avait pu venir à bout de la nomination de ce grand nombre de bénéfices
vacants\,; il ne craignait plus rien sur le cardinal de Noailles depuis
que Bissy et lui, avec M\textsuperscript{me} de Maintenon, avaient paré
son retour. Ainsi, n'ayant plus rien à craindre ni à espérer du roi, il
se donna à d'autres soins, tellement que tout cet intérieur de chambre
du roi, et les cabinets même, étaient scandalisés de ses absences, et
qu'il y en avait qui ne s'en contraignaient pas, comme Bloin et
Maréchal, qui quelquefois l'envoyaient chercher d'eux-mêmes. Le roi le
demandait souvent sans qu'il fût là à portée, et quelquefois sans qu'il
vînt du tout, parce qu'on ne le trouvait ni chez lui ni où on le
cherchait. Quand il s'approchait du roi, c'était toujours de lui-même
qu'il s'en retirait, et presque toujours en fort peu de moments. Les
derniers jours, et dans cet état extrême, il parut encore bien moins,
quoiqu'un confesseur, et qui n'était doublé de personne, ne dût point
alors quitter les environs du lit. Mais il ne parut pas que la charité,
la sollicitude, non plus que l'affection ni la reconnaissance, fussent
les vertus distinctives de ce maître imposteur, à qui ses profondeurs et
ses artifices n'avaient pas donné le goût, l'onction, ni le talent
d'assister les mourants. Il fallait l'envoyer chercher sans cesse\,; il
s'échappait sans cesse aussi, et par une aussi indigne conduite, il
scandalisa tout ce qui y était, et tout ce qui y pouvait être y était,
depuis que, par la retraite de M\textsuperscript{me} de Maintenon et de
M. du Maine, l'accès de la chambre fut rendu et devenu libre.

Mais, à propos du P. Tellier, la vérité veut que j'ajoute que je me suis
depuis informé curieusement à Maréchal de l'opinion que le roi avait
fait le vœu de jésuite et de ce que j'ai raconté là-dessus. Maréchal,
qui était fort vrai, et qui n'estimait pas le P. Tellier, m'a assuré
qu'il ne s'était jamais aperçu de rien qui eût trait à cela, ni de
formule de prières ou de bénédiction particulière, ni que le roi ait eu
aucune marque ni manière de scapulaire sur lui, et qu'il était très
persuadé qu'il n'y avait pas la moindre vérité dans tout ce qui s'était
dit là-dessus Maréchal, quoique très assidu, n'était pas toujours ni
dans la chambre, ni près du lit. Le P. Tellier pouvait aussi s'en
défiier et se cacher de lui\,; mais je ne puis croire, malgré tout cela,
que s'il y avait quelque chose de vrai là-dessus, Maréchal n'en eût pas
eu la moindre connaissance, et que jusqu'aux soupçons lui eussent
échappé.

\hypertarget{chapitre-v.}{%
\chapter{CHAPITRE V.}\label{chapitre-v.}}

~

{\textsc{Vie publique du roi.}} {\textsc{- Où seulement et quels hommes
mangeaient avec le roi.}} {\textsc{- Matinées du roi.}} {\textsc{-
Conseils.}} {\textsc{- Dîner du roi.}} {\textsc{- Service.}} {\textsc{-
Promenades du roi.}} {\textsc{- Soirs du roi.}} {\textsc{- Jours de
médecine.}} {\textsc{- Dévotions.}} {\textsc{- Autres bagatelles.}}
{\textsc{- Le roi peu regretté.}}

~

Après avoir exposé avec la vérité et la fidélité la plus exacte tout ce
qui est venu à ma connaissance par moi-même, ou par ceux qui ont vu ou
manié les choses et les affaires pendant les vingt-deux dernières années
de Louis XIV, et l'avoir montré tel qu'il a été, sans aucune passion,
quoique je me sois permis les raisonnements résultant naturellement des
choses, il ne me reste plus qu'à exposer l'écorce extérieure de la vie
de ce monarque, depuis que j'ai continuellement habité à sa cour.

Quelque insipide et peut-être superflu qu'un détail, encore si public,
puisse paraître après tout ce qu'on a vu d'intérieur, il s'y trouvera
encore des leçons pour les rois qui voudront se faire respecter et qui
voudront se respecter eux-mêmes. Ce qui m'y détermine encore, c'est que
l'ennuyeux, je dirai plus, le dégoûtant pour un lecteur instruit de ce
dehors public, par ceux qui auront pu encore en avoir été témoins,
échappe bientôt à la connaissance de la postérité, et que l'expérience
nous apprend que nous regrettons de ne trouver personne qui se soit
donné une peine pour leur temps si ingrate, mais, pour la postérité,
curieuse, et qui ne laisse pas de caractériser les princes qui ont fait
autant de bruit dans le monde que celui dont il s'agit ici. Quoiqu'il
soit difficile de ne pas tomber en quelques redites, je m'en défendrai
autant qu'il me sera possible.

Je ne parlerai point de la manière de vivre du roi quand il s'est trouvé
dans ses armées. Ses heures y étaient déterminées par ce qui se
présentait à faire, en tenant néanmoins régulièrement ses conseils\,; je
dirai seulement qu'il n'y mangeait soir et matin qu'avec des gens d'une
qualité à pouvoir avoir cet honneur. Quand on y pouvait prétendre, on le
faisait demander au roi par le premier gentilhomme de la chambre en
service. Il rendait la réponse, et dès le lendemain, si elle était
favorable, on se présentait au roi lorsqu'il allait dîner, qui vous
disait\,: «\,Monsieur, mettez-vous à table.\,» Cela fait, c'était pour
toujours, et on avait après l'honneur d'y manger quand on voulait, avec
discrétion. Les grades militaires, même d'ancien lieutenant général, ne
suffisaient pas. On a vu que M. de Vauban, lieutenant général si
distingué depuis tant d'années, y mangea pour la première fois à la fin
du siège de Namur, et qu'il fut comblé de cette distinction, comme aussi
les colonels de qualité distinguée y étaient admis sans difficulté. Le
roi fit le même honneur à Namur à l'abbé de Grancey, qui s'exposait
partout à confesser les blessés et à encourager les troupes. C'est
l'unique abbé qui ait eu cet honneur. Tout le clergé en fut toujours
exclu, excepté les cardinaux et les évêques-pairs, ou les
ecclésiastiques ayant rang de prince étranger. Le cardinal de Coislin,
avant d'avoir la pourpre, étant évêque d'Orléans, premier aumônier et
suivant le roi en toutes ses campagnes\,; et l'archevêque de Reims qui
suivait le roi comme maître de sa chapelle, y voyait manger le duc et le
chevalier de Coislin, ses frères, sans y avoir jamais prétendu. Nul
officier des gardes du corps n'y a mangé non plus, quelque préférence
que le roi eût pour ce corps, que le seul marquis d'Urfé par une
distinction unique\,; je ne sais qui la lui valut en ces temps reculés
de moi\,; et du régiment des gardes, jamais que le seul colonel, ainsi
que les capitaines des gardes du corps.

À ces repas tout le monde était couvert\,; c'eût été un manque de
respect dont on vous aurait averti sur-le-champ de n'avoir pas son
chapeau sur sa tête. Monseigneur même l'avait\,; le roi seul était
découvert. On se découvrait quand le roi vous parlait, ou pour parler à
lui, et on se contentait de mettre la main au chapeau pour ceux qui
venaient faire leur cour le repas commencé, et qui étaient de qualité à
avoir pu se mettre à table. On se découvrait aussi pour parler à
Monseigneur et à Monsieur, ou quand ils vous parlaient. S'il y avait des
princes du sang, on mettait seulement la main au chapeau pour leur
parler ou s'ils vous parlaient. Voilà ce que j'ai vu au siège de Namur,
et ce que j'ai su de toute la cour. Les places qui approchaient du roi
se laissaient aussi aux titres, et après aux grades\,; si on en avait
laissé qui ne s'en remplissent pas, on se rapprochait. Quoiqu'à l'armée,
les maréchaux de France n'y avaient point de préférence sur les ducs, et
ceux-ci, et les princes étrangers, ou qui en avaient rang, se plaçaient
les uns avec les autres comme ils se rencontraient, sans affectation.
Mais duc, prince ou maréchal de France, si le hasard faisait qu'ils
n'eussent pas encore mangé avec le roi, il fallait s'adresser au premier
gentilhomme de la chambre. On juge bien que cela ne faisait pas de
difficulté. Il n'y avait là-dessus que les princes du sang exceptés. Le
roi seul avait un fauteuil. Monseigneur même, et tout ce qui était à
table, avaient des sièges à dos de maroquin noir, qui se pouvaient
briser pour les voiturer, qu'on appelait des perroquets. Ailleurs qu'à
l'armée, le roi n'a jamais mangé avec aucun homme, en quelque cas que
ç'ait été, non pas même avec aucun prince du sang, qui n'y ont mangé
qu'à des festins de leurs noces, quand le roi les a voulu faire, comme
on en a vu le oui et le non en leur temps. Revenons maintenant à la
cour.

À huit heures le premier valet de chambre en quartier, qui avait couché
seul dans la chambre du roi, et qui s'était habillé, l'éveillait. Le
premier médecin, le premier chirurgien, et sa nourrice, tant qu'elle a
vécu, entraient en même temps. Elle allait le baiser\,; les autres le
frottaient et souvent lui changeaient de chemise, parce qu'il était
sujet à suer. Au quart, on appelait le grand chambellan, en son absence
le premier gentilhomme de la chambre d'année, avec eux les grandes
entrées. L'un de ces deux ouvrait le rideau qui était refermé, et pré
sentait l'eau bénite du bénitier du chevet du lit. Ces messieurs étaient
là un moment, et c'en était un de parler au roi s'ils avaient quelque
chose à lui dire ou à lui demander, et alors les autres s'éloignaient.
Quand aucun d'eux n'avait à parler comme d'ordinaire, ils n'étaient là
que quelques moments. Celui qui avait ouvert le rideau et présenté l'eau
bénite présentait le livre de l'office du Saint-Esprit, puis passaient
tous dans le cabinet du conseil. Cet office fort court dit, le roi
appelait\,; ils rentraient. Le même lui donnait sa robe de chambre, et
cependant les secondes entrées ou brevets d'affaires entraient\,; peu de
moments après, la chambre\,; aussitôt ce qui était là de distingué, puis
tout le monde, qui trouvait le roi se chaussant\,; car il se faisait
presque tout lui-même avec adresse et grâce. On lui voyait faire la
barbe de deux jours l'un, et il avait une petite perruque courte, sans
jamais en aucun temps, même au lit, les jours de médecine, paraître
autrement en public. Souvent il parlait de chasse, et quelquefois
quelque mot à quelqu'un. Point de toilette à portée de lui, on lui
tenait seulement un miroir.

Dès qu'il était habillé, il allait prier Dieu à la ruelle de son lit, où
tout ce qu'il y avait de clergé se mettait à genoux, les cardinaux sans
carreaux\,; tous les laïques demeuraient debout, et le capitaine des
gardes venait au balustre pendant la prière, d'où le roi passait dans
son cabinet.

Il y trouvait ou y était suivi de tout ce qui avait cette entrée, qui
était fort étendue par les charges qui l'avaient toutes. Il y donnait
l'ordre à chacun pour la journée\,; ainsi on savait, à un demi-quart
d'heure près, tout ce que le roi devait faire. Tout ce monde sortait
ensuite. Il ne demeurait que les bâtards, MM. de Montchevreuil et d'O,
comme ayant été leurs gouverneurs, Mansart, et après lui d'Antin, qui
tous entraient, non par la chambre mais par les derrières, et les valets
intérieurs. C'était là leur bon temps aux uns et aux autres, et celui de
raisonner sur les plans des jardins et des bâtiments, et cela durait
plus ou moins, selon que le roi avait affaire.

Toute la cour attendait cependant dans la galerie, le capitaine des
gardes seul dans la chambre, assis à la porte du cabinet, qu'on
avertissait quand le roi voulait aller à la messe, et qui alors entrait
dans le cabinet. À Marly, la cour attendait dans le salon\,; à Trianon,
dans les pièces de devant, comme à Meudon. À Fontainebleau, on demeurait
dans la chambre et l'antichambre.

Cet entre-temps était celui des audiences, quand le roi en accordait, ou
qu'il voulait parler à quelqu'un, et des audiences secrètes des
ministres étrangers, en présence de Torcy. Elles n'étaient appelées
secrètes que pour les distinguer de celles qui se donnaient sans
cérémonie à la ruelle du lit, au sortir de la prière, qu'on appelait
particulières, où celles de cérémonie se donnaient aussi aux
ambassadeurs.

Le roi allait à la messe, où sa musique chantait toujours un motet. Il
n'allait en bas qu'aux grandes fêtes, ou pour des cérémonies. Allant et
revenant de la messe, chacun lui parlait, qui voulait, après l'avoir dit
au capitaine des gardes, si ce n'était gens distingués, et il y allait
et rentrait par la porte des cabinets dans la galerie. Pendant la messe,
les ministres étaient avertis et s'assemblaient dans la chambre du roi,
où les gens distingués pouvaient aller leur parler ou causer avec eux.
Le roi s'amusait peu au retour de la messe, et demandait presque
aussitôt le conseil. Alors la matinée était finie.

Le dimanche il y avait conseil d'État, et souvent les lundis. Les
mardis, conseil de finance\,; les mercredis, conseil d'État\,; les
samedis, conseil de finance. Il était rare qu'il y en eût deux par jour,
et qu'il s'en tînt les jeudis ni les vendredis. Une ou deux fois le
mois, il y avait un lundi matin conseil de dépêches\,; mais les ordres
que les secrétaires d'État prenaient tous les matins, entre le lever et
la messe, abrégeaient et diminuaient fort ces sortes d'affaires. Tous
les ministres étaient assis en rang entre eux, excepté au conseil des
dépêches, où tous étaient debout, tout du long, excepté les fils de
France quand il y en avait, le chancelier et le duc de Beauvilliers\,;
rarement pour des affaires extraordinaires évoquées, et vues dans un
bureau de conseillers d'État. Ces mêmes conseillers d'État venaient à un
conseil donné exprès de finance ou de dépêches, mais où on ne parlait
que de cette seule affaire. Alors tous étaient assis, et les conseillers
d'État y coupaient les secrétaires d'État et le contrôleur général,
suivant leur ancienneté de conseiller d'État entre eux, et un maître des
requêtes rapportait debout, lui et les conseillers d'État en robes. Le
jeudi matin était presque toujours vide. C'était le temps des audiences
que le roi voulait donner, et le plus souvent des audiences inconnues,
par les derrières. C'était aussi le grand jour des bâtards, des
bâtiments, des valets intérieurs, parce que le roi n'avait rien à faire.
Le vendredi après la messe était le temps du confesseur, qui n'était
borné par rien, et qui pouvait durer jusqu'au dîner. À Fontainebleau,
ces matins-là qu'il n'y avait point de conseil, le roi passait très
ordinairement de la messe chez M\textsuperscript{me} de Maintenon\,; et
de même à Trianon et à Marly\,; quand elle n'était pas allée dès le
matin à Saint-Cyr. C'était le temps de leur tête-à-tête sans ministre et
sans interruption, et à Fontainebleau jusqu'au dîner. Souvent, les jours
qu'il n'y avait pas de conseil, le dîner était avancé plus ou moins pour
la chasse ou la promenade. L'heure ordinaire était une heure\,; si le
conseil durait encore, le dîner attendait et on n'avertissait point le
roi. Après le conseil de finance, Desmarets restait souvent seul à
travailler avec le roi.

Le dîner était toujours au petit couvert, c'est-à-dire seul dans sa
chambre, sur une table carrée vis-à-vis la fenêtre du milieu. Il était
plus ou moins abondant\,; car il ordonnait le matin petit couvert ou
très petit couvert. Mais ce dernier était toujours de beaucoup de plats,
et de trois services sans le fruit. La table entrée, les principaux
courtisans entraient, puis tout ce qui était connu, et le premier
gentilhomme de la chambre en année allait avertir le roi. Il le servait
si le grand chambellan n'y était pas.

Le marquis de Gesvres, depuis duc de Tresmes, prétendit que, le dîner
commencé, M. de Bouillon arrivant ne lui pouvait ôter le service, et fut
condamné. J'ai vu M. de Bouillon arriver derrière le roi au milieu du
dîner, et M. de Beauvilliers qui servait lui vouloir donner le service,
qu'il refusa poliment, et dit qu'il toussait trop et était trop enrhumé.
Ainsi il demeura derrière le fauteuil, et M. de Beauvilliers continua le
service, mais à son refus public. Le marquis de Gesvres avait tort. Le
premier gentilhomme de la chambre n'a que le commandement dans la
chambre, etc., et nul service. C'est le grand chambellan qui l'a tout
entier, et nul commandement. Ce n'est qu'en son absence que le premier
gentilhomme de la chambre sert\,; mais si le premier gentilhomme de la
chambre est absent, et qu'il n'y en ait aucun autre, ce n'est point le
grand chambellan qui commande dans la chambre, c'est le premier valet de
chambre.

J'ai vu, mais fort rarement, Monseigneur et Mgrs ses fils au petit
couvert, debout, sans que jamais le roi leur ait proposé un siège. J'y
ai vu continuellement les princes du sang et les cardinaux tout du long.
J'y ai vu assez souvent Monsieur, ou venant de Saint-Cloud voir le roi,
ou sortant du conseil des dépêches, le seul où il entrait. Il donnait la
serviette et demeurait debout. Un peu après, le roi, voyant qu'il ne
s'en allait point, lui demandait s'il ne voulait point s'asseoir\,; il
faisait la révérence, et le roi ordonnait qu'on lui apportât un siège.
On mettait un tabouret derrière lui. Quelques moments après, le roi lui
disait\,: «\,Mon frère, asseyez-vous donc.\,» Il faisait la révérence et
s'asseyait jusqu'à la fin du dîner, qu'il présentait la serviette.
D'autres fois, quand il venait de Saint-Cloud, le roi en arrivant à
table demandait un couvert pour Monsieur, ou bien lui demandait s'il ne
voulait pas dîner. S'il le refusait, il s'en allait un moment après sans
qu'il fût question de siège\,; s'il l'acceptait, le roi demandait un
couvert pour lui. La table était carrée\,; il se mettait à un bout, le
dos au cabinet. Alors le grand chambellan, s'il servait, ou le premier
gentilhomme de la chambre, donnait à boire et des assiettes à Monsieur,
et prenait de lui celles qu'il ôtait, tout comme il faisait au roi\,;
mais Monsieur recevait tout ce service avec une politesse fort marquée.
S'ils allaient à son lever, comme cela leur arrivait quelquefois, ils
ôtaient le service au premier gentilhomme de sa chambre, et le
faisaient, dont Monsieur se montrait fort satisfait. Quand il était au
dîner du roi, il remplissait et il égayait fort la conversation. La,
quoique à table, il donnait la serviette au roi en s'y mettant et en
sortant\,; et en la rendant au grand chambellan, il y lavait. Le roi,
d'ordinaire, parlait peu à son dîner, quoique par-ci par-là quelques
mots, à moins qu'il n'y eût de ces seigneurs familiers avec qui il
causait un peu plus, ainsi qu'à son lever.

De grand couvert à dîner, cela était extrêmement rare\,: quelques
grandes fêtes, ou à Fontainebleau quelquefois, quand la reine
d'Angleterre y était. Aucune dame ne venait au petit couvert. J'y ai
seulement vu très rarement la maréchale de La Mothe, qui avait conservé
cela d'y avoir amené les enfants de France, dont elle avait été
gouvernante. Dès qu'elle y paraissait, on lui apportait un siège, et
elle s'asseyait, car elle était duchesse à brevet.

Au sortir de table, le roi rentrait tout de suite dans son cabinet.
C'était là un des moments de lui parler, pour des gens distingués. Il
s'arrêtait à la porte un moment à écouter, puis il entrait, et très
rarement l'y suivait-on, jamais sans le lui demander, et c'est ce qu'on
n'osait guère. Alors il se mettait avec celui qui le suivait dans
l'embrasure de la fenêtre la plus proche de la porte du cabinet, qui se
fermait aussitôt, et que l'homme qui parlait au roi rouvrait lui-même
pour sortir, en quittant le roi. C'était encore le temps des bâtards et
des valets intérieurs, quelquefois des bâtiments, qui attendaient dans
les cabinets de derrière, excepté le premier médecin qui était toujours
au dîner, et qui suivait dans les cabinets. C était aussi le temps où
Monseigneur se trouvait quand il n'avait pas vu le roi le matin. Il
entrait et sortait par la porte de la galerie.

Le roi s'amusait à donner à manger à ses chiens couchants, et
{[}restait{]} avec eux plus ou moins, puis demandait sa garde-robe, et
changeait devant le très peu de gens distingués qu'il plaisait au
premier gentilhomme de la chambre d'y laisser entrer, et tout de suite
le roi sortait par derrière et par son petit degré dans la cour de
Marbre pour monter en carrosse\,; depuis le bas de ce degré jusqu'à son
carrosse, lui parlait qui voulait, et de même en revenant.

Le roi aimait extrêmement l'air, et quand il en était privé, sa santé en
souffrait par des maux de tête et par des vapeurs que lui avait causées
un grand usage des parfums autrefois, tellement qu'il y avait bien des
années que, excepté l'odeur de la fleur d'orange, il n'en pouvait
souffrir aucune, et qu'il fallait être fort en garde de n'en avoir
point, pour peu qu'on eût à l'approcher.

Comme il était peu sensible au froid et au chaud, même à la pluie, il
n'y avait que des temps extrêmes qui l'empêchassent de sortir tous les
jours. Ces sorties n'avaient que trois objets\,: courre le cerf, au
moins une fois la semaine, et souvent plusieurs, à Marly et à
Fontainebleau, avec ses meutes et quelques autres\,; tirer dans ses
parcs, et homme en France ne tirait si juste, si adroitement ni de si
bonne grâce, et il y allait aussi une ou deux fois la semaine, surtout
les dimanches et les fêtes qu'il ne voulait point de grandes chasses, et
qu'il n'avait point d'ouvriers\,; les autres jours voir travailler et se
promener dans ses jardins et ses bâtiments\,; quelquefois des promenades
avec des dames, et la collation pour elles, dans la forêt de Marly et
dans celle de Fontainebleau, et, dans ce dernier lieu, des promenades
avec toute la cour autour du canal, qui était un spectacle magnifique où
quelques courtisans se trouvaient à cheval. Aucuns ne le suivaient en
ses autres promenades que ceux qui étaient en charges principales qui
approchaient le plus de sa personne, excepté lorsque, assez rarement, il
se promenait dans ses jardins de Versailles, où lui seul était couvert,
ou dans ceux de Trianon, lorsqu'il y couchait et qu'il y était pour
quelques jours, non quand il y allait de Versailles s'y promener et
revenir après. À Marly, de même\,; mais s'il y demeurait, tout ce qui
était du voyage avait toute liberté de l'y suivre dans les jardins, l'y
joindre, l'y laisser, en un mot, comme ils voulaient.

Ce lieu avait encore un privilège qui n'était pour nul autre. C'est
qu'en sortant du château, le roi disait tout haut\,: \emph{Le chapeaux
messieurs\,!} et aussitôt courtisans, officiers des gardes du corps,
gens des bâtiments se couvraient tous, en avant, en arrière, à côté de
lui, et il aurait trouvé mauvais si quelqu'un eût non seulement manqué,
mais différé à mettre son chapeau\,; et cela durait toute la promenade,
c'est-à-dire quelquefois quatre et cinq heures en été, ou en d'autres
saisons, quand il mangeait de bonne heure à Versailles pour s'aller
promener à Marly, et n'y point coucher.

La chasse du cerf était plus étendue. Y allait à Fontainebleau qui
voulait\,; ailleurs, il n'y avait que ceux qui en avaient obtenu la
permission une fois pour toutes, et ceux qui en avaient obtenu le
justaucorps, qui était uniforme, bleu, avec des galons, un d'argent
entre deux d'or, doublé de rouge. Il y en avait un assez grand nombre,
mais jamais qu'une partie à la fois que le hasard rassemblait. Le roi
aimait à y avoir une certaine quantité, mais le trop l'importunait et
troublait la chasse. Il se plaisait qu'on l'aimât, mais il ne voulait
pas qu'on y allât sans l'aimer\,; il trouvait cela ridicule, et ne
savait aucun mauvais gré à ceux qui n'y allaient jamais.

Il en était de même du jeu, qu'il voulait gros et continuel dans le
salon de Marly pour le lansquenet, et force tables d'autres jeux par
tout le salon. Il s'amusait volontiers à Fontainebleau les jours de
mauvais temps à voir jouer les grands joueurs à la paume où il avait
excellé autrefois, et à Marly très souvent, à voir jouer au mail, où il
avait aussi été fort adroit.

Quelquefois les jours qu'il n'y avait point de conseil, qui n'étaient
pas maigres, et qu'il était à Versailles, il allait dîner à Marly ou à
Trianon avec M\textsuperscript{me} la duchesse de Bourgogne,
M\textsuperscript{me} de Maintenon et des dames, et cela devint beaucoup
plus ordinaire ces jours-là les trois dernières années de sa vie. Au
sortir de table, en été, le ministre qui devait travailler avec lui
arrivait, et quand le travail était fini, il passait jusqu'au soir à se
promener avec les dames, à jouer avec elles, et assez souvent à leur
faire tirer une loterie toute de billets noirs, sans y rien mettre\,;
c'était ainsi une galanterie de présents qu'il leur faisait, au hasard,
de choses à leur usage, comme d'étoffes et d'argenterie, ou de joyaux ou
beaux ou jolis, pour donner plus au hasard. M\textsuperscript{me} de
Maintenon tirait comme les autres, et donnait presque toujours
sur-le-champ ce qu'elle avait gagné. Le roi ne tirait point, et souvent
il y avait plusieurs billets sous le même lot. Outre ces jours-là, il y
avait assez souvent de ces loteries quand le roi dînait chez
M\textsuperscript{me} de Maintenon. Il s'avisa fort tard de ces dîners,
qui furent longtemps rares, et qui, sur la fin, vinrent à une fois la
semaine avec les dames familières, avec musique et jeu. À ces loteries,
il n'y avait que des dames du palais et des dames familières, et plus de
dames du palais depuis la mort de M\textsuperscript{me} la Dauphine\,;
mais il y en avait trois, M\textsuperscript{me}s de Lévi, Dangeau et
d'O, qui étaient familières. L'été, le roi travaillait chez lui, au
sortir de table, avec les ministres, et lorsque les jours
s'accourcissaient, il y travaillait le soir chez M\textsuperscript{me}
de Maintenon.

À son retour de dehors, lui parlait qui voulait, depuis son carrosse
jusqu'au bas de son petit degré. Il se rhabillait comme il avait changé
d'habit, et restait dans son cabinet. C'était le meilleur temps des
bâtards, des valets intérieurs et des bâtiments. Ces intervalles-là, qui
arrivaient trois fois par jour, étaient leur temps, celui des
rapporteurs de vive voix ou par écrit, celui où le roi écrivait, s'il
avait à écrire lui-même. Au retour de ses promenades, il était une heure
et plus dans ses cabinets\,; puis passait chez M\textsuperscript{me} de
Maintenon, et en chemin lui parlait encore qui voulait.

À dix heures il était servi. Le maître d'hôtel en quartier, ayant son
bâton, allait avertir le capitaine des gardes en quartier dans
l'antichambre de M\textsuperscript{me} de Maintenon, où, averti lui-même
par un garde de l'heure, il venait d'arriver. Il n'y avait que les
capitaines des gardes qui entrassent dans cette antichambre, qui était
fort petite, entre la chambre où était le roi et M\textsuperscript{me}
de Maintenon, et une autre très petite antichambre pour les officiers,
et le dessus public du degré où le gros était. Le capitaine des gardes
se montrait à l'entrée de la chambre, disant au roi qu'il était servi,
revenait dans l'instant dans l'antichambre. Un quart d'heure après, le
roi venait souper, toujours au grand couvert\,; et depuis l'antichambre
de M\textsuperscript{me} de Maintenon jusqu'à sa table, lui parlait
encore qui voulait.

À son souper, toujours au grand couvert, avec la maison royale,
c'est-à-dire uniquement les fils et filles de France et les petits-fils
et petites-filles de France, étaient toujours grand nombre de
courtisans, et de dames tant assises que debout, et la surveille des
voyages de Marly toutes celles qui voulaient y aller. Cela s'appelait se
présenter pour Marly. Les hommes demandaient le même jour le matin, en
disant au roi seulement\,: «\,Sire, Marly\,! » Les dernières années le
roi s'en importuna. Un garçon bleu écrivait dans la galerie les noms de
ceux qui demandaient, et qui y allaient se faire écrire. Pour les dames,
elles continuèrent toujours à se présenter.

Après souper, le roi se tenait quelques moments debout, le dos au
balustre du pied de son lit, environné de toute la cour\,; puis avec des
révérences aux dames passait dans son cabinet où, en arrivant, il
donnait l'ordre. Il y passait un peu moins d'une heure avec ses enfants
légitimes et bâtards, ses petits-enfants légitimes et bâtards, et leurs
maris ou leurs femmes, tous dans un cabinet, le roi dans un fauteuil,
Monsieur dans un autre, qui dans le particulier vivait avec le roi en
frère, Monseigneur debout ainsi que tous les autres princes, et les
princesses sur des tabourets. Madame y fut admise après la mort de
M\textsuperscript{me} la Dauphine. Ceux qui entraient par les derrières
s'y trouvaient, et qu'on a nommés, et les valets intérieurs avec
Chamarande, qui avait été premier valet de chambre en survivance de son
père, et qui était devenu depuis premier maître d'hôtel de
M\textsuperscript{me} la Dauphine de Bavière, et lieutenant général
distingué, fort à la mode dans le monde, et avec fort peu d'esprit un
fort galant homme et bien reçu partout.

Les dames d'honneur des princesses, et les dames du palais de jour,
attendaient dans le cabinet du conseil qui précédait celui où était le
roi, à Versailles et ailleurs. À Fontainebleau, où il n'y avait qu'un
grand cabinet, les dames des princesses, qui étaient assises, achevaient
le cercle avec les princesses, au même niveau et sur mêmes tabourets\,;
les autres dames étaient derrière, en liberté de demeurer debout, ou de
s'asseoir par terre sans carreau, comme plusieurs faisaient. La
conversation n'était guère que de chasse ou de quelque autre chose aussi
indifférente.

Le roi, voulant se retirer, allait donner à manger à ses chiens, puis
donnait le bonsoir, passait dans sa chambre à la ruelle de son lit, où
il faisait sa prière comme le matin\,; puis se déshabillait. Il donnait
le bonsoir d'une inclination de tête, et tandis qu'on sortait, il se
tenait debout au coin de la cheminée, où il donnait l'ordre au colonel
des gardes seul\,; puis commençait le petit coucher, où restaient les
grandes et secondes entrées ou brevets d'affaires. Cela était court. Ils
ne sortaient que lorsqu'il se mettait au lit. Ce moment en était un de
lui parler pour ces privilégiés. Alors tous sortaient quand ils en
voyaient un attaquer le roi, qui demeurait seul avec lui.

Lorsque le roi mourut, il y avait dix ou douze ans que ce qui n'avait
point ces entrées ne demeurait plus au coucher, depuis une longue
attaque de goutte que le roi avait eue, en sorte qu'il n'y avait plus de
grand coucher, et que la cour était finie au sortir du souper. Alors le
colonel des gardes prenait l'ordre, avec tous les autres\,; les
aumôniers de quartier, et le grand et le premier aumônier sortaient
après la prière.

Les jours de médecine, qui revenaient tous les mois au plus loin, il la
prenait dans son lit, puis entendait la messe où il n'y avait que les
aumôniers et les entrées. Monseigneur et la maison royale venaient le
voir un moment\,; puis M. du Maine, M. le comte de Toulouse, lequel y
demeurait peu, et M\textsuperscript{me} de Maintenon venaient
l'entretenir. Il n'y avait qu'eux et les valets intérieurs dans le
cabinet, la porte ouverte. M\textsuperscript{me} de Maintenon s'asseyait
dans le fauteuil au chevet du lit. Monsieur s'y mettait quelquefois,
mais avant que M\textsuperscript{me} de Maintenon fût venue, et
d'ordinaire, après qu'elle était sortie\,; Monseigneur toujours
debout\,:, et les autres de la maison royale un moment. M. du Maine qui
y passait toute la matinée, et qui était fort boiteux, se mettait auprès
du lit sur un tabouret, quand il n'y avait personne que
M\textsuperscript{me} de Maintenon et son frère. C'était où il tenait le
dé à les amuser tous deux, et où souvent il en faisait de bonnes. Le roi
dînait dans son lit, sur les trois heures où tout le monde entrait, puis
se levait, et il n'y demeurait que les entrées. Il passait après dans
son cabinet où il tenait conseil, et après il allait à l'ordinaire chez
M\textsuperscript{me} de Maintenon, et soupait à dix heures au grand
couvert.

Le roi n'a de sa vie manqué la messe qu'une fois à l'armée, un jour de
grande marche, ni aucun jour maigre, à moins de vraie et très rare
incommodité. Quelques jours avant le carême, il tenait un discours
public à son lever, par lequel il témoignait qu'il trouverait fort
mauvais qu'on donnât à manger gras à personne, sous quelque prétexte que
ce fût, et ordonnait au grand prévôt d'y tenir la main, et de lui en
rendre compte. Il ne voulait pas non plus que ceux qui mangeaient gras
mangeassent ensemble, ni autre chose que bouilli et rôti fort court, et
personne n'osait outrepasser ses défenses, car on s'en serait bientôt
ressenti. Elles s'étendaient à Paris, où le lieutenant de police y
veillait et lui en rendait compte. Il y avait douze ou quinze ans qu'il
ne faisait plus de carême. D'abord quatre jours maigre, puis trois, et
les quatre derniers de la semaine sainte. Alors son très petit couvert
était fort retranché les jours qu'il faisait gras\,; et le soir au grand
couvert tout était collation, et le dimanche tout était en poisson\,;
cinq ou six plats gras tout au plus, tant pour lui que pour ceux qui à
sa table mangeaient gras. Le vendredi saint grand couvert matin et soir,
en légumes, sans aucun poisson, ni à pas une de ses tables. Il manquait
peu de sermons l'avent et le carême, et aucune des dévotions de la
semaine sainte, des grandes fêtes, ni les deux processions du saint
sacrement, ni celles des jours de l'ordre du Saint-Esprit, ni celle de
l'Assomption. Il était très respectueusement à l'église. À sa messe tout
le monde était obligé de se mettre à genoux au \emph{Sanctus}, et d'y
demeurer jusqu'après la communion du prêtre\,; et s'il entendait le
moindre bruit ou voyait causer pendant la messe, il le trouvait fort
mauvais. Il manquait rarement le salut les dimanches, s'y trouvait
souvent les jeudis, et toujours pendant toute l'octave du saint
sacrement. Il communiait toujours en collier de l'ordre, rabat et
manteau, cinq fois l'année, le samedi saint à la paroisse, les autres
jours à la chapelle, qui étaient la veille de la Pentecôte, le jour de
l'Assomption, et la grand'messe après, la veille de la Toussaint et la
veille de Noël, et une messe basse après celle où il avait communié, et
ces jours-là point de musique à ses messes, et à chaque fois il touchait
les malades. Il allait à vêpres les jours de communion, et après vêpres
il travaillait dans son cabinet, avec son confesseur, à la distribution
des bénéfices qui vaquaient. Il n'y avait rien de plus rare que de lui
voir donner aucun bénéfice en d'autres temps. Il allait le lendemain à
la grand'messe et à vêpres, à matines et à trois messes de minuit en
musique, et c'était un spectacle admirable que la chapelle\,; le
lendemain à la grand'messe, à vêpres, au salut. Le jeudi saint, il
servait les pauvres à dîner, et après la collation, il ne faisait
qu'entrer dans son cabinet, passait à la tribune adorer le saint
sacrement, et se venait coucher tout de suite. À la messe, il disait son
chapelet (il n'en savait pas davantage), et toujours à genoux, excepté à
l'évangile. Aux grand'messes, il ne s'asseyait dans son fauteuil qu'aux
temps où on a coutume de s'asseoir. Aux jubilés, il faisait presque
toujours ses stations à pied\,; et tous les jours de jeune, et ceux du
carême où il mangeait maigre, il faisait seulement collation.

Il était toujours vêtu de couleur plus ou moins brune avec une légère
broderie, jamais sur les tailles, quelquefois rien qu'un bouton d'or,
quelquefois du velours noir. Toujours une veste de drap ou de satin
rouge, ou bleue ou verte, fort brodée. Jamais de bague, et jamais des
pierreries qu'à ses boucles de souliers, de jarretières, et de chapeau
toujours bordé de point d'Espagne avec un plumet blanc. Toujours le
cordon bleu dessous, excepté des noces ou autres fêtes pareilles qu'il
le portait par-dessus, fort long avec pour huit ou dix millions de
pierreries. Il était le seul de la maison royale et des princes du sang
qui portât l'ordre dessous, en quoi fort peu de chevaliers de l'ordre
l'imitaient, et aujourd'hui presque aucun ne le porte dessus, les bons
par honte de leurs confrères, et ceux-là embarrassés de le porter.

Jusqu'à la promotion de 1661 inclusivement, les chevaliers de l'ordre en
portaient tous le grand habit à toutes les trois cérémonies de l'ordre,
y allaient à l'offrande, et y communiaient. Le roi retrancha lors le
grand habit, l'offrande et la communion. Henri III l'avait prescrite à
cause des huguenots et de la Ligue. La vérité est qu'une communion
générale, publique, en pompe, prescrite à jour nommé trois fois l'an à
des courtisans, devient une terrible et bien dangereuse pratique, qu'il
a été très bon d'ôter\,; mais pour l'offrande, qui était majestueuse où
il n'y a plus que le roi qui y aille, et le grand habit de l'ordre
réduit aux jours de réception, et le plus souvent encore seulement pour
ceux qui sont reçus, cela ôte toute la beauté de la cérémonie. À l'égard
du repas en réfectoire avec le roi, on a dit ailleurs ce qui l'a fait
supprimer.

Il ne se passait guère quinze jours que le roi n'allât à Saint-Germain,
même après la mort du roi Jacques II. La cour de Saint-Germain venait
aussi à Versailles, mais plus souvent à Marly, et souvent y souper, et
nulle fête de cérémonie ou de divertissement qu'elle n'y fût invitée,
qu'elle vînt et dont elle ne reçût tous les honneurs. Ils étaient
réciproquement convenus de se recevoir et se conduire dans le milieu de
leur appartement. À Marly, le roi les recevait et les conduisait à la
porte du petit salon du côté de la Perspective, et les y voyait
descendre et monter dans leur chaise à porteurs\,; à Fontainebleau, tous
les voyages, au haut de l'escalier à fer à cheval, depuis que le roi
leur eut accordé de ne les aller plus recevoir et conduire au bout de la
forêt. Rien n'était pareil aux soins, aux égards, à la politesse du roi
pour eux, ni à l'air de majesté et de galanterie avec lequel cela se
passait à chaque fois. On en a parlé ailleurs plus au long. À Marly, ils
demeuraient en arrivant un quart d'heure dans le salon, debout, au
milieu de toute la cour, puis passaient chez le roi ou chez
M\textsuperscript{me} de Maintenon. Le roi n'entrait jamais dans le
salon que pour le traverser, pour des bals, ou pour y voir jouer un
moment le jeune roi d'Angleterre ou l'électeur de Bavière. Les jours de
naissance, ou de la fête du roi et de sa famille, si observés dans les
cours de l'Europe, ont toujours été inconnus dans celle du roi\,; en
sorte que jamais il n'y en a été fait la moindre mention en rien, ni
différence aucune de tous les autres jours de l'année.

Louis XIV ne fut regretté que de ses valets intérieurs, de peu d'autres
gens, et des chefs de l'affaire de la constitution. Son successeur n'en
était pas en âge. Madame n'avait pour lui que de la crainte et de la
bienséance. M\textsuperscript{me} la duchesse de Berry ne l'aimait pas,
et comptait aller régner. M. le duc d'Orléans n'était pas payé pour le
pleurer, et ceux qui l'étaient n'en firent pas leur charge.
M\textsuperscript{me} de Maintenon était excédée du roi depuis la perte
de la Dauphine\,; elle ne savait qu'en faire ni à quoi l'amuser\,; sa
contrainte en était triplée, parce qu'il était beaucoup plus chez elle,
ou en parties avec elle. Sa santé, ses affaires, les manèges qui avaient
fait tout faire, ou pour parler plus exactement, qui avaient tout
arraché pour le duc du Maine, avaient fait essuyer continuellement
d'étranges humeurs, et souvent des sorties à M\textsuperscript{me} de
Maintenon. Elle était venue à bout de ce qu'elle avait voulu\,; ainsi,
quoi qu'elle perdît en perdant le roi, elle se sentit délivrée, et ne
fut capable que de ce sentiment. L'ennui et le vide dans la suite
rappelèrent les regrets\,; mais comme elle n'influa plus rien de sa
retraite, il n'est pas temps de parler d'elle, ni des occupations
qu'elle s'y fit.

On a vu jusqu'à quelle joie, à quelle barbare indécence le prochain
point de vue de la toute-puissance jeta le duc du Maine. La tranquillité
glacée de son frère ne s'en haussa ni baissa. M\textsuperscript{me} la
Duchesse, affranchie de tous ses liens, n'avait plus besoin de l'appui
du roi, elle n'en sentait que la crainte et la contrainte, elle ne
pouvait souffrir M\textsuperscript{me} de Maintenon\,; elle ne pouvait
douter de la partialité du roi pour le duc du Maine dans leur procès de
la succession de M. le Prince\,; on lui reprochait depuis toute sa vie
qu'elle n'avait point de cœur, mais seulement un gésier\,; elle se
trouva donc fort à son aise et en liberté, et n'en fit pas grandes
façons.

M\textsuperscript{me} la duchesse d'Orléans me surprit. Je m'étais
attendu à de la douleur\,; je n'aperçus que quelques larmes qui, sur
tous sujets, lui coulaient très aisément des yeux, et qui furent bientôt
taries. Son lit, qu'elle aimait fort, suppléa à tout pendant quelques
jours, avec la façon de l'obscurité qu'elle ne haïssait pas. Mais
bientôt les rideaux des fenêtres se rouvrirent, et il n'y parut plus
qu'en rappelant de fois à autre quelque bienséance.

Pour les princes du sang, c'étaient des enfants.

La duchesse de Ventadour et le maréchal de Villeroy donnèrent un peu la
comédie\,; pas un autre n'en prit même la peine. Mais quelques vieux et
plats courtisans comme Dangeau, Cavoye, et un très petit nombre
d'autres, qui se voyaient hors de toute mesure, quoique tombés d'une
fort commune situation, regrettèrent de n'avoir plus à se
cuider\footnote{Vieux mot synonyme de \emph{croire, penser}.} parmi les
sots, les ignorants, les étrangers, dans les raisonnements et
l'amusement journalier d'une cour qui s'éteignait avec le roi.

Tout ce qui la composait était de deux sortes\,: les uns, en espérance
de figurer, de se mêler, de s'introduire, étaient ravis de voir finir un
règne sous lequel il n'y avait rien pour eux à attendre\,; les autres,
fatigués d'un joug pesant, toujours accablant, et des ministres bien
plus que du roi, étaient charmés de se trouver au large\,; tous, en
général, d'être délivrés d'une gêne continuelle, et amoureux des
nouveautés.

Paris, las d'une dépendance qui avait tout assujetti, respira dans
l'espoir de quelque liberté, et dans la joie de voir finir l'autorité de
tant de gens qui en abusaient. Les provinces, au désespoir de leur ruine
et de leur anéantissement, respirèrent et tressaillirent de joie\,; et
les parlements et toute espèce de judicature, anéantie par les édits et
par les évocations, se flatta, les premiers de figurer, les autres de se
trouver affranchis. Le peuple ruiné, accablé, désespéré, rendit grâces à
Dieu, avec un éclat scandaleux, d'une délivrance dont ses plus ardents
désirs ne doutaient plus.

Les étrangers ravis d'être enfin, après un si long cours d'années,
défaits d'un monarque qui leur avait si longuement imposé la loi, et qui
leur avait échappé par une espèce de miracle au moment qu'ils comptaient
le plus sûrement de l'avoir enfin subjugué, se continrent avec plus de
bienséance que les Français. Les merveilles des trois premiers quarts de
ce règne de plus de soixante-dix ans, et la personnelle magnanimité de
ce roi jusqu'alors si heureux, et si abandonné après de la fortune
pendant le dernier quart de son règne, les avait justement éblouis. Ils
se firent un honneur de lui rendre après sa mort ce qu'ils lui avaient
constamment refusé pendant sa vie. Nulle cour étrangère n'exulta\,;
toutes se piquèrent de louer et d'honorer sa mémoire.

L'empereur en prit le deuil comme d'un père\,; et quoiqu'il y eût quatre
ou cinq mois depuis la mort du roi jusqu'au carnaval, toute espèce de
divertissement fut défendu à Vienne\,; et observé exactement. Le
monstrueux fut que, sur la fin du carnaval, il y eut un bal unique, avec
une espèce de fête, que le comte du Luc, ambassadeur de France, n'eut
pas honte de donner aux dames qui le séduisirent par l'ennui d'un
carnaval si triste. Cette complaisance ne le fit pas estimer à Vienne ni
ailleurs. En France on se contenta de l'ignorer. Pour nos ministres et
les intendants des provinces, les financiers, et ce qu'on peut appeler
la canaille, ceux-là sentirent toute l'étendue de leur perte. Nous
allons voir si le royaume eut tort ou raison des sentiments qu'il
montra, et s'il trouva bientôt après qu'il eût gagné ou perdu.

\hypertarget{chapitre-vi.}{%
\chapter{CHAPITRE VI.}\label{chapitre-vi.}}

1715

~

{\textsc{M. le duc d'Orléans surpris par la mort du roi.}} {\textsc{- La
pompe funèbre réduite au plus simple.}} {\textsc{- Points d'états
généraux.}} {\textsc{- Liberté accordée aux pairs sur les usurpations du
parlement, puis commuée en protestations et promesses de décision.}}
{\textsc{- Séance au parlement pour la régence.}} {\textsc{- Le duc de
La Rochefoucauld reçu au parlement.}} {\textsc{- Scélératesse et piège
du premier président, que le duc de La Rochefoucauld évite avec
noblesse.}} {\textsc{- Duc du Maine arrive en séance.}} {\textsc{-
Protestation des pairs sur les usurpations du parlement à leur égard, et
interpellation à M. le duc d'Orléans sur sa promesse de les juger dès
que les affaires du gouvernement seraient réglées, à laquelle il
acquiesce en pleine séance.}} {\textsc{- Députation du parlement va
quérir le testament et le codicille du roi.}} {\textsc{- Stairs dans une
lanterne\,; le duc de Guiche, bien payé, dans une autre.}} {\textsc{- Le
régiment des gardes aux avenues.}} {\textsc{- Dreux, conseiller de la
grand'chambre, fait à haute voie lecture du testament, et l'abbé Menguy,
conseiller clerc de la grand'chambre, du codicille.}} {\textsc{-
Discours de M. le duc d'Orléans.}} {\textsc{- Le testament du roi abrogé
quant à l'administration de l'État.}} {\textsc{- Forte dispute publique,
puis particulière, entre M. le duc d'Orléans et le duc du Maine sur le
codicille du roi.}} {\textsc{- Sur l'avis du duc de La Force, je fais
passer la dispute dans la quatrième des enquêtes.}} {\textsc{- Je l'y
fais après suspendre, et fais lever la séance et remettre à
l'après-dînée.}} {\textsc{- M\textsuperscript{me} la Duchesse\,; en
haine des bâtards, en récente et secrète mesure avec M. le duc
d'Orléans, qui déclare M. le Duc, en séance, chef du conseil de
régence.}} {\textsc{- Le régent rend au parlement les remontrances, lui
promet de lui parler de la forme du gouvernement, et lève la séance avec
grand applaudissement.}} {\textsc{- Mesures au Palais-Royal, où je vais
dîner.}} {\textsc{- Courte joie du maréchal de Villeroy, etc.}}
{\textsc{- Séance de l'après-dînée.}} {\textsc{- Discours de M. le duc
d'Orléans.}} {\textsc{- Le duc du Maine ose à peine répondre.}}
{\textsc{- Le codicille est en tout abrogé.}} {\textsc{- Le régent est
revêtu de tout pouvoir.}} {\textsc{- Contenance des bâtards.}}
{\textsc{- Acclamations.}} {\textsc{- Compliment du régent, qui propose
six conseils et s'y appuie de Mgr le duc de Bourgogne, et pourquoi.}}
{\textsc{- Applaudissements.}} {\textsc{- Fin de la séance.}} {\textsc{-
Le régent retourne à Versailles, où, en arrivant, Madame lui demande
pour grâce unique l'exclusion entière de l'abbé Dubois de tout, et en
tire publiquement sa parole.}}

~

La mort du roi surprit la paresse de M. le duc d'Orléans, comme si elle
n'avait pu être prévue\,; il en était demeuré où on a vu que je l'avais
laissé. Il n'avait fait aucun progrès dans aucune des résolutions qu'il
fallait avoir prises, tant sur les affaires que sur les divers choix\,;
et il fut noyé d'ordres à donner, et de choses à régler, toutes plus
petites ou plus médiocres les unes que les autres, mais toutes si
provisoires et si instantes qu'il lui arriva ce que je lui avais prédit
pour ses premiers jours, qu'il n'aurait pas le temps de penser à rien
d'important.

Deux jours auparavant M\textsuperscript{me} Sforce m'avait envoyé prier
de passer chez elle un matin. Elle était inquiète, et
M\textsuperscript{me} la duchesse d'Orléans encore plus, des résolutions
de M. le duc d'Orléans et de ses choix. Ni l'une ni l'autre ne pouvaient
croire qu'il fût demeuré dans l'inaction intérieure. J'assurai
M\textsuperscript{me} Sforce qu'elle n'en serait que trop tôt
convaincue, et elle et M\textsuperscript{me} la duchesse d'Orléans le
furent en effet pleinement quatre jours après.

J'appris la mort du roi à mon réveil. J'allai aussitôt faire ma
révérence au nouveau monarque. Le premier flot y avait déjà passé\,; je
m'y trouvai presque seul. Je fus de là chez M. le duc d'Orléans que je
trouvai enfermé, et tout son appartement plein à n'y pas pouvoir faire
tomber une épingle par terre. Je le pris à part dans son cabinet pour
faire un dernier effort sur la convocation des états généraux, qui fut
entièrement inutile, et pour le faire souvenir de la parole qu'il
m'avait donnée, et à dix ou douze pairs avec moi, de trouver bon que
nous demeurassions couverts lorsque nos voix seraient demandées, et pour
les autres indécences des séances du parlement, dont il convint avec
moi. Je le fis souvenir aussi de ce que je lui avais proposé sur ce qui
regardait la totalité de la pompe funèbre, et qu'il avait agréé\,:
c'était d'épargner la dépense, la longueur et les disputes que ferait
naître une si longue cérémonie, et d'en user, quoique le roi n'eût rien
ordonné là-dessus, comme il avait été pratiqué pour Louis XIII, qui
avait tout défendu et réduit au plus simple. M. le duc d'Orléans s'y
conforma en effet, et il ne se trouva personne qui se souciât assez du
feu roi pour relever un retranchement si entier, et qu'il n'avait point
ordonné.

Je montai de là chez le duc de La Trémoille, où nous devions nous
assembler aussitôt après la mort du roi, et où presque tous les ducs qui
étaient à Versailles étaient déjà en très grand nombre. M. de La
Trémoille était l'ancien de tous ceux qui avaient un appartement au
château. M. de Reims, le premier des dix ou douze ensemble qui avaient
vu M. le duc d'Orléans sur le bonnet, rendit compte de la liberté qu'il
nous avait accordée, et moi après, du renouvellement que j'en venais de
prendre tout à l'instant. L'union et les résolutions furent bien
confirmées, et la totale séparation du premier président sur le pied
sans mesure où nous étions avec lui\,; après quoi on se sépara.

Je revis bientôt après M. le duc d'Orléans qui se trouva un peu moins
accablé, pendant l'heure du dîner, de tout le monde, qui m'avoua qu'il
n'avait fait aucune liste, ni aucun choix par delà ceux dont j'ai parlé,
ni pris son parti sur rien. Ce n'était pas le temps de gronder ni de
reproches. Je me contentai de hausser les épaules, et de l'exhorter
d'être au moins en garde contre les sollicitations et les ministres. Je
m'assurai encore de la totale expulsion de Pontchartrain et de
Desmarets, sitôt que les conseils seraient formés et déclarés, et que le
nouveau gouvernement commencerait. Puis je le mis sur le testament et
sur le codicille, et je lui demandai comment il prétendait se conduire
là-dessus au parlement, où nous allions le lendemain, et où la lecture
de ces deux pièces serait faite.

C'était l'homme du monde le plus ferme dans son cabinet tête à tête, et
qui l'était le moins ailleurs. Il me promit merveilles\,; je lui en
remontrai l'importance et tout ce dont il y allait pour lui. Je fus près
de deux heures avec lui. Je passai un moment chez M\textsuperscript{me}
la duchesse d'Orléans, qui était entre ses rideaux avec force femmes en
silence, et m'en vins dîner avec gens qui m'attendaient chez moi, pour
m'en aller après à Paris. Il était fort tard, nous eûmes à raisonner
après le dîner, et j'allais partir, lorsque M. le duc d'Orléans m'envoya
chercher, et quelques ducs qui se trouvèrent chez moi, qu'on n'eut pas
la peine d'aller trouver ailleurs. Nous fûmes donc chez lui. Il était
dans son entresol avec le duc de Sully, M. de Metz, et quelques autres
ducs qu'il avait mandés, car il avait envoyé chercher tous ceux qu'on ne
trouverait pas partis. Il était huit heures du soir.

Là M. le duc d'Orléans nous fit un discours bien doré pour nous
persuader de n'innover rien le lendemain comme il nous avait permis de
le faire, en représentant le trouble que cela pourrait apporter dans les
plus grandes affaires de l'État qui devaient y être réglées, telles que
la régence et l'administration du royaume, et l'indécence qui
retomberait sur nous de les arrêter, et au moins les retarder, pour nos
intérêts particuliers.

Plusieurs de ceux qui étaient là se trouvèrent bien étonnés d'un
changement si subit depuis la fin de la matinée.

D'Antin, M. de Metz, et quelques autres insistèrent sur la situation où
nous jetait l'étrange tour qu'on avait su donner à une affaire qu'on
nous avait fait entreprendre malgré nous\,; tout cela fut rappelé en peu
mots. M. de Sully, Charost, moi et quelques autres, M. de Reims sur tous
à qui la permission avait été donnée, et qui l'avions portée à tous de
sa part, moi tout récemment, et en la réitérant le matin de ce même jour
à la nombreuse assemblée chez le duc de La Trémoille, demandâmes quel
effet il pouvait attendre d'une telle variation, et de la considération
que la première dignité du royaume si blessée, et les personnes qui en
étaient revêtues croyaient au moins, pour la plupart, mériter de lui.
Son embarras fut extrême, mais sans s'ébranler. Nous nous regardâmes
tous, et nous nous dîmes les uns aux autres que ce qui nous était
demandé était impossible après ce qui s'était passé.

M. le duc d'Orléans parut fort peiné, avoua plusieurs fois que ce bonnet
était une usurpation insoutenable, que les autres dont nous nous
plaignions ne l'étaient pas moins\,; mais qu'il fallait y pourvoir en
temps et lieu, et ne pas troubler une séance si importante par une
querelle particulière\,; que plus elle était juste, puis il nous serait
obligé de la suspendre, plus nous mériterions de l'État, plus nous
serions approuvés du public de préférer les affaires générales aux
nôtres. «\,Mais, lui dis-je, monsieur, quand les publiques seront
réglées, vous vous moquerez de nous et des nôtres\,; et si nous ne
prenons une conjoncture telle que celle-ci, vous nous remettrez sans
fin, et nous vous aurons sacrifié nos intérêts en vain.\,» M. le duc
d'Orléans nous protesta merveilles, et nous engagea sa parole positive,
formelle, solennelle, de juger en notre faveur toutes nos disputes sur
les usurpations du parlement\,: bonnet, conseillers sur le banc, etc.,
aussitôt que les affaires publiques seraient débouchées. Je le suppliai
de prendre garde à l'engagement, de ne promettre que ce qu'il voudrait
tenir, et de ne se pas mettre à portée des plaintes et des sommations
qu'il pouvait s'assurer que nous ne lui épargnerions pas, si nous nous
apercevions qu'il cherchât à éluder sa parole. Il nous la donna bien
authentiquement de nouveau, et nous demanda la nôtre de ne rien innover
de nouveau le lendemain au parlement.

Ces messieurs étaient également faibles et mécontents. Ils grommelaient
sans oser s'expliquer. Ils sentaient l'importance de manquer la
conjoncture\,; mais accoutumés à la servitude, pas un n'osait hocher le
mors au prince qui représentait le feu roi, dont l'ombre leur faisait
encore frayeur. Ce murmure sourd dura quelque temps.

Comme je désespérai qu'il en sortît rien de résolu, je repris la parole.
Je dis à M. le duc d'Orléans que ce serait un grand embarras que
d'arrêter le lendemain tous les pairs qui s'étaient trouvés ce matin
chez le duc de La Trémoille, et ceux qu'ils auraient avertis en arrivant
à Paris\,; que de plus, je ne voyais pas comment les persuader de la
parole qu'il nous donnait de juger en notre faveur le bonnet et les
autres usurpations dont nous avions tant à nous plaindre, à moins qu'il
ne trouvât bon que, en entrant en séance le lendemain, un de nous
déclarât, avant toute affaire, la résolution que nous avions prise, en
même temps que, par respect pour ce qu'il nous venait de marquer qu'il
désirait de nous, et pour ne pas retarder les affaires publiques pour
notre intérêt particulier, nous consentions à laisser les choses comme
elles étaient jusqu'à ce que les affaires publiques fussent réglées\,;
que cependant nous protestions contre les usurpations, nommément du
bonnet, du conseiller sur les bouts des bancs, etc.\,; que néanmoins
nous ne les aurions pas souffertes davantage sans la parole positive,
expresse, nette, authentique qu'il nous avait donnée de juger, et de
nous faire pleine justice de toutes ces usurpations, aussitôt après que
les affaires publiques seraient réglées\,; et en même temps que celui
qui ferait la protestation se tournât vers lui et l'interpellât
d'affirmer la vérité de ce qui était avancé, et de la confirmer en
donnant de nouveau en pleine séance la même parole.

M. le duc d'Orléans commença lors à respirer, et ne fit nulle difficulté
sur la protestation, ni sur la réitération de sa parole. Il ajouta qu'il
me chargeait de faire la protestation, et toutes les plus fortes
assurances d'un jugement prompt, net et favorable, dès que les affaires
publiques se trouveraient réglées, et que la régence aurait pris une
forme stable et permanente pour le gouvernement de l'État.

M. de Reims et quelques autres avaient bien envie d'attaquer les bâtards
dès cette première séance\,; je les avais arrêtés avec peine par la
considération de trop d'entreprises à la fois, et la nécessité de nous
tirer d'abord de celles du parlement contre nous\,; mais dès qu'ils
virent la remise que M le duc d'Orléans en exigeait, ils voulurent
revenir aux bâtards. M. le duc d'Orléans remontra qu'avant toutes
choses, il était nécessaire d'empêcher qu'ils usurpassent une autorité
sous laquelle tout succomberait, et avec laquelle, si elle passait telle
qu'il était plus que vraisemblable que le testament du roi et son
codicille la leur donnait, il n'y avait personne, à commencer par lui,
qui pût leur résister en rien, bien moins leur contester ce dont ils se
trouvaient déjà en possession, sur laquelle il fallait attendre d'autres
temps et d'autres conjonctures. Ce raisonnement était vrai\,; je
l'appuyai d'autant plus que la vérité, qui m'en avait frappé, m'avait
rendu facile à m'engager, comme on l'a vu, à M\textsuperscript{me} la
duchesse d'Orléans qu'il ne se ferait rien contre les bâtards en ces
premières séances. Tout ce qui était présent s'y rendit, mais en prit
occasion d'insister su les usurpations du parlement.

M. le duc d'Orléans ne laissa rien à désirer là-dessus par les
engagements qu'il prit de nouveau, en conséquence de ce qui venait
d'être dit, et me chargea de nouveau de faire la protestation. Je m'en
défendis sur ce qu'elle serait plus dignement faite par M. de Reims, qui
par acclamation avec les autres me la remit. Je résistai, et, après
avoir demandé un moment de silence, je dis que j'avais trois raisons de
m'en excuser\,: la première, parce qu'il convenait qu'elle se fît par le
plus ancien, qui était M. de Reims, dès qu'il était présent\,; la
seconde, parce que, si on convenait qu'elle se fît par un autre, cela ne
pouvait regarder que M. d'Antin, ou un de ceux par qui l'affaire du
bonnet avait principalement passé\,; la troisième, parce que la
connaissance que j'avais de moi-même me faisait craindre de la faire
trop fortement, surtout dans l'interpellation à M. le duc d'Orléans.

On se moqua de moi sur tous les trois. M. de Reims déclara qu'il ne la
ferait point\,; qu'il fallait me la laisser, parce que je m'en
acquitterais mieux que personne, comme on dit toujours quand on veut se
décharger. D'Antin prétendit que d'être entré dans le détail de
l'affaire du bonnet n'avait aucun trait à rendre plus propre à faire la
protestation\,; M. le duc d'Orléans déclara qu'il agissait de si bonne
foi, qu'il trouverait bonne toute manière d'interpellation qui lui
pourrait être faite. Le bruit confus recommença\,; plusieurs me dirent à
l'oreille que la protestation et l'interpellation auraient tout un autre
poids dans ma bouche par la situation où personne n'ignorait que j'étais
avec M. le duc d'Orléans\,; en un mot, personne ne voulut s'en charger.
M. le duc d'Orléans se mit de plus belle à me presser de la faire\,; il
n'y eut pas moyen de m'en délivrer.

Tout réglé et convenu de la sorte, à notre grand regret a tous, il
fallut voir comment avertir les absents dans un terme aussi court d'un
changement si considérable, et dont il fallait qu'ils fussent instruits
avant d'entrer le lendemain matin au parlement. Nous convînmes que
chacun de nous enverrait chez les plus à portée de chez soi, les prier
le soir même de se rendre chez l'archevêque de Reims, le lendemain à
cinq heures du matin, en habit de parlement, pour chose très importante
et très pressée. Il était dix heures du soir lorsque nous arrivâmes à
Paris, et aussitôt chacun de nous fit à l'égard des autres ce qui était
convenu.

Presque tous se trouvèrent entre cinq et six heures du matin chez
l'archevêque de Reims, au bout du pont Royal, derrière l'hôtel de
Mailly. Il rendit compte de ce qui s'était passé la veille au soir chez
M. le duc d'Orléans. Le murmure fut grand, mais il n'y eut pas de
remède, il fallut bien s'y conformer.

J'essayai encore de me décharger de la protestation sur quelque autre.
Ce fut très inutilement\,; l'acclamation fut unanime. On m'opposa ce qui
était convenu la veille, qu'il ne s'y pouvait rien changer sans l'aveu
de M. le duc d'Orléans, qui avait voulu le premier, et toujours persisté
depuis à m'en charger\,; qu'il n'y avait ni temps de l'aller trouver ni
raison pour le faire changer là-dessus\,; et on finit par m'exhorter à
m'en acquitter avec courage, et à ne pas ménager dans l'interpellation
M. le duc d'Orléans, qui nous ménageait lui-même si peu, et sitôt par
une si subite variation, qui se pouvait nommer un manquement de parole.

Ces derniers propos me firent sentir la nécessité de tâcher de ramener
les esprits. Je représentai la situation embarrassante de M. le duc
d'Orléans entre le parlement dépositaire du testament et du codicille du
roi, et les bâtards pour la grandeur et l'autorité desquels il n'y avait
personne qui doutât qu'ils ne fussent faits\,; qu'il y allait du tout
pour lui, pour l'État, pour nous-mêmes que les bâtards ne remportassent
pas ce que le roi leur avait très vraisemblablement attribué\,; que la
permission que M. le duc d'Orléans nous avait donnée et réitérée était
un effet de son équité, de sa bonne volonté pour nous, de sa persuasion
de nos raisons\,; que ce qui s'était passé le soir était un effet de ses
réflexions\,; que nous ne pouvions le blâmer de ne vouloir pas hasarder
pour nous de réunir contre lui le parlement avec les bâtards, dans le
moment critique de décider du pouvoir du régent, ou de hasarder un éclat
et une suspension d'affaires si majeures et si instantes, où il n'aurait
qu'à perdre et nous encore plus, à qui le public, disposé comme il était
à notre égard, se prendrait de tout pour avoir voulu mêler nos querelles
particulières avec le règlement du gouvernement\,; qu'il était des temps
et des conjonctures où il était force de se prêter\,; et que rien ne
pouvait nous être plus dommageable que de souffrir la moindre autorité
dans l'État à des bâtards que nous ne pouvions ignorer être les plus
intéressés ennemis de notre dignité, et les plus grands de la plupart de
nos personnes\,; qu'enfin M. le duc d'Orléans, établi une fois dans
toute l'autorité qui appartenait à sa naissance et à sa régence, ne
pourrait ne nous pas savoir gré d'une différence qui lui devenait si
nécessaire pour y parvenir, ni cesser de penser comme il avait toujours
fait sur les usurpations du parlement à notre égard, ni nous manquer de
parole si solennellement donnée, comme il allait faire en plein
parlement, de nous juger et de nous rendre justice, dès qu'il aurait
donné ordre aux affaires publiques.

Ce petit discours me parut avoir ramené les esprits. Il était plus de
sept heures du matin, et nous nous en allâmes tous ensemble tout droit
au parlement avec tous nos carrosses et notre cortège à notre suite.

Nous le trouvâmes tout entier en séance avec M. de La Rochefoucauld, M.
d'Harcourt, et deux ou trois autres seulement, qui avaient mandé à M. de
Reims qu'ils se rapporteraient à ce qui serait réglé chez lui entre
nous, mais qu'ils n'y pouvaient venir, parce qu'ils étaient obligés de
se trouver à la réception de M. de La Rochefoucauld. Ce duc, qui n'avait
pu encore digérer ma préséance, avait toujours différé sa réception.
Mais il ne voulut pas se priver d'assister à tout ce qui devait se
passer au parlement à la mort du roi, et il s'était fait recevoir ce
même matin avant que personne arrivât. Je sus, presque aussitôt que je
fus entré en séance, que le premier président avait eu la hardiesse de
lui proposer, et encore en plein parlement, ce matin-là même, de
protester contre le jugement rendu par le feu roi entre lui et moi, et
d'en appeler au parlement, avec assurance qu'on y serait bien aise de
lui faire justice, et que le duc de La Rochefoucauld lui avait très
dignement répondu qu'il se tenait pour bien jugé par le roi, qu'il ne
songerait jamais à en appeler, et qu'il n'était plus question d'une
affaire finie et consommée\,; dont le premier président demeura confus.
Cet honnête homme ne cherchait qu'à mettre la discorde parmi nous. M. de
La Rochefoucauld en sentit le piège, et quel pas ce serait qu'appeler du
roi au parlement, et sagement se garda d'y tomber. En effet, dès que je
parus, il se baissa pour me laisser place au-dessus de lui, où je me mis
tout de suite, et je lui parlai de ce qui s'était passé la veille au
soir chez M. le duc d'Orléans, et le matin chez M. de Reims, que je vis
être fort peu de son goût. Je glissai avec lui, parce que nous n'étions
plus, depuis le jugement de préséance, sur le pied où nous avions été
autrefois, et parce que, sans savoir pourquoi, il était éloigné de M. le
duc d'Orléans.

Lorsque je sus ce qui se venait de passer, à mon égard, entre lui et le
premier président, je fus tenté de lui en faire une honnêteté\,; mais je
m'en retins pour laisser vieillir la rancune et l'habitude de ma
préséance, et ne rien hasarder avec un homme rogue, piqué encore et de
peu d'esprit, qui peut-être n'aurait pas trop bien reçu ce compliment.

Moins de demi-quart d'heure après que nous fûmes en séance, arrivèrent
les bâtards. M. du Maine crevait de joie. Le terme est étrange, mais on
ne peut rendre autrement son maintien. L'air riant et satisfait
surnageait à celui d'audace, de confiance, qui perçaient néanmoins, et à
la politesse qui semblait les combattre. Il saluait à droite et à
gauche, et perçait chacun de ses regards. Entré dans le parquet quelques
pas, son salut aux présidents eut un air de jubilation, que celui du
premier président réfléchissait d'une manière sensible. Aux pairs le
sérieux, ce n'est point trop dire le respectueux, la lenteur, la
profondeur de son inclination vers eux de tous les trois côtés fut
parlante. Sa tête demeura abaissée même en se relevant, tant est forte
la pesanteur des forfaits aux jours mêmes qu'on ne doute plus du
triomphe. Je le suivis exactement partout de mes regards, et je
remarquai sur les trois côtés également que l'inclination du salut qui
lui fut rendu fut roide et courte. Pour son frère, il n'y parut que son
froid ordinaire.

À peine étions-nous rassis que M. le Duc arriva, et l'instant d'après M.
le duc d'Orléans. Je laissai rasseoir le bruit qui accompagna son
arrivée, et comme je vis que le premier président se mettait en devoir
de vouloir parler, en se découvrant, je fis signe de la main, me
découvris et me couvris tout de suite, et je dis que j'étais chargé par
MM. les pairs de déclarer à la compagnie assemblée que ce n'était qu'en
considération des importantes et pressantes affaires publiques qu'il
s'agissait maintenant de régler, que les pairs voulaient bien encore
souffrir l'usurpation plus qu'indécente du bonnet, et les autres dont
ils avaient à se plaindre, et montrer par ce témoignage public la juste
préférence qu'ils donnaient aux affaires de l'État sur les leurs les
plus particulières, les plus chères et les plus justes, qu'ils ne
voulaient pas retarder d'un instant\,; mais qu'en même temps je
protestais au nom des pairs contre ces usurpations, et contre leur
durée, de la manière la plus expresse, la plus formelle, la plus
authentique, au milieu et en face de la plus auguste assemblée, et
autorisé de l'aveu et de la présence de tous les pairs\,; et que je
protestais encore que ce n'était qu'en considération de la parole
positive et authentique que M. le duc d'Orléans ci-présent nous donna
hier au soir dans son appartement, à Versailles, de décider et juger
nettement ces usurpations aussitôt que les affaires publiques du
gouvernement seront réglées\,; et qu'il a trouvé bon que je l'énonçasse
clairement ici comme je fais, et (me découvrant et me recouvrant
aussitôt) que j'eusse l'honneur de l'interpeller ici lui-même d'y
déclarer que telle est la parole qu'il nous a donnée, et sur laquelle
uniquement nous comptons, et en conséquence nous {[}nous{]} bornons
présentement à ce qui vient d'être dit et déclaré par moi, de son aveu
et permission expresse et formelle, en présence de quinze ou seize pairs
ci-présents qu'il manda hier au soirchez lui\footnote{}.

Le silence profond avec lequel je fus écouté témoigna la surprise de
toute l'assistance. M. le duc d'Orléans se découvrit, en affirmant ce
que je venais de dire, assez bas et l'air embarrassé, et se recouvrit.

Aussitôt après je regardai M. du Maine, qui me parut avoir un air
content d'en être quitte à si bon marché, et que mes voisins me dirent
avoir eu l'air fort en peine à mon début.

Un silence fort court suivit ma protestation, après quoi je vis le
premier président dire quelques mots assez bas à M. le duc d'Orléans,
puis faire tout haut la députation du parlement pour aller chercher le
testament du roi et son codicille, qui avait été mis au même lieu. Le
silence continua pendant cette grande et courte attente\,; chacun se
regardait sans se remuer. Nous étions tous aux sièges bas, les portes
étaient censées fermées, mais la grand'chambre était pleine de curieux
de qualité et de tous états, et de la suite nombreuse de ce qui était en
séance. M. le duc d'Orléans avait eu la facilité de se laisser leurrer,
en cas de besoin, du secours d'Angleterre, et pour cela de faire placer
milord Stairs dans une des lanternes. Ce fut l'ouvrage du duc de
Noailles, de Canillac, de l'abbé Dubois.

Il y en avait un autre plus présent. Le régiment des gardes occupait
sourdement toutes les avenues, et tous les officiers, avec des soldats
d'élite dispersés, l'intérieur du palais. Le duc de Guiche, démis à son
fils, était dans la lanterne basse de la cheminée. Il avait capitulé
avec M. le duc d'Orléans, et en avait tiré six cent mille livres pour ce
service qu'il avait eu le talent de lui faire valoir. Il s'était donné
pendant la vie du roi pour un homme attaché aux bâtards. Ils y avaient
compté, et comme on le voit, ne tardèrent pas à se mécompter. La
précaution ne fut utile qu'au duc de Guiche\,; tout se passa, il est
vrai, peu doucement, mais sans la plus légère apparence de donner la
moindre atteinte à la tranquillité parfaite.

La députation ne fut pas longtemps à revenir. Elle remit le testament et
le codicille entre les mains du premier président qui les présenta, sans
s'en dessaisir, à M. le duc d'Orléans, puis les fit passer de main en
main par les présidents à mortier à Dreux, conseiller au parlement, père
du grand maître des cérémonies, disant qu'il lisait bien, et d'une voix
forte qui serait bien entendue de tous, de la place où il était sur les
sièges hauts derrière les présidents près de la lanterne de la buvette.
On peut juger avec quel silence il fut écouté, et combien les yeux et
les oreilles se dressèrent vers ce lecteur. À travers toute sa joie, le
duc du Maine montra une âme en peine\,; il se trouvait au moment d'une
forte opération qu'il fallait soutenir. M. le duc d'Orléans ne marqua
qu'une application tranquille.

Je ne m'arrêterai point à ces deux pièces, où il n'est question que de
la grandeur et de la puissance des bâtards, de M\textsuperscript{me} de
Maintenon et de Saint-Cyr, du choix de l'éducation du roi, et du conseil
de régence au pis pour M. le duc d'Orléans, et de le livrer entièrement
dépouillé de tout pouvoir au pouvoir sans bornes du duc du Maine.

Je remarquai un morne et une sorte d'indignation qui se peignit sur tous
les visages, à mesure que la lecture avançait, et qui se tourna en une
sorte de fermentation muette à la lecture du codicille que fit l'abbé
Menguy, autre conseiller de la grand'chambre, mais clerc, et en la même
place de Dreux pour être mieux entendu. Le duc du Maine la sentit et en
pâlit, car il n'était appliqué qu'à jeter les yeux sur tous les visages,
et les miens le suivaient de près tout en écoutant, et regardant de fois
à autre la contenance de M. le duc d'Orléans.

La lecture achevée, ce prince prit la parole, et passant les yeux sur
toute la séance, se découvrit, se recouvrit, et dit un mot de louange et
de regret du feu roi. Élevant après la voix davantage, il déclara qu'il
n'avait qu'à approuver tout ce qui regardait l'éducation du roi, quant
aux personnes, et ce qui se trouvait sur un établissement aussi beau et
aussi utile que l'était celui de Saint-Cyr, dans les dispositions qu'on
venait d'entendre\,; qu'à l'égard de celles qui regardaient le
gouvernement de l'État, il parlerait séparément de ce qui en était
contenu dans le testament et dans le codicille\,; qu'il avait peine à
les concilier avec ce que le roi lui avait dit dans les derniers jours
de sa vie, et avec les assurances qu'il lui avait données publiquement
qu'il ne trouverait rien dans ses dispositions dont il pût n'être pas
content, en conséquence de quoi il avait lui-même toujours depuis
renvoyé à lui pour tous les ordres à donner, et ses ministres pour les
recevoir sur les affaires\,; qu'il fallait qu'il n'eût pas compris la
force de ce qu'on lui avait fait faire, regardant du côté du duc du
Maine, puisque le conseil de régence se trouvait choisi, et son autorité
tellement établie par le testament qu'il ne lui en demeurait plus aucune
à lui\,; que ce préjudice fait au droit de sa naissance, à son
attachement pour la personne du roi, à son amour et à {[}sa{]} fidélité
pour l'État, était de nature à ne pouvoir le souffrir avec la
conservation de son honneur\,; et qu'il espérait assez de l'estime de
tout ce qui était là présent pour se persuader que sa régence serait
déclarée telle qu'elle devait être, c'est-à-dire entière, indépendante,
et le choix du conseil de régence, à qui il ne disputait pas la voix
délibérative pour les affaires, à sa disposition, parce qu'il ne les
pouvait discuter qu'avec des personnes qui, étant approuvées du public,
pussent aussi avoir sa confiance. Ce court discours parut faire une
grande impression.

Le duc du Maine voulut parler. Comme il se découvrait, M. le duc
d'Orléans avança la tête par-devant M. le Duc, et dit au duc du Maine
d'un ton sec\,: «\, Monsieur, vous parlerez à votre tour.\,» En un
moment l'affaire tourna selon les désirs de M. le duc d'Orléans. Le
pouvoir du conseil de régence et sa composition tombèrent. Le choix du
conseil de régence fut attribué a M. le duc d'Orléans, régent du
royaume, avec toute l'autorité de la régence, et à la pluralité des voix
du conseil de régence la décision des affaires seulement, avec la voix
du régent comptée pour deux, en cas de partage. Ainsi toutes les grâces
et les punitions demeurèrent en la main seule de M. le duc d'Orléans.
L'acclamation fut telle que le duc du Maine n'osa dire une parole. Il se
réserva pour soutenir le codicille, dont la conservation, en effet, eût
annulé par soi-même tout ce que M. le duc d'Orléans venait d'obtenir.

Après quelques moments de silence, M. le duc d'Orléans reprit la parole.
Il témoigna une nouvelle surprise que les dispositions du testament
n'eussent pas suffi à qui les avait suggérées, et que, non contents de
s'y être établis les maîtres de l'État, ils en eussent eux-mêmes trouvé
les clauses si étranges qu'il avait fallu, pour se rassurer, devenir
encore les maîtres de la personne du roi, de la sienne à lui, de la cour
et de Paris. Il ajouta que si son honneur se trouvait blessé au point où
il lui paraissait que la compagnie l'avait senti elle-même par les
dispositions du testament, ainsi que toutes les lois et les règles, les
mêmes étaient encore plus violées par celles du codicille, qui ne
laissait ni sa liberté ni sa vie même en sûreté, et mettait la personne
du roi dans l'absolue dépendance de qui avait osé profiter de l'état de
faiblesse d'un roi mourant pour lui arracher ce qu'il n'avait pu
entendre. Il conclut par déclarer que la régence était impossible à
exercer avec de telles conditions, et qu'il ne doutait pas que la
sagesse de la compagnie n'annulât un codicille qui ne se pouvait
soutenir, et dont les règlements jetteraient la France dans les malheurs
les plus grands et les plus indispensables. Tandis que ce prince
parlait, un profond et morne silence lui applaudissait, sans
s'expliquer.

Le duc du Maine, devenu de toutes les couleurs, prit la parole, qui pour
cette fois lui fut laissée. Il dit que l'éducation du roi, et par
conséquent sa personne, lui étant confiée, c'était une suite toute
naturelle qu'il eût, privativement à tout autre, l'entière autorité sur
sa maison civile et militaire, sans quoi il ne pouvait se charger de le
faire servir, ni répondre de sa personne\,; et de là à vanter son
attachement, si connu du feu roi, qu'il y avait mis toute sa confiance.

M. le duc d'Orléans l'interrompit à ce mot, qu'il releva. M. du Maine
voulut le tempérer par les louanges du maréchal de Villeroy adjoint à
lui, mais sous lui dans la même charge et la même confiance. M. le duc
d'Orléans reprit qu'il serait étrange que la première et plus entière
confiance ne fût pas en lui, et plus encore qu'il ne pût vivre auprès du
roi que sous l'autorité et la protection de ceux qui se seraient rendus
les maîtres absolus du dedans et du dehors, et de Paris même par les
régiments des gardes.

La dispute s'échauffait, se morcelait par phrases coupées de l'un à
l'autre, lorsque en peine de la fin d'une altercation qui devenait
indécente, et cédant à l'ouverture que le duc de La Force venait de me
faire par-devant le duc de La Rochefoucauld qui siégeait entre nous
deux, je fis signe de la main à M. le duc d'Orléans de sortir et d'aller
achever cette discussion dans la quatrième des enquêtes, qui a une porte
de communication dans la grand'chambre, et où il n'y avait personne. Ce
qui me détermina à cette action fut que je m'aperçus que M. du Maine
s'affermissait, qu'il se murmurait confusément de partage, et que M. le
duc d'Orléans ne faisait pas le meilleur personnage, puisqu'il
descendait et plaider pour ainsi dire sa cause contre le duc du Maine.
Il avait la vue basse. Il était tout entier à attaquer et à répondre, en
sorte qu'il ne vit point le signe que je lui faisais. Quelques moments
après je redoublai, et n'en ayant pas plus de succès, je me levai et
m'avançai quelques pas, et lui dis, quoique d'assez loin\,: «\,Monsieur,
si vous passiez dans la quatrième des enquêtes, avec M. du Maine, vous y
parleriez plus commodément,\,» et m'avançant au même instant davantage,
je l'en pressai par un signe de la main et des yeux qu'il put
distinguer. Il m'en rendit un de la tête, et à peine fus-je rassis que
je le vis s'avancer par-devant M. le duc à M. du Maine\,; et aussitôt
après, tous deux se levèrent et s'en allèrent dans la quatrième des
enquêtes. Je ne pus voir qui, de ce qui était épars hors de la séance,
les y suivit, car toute la séance se leva à leur sortie, et se rassit en
même temps sans bouger, et tout en grand silence. Quelque temps après M.
le comte de Toulouse sortit de place, et alla dans cette chambre. M. le
duc l'y suivit un peu après. Au bout de quelque temps le duc de La Force
en fit autant.

Il y fut assez peu. Revenant en séance, il dépassa le duc de La
Rochefoucauld et moi, mit sa tête entre celle du duc de Sully et la
mienne, parce qu'il ne voulut pas être entendu par La Rochefoucauld, et
me dit\,: «\,Au nom de Dieu, allez-vous-en là dedans, cela va fort mal.
M. le duc d'Orléans mollit, rompez la dispute, faites rentrer M. le duc
d'Orléans\,; et dès qu'il sera en place, qu'il dise qu'il est trop tard
pour achever, qu'il faut laisser la compagnie aller dîner, et revenir
achever au sortir de table\,; et pendant cet intervalle, ajouta La
Force, mander les gens du roi au Palais-Royal, et faire parler aux pairs
dont on pourrait douter, et aux chefs de meute parmi les magistrats.\,»
L'avis me parut bon et important. Je sortis de séance et allai à la
quatrième des enquêtes. Je trouvai un grand cercle assez fourni de
spectateurs, M. le comte de Toulouse vers l'entrée en avant, mais collé
à ce cercle, M. le Duc vers le milieu en même situation, tous assez
éloignés de la cheminée, devant laquelle M. le duc d'Orléans et le duc
du Maine étaient seuls, disputant d'action à voix basse, avec l'air fort
allumé tous deux. Je considérai quelques moments ce spectacle, puis je
m'approchai de la cheminée, en homme qui voulait parler. «\,Qu'y a-t-il,
monsieur\,? me dit M. le duc d'Orléans d'un air vif d'impatience. --- Un
mot pressé, monsieur, lui dis-je, que j'ai à vous dire.\,» Il continuait
à parler au duc du Maine, moi presque en tiers\,; je redoublai, il me
tendit l'oreille. «\,Non pas cela, lui dis-je, et lui prenant la main\,:
venez-vous-en ici.\,» Je le tirai au coin de la cheminée. Le comte de
Toulouse qui était là auprès se recula beaucoup, et tout le cercle de ce
côté-là. Le duc du Maine se recula aussi d'où il était en arrière.

Je dis à l'oreille à M. le duc d'Orléans qu'il ne devait pas espérer de
rien gagner sur M. du Maine, qui ne sacrifierait pas le codicille à ses
raisons, que la longueur de cette conférence devenait indécente,
inutile, dangereuse\,; qu'il était là en spectacle à tout ce qui y était
entré comme en séance, et encore mieux vu et examiné\,; qu'il n'avait de
parti que de rentrer en séance, et dès qu'il y serait, la rompre, etc.
«\,Vous avez raison, me dit-il, je vais le faire. --- Mais, repris-je,
faites-le donc sur-le-champ, et ne vous laissez point amuser. C'est M.
de La Force à qui vous devez cet avis, et qui m'envoie vous le
donner.\,» Il me quitta sans plus rien dire, alla à M. du Maine, lui dit
en deux mots qu'il était trop tard, et qu'on finirait l'après-dînée.

J'étais demeuré où il m'avait laissé. Je vis aussitôt le duc du Maine
lui faire la révérence, comme se séparant tous deux, et se retirer, et
dans le même moment M. le Duc venir joindre M. le duc d'Orléans, et se
parler, tandis que M. du Maine joignit le cercle, et s'arrêta le dos
dedans pour voir apparemment ce colloque. Il dura assez peu, et fut fort
en douceur, quoique M. le Duc en air d'empressement. Comme il fallait
passer à peu près où j'étais pour rentrer dans la grand'chambre, tous
deux vinrent vers moi.

En ce moment je sus que M. le Duc venait de demander à M. le duc
d'Orléans d'entrer au conseil de régence, puisqu'on n'avait point égard
au testament, et d'en être déclaré chef, et qu'il l'avait obtenu. La
haine des bâtards, et par le rang de prince du sang, etc., et par le
procès de la succession de M. le Prince, avait engagé
M\textsuperscript{me} la Duchesse à faire des pas auprès de M. le duc
d'Orléans dans les dernières semaines de la vie du roi, et M. le duc
d'Orléans à les bien recevoir, pour se fortifier contre M. du Maine. Il
n'avait, je pense, osé me dire qu'il s'était engagé à cette place de
chef du conseil de régence, mais je crois que l'engagement en était
pris, et que M. le Duc l'en somma plutôt qu'il ne lui demanda. Bref, M.
le duc d'Orléans me dit qu'il en allait parler au parlement avant de
lever la séance\,; j'en fis un air de félicitation et d'approbation à M.
le Duc, et nous rentrâmes aussitôt en séance.

Le bruit qui accompagne toujours ces rentrées étant apaisé, M. le duc
d'Orléans dit qu'il était trop tard pour abuser plus longtemps de la
compagnie, qu'il fallait aller dîner, et rentrer au sortir de table pour
achever. Tout de suite il ajouta qu'il croyait convenable que M. le Duc
entrât dès lors au conseil de régence et que ce fût avec la qualité de
chef de ce conseil\,; et que, puisque la compagnie avait rendu à cet
égard la justice qui était duc à sa naissance et à la qualité de régent,
il lui expliquerait ce qu'il pensait sur la forme à donner au
gouvernement, et qu'en attendant il profitait du pouvoir de sa régence
pour profiter des lumières et de la sagesse de la compagnie, et lui
rendait dès maintenant l'ancienne liberté des remontrances. Ces paroles
furent suivies d'un applaudissement éclatant et général, et la séance
fut aussitôt levée.

J'étais prié à dîner ce jour-là chez le cardinal de Noailles, mais je
sentis l'importance d'employer le temps si court et si précieux de
l'intervalle jusqu'à la rentrée de l'après-dînée, et de ne pas quitter
M. le duc d'Orléans, dont le duc de La Force me pressa dès que je fus
rentré en séance. Je m'approchai de M. le duc d'Orléans dans la fin du
parquet, et lui dis a l'oreille\,: «\,Les moments sont chers, je vous
suis au Palais-Royal\,;» et me remis après où je devais être pour sortir
avec les pairs. Montant en carrosse, j'envoyai un gentilhomme m'excuser
au cardinal de Noailles, et lui dire que je lui en dirais la raison. Je
m'en allai au Palais-Royal, où la curiosité avait rassemblé tout ce qui
n'était pas au palais, et où vint encore une partie de ce qui y avait
été spectateur. Tout ce que j'y trouvai de ma connaissance me demanda
des nouvelles avec empressement. Je me contentai de répondre que tout
allait bien, et dans la règle, mais que tout n'était pas encore fini.

M. le duc d'Orléans était passé dans un cabinet où je le trouvai seul
avec Canillac qui l'avait attendu. Nous primes là nos mesures, et M. le
duc d'Orléans envoya chercher le procureur général d'Aguesseau, depuis
chancelier, et le premier avocat général Joly de Fleury, depuis
procureur général. Il était près de deux heures. On servit une petite
table de quatre couverts où Canillac, Conflans, premier gentilhomme de
la chambre de M. le duc d'Orléans, et moi nous mîmes avec ce prince, et
pour le dire en passant, je n'ai jamais mangé avec lui depuis qu'une
fois, chez M\textsuperscript{me} la duchesse d'Orléans à Bagnolet.

Le maréchal de Villeroy était demeuré à Versailles. Il avait chargé
Goesbriant, gendre de Desmarets, de venir au palais, et de lui mander
souvent des nouvelles. Il en reçut trois courriers fort près à près qui
le remplirent tellement de joie et d'espérance, lui et la duchesse de
Ventadour, son ancienne bonne amie, qu'ils ne doutèrent pas que ce qui
se passait sur le codicille ne le soutînt, et ne rétablit le testament,
de sorte qu'ils ne purent se contenir, et répandirent la victoire
complète du duc du Maine sur M. le duc d'Orléans dans Versailles. Paris
fut aussi dans la même erreur, répandue par les émissaires du duc du
Maine de tous côtés, mais le triomphe ne fut pas de longue durée.

Nous retournâmes au parlement un peu avant quatre heures. J'y allai seul
dans mon carrosse un moment avant M. le duc d'Orléans, et j'y trouvai
tout en séance. J'y fus regardé avec grande curiosité, à ce qu'il me
parut\,; je ne sais si on était instruit d'où je venais. J'eus soin que
mon maintien ne montrât rien. Je dis seulement en passant au duc de La
Force que son conseil avait été salutaire, que j'avais lieu d'en espérer
tout succès, et que j'avais dit à M. le duc d'Orléans que c'était lui
qui l'avait pensé et me l'avait dit. M. le duc d'Orléans arrivé, et le
bruit inséparable d'une nombreuse suite apaisée, il dit qu'il fallait
reprendre les choses où elles en étaient demeurées le matin\,; qu'il
devait dire à la cour qu'il n'était demeuré d'accord de rien avec M. du
Maine, en même temps lui remettre devant les yeux les clauses
monstrueuses d'un codicille arraché à un prince mourant, clauses bien
plus étranges encore que les dispositions du testament que la cour
n'avait pas jugé devoir être exécutées, et que la cour ne pouvait passer
à M. du Maine d'être maître de la personne du roi, de la cour, de Paris,
par conséquent de l'État, de la personne, de la liberté, de la vie du
régent, qu'il serait en état de faire arrêter à toute heure, dès qu'il
serait le maître absolu et indépendant de la maison du roi civile et
militaire\,; que la cour voyait ce qui devait nécessairement résulter
d'une nouveauté inouïe qui mettait tout entre les mains de M. du Maine,
et qu'il laissait aux lumières, à la prudence de la compagnie, à sa
sagesse, son équité, à son amour pour l'État, à déclarer ce qu'elle en
pensait.

M. du Maine parut alors aussi méprisable sur le pré, qu'il était
redoutable dans l'obscurité des cabinets. Il avait l'air d'un condamné,
et lui toujours si vermeil, avec la pâleur de la mort sur le visage. Il
répondit à voix fort basse et peu intelligible, et avec un air aussi
respectueux et aussi humble qu'il l'avait été audacieux le matin.

On opinait cependant sans l'écouter, et il passa tout d'une voix comme
en tumulte à l'entière abrogation du codicille. Cela fut prématuré comme
l'abrogation du testament l'avait été le matin, l'un et l'autre par une
indignation soudaine. Les gens du roi devaient parler, et ils étaient
là, avant que personne opinât\,; aussi le premier président n'avait
point demandé les voix\,: elles avaient prévenu l'ordre. D'Aguesseau,
quoique procureur général, et Fleury, premier avocat général, parlèrent
donc\,: le premier en peu de mots\,; l'autre avec plus d'étendue, et fit
un fort beau discours. Comme il existe dans les bibliothèques, je ne
parlerai que des conclusions conformes de tous deux, en tout et partout
favorables à M. le duc d'Orléans.

Après qu'ils eurent parlé, le duc du Maine, se voyant totalement tondu,
essaya une dernière ressource. Il représenta avec plus de force qu'on
n'en attendait de ce qu'il avait montré en cette seconde séance, mais
pourtant avec mesure, que s'il était dépouillé de l'autorité qui lui
était donnée par le codicille, il demandait a être déchargé de la garde
du roi, de répondre de sa personne, et de conserver seulement la
surintendance de son éducation. M. le duc d'Orléans répondit\,: «\,Très
volontiers, monsieur, il n'en faut pas aussi davantage.\,» Là-dessus, le
premier président, aussi abattu que le duc du Maine, prit les voix.

Chacun répondit de l'avis des conclusions, et l'arrêt fut prononcé en
sorte qu'il ne resta nulle sorte de pouvoir au duc du Maine, qui fut
totalement remis entre les mains du régent, avec le droit de mettre dans
la régence qui il voudrait, d'en ôter qui bon lui semblerait, et de
faire tout ce qu'il jugerait à propos sur la forme à donner au
gouvernement, l'autorité toutefois des affaires demeurant au conseil de
régence, à la pluralité des voix, celle du régent comptée pour deux en
cas seulement de partage, et M. le Duc déclaré chef sous lui du conseil
de régence, avec, dès à présent, la faculté d'y entrer et d'y opiner.

Pendant les opinions, le prononcé et le reste de la séance, le duc du
Maine eut toujours les yeux baissés, l'air plus mort que vif, et parut
immobile. Son fils et son frère ne donnèrent aucun signe de prendre part
à rien.

L'arrêt fut suivi de fortes acclamations de la foule qui était éparse
hors de la séance\,; et celle qui remplissait le reste du palais y
répondit à mesure qu'elle fut instruite de ce qui avait été décidé.

Ce bruit un peu long apaisé, le régent fit un remerciement court, poli,
majestueux à la compagnie, protesta du soin qu'il aurait d'employer au
bien de l'État l'autorité de laquelle il était revêtu, puis dit à la
compagnie qu'il était temps de l'informer de ce qu'il jugeait nécessaire
d'établir pour lui aider dans l'administration de l'État. Il ajouta
qu'il le faisait avec d'autant plus de confiance, que ce qu'il se
proposait n'était que l'exécution de ce que M. le duc de Bourgogne, car
il le nomma ainsi, avait résolu, et qu'on avait trouvé parmi les papiers
de sa cassette. Il fit un court et bel éloge des lumières et des
intentions de ce prince, puis déclara qu'outre le conseil de régence qui
serait le suprême où toutes les affaires du gouvernement ressortiraient,
il se proposait d'en établir un pour les affaires étrangères, un pour
les affaires de la guerre, un pour celles de la marine, un pour celles
des finances, un pour les affaires ecclésiastiques, et un pour celles du
dedans du royaume, et de choisir quelques-uns des magistrats de la
compagnie pour entrer dans ces deux derniers conseils, et les aider de
leurs lumières sur la police du royaume, la jurisprudence, et ce qui
regardait les libertés de l'Église gallicane.

L'applaudissement des magistrats éclata, et toute la foule y répondit.
Le premier président conclut la séance par un compliment fort court au
régent, qui se leva, et en même temps toute la séance, et on s'en alla.

Il faut ici se souvenir de la très singulière rencontre en même pensée
sur ces conseils entre le duc de Chevreuse et moi, conseils destinés et
adoptés par M. le duc de Bourgogne, et donnés en cette seconde séance
par le régent pour avoir été trouvés dans ses papiers. On ne peut
exprimer l'impression que fit ce nom auguste, ni à quel point la mémoire
de ce prince parut chère, et sa personne regrettée et respectée avec la
plus sincère vénération.

Il alla droit du palais à Versailles, parce qu'il était fort tard, et
qu'il voulait voir le roi avant qu'il se couchât, comme pour lui rendre
compte de ce qui s'était passé. Il y reçut les compliments forcés des
deux vieux amants, et de là s'en alla chez Madame. Elle fut au-devant de
lui l'embrasser, ravie de joie, et après les premières questions et
conjouissances, elle lui dit qu'elle ne désirait rien autre chose que le
bonheur de l'État par un bon et sage gouvernement, et sa gloire à lui\,;
qu'elle ne lui demanderait jamais rien qu'une seule chose qui n'était
que pour son bien et son honneur, mais qu'elle lui en demandait sa
parole précise\,: c'était de n'employer jamais en rien du tout, pour peu
que ce fût, l'abbé Dubois, qui était le plus grand coquin et le plus
insigne fripon qu'il y eût au monde, ce dont elle avait mille et mille
preuves, qui, pour peu qu'il pût se fourrer, voudrait aller à tout, et
le vendrait lui et l'État pour son plus léger intérêt. Elle en dit bien
d'autres sur son compte, et pressa tant M. son fils qu'elle en tira
parole positive de ne l'employer jamais.

J'arrivai une heure après à Versailles. J'allai chez
M\textsuperscript{me} la duchesse de Berry, qui était ravie. M. le duc
d'Orléans en sortait. Je vis après M\textsuperscript{me} la duchesse
d'Orléans qui me parut tâcher d'être bien aise. J'évitai les détails
avec elle sous prétexte de m'aller reposer. Ce n'était pas sans besoin.
J'appris le lendemain la parole exigée et donnée de l'exclusion totale
de l'abbé Dubois. On ne verra que trop tôt que les paroles de M. le duc
d'Orléans ne furent jamais que des paroles, c'est-à-dire des sons qui
frappent l'air.

\hypertarget{chapitre-vii.}{%
\chapter{CHAPITRE VII.}\label{chapitre-vii.}}

1715

~

{\textsc{Conseils à l'ordinaire.}} {\textsc{- Les entrailles du roi
portées à Notre-Dame tout simplement.}} {\textsc{- Harangues des
compagnies au roi.}} {\textsc{- Force réformes civiles.}} {\textsc{- Le
coeur du roi fort simplement porté aux Grands-Jésuites.}} {\textsc{-
Merveilleuse et prompte ingratitude.}} {\textsc{- Le régent visite à
Saint-Cyr M\textsuperscript{me} de Maintenon, et lui continue sa
pension.}} {\textsc{- Madame l'y visite aussi le même jour.}} {\textsc{-
Le parlement continué pour un mois.}} {\textsc{- Le roi va à
Vincennes.}} {\textsc{- Le corps du roi porté à Saint-Denis.}}
{\textsc{- Entreprise de M. le Duc, qui fait monter avec lui dans le
carrosse du roi le chevalier de Dampierre, son écuyer.}} {\textsc{- Le
régent permet à tous les carrosses d'entrer dans la dernière cour du
Palais-Royal, et à qui voulut de draper, jusqu'au premier président du
parlement.}} {\textsc{- Nouveauté pour les magistrats de draper des plus
grands deuils de famille et de porter des pleureuses.}} {\textsc{-
Prisons ouvertes\,; horreurs.}} {\textsc{- Duc du Maine et comte de
Toulouse admis au conseil avec les seuls ministres du feu roi.}}
{\textsc{- Mort de M\textsuperscript{me} de La Vieuville.}} {\textsc{-
M\textsuperscript{me} la duchesse de Berry, à Saint-Cloud, fait
M\textsuperscript{me} de Pons sa dame d'atours, et la remplace de
M\textsuperscript{me} de Beauvau.}} {\textsc{- Duc d'Albret est grand
chambellan sur la démission du duc de Bouillon, son père.}} {\textsc{-
Le roi tient son premier lit de justice.}} {\textsc{- Le roi harangué
par les compagnies à Vincennes.}} {\textsc{- Le chancelier se démet,
pour quatre cent mille livres, de sa charge de secrétaire d'État.}}
{\textsc{- Crosat\,; quel\,; fait grand trésorier de l'ordre pour des
avances.}} {\textsc{- Térat\,; quel\,; en a le râpé.}} {\textsc{-
Conseils, d'où pris, comment pervertis.}} {\textsc{- Je fais déclarer le
cardinal et le duc de Noailles chef du conseil de conscience et
président de celui des finances.}} {\textsc{- Réflexion sur le pouvoir
et le grand nombre en matière de religion.}} {\textsc{- Conseil de
conscience.}} {\textsc{- Caractère de Besons, archevêque de Bordeaux,
puis de Rouen, de Pucelle et de Joly de Fleury.}} {\textsc{- Dorsanne\,;
son caractère et sa fin.}} {\textsc{- Conseil des finances.}} {\textsc{-
Le chancelier de Pontchartrain raffermit secrètement son fils.}}
{\textsc{- Conseil des affaires étrangères.}} {\textsc{- Conseil de
guerre.}} {\textsc{- Caractère du duc de Guiche.}} {\textsc{- Les
fortifications données à Asfeld.}} {\textsc{- Caractère de Saint-Contest
et de Le Blanc.}} {\textsc{- Conseil de marine.}} {\textsc{- Conseil des
affaires du dedans du royaume.}} {\textsc{- Caractère de Beringhen,
premier écuyer, et du marquis de Brancas.}}

~

Le lendemain, mardi 3 septembre, le régent tint à Versailles deux
conseils\,: un le matin, l'autre l'après-dînée, où il n'y eut que les
ministres du feu roi, c'est-à-dire le maréchal de Villeroy, Voysin,
chancelier de France et secrétaire d'État de la guerre, Torcy,
secrétaire d'État des affaires étrangères, qui avait les postes, et
Desmarets, contrôleur général des finances. Ils étaient tous nommés par
le testament du roi, avec ses deux bâtards, et les maréchaux de Villars,
d'Harcourt, de Tallard et d'Huxelles, pour composer le conseil de
régence avec M. le duc d'Orléans, et avec M. le Duc dans un an, à
vingt-quatre ans. Mais par ce qui avait été décidé la veille au
parlement, le régent était pleinement le maître de le composer tout
comme il lui plairait, et tous ces messieurs fort en peine. Il eut
encore conseil le lendemain avec les mêmes ministres du feu roi
seulement\,; et les entrailles du roi furent portées sans aucune
cérémonie à Notre-Dame, par deux aumôniers du roi, dans un de ses
carrosses, sans personne d'accompagnement. Elles le devaient être à
Saint-Denis, mais cela fut changé sur la représentation que fut le
cardinal de Noailles que les entrailles des derniers rois étaient toutes
à Notre-Dame.

Le jeudi, 5 septembre, le parlement et les autres compagnies
haranguèrent le roi. Ce même jour il parut de grandes réformes dans la
maison du roi et les bâtiments\,; et ses équipages de chasse furent
réduits sur le pied qu'ils avaient été sous Louis XIII.

Le vendredi, 6 septembre, le cardinal de Rohan porta le cœur aux
Grands-Jésuites avec très peu d'accompagnement et de pompe. Outre le
service purement nécessaire, on remarqua qu'il ne se trouva pas six
personnes de la cour aux Jésuites à cette cérémonie. Ce n'est pas à moi,
qui après mon père n'ai de ma vie manqué d'assister tous les ans à
l'anniversaire de Louis XIII à Saint-Denis, et qui y ai déjà été
cinquante-deux fois sans y avoir jamais vu personne, à relever une si
prompte ingratitude.

Ce même jour le régent fit une action du mérite le plus exquis, si la
vue de Dieu l'eût conduit, mais de la dernière misère parce que la
religion n'y eut aucune part, et qu'alors il se devait garder plus de
respect à soi-même, et n'afficher pas au moins si subitement avec quelle
sécurité il était permis de le persécuter de la manière la plus
opiniâtre et la plus cruelle. Il alla à huit heures du matin voir
M\textsuperscript{me} de Maintenon à Saint-Cyr. Il fut près d'une heure
avec cette ennemie qui lui avait voulu faire perdre la tête, et qui tout
récemment l'avait voulu livrer pieds et poings liés au duc du Maine, par
les monstrueuses dispositions du testament et du codicille du roi.

Le régent l'assura dans cette visite que les quatre mille livres que le
roi lui donnait tous les mois lui seraient continuées, et lui seraient
portées tous les premiers jours de chaque mois par le duc de Noailles,
qui avait apparemment engagé ce prince à cette visite et à ce présent.
Il dit à M\textsuperscript{me} de Maintenon que si elle en voulait
davantage, elle n'avait qu'à parler, et l'assura de toute sa protection
pour Saint-Cyr, où il vit les classes des demoiselles toutes ensemble en
sortant.

Il faut savoir qu'outre la terre de Maintenon et les autres biens de
cette fameuse et trop funeste fée, la maison de Saint-Cyr, qui avait
plus de quatre cent mille livres de rente et beaucoup d'argent en
réserve, était obligée par son établissement à y recevoir
M\textsuperscript{me} de Maintenon, si elle venait à vouloir s'y
retirer\,; à lui obéir en tout comme à la supérieure unique et absolue
en tout, à l'entretenir elle et tout ce qu'elle y aurait auprès d'elle,
ses domestiques, ses équipages dedans et au de hors, de toutes choses,
sans exception, à son gré, sa table et les autres nourritures aussi à
son gré, aux dépens de la maison, ce qui a été très ponctuellement
exécuté jusqu'à sa mort. Ainsi elle n'avait pas besoin de cette belle
libéralité d'une continuation de pension de quarante-huit mille livres.
C'était bien assez que M. le duc d'Orléans daignât oublier qu'elle fût
au monde, et ne pas troubler son repos à Saint-Cyr. Madame la fut voir
aussi le même matin sur les onze heures. Pour elle, on a vu qu'elle lui
dut tout à la mort de Monsieur, et Madame lui devait au moins cette
marque de reconnaissance.

Le régent se garda bien de me parler de sa visite, ni devant ni après,
et je ne pris pas non plus la peine de la lui reprocher et de lui en
faire honte. Elle fit grand bruit dans le monde et n'en fut pas
approuvée. L'affaire d'Espagne n'était pas encore oubliée, et le
testament et le codicille fournissaient alors à toutes les
conversations.

Le samedi 7 septembre était le jour pris pour le premier lit de justice
du roi, mais il se trouva enrhumé la nuit, qu'il ne passa pas trop bien.
Le régent vint seul à Paris. Le parlement était assemblé, et j'allai
jusqu'à une porte du palais, où je fus averti du contre-ordre qui ne
venait que d'arriver, et qui ne put nous trouver chez nous. Le premier
président et les gens du roi furent aussitôt mandés au Palais-Royal\,;
et le parlement, qui allait entrer en vacance, fut continué pour huit
jours à l'égard des procès, et pour tout le reste du mois quant aux
affaires générales. Le lendemain, le régent qui était importuné du
séjour de Versailles, parce qu'il aimait à demeurer à Paris où il avait
tous ses plaisirs sous sa main, et trouvant de l'opposition dans les
médecins de la cour, tous commodément logés à Versailles, au transport
de la personne du roi à Vincennes sous prétexte d'un petit rhume, fit
venir tous ceux de Paris qui avaient été mandés à voir le feu roi.
Ceux-là qui n'avaient rien à gagner au séjour de Versailles se moquèrent
des médecins de la cour, et sur leur avis il fut résolu qu'on mènerait,
le lendemain lundi 9 septembre, le roi à Vincennes, ou tout était prêt à
le recevoir.

Il partit donc ce jour-là sur les deux heures après midi de Versailles,
entre le régent et la duchesse de Ventadour au fond, le duc du Maine et
le maréchal de Villeroy au devant, et le comte de Toulouse à une
portière, qui l'aima mieux que le devant. Il passa sur les remparts de
Paris sans entrer dans la ville, et arriva sur les cinq heures à
Vincennes, ayant trouvé beaucoup de monde et de carrosses sur le chemin
pour le voir passer.

Le même jour, le corps du feu roi fut porté à Saint-Denis. On a déjà dit
qu'il n'avait rien réglé ni défendu pour ses obsèques, et qu'on se
conforma au dernier exemple pour éviter la dépense, l'embarras, la
longueur des cérémonies\,: Louis XIII, par modestie et par humilité,
avait lui-même ordonné des siennes au moindre état qu'il avait pu. Ces
vertus, ainsi que tant d'autres héroïques ou chrétiennes, il ne les
avait pas transmises à son fils. Mais on se servit de l'autorité du
dernier exemple, et personne ne le releva ni le trouva mauvais, tant il
est vrai que l'attachement et la reconnaissance sont des vertus qui se
sont envolées au ciel avec Astrée, comme il y avait paru aux
Grands-Jésuites depuis si peu de jours, lorsque le cœur du roi y fut
porté, ce cœur qui n'aima personne et qui fut aussi si peu aimé. M. le
Duc, au lieu de M. le duc d'Orléans, qui n'était pas payé pour en
prendre la fatigue, mena le convoi. Il fit monter dans le carrosse du
roi où il était le chevalier de Dampierre, son écuyer, ce qui surprit
étrangement.

Je ne m'arrêterai pas ici à cette entreprise qui ne fut que de légères
prémices de toutes celles qui se succédèrent bientôt les unes aux
autres. Dampierre était Cugnac, et pouvait entrer dans les carrosses par
sa naissance, mais on a vu ailleurs combien les principaux domestiques
des princes du sang en étaient exclus par cette qualité, de quelque
naissance qu'ils pussent être, à la différence de ceux des fils et
petits-fils de France\,; combien le feu roi était jaloux et attentif
là-dessus, et divers exemples. Cette hardiesse fit grand bruit, et ce
fut tout. M. le duc d'Orléans n'était pas fait pour les règles ni pour
les bienséances, mais pour laisser usurper chacun contre les unes et les
autres, sans droit, et contre tout exemple constant.

Ainsi il permit l'entrée de la seconde cour du Palais-Royal à toutes
sortes de carrosses, jusqu'alors réservée comme la seconde cour de
Versailles, et il souffrit que drapât du roi qui voulut, jusqu'au
premier président de Mesmes. Jusqu'alors cette distinction n'avait point
passé au delà des officiers de la couronne et des grands officiers des
maisons du roi, de la reine et des fils de France. Il n'y avait pas même
plus de cinquante ans que les magistrats, quels qu'ils fussent, avaient
commencé à draper de leurs pères, mères et femmes, et rien n'avait paru
plus nouveau ni plus ridicule au deuil de Monseigneur que quelques
magistrats du conseil, en fort petit nombre, qui hasardèrent de paraître
en pleureuses, et qui ne furent point imités par les autres. Le régent
crut apparemment se dévouer le parlement et le premier président, en
flattant son orgueil extrême\,; il ne fit que faire mépriser son extrême
facilité. On en verra bien d'autres et en tousgenres dans la
suite\footnote{{[}12{]}}.

Le lendemain de l'arrivée du roi à Vincennes, le régent travailla tout
le matin séparément avec les secrétaires d'État qu'il avait chargés de
lui apporter la liste de toutes les lettres de cachet de leurs bureaux,
et leurs causes, qui sur ces dernières se trouvèrent souvent courts. La
plupart des lettres de cachet, d'exil et de prison avaient été expédiées
pour jansénisme et pour la constitution, quantité dont les raisons
étaient connues du feu roi seul et de ceux qui les lui avaient fait
donner, d'autres du temps des précédents ministres, parmi lesquelles
beaucoup étaient ignorées et oubliées depuis longtemps. Le régent leur
rendit à tous pleine liberté, exilés et prisonniers, excepté ceux qu'il
connut être arrêtés pour crime effectif et affaires d'État, et se fit
donner des bénédictions infinies pour cet acte de justice et d'humanité.

Il se débita là-dessus des histoires très singulières, et d'autres fort
étranges, ce qui fit déplorer le malheur des prisonniers, et la tyrannie
du dernier règne et de ses ministres. Parmi ceux de la Bastille il s'en
trouva un arrêté depuis trente-cinq {[}ans{]}, le jour qu'il arriva à
Paris d'Italie d'où il était, et qui venait voyager. On n'a jamais su
pourquoi, et sans qu'il eût jamais été interrogé, ainsi que la plupart
des autres. On se persuada que c'était une méprise. Quand on lui annonça
sa liberté, il demanda tristement ce qu'on prétendait qu'il en pût
faire. Il dit qu'il n'avait pas un sou, qu'il ne connaissait qui que ce
fût à Paris, pas même une seule rue, personne en France, que ses parents
d'Italie étaient apparemment morts depuis qu'il en était parti, que ses
biens apparemment aussi avaient été partagés depuis tant d'années qu'on
n'avait point eu de nouvelles de lui\,; qu'il ne savait que devenir. Il
demanda de rester à la Bastille le reste de ses jours avec la nourriture
et le logement. Cela lui fut accordé avec la liberté qu'il y voudrait
prendre.

Pour ceux qui furent tirés des cachots où la haine des ministres et
celle des jésuites et des chefs de la constitution les avait fait jeter,
l'horreur de l'état où ils parurent épouvanta et rendit croyables toutes
les cruautés qu'ils racontèrent dès qu'ils furent en pleine liberté. Le
même jour le régent tint conseil avec les ministres du feu roi, et il y
fit entrer le duc du Maine et le comte de Toulouse.

Ce même jour mourut M\textsuperscript{me} de La Vieuville dans un âge
peu avancé, d'un cancer au sein, dont jusqu'à deux jours avant sa mort
elle avait gardé le secret avec un courage égal à la folie de s'en
cacher, et de se priver par là des secours. Une seule femme de chambre
le savait et la pansait. On a suffisamment parlé d'elle et de son mari,
lorsqu'elle fut faite dame d'atours de M\textsuperscript{me} la duchesse
de Berry. Cette princesse était à Saint-Cloud avec sa petite cour, en
attendant que le Luxembourg fût en état qu'elle y vint loger. Elle
disposa de la charge de sa dame d'atours en faveur de
M\textsuperscript{me} de Pons qui était une de ses dames, qu'elle
remplaça de M\textsuperscript{me} de Beauvau, dont le mari fut chevalier
de l'ordre en 1724, et son frère aussi qui était archevêque de Narbonne.
Cette dame était aussi Beauvau, d'une autre branche\,; son père avait
été capitaine des gardes autrefois de Monsieur. On donna au duc et à
M\textsuperscript{me} la duchesse du Maine un magnifique appartement en
bas, aux Tuileries\,; et M. de Bouillon obtint pour le duc d'Albret, son
fils, la charge de grand chambellan sur sa démission, en ayant vainement
tenté la survivance.

Le jeudi la septembre le roi vint tenir son premier lit de justice, où
il n'y eut point de foi et hommage et rien de particulier, sinon que la
duchesse de Ventadour y eut un petit siège, et que le maréchal de
Villeroy en eut un aussi fort bas, hors de rang, entre le trône et la
première place des pairs ecclésiastiques. Ce fut une tolérance, car il
ne pouvait être en fonctions tant que le roi était entre les mains des
femmes. Le premier chambellan, comme grand écuyer, le porta depuis le
carrosse jusqu'à la porte de la grand'chambre, où le duc de Tresmes le
prit et le porta sur son trône. Il servit de grand chambellan, et en eut
la place comme premier gentilhomme de la chambre en année, parce que le
duc d'Altret, qui ne l'était que de la veille, n'avait pas prêté
serment.

Le samedi 14 septembre les compagnies allèrent haranguer le roi à
Vincennes, et le chancelier donna la démission de sa charge de
secrétaire d'État de la guerre, suivant l'engagement qu'on a vu qu'il en
avait pris avec M. le duc d'Orléans pour se conserver les sceaux. On en
a assez dit sur cette belle convention pour n'avoir rien à y ajouter. Il
en eut encore quatre cent mille livres, outre tout ce qu'il en avait
tiré du feu roi.

Peu de jours après, la facilité du régent, et l'extrême et pressant
besoin des finances fit accorder à Crosat l'agrément de la charge de
trésorier de l'ordre, à rembourser aux héritiers de l'avocat général
Chauvelin. Le régent y trouva le prêt d'un million au roi en barres
d'argent, et l'engagement pour deux autres millions que fit Crosat.
Térat eut le râpé de cette charge Il était depuis longtemps chancelier
et surintendant des affaires de Monsieur, et de M. le duc d'Orléans
ensuite, exact, appliqué, désintéressé, vertueux et fort honorable, qui
faisait sa charge avec dignité, au profit de son maître, et à la
satisfaction de tout ce qui avait affaire à lui\,: \emph{rara avis}
certes au Palais-Royal. Le mérite fit passer ce râpé au public\,; mais
pour Crosat, ce fut un cri général.

Crosat était de Languedoc, où il s'était fourré chez Penautier en fort
bas étage\,; on a dit même qu'il avait été son laquais. Il fut petit
commis et parvint par degrés à devenir son caissier. On a vu quel était
Penautier. Enrichi dans ce poste, il nagea en plus grande eau\,; mais il
ne voulut point tâter de la finance ordinaire. Il donna dans la banque,
dans les armateurs, et devint le plus riche homme de Paris. Le roi
voulut qu'il fût intendant du duc de Vendôme, quand il ôta le maniement
de ses affaires délabrées des mains et du pillage du grand prieur et de
l'abbé de Chaulieu, à qui il les avait confiées depuis longtemps\,;
enfin Crosat fut trésorier ou receveur du clergé, qui est un emploi fort
lucratif. On peut juger qu'il était énormément riche et glorieux à
proportion, par le mariage qu'il fit de sa fille avec le comte d'Évreux,
qui devint le repentir et la douleur de tout le reste de sa vie\,; mais
il eut aussi de quoi se consoler par le mérite de ses trois fils, qui a
fait oublier tout le reste en leurs personnes.

La Bazinière, trésorier de l'épargne, qui ne valait pas mieux que
Crosat, avait eu sous le feu roi la charge de prévôt et grand maître des
cérémonies de l'ordre, qui est à preuves\footnote{{[}13{]}}, et par là,
grâce bien plus étrange, et le roi avait fait, surtout en 1688, bien des
chevaliers de l'ordre plus étranges encore en leur genre, dont on avait
crié, mais jamais au point qu'on fit sur le cordon bleu de Crosat. Rien
de si court en robe que les Chauvelin, qui étaient des va-nu-pieds\,;
sans magistrature, quand la fortune du chancelier Le Tellier les
débourba, parce que lui et le père de Chauvelin, conseiller d'État,
avaient épousé les deux soeurs, lorsque Le Tellier était encore petit
compagnon au Châtelet, et Chauvelin, conseiller d'État, était père de
l'avocat général par la mort duquel la charge de trésorier de l'ordre
vaquait. Or, dans la robe, ces charges n'étaient jamais tombées qu'aux
premiers présidents du parlement, très rarement à des présidents à
mortier. On fut surpris, lorsque le roi permit à Pontchartrain de vendre
la sienne de prévôt et grand maître des cérémonies de l'ordre à Le
Camus, premier président de la cour des aides. Un avocat général en
cordon bleu, cela parut un monstre qui révolta le parlement même\,; mais
cet avocat général, qui n'avait pas moins d'ambition qu'en a montré
depuis le garde des sceaux, son frère cadet, avec bien plus de talents
que lui, était le mignon des jésuites, le favori de la constitution, par
conséquent du roi avec qui il avait secrètement des rapports continuels,
et entrait fort souvent chez lui par les derrières.

Crosat était loin de tout cela, et on se donnait plus de liberté avec M.
le duc d'Orléans qu'avec Louis XIV. Ces charges étaient pour les
ministres, et leur indignation de voir Crosat paré comme eux passa au
public, qui fit leur écho sans y avoir intérêt, lequel a vu depuis avec
beaucoup plus de silence et de tranquillité les énormes choix de la
promotion de 1724, et de beaucoup encore depuis. Ainsi est fait le
public et le monde.

J'ai passé légèrement sur les cérémonies depuis la mort du roi jusqu'à
présent, parce que le retranchement ôta l'occasion des grandes disputes,
et que tout s'y passa sans rien de particulier, et je me suis arrêté au
reste, le moins qu'il a été possible, comme peu important. Il faut
maintenant venir aux conseils pris sur le plan que j'en avais donné
autrefois au duc de Chevreuse, si singulièrement conforme à son idée,
sans nous en être jamais parlé auparavant. Il avait passé entre les
mains de Mgr le duc de Bourgogne par celles du duc de Beauvilliers, et
avait été agréé de ce prince comme la meilleure forme de gouvernement,
dont il avait résolu de se servir quand Dieu l'y aurait appelé. Mais il
s'en fallut bien que ce premier plan fût suivi par M. le duc d'Orléans.
Il n'en prit que la plus faible écorce. J'expliquerai comment ce malheur
arriva, sous lequel la France gémit encore et gémira longtemps, parce
que, pour les États ainsi que pour les corps humains, il n'y a rien de
plus pernicieux que les meilleurs remèdes tournés en poisons.

M. le duc d'Orléans qui, avant la mort du roi, devait, comme on l'a vu
en son temps, avoir fait ses choix à tête reposée, et n'avoir plus qu'à
les déclarer, n'y avait rien déterminé, ni peut-être pas songé, quoique
je l'en eusse fait souvenir souvent. Il se trouva donc à la mort du roi
comme surpris d'un événement annoncé depuis si longtemps, et comme je le
lui avais prédit, noyé alors d'affaires et de bagatelles, d'ordres à
donner et de choses sans nombre à régler. Il se trouva en même temps
assiégé de gens qui voulaient être de ses conseils qu'il avait annoncés
au parlement.

Il y en avait d'indispensables pour celui de régence par leur état, et
ceux-là lui étaient ennemis ou suspects. Il les fallut balancer par
d'autres, ce qui était d'autant plus important que c'était en ce conseil
où ressortissaient tous les autres, où aboutissaient toutes les affaires
d'État et du gouvernement, et qu'elles y devaient être réglées à la
pluralité des voix. C'est ce qui causa l'extrême lenteur de sa
formation.

L'indigeste composition et formation de tout le nouveau gouvernement fut
duc à l'ambition, à l'astuce et aux persévérantes adresses du duc de
Noailles, qui n'oublia rien pour mettre le plus grand désordre qu'il put
dans l'économie des districts et des fonctions des conseils, pour les
rendre en eux-mêmes ridicules et odieux encore par le mélange et
l'enchevêtrement des matières, et la difficulté de l'expédition, pour
les faire tomber le plus tôt qu'il pourrait, et demeurer lui premier
ministre\,: tellement que choix, rangs, administration, décisions, il y
mit tous les obstacles qu'il put y faire naître pour fatiguer M. le duc
d'Orléans, rebuter le public, qui fut d'abord ravi de ces
établissements, lasser même ceux qui en seraient, en les commettant tous
les uns avec les autres, et les corps aussi des conseils entre eux. Il
en résulta beaucoup d'embarras, de désordres, de maux dans les affaires,
et ce pernicieux homme en eut tout le succès qu'il s'en était proposé,
excepté celui pour lequel il brassa tous les autres, et après lequel il
ne s'est jamais lassé de courir, et court encore plus de trente ans
après, à travers tous les opprobres qu'il a recueillis en ces dernières
guerres, et qu'il avale sans cesse dans son néant à la cour et dans le
conseil, noyé qu'il est dans le mépris universel\footnote{{[}14{]}}.

Dès les premiers jours que nous fûmes à demeure à Paris, c'est-à-dire
aussitôt que le roi fut à Vincennes, il fut question des conseils entre
M. le duc d'Orléans et moi. Ce ne fut pas sans quelques reproches de ma
part de ce que les choix étaient à faire. Il me parla douteusement sur
la place de président des finances, quoiqu'il l'eût promise au duc de
Noailles, comme je l'ai dit, dès avant la mort du roi. Je savais de
reste alors à quoi m'en tenir avec ce galant homme, mais je crus devoir
plus à l'État et à mon premier plan qu'à moi. Je le croyais encore
capable de travail par lui-même, instruit surtout comme il l'était
depuis deux ans par Desmarets. Ses richesses et ses établissements
m'assuraient de la netteté de ses mains\,; son ambition même, de tous
ses efforts à bien faire dans une place si considérable où je voulais un
seigneur, et pour laquelle je n'en voyais point qui l'égalât. Je
raffermis donc M. le duc d'Orléans dans la résolution de la lui donner.

En même temps j'achevai de le fortifier contre les efforts qui se
faisaient contre le cardinal de Noailles. Les cardinaux de Rohan et de
Bissy, le nonce Bentivoglio et les autres chefs de la constitution
étaient dans les plus vives alarmes du traitement que le cardinal de
Noailles recevait depuis la mort du roi. Ils mouraient de frayeur de le
voir à la tête des affaires ecclésiastiques\,; ils remuaient tout pour
l'empêcher, ils criaient à l'aide à tout le monde, ils demandaient aux
gens principaux leur protection pour la religion et pour la bonne cause.
Bissy, dès Versailles, me l'avait demandée tout éperdu, je lui avais
répondu avec une très froide modestie. Un soir qu'il y avait assez de
monde, mais trayé, chez M. le duc d'Orléans, de ces premiers jours à
Paris, je vis le duc de Noailles parler à Canillac, tous deux raisonner
ensemble, me regarder, et tout de suite Canillac venir à moi et me tirer
à part. C'était pour me représenter le danger du délai de déclarer le
cardinal de Noailles chef du conseil de conscience ou des affaires
ecclésiastiques (car ce conseil eut ces deux noms), les mouvements et
les intrigues du parti opposé, et l'embarras où se trouverait M. le duc
d'Orléans, s'il donnait le temps au pape de lui écrire un bref d'amitié
par lequel il lui demanderait comme une grâce de ne pas mettre le
cardinal de Noailles à la tête de ce conseil. Cette raison me frappa\,;
je convins avec Canillac qu'il n'y avait point de temps à perdre. Il me
proposa d'en parler à l'heure même au régent. Quelques moments après je
le fis.

Je lui fis peur de l'embarras où il se trouverait entre désobliger si
formellement le pape, ou lui donner pied à se mêler du gouvernement
intérieur, avec les conséquences pernicieuses qui en résulteraient. Il
les sentit, mais il avait peine à finir. Je lui proposai alors, pour
éviter toute affectation, de déclarer tout à la fois les places du duc
et du cardinal de Noailles, d'appeler le duc sur-le-champ, de faire la
déclaration tout haut, en présence de tout ce monde, et de le charger de
l'aller dire à son oncle. Le régent balança encore, je le pressai, et
j'en vins à bout. Il appela le duc de Noailles, en s'approchant du
monde, et fit la déclaration. Noailles me parut également surpris et
ravi de joie, fit son remerciement pour soi et pour son oncle.

Tout retentit de cette nouvelle aussitôt après dans le Palais-Royal, et
dès le soir à Paris. Le lendemain toute la ville le sut, et la joie et
les applaudissements parurent universels, autant que la douleur et le
dépit furent extrêmes dans le parti opposé, naguère si gros et si
triomphant, alors si réduit en nombre et en crédit. Le remerciement du
cardinal de Noailles, le lendemain au régent, acheva de consommer la
chose.

Il en était temps. On sut que la prière du pape était résolue. Il la
changea en plaintes, mais assez douces, auxquelles le régent répondit
plus doucement encore, mais avec une fermeté sur la chose, mêlée de
force compliments et respects. On vit alors bien à clair le pouvoir de
la puissance temporelle sur les matières ecclésiastiques, et bien à nu
la gaze délice de ce manteau de religion qui couvre tant d'ambition, de
cabales, de brigues et d'infamies.

Cette bonne cause, dont sous le feu roi la foi et toute la religion
semblait dépendre, cette constitution qui avait obscurci l'Évangile
compté pour peu en comparaison, et ce que j'avance en soi n'est point
exagération, changea tout à coup de situation avec ce parti de
mécroyants, de révoltés, de schismatiques, d'hérétiques proscrits,
persécutés, dont les plus hautes têtes abattues sous la plus profonde
disgrâce se voyaient au moment de leur dégradation, et les membres
livrés à la persécution la plus ouverte, dispersés en exil, jetés dans
les prisons et les cachots sans pouvoir trouver de refuge dans les cas
où la justice et l'humanité réclamait inutilement pour eux, sans qu'il
fût permis à aucun tribunal réglé d'admettre la connaissance de leurs
causes. Il ne fallut que ce grand coup à la suite du retour du cardinal
de Noailles et des siens en considération à la mort du roi, pour
atterrer leurs ennemis, écrire sur leur front l'ignominie de leur
ambition, de leurs complots, de leurs violences\,; décrier leur
constitution comme l'opprobre de la religion, l'ennemie de la bonne
doctrine, de l'Écriture, des Pères\,; leur cause comme la plus odieuse
et la plus dangereuse pour la religion et pour l'État.

Je me garde bien ici de prétendre décider rien\,; mon état laïque et la
nature de ces Mémoires purement historiques ne le pourraient souffrir.
Mais je rapporte avec la plus fidèle exactitude quelle fut l'opinion
générale et transcendante du monde laïque et ecclésiastique du vivant et
après la mort du roi, et je m'y arrête d'autant plus volontiers,
qu'outre que ce fait est trop marqué pour ne le pas rapporter, il prouve
avec la dernière évidence le cas qu'on doit faire, en choses d'opinion
et de religion, de ce que la cour appuie ouvertement, jusqu'à y mettre
toute son autorité et son honneur, et à y déployer toute sa puissance et
sa violence, par conséquent le cas qu'on doit faire du grand nombre,
lorsque pendant tant d'années les grâces, les tolérances, toutes sortes
de bienfaits, encore plus d'espérance se trouvent d'un côté\,; toute
persécution, déni de justice, exclusion radicale de tout, prisons,
cachots, expatriations sont de l'autre, sans qu'aucune voix puisse être
écoutée, sans qu'aucun crédit ose s'y hasarder, sans que le plus léger
doute ou soupçon soit moins qu'un crime irrémissible.

Vingt-quatre heures suffirent à un si grand changement\,; quinze jours y
mirent le comble. L'herbe croissait à l'archevêché, il n'y paraissait
que quelques Nicodèmes tremblants sous l'effroi de la synagogue. En un
moment on s'en rapprocha, en un autre tout y courut. Les évêques qui
s'étaient le plus prostitués à la cour, ceux du second ordre qui
s'étaient le plus fourrés pour faire leur fortune, les gens du monde qui
avaient eu le plus d'empressement de plaire, et de s'appuyer des
dictateurs ecclésiastiques, n'eurent pas honte de grossir la cour du
cardinal de Noailles, et il y en eut d'assez impudents pour essayer de
lui vouloir persuader qu'ils l'avaient toujours aimé et respecté, et que
leur conduite avait été innocente. Il en eut lui-même honte pour eux\,;
il les reçut tous en véritable père, et ne montra quelque froideur qu'à
ceux où la duperie aurait été trop manifeste, mais sans aigreur et sans
reproches, peu ému, au reste, de ce subit changement qu'il voyait être
la preuve d'un autre contraire, si la cour venait à cesser la faveur
qu'elle lui montrait.

L'abattement de ses ennemis fut incroyable. Il montra bien qu'ils ne
pouvaient s'appuyer que sur un bras de chair, et ils en étaient si
convaincus, qu'après le premier étourdissement, les plus furieux se
réunirent pour chercher à conjurer l'orage, et à revenir avec le temps
d'où ils étaient tombés, par les mêmes intrigues qui les y avaient
portés la première fois. Dieu qui veut éprouver les siens, dont le règne
n'est pas de ce monde, et pour lequel Jésus-Christ a déclaré qu'il ne
priait pas, permit que ce même monde vînt enfin à bout de ses complots,
et que la bonace fût de peu de durée.

Cette déclaration faite, il devint pressé de former ce conseil, et d'en
choisir les membres. Les matières de Rome, les affaires des divers
diocèses, de nature à avoir besoin de la main du roi, celles des divers
ordres et communautés qui pouvaient passer pour majeures, certaines
matières bénéficiales particulières, quelques dépendances de celles de
la constitution, étaient du ressort de ce conseil\,; car pour la
distribution des bénéfices, le cardinal de Noailles en eut en même temps
la feuille. Le régent crut avec raison le devoir composer de peu de
personnes, et que les unes fussent du métier, c'est-à-dire
ecclésiastiques, les autres du parlement, à cause des matières
bénéficiales, de celles de Rome, et des libertés de l'Église gallicane.
Le cardinal de Noailles fut du même avis, et j'en avais parlé de même à
M. le duc d'Orléans avant la mort du roi.

On choisit donc, de concert avec le cardinal de Noailles, l'archevêque
de Bordeaux, qui le fut après de Rouen, l'abbé Pucelle, conseiller clerc
de la grand'chambre, de la première réputation pour la capacité et
l'intégrité, et qui l'a bien montré depuis avec un sage, niais insigne
courage, d'Aguesseau, procureur général, et Joly de Fleury, premier
avocat général, l'un aujourd'hui chancelier, l'autre procureur général.
L'archevêque était frère du maréchal de Besons, et avait été évêque
d'Aire, le même que j'avais fait travailler sous Mgr le duc de
Bourgogne, comme on l'a vu en son temps, la première fois que le roi lui
envoya l'affaire de la constitution. Par être frère de Besons, il était
agréable au régent, avait toujours tenu une conduite honnête avec le
cardinal de Noailles, et avec les cardinaux de Rohan et de Bissy et les
jésuites, sans bassesse d'aucun côté, ni prostitution\,; il était en
réputation d'homme d'honneur, et du plus capable dans toutes les
affaires temporelles et bénéficiales du clergé, aux assemblées duquel il
était fort rompu, et fort considéré, et sous un extérieur fort rude, il
avait un liant et une douceur fort propre à conciliation. Avec cela
point faux, bon homme et bonne tête pour tout, et ne s'en faisant
accroire sur rien, respectueux et fort courtisan, sans être néanmoins
corrompu, mais complaisant autant qu'il pouvait l'être honnêtement\,:
avec assez d'esprit pour se savoir bien tirer d'affaires.

La composition de ce conseil déplut horriblement aux chefs du parti de
la constitution\,; ils n'avaient pu, dans leur puissance, s'assujettir
l'archevêque de Bordeaux, et en même temps ils ne pouvaient s'en
plaindre\,; mais les trois magistrats leur étaient insupportables par
leurs lumières, par l'expérience qu'ils avaient de leurs artifices, de
leurs détours, de leur violence, et par la fermeté et la capacité avec
laquelle Pucelle s'était conduit contre eux au parlement et {[}avait{]}
donné courage à cette compagnie de leur résister sans cesse, et avec
laquelle d'Aguesseau avait résisté au feu roi, jusqu'à s'exposer à
perdre sa charge. Ils n'étaient pas plus contents de Joly de Fleury, qui
avec plus d'art, de douceur, d'adresse et de finesse, ne leur était pas
moins opposé, et doucement ralliait ses confrères et tout le parlement,
et leur fournissait des armes sans y paraître que le moins qu'il
pouvait, mais se montrant dans le besoin avec une capacité très
supérieure, et des lumières infinies.

Les chefs de la constitution crurent tout perdu par la feuille et par ce
conseil ainsi composé. Ils n'y trouvèrent de remède que par Rome, et
n'oublièrent rien pour irriter le pape, et l'engager d'en demander la
destruction, et de la procurer par toutes sortes de voies. Ils eurent le
dépit de trouver Rome plus sage qu'eux, et un pape qui, bien que très
affligé, prit le parti du silence, et ne voulut jamais se commettre.

Le parlement transporté de joie de voir ceux de ses membres qu'il
estimait le plus employés dans ce conseil, et avant tous autres, se
répandit en applaudissements, et le public entier y répondit par les
siens, dans l'espérance de voir enfin en tout genre la fin de la
tyrannie qui commençait par celle de la religion, et par un choix
justement applaudi de tout le monde.

Ce conseil se tint à l'archevêché. Le cardinal de Noailles proposa au
régent l'abbé Dorsanne pour en être le secrétaire. C'était un saint
prêtre et fort instruit, qui dans la place d'official de Paris avait
mérité l'estime et l'approbation publique. Il s'acquitta très dignement
de cet emploi, et fut toujours semblable à soi-même. Il n'était pas
favorable à la constitution. Ses ennemis prétendirent que le cardinal de
Noailles puisait dans ses lumières, et que Dorsanne le retenait dans sa
fermeté. Il mourut d'une manière fort prompte et fort singulière qui ne
fit pas honneur dans l'opinion publique à MM. de la constitution.

Ce conseil réglé, le plus pressé à former parut être celui des finances.
Le maréchal de Villeroy en demeura chef, mais sans s'en mêler
directement, et il demeura à cet égard comme il était du temps du feu
roi. Noailles, qui sous le titre de président s'en arrogea toute
l'autorité en repaissant le maréchal de toutes sortes de bassesses,
avait hâte de se voir en fonction. Il y avait sept intendants des
finances qui, pour six cent mille livres que leurs charges leur avaient
coûtées, touchaient chacun quatre-vingt mille livres de rente, sans le
tour du bâton que personne ne pouvait supputer. On les supprima tous
sept, en leur payant l'intérêt de leur finance, c'est-à-dire trente
mille livres de rente à chacun en attendant leur remboursement de six
cent mille livres.

Ces sept étaient Caumartin et des Forts, conseillers d'État, Le Rebours
et Guyet, que Chamillart y avait mis, et qui n'avaient qu'une suprême
impertinence\,; Bercy, gendre de Desmarets, d'une humeur étrange et de
mains fort soupçonnées\,; Poulletier, fils d'un riche financier, qui
avait donné huit cent mille livres, c'est-à-dire deux cent mille livres
plus que les autres, et Fagon, tous maîtres des requêtes, qui fut
presque le seul qui entra dans le nouveau conseil des finances. C'était
le fils du premier médecin du feu roi qui, en ce genre, était d'une
grande capacité, et qui le montra bien dans la suite.

Noailles, ami après son père de Rouillé du Coudray, conseiller d'État,
qui avait été directeur des finances, l'y fit entrer, et d'Ormesson,
maître des requêtes, frère de la femme du procureur général d'Aguesseau,
qui était tout aux Noailles. Le régent y joignit Effiat que je lui avais
proposé pendant la vie du roi pour ce conseil, par la richesse dont il
était, et le grand ordre qu'il tenait dans ses affaires, et qui était
fort propre à bien voir tout ce qu'il s'y passerait, et à en tenir M. le
duc d'Orléans bien averti. Le duc de Noailles choisit La Blinière,
ancien avocat, pour secrétaire, qui s'était acquis de l'estime au
barreau. Pelletier des Forts, Gaumont, Gilbert de Voisins et Baudry y
furent joints.

Ces établissements, parmi lesquels on ne disait mot à Pontchartrain, le
mirent en grande inquiétude. Il s'était bassement mis sous la protection
du maréchal de Besons dont il réclamait la parenté, et d'Effiat par lui,
à qui Besons s'était de longue main amalgamé. Ils ne se trouvèrent pas
assez forts pour se promettre de le maintenir. Ils firent donc venir son
père de Pontchartrain, à qui ils procurèrent une audience secrète de M.
le duc d'Orléans au Palais-Royal par les derrières, qui conservait de la
considération pour lui. L'ex-chancelier lui parla si bien qu'il en
obtint que son fils ne serait point chassé, tellement que lorsque j'en
voulus presser le régent, je trouvai un changement que je ne pouvais
prévoir. Je fus quelque temps à découvrir cette visite\,; il fallut
attendre, mais je ne perdis pas mon dessein de vue, et bientôt après
j'en vins à bout.

Peu après le maréchal d'Huxelles, avec qui le régent avait déjà
travaillé, fut déclaré chef du conseil des affaires étrangères. Le
maréchal et l'abbé d'Estrées s'intriguaient depuis longtemps auprès de
M. le duc d'Orléans, je n'oserait ajouter auprès de moi, mais avec une
crainte et des mystères tout à fait plaisants. L'abbé avait donné
plusieurs mémoires historiques sur le gouvernement de l'État à M. le duc
d'Orléans et à moi. Il parvint donc à être de ce conseil des affaires
étrangères, porté par ses ambassades, par la haine de
M\textsuperscript{me} des Ursins, par les Noailles et par moi. J'y fis
entrer Cheverny, dont j'ai parlé ailleurs, qui avait été envoyé
extraordinaire à Vienne, et ambassadeur en Danemark, et M. le duc
d'Orléans y ajouta Canillac. Pecquet, le principal chef des bureaux de
Torcy, en fut le secrétaire.

Villars, second maréchal de France, fut chef du conseil de guerre. Il ne
pouvait ne l'être point dans le brillant où il était, dès que Villeroy,
doyen des maréchaux de France, lui en laissait la place libre par son
titre de chef du conseil des finances, et ses autres futurs emplois. Le
duc de Guiche, longtemps depuis maréchal de France, en fut fait
président, parce qu'il était beau-frère du duc de Noailles, et beaucoup
plus parce qu'il était colonel du régiment des gardes, et que le régent
compta se le dévouer.

Avec moins d'esprit qu'il n'est possible de l'imaginer, fort peu de
sens, une parfaite ignorance, une longue et cruelle indigence et de
grands airs, et un grand usage du monde lui avait appris à se retourner.
Valet des bâtards avec la dernière bassesse, qui comptaient sur lui, et
de toute faveur, comme les Noailles, ses beau-père et beau-frère, il
sut, dans les dernières semaines de la vie du roi, faire accroire à M.
le duc d'Orléans qu'il se tenait caché pour éviter de recevoir des
ordres qui lui fussent contraires, comme si un homme comme lui eût pu
être difficile à trouver. Il sut si bien faire valoir ce service et ceux
qu'il était en situation de pouvoir rendre, qu'il tira pour soi et pour
les siens tout ce qu'il voulut en tout genre, et pour de l'argent\,; on
ne serait pas cru si on articulait le quart de ce qu'il en eut du
régent, puis de Law, lorsque celui-ci exista. Du reste inepte à tout,
payant de grandes manières et de sottise, il n'eut de dupe que le régent
du royaume, et si\footnote{{[}15{]}} ce n'était pas manque d'esprit ni
de connaissance. Mais la parentelle et le régiment des gardes tinrent
lieu de tout.

J'y fis entrer un peu à force Biron et Lévi, tous deux depuis devenus
ducs et pairs, et le premier maréchal de France. Biron était neveu de M.
de Lauzun par sa femme, fille de sa sœur, et il en avait deux,
M\textsuperscript{me}s de Nogaret et d'Urfé, avec qui
M\textsuperscript{me} de Saint-Simon et moi avions intimement vécu à la
cour. Lévi était gendre du feu duc de Chevreuse, neveu par conséquent du
feu duc de Beauvilliers, mérite transcendant pour moi. Puységur, trop
tard maréchal de France, n'y dut une place qu'à son rare mérite, qui a
fait l'honneur des quatre ou cinq dernières campagnes de M. de
Luxembourg, et qui avait servi depuis toujours très utilement. M. le duc
d'Orléans y mit aussi Joffreville, Saint-Hilaire, Reynold et le
chevalier d'Asfeld, longtemps depuis maréchal de France. Le premier et
le dernier étaient gens de talent et de mérite, d'un grand soulagement
pour un général, dont le maréchal de Berwick, qui les estimait et aimait
fort, s'était fort utilement servi en Espagne, et avec toute confiance.
Ils étaient aussi fort gens d'honneur, avec des mains fort nettes, et
ils s'étaient fort attachés à M. le duc d'Orléans en Espagne. Il les
avait fort employés, avait pris pour eux beaucoup d'estime et d'amitié,
et disait qu'Asfeld était le meilleur intendant d'armée par ses soins et
sa prévoyance.

Louvois, qui voulait surtout avec jalousie ce qui avait trait à la
guerre, avait pris les fortifications avec le titre de surintendant. À
son exemple, Seignelay en avait fait autant de celles de places
maritimes. À sa mort Louvois se les fit donner. Il ne les garda qu'un an
et mourut. Le roi, qui ne voulait partout que des gens de robe, et de
qui Pelletier de Sousy était fort connu par son intendance de Lille, du
temps des campagnes du roi en Flandre, et que Louvois son ami lui avait
vanté, crut que ce conseiller d'État et intendant des finances
entendrait bien les fortifications, parce que ses yeux en avaient vu, et
les lui donna avec le titre de directeur général. Il devint ainsi le
maître de cette dépense, l'arbitre du mérite des ingénieurs, le seul
ministre de ce district à part, et de leurs promotions, avec un travail
réglé avec le roi tête-à-tête toutes les semaines, qui lui en faisait
toujours passer une partie à Marly. Rien peut-être n'était plus ridicule
qu'un magistrat arbitre des fortifications et des ingénieurs. Le régent
ôtant la guerre à la robe lui en ôta aussi cette partie si principale,
et je l'engageai assez aisément de la donner à Asfeld.

Saint-Hilaire, lieutenant général de l'artillerie, en eut le département
au conseil de guerre. Il était fils de celui qui eut le bras emporté du
même coup de canon qui tua M. de Turenne, et il y était présent. C'était
un homme fort lourd, mais qui entendait bien l'artillerie\footnote{}.
Lui et Reynold furent regardés comme deux {[}personnes{]} nulles. Ce
dernier était colonel du régiment des gardes suisses, et eut le corps
des Suisses pour son département au conseil de guerre. Il s'était offert
de très bonne grâce à M. le duc d'Orléans tout d'abord, et sans autre
ménagement pour M. du Maine, avec qui il était bien, que de respect,
cela en galant homme qui va droit où l'autorité doit être. L'autre en
avait fait autant pour l'artillerie. Tous ces messieurs étaient
lieutenants généraux.

Il fallut songer aux vivres, étapes, fourrages, et aux divers marchés,
par conséquent à des gens dont ce fût plus particulièrement le métier.
C'est ce qui fit choisir deux intendants de frontière distingués en ce
genre\,: Le Blanc, de la partie maritime de la Flandre, et Saint-Contest
de Metz, qui était de mes amis, et qu'on a vu ici aller signer en
troisième la paix de l'empereur et de l'empire à Bade. C'était un homme
d'un extérieur lourd et grossier, avec toutes les manières ridiculement
bourgeoises, qui avait tout l'art, la finesse, la souplesse, les vues et
les tours pour arriver à ses fins sans avoir l'air de penser à rien,
lors même qu'il y travaillait le plus. Cela lui était naturel. Avec cela
doux, liant, accessible et honnête homme. Il fut enfin reconnu à Cambrai
par les ministres étrangers du congres, où il était l'un des
ambassadeurs de France. Ils l'aimaient tous, mais ils le craignaient.
L'autre était plein d'esprit, de capacité, d'expédients, fort liant
aussi, tous deux gens de travail et d'expérience, qui connaissaient le
monde, et qui avaient toujours su contenter tous ceux qui avaient eu
affaire à eux. Leur choix aussi fut fort applaudi. Je ne connaissais
point du tout Le Blanc, je m'en accommodai fort. Il y aura beaucoup lieu
d'en parler dans la suite, et l'histoire de son temps ne se pourra taire
de sa fortune, de sa catastrophe, et de son triomphant retour. Ce sont
des événements que tout le poids d'un prince du sang premier ministre ne
saurait étouffer.

Le conseil de marine fut aisé à composer. Le comte de Toulouse, comme
amiral, en fut chef\,; le maréchal d'Estrées, premier vice-amiral, en
fut président\,; le maréchal de Tessé y entra comme général des
galères\,; Coetlogon, mort maréchal de France, et d'O, comme lieutenants
généraux de mer\,; Bonrepos qui avait été intendant général de la
marine, que j'aidai à en être\,; Vauvray et un autre intendant de
marine, avec La Grandville, maître des requêtes, pour rapporteur des
prises. J'y fis mettre pour secrétaire ce même La Chapelle que
Pontchartrain avait chassé de ses bureaux et dont j'ai parlé plus d'une
fois.

La place de chef du conseil des affaires du dedans du royaume, qui était
proprement le conseil des dépêches, celles des départements des
provinces des quatre secrétaires d'État, et quelques autres encore de
pareille nature, fut offerte au maréchal d'Harcourt. Il s'en excusa sur
le travail de cet emploi, et sur la difficulté de parler bien librement,
qui lui était demeurée de ses apoplexies, et qui le mettait hors d'état
de rapporter souvent et longuement les affaires de ce conseil à la
régence. Ces raisons étaient vraies et solides. Harcourt, dans la
considération où il s'était mis, voyait bien que le régent ne pourrait
se dispenser de l'admettre au conseil de régence, et se tint ferme aux
refus réitérés. Je ne voyais que d'Antin à mettre à la tête du conseil
du dedans\,; je le proposai, je fus refusé.

C'est le seul homme pour qui M. le duc d'Orléans, si fort sans aucun
fiel pour ses plus mortels ennemis, ait conservé rancune, et le seul
encore pour qui ce prince, si indifférent à la vertu, n'ait pu vaincre
son mépris. On a vu les raisons de l'un et de l'autre dans le cours de
ces Mémoires. D'ailleurs lié étroitement aux bâtards par état et par
besoin sous le feu roi, et tout à M\textsuperscript{me} la Duchesse, ce
prince si aisément soupçonneux ne le pouvait souffrir.

D'Antin, depuis qu'il était due, s'était peu à peu jeté à moi. M. {[}le
Dauphin{]} et M\textsuperscript{me} la Dauphine, les ducs de Chevreuse
et de Beauvilliers, le maréchal de Boufflers étaient disparus\,; il n'y
avait plus trace de Monseigneur ni de la cabale de Meudon\,; le mariage
de M\textsuperscript{me} la duchesse de Berry était fait\,; elle était
veuve, et M\textsuperscript{me} la Duchesse l'était aussi depuis
longtemps. Mon éloignement pour d'Antin avait cessé avec les personnes
et les causes qui le formaient. Je sentais également tout son fumier,
mais je n'en pouvais ignorer les perles qui y étaient semées, et je ne
voyais personne de rang qui eût plus de talents pour bien remplir cette
place. D'Antin d'ailleurs avait trop d'esprit et trop peu de courage
pour se laisser engager contre le régent\,; il connaissait trop aussi M.
et M\textsuperscript{me} du Maine pour s'attacher véritablement à eux.
Il tenait trop d'ailleurs de tout temps à M\textsuperscript{me} la
Duchesse qui les détestait souverainement. Par cette liaison intime, il
était propre à en former une entre le régent et M. le Duc, sur qui l'âge
et la confiance de M\textsuperscript{me} la Duchesse lui donnait de
l'autorité, qui demeurerait crédit et créance quand ce prince viendrait
à l'âge d'être compté, ce qui arriverait bientôt\,; enfin l'esprit
courtisan de d'Antin, et la servitude tournée en lui en nature, me
rassurait pleinement. C'était un homme naturellement brutal et livré à
tous les vices, mais si maître de soi qu'il était doux, liant, patient,
plein de ressources. Personne n'avait plus d'esprit, ni de toutes sortes
d'esprit, et avec un air tout grossier et tout naturel, plus d'art, de
tour, de persuasion, de finesse, de souplesse. Il était et il disait
tout ce qu'il voulait, et comme il le voulait\,; et hors d'intérêt, il
était bon homme, et aimait à faire plaisir. Toutes ces raisons me
déterminèrent à m'opiniâtrer pour lui.

La défense du régent dura plus de douze ou quinze jours. Il se rendit
enfin, mais de mauvaise grâce\,; d'Antin fut déclaré chef du conseil des
affaires du dedans du royaume\,; mais quelque soin, quelques contours
qu'il put employer, jamais il ne prit bien avec M. le duc d'Orléans.

Je proposai à ce prince le marquis de Brancas et Beringhen, premier
écuyer du roi, pour entrer dans ce conseil. Je réussis aisément pour le
premier des deux qui s'était bien conservé avec lui, et à qui sa
brouillerie ouverte avec la princesse des Ursins avait ajouté du mérite.
Je n'obtins pas l'autre avec tant de facilité.

C'était un personnage de ce qu'on appelait alors de la vieille cour,
mais plus par ses amis et ses liaisons, le soutien de sa charge, et
l'habitude de la cour et du grand monde, que par lui-même. Il était fort
honnête homme, court d'esprit, pesant de langage, fort bien avec le roi,
avec le duc du Maine, avec le maréchal de Villeroy, avec Harcourt, avec
son cousin germain le maréchal d'Huxelles, avec le premier président,
intime de ces deux derniers, fort lié encore avec le duc d'Aumont, son
beau-frère, que j'empêchai d'arriver à rien, assez aussi avec le duc
d'Humières, son autre beau-frère, pour qui M. le duc d'Orléans m'avait
promis merveilles, et à lui-même aussi, car je les avais abouchés tous
deux dans les derniers jours de la vie du roi en rendez-vous pris exprès
dans un bosquet de Versailles près de l'Orangerie. Je n'ai pu démêler ce
qui nous fit manquer de parole, mais jamais je n'ai pu parvenir à rien
pour lui, quelque travail que je m'en sois donné. Enfin je résolus le
régent à mettre Beringhen dans le conseil du dedans. On a vu qu'il était
intimement avec le chancelier de Pontchartrain, que je l'y avais connu,
et que nous étions ensemble sur le pied de confiance.

J'étais aussi ami du marquis de Brancas, longtemps depuis grand
d'Espagne et maréchal de France. On a vu en son temps l'origine et les
chemins de sa fortune. Jamais il ne négligea aucun des chemins qui l'y
pouvaient conduire. M\textsuperscript{me} de Maintenon fut sa
protectrice\,; il fut très bien avec M. et M\textsuperscript{me} du
Maine, qu'il cultiva dans tous les temps, et sut n'en être pas moins
bien avec M. le duc d'Orléans. Il parvint à manger également au râtelier
de la guerre et à celui de la cour, et les faire servir réciproquement
l'un à l'autre. Aussi avait-il de l'esprit, encore plus d'art, d'adresse
et de manège, avec une ambition insatiable qui ne lui a jamais laissé de
repos. C'était un grand homme, fort bien fait, d'une figure avenante,
avec des manières polies, aisées, entrantes, qui ne faisait jamais rien
sans dessein, et qui aîné de quinze ou seize frères ou sœurs, avec sept
ou huit mille livres de rente entre eux tous, devenu conseiller d'État
d'épée, chevalier du Saint-Esprit et de la Toison, lieutenant général de
Provence, gouverneur de Nantes et tenant les états de Bretagne, grand
d'Espagne et maréchal de France, avec un grand mariage pour son fils,
l'archevêché d'Aix et l'évêché de Lisieux pour ses frères, se mourait de
douleur de n'être pas ministre d'État, duc et pair, et gouverneur de Mgr
le Dauphin.

J'en parle comme d'un homme mort par les apoplexies dont il est accablé,
qui apparemment ne le laisseront pas vivre longtemps. Il a la main
droite toujours gantée, même en mangeant\,; les doigts en paraissent
vides, il n'y a qu'un mouvement léger du pouce\,: homme vivant ne l'a
jamais vue. À la grosseur du dedans, et à tout ce qu'on en voit, il
paraît que c'est une patte de crabe ou de homard. Ses façons et sa
conversation étaient agréables, et il était fort instruit de tout ce qui
se passait au dedans et au dehors. Dévot et constitutionnaire jusqu'au
fanatisme, et du petit troupeau de Fénelon qui n'empêche pars l'ambition
à pas un des disciples de cette école.

Brancas eut les haras qui furent d'abord ôtés à Pontchartrain, et le
premier écuyer les grands chemins, ponts et chaussées, pavé de Paris,
etc., dont il s'acquitta en perfection. Il n'en fut pas de même des
haras, que Brancas acheva de laisser perdre, quoiqu'il en eût douze
mille livres d'appointements particuliers.

À ces messieurs on joignit Rougeault, intendant de Rouen, avec un autre
ou deux, et l'abbé Menguy et Goeslard, tous deux conseillers de la
grand'chambre, à cause des procès fréquents en ce conseil, et des
évocations qu'on y en pouvait faire. Le choix fut aussi fort applaudi.
La Roque, attaché à d'Antin, homme d'esprit et capable, fut, à sa
recommandation, secrétaire de ce conseil.

\hypertarget{chapitre-viii.}{%
\chapter{CHAPITRE VIII.}\label{chapitre-viii.}}

1715

~

{\textsc{Conseil de régence.}} {\textsc{- Caractère de Besons.}}
{\textsc{- Torcy.}} {\textsc{- Bouthillier-Chavigny, ancien évêque de
Troyes.}} {\textsc{- La Vrillière sans voix\,; son caractère et ses
fonctions.}} {\textsc{- Pontchartrain sans voix ni fonction.}}
{\textsc{- Rage et conduite de Tallard.}} {\textsc{- Personnages des
conseils.}} {\textsc{- Desmarets congédié avec une gratification de
trois cent cinquante mille livres.}} {\textsc{- Trop juste augure de M.
le duc d'Orléans.}} {\textsc{- Catastrophe de M\textsuperscript{me}
Desmarets.}} {\textsc{- Bercy, son gendre, chassé.}} {\textsc{- Lieux
des divers conseils.}} {\textsc{- Leurs appointements.}} {\textsc{-
Règlements particuliers.}} {\textsc{- Prétention des conseillers d'État
de ne céder qu'aux ducs et aux officiers de la couronne.}} {\textsc{-
Noailles et Canillac avocats des conseillers d'État contre les gens de
qualité.}} {\textsc{- J'expose au régent la qualité et le ridicule de
cette prétention.}} {\textsc{- Mollesse du régent.}} {\textsc{- Adresse
des conseillers d'État.}} {\textsc{- Effiat vice-président.}} {\textsc{-
Forme des conseils du feu roi adoptée au conseil de régence.}}
{\textsc{- Les maîtres des requêtes refusent de rapporter au conseil de
régence, s'ils n'y sont assis, ou si ceux de ce conseil qui ne sont ni
ducs, ni maréchaux de France, ou conseillers d'État, n'y sont debout
tant que les maîtres des requêtes y seraient.}} {\textsc{- Les
conseillers au parlement mis dans les conseils imitent les maîtres des
requêtes, et le régent le souffre.}} {\textsc{- Deux exemples de
l'inconvénient qui en résulte pour les affaires.}} {\textsc{- Les
maîtres des requêtes cèdent enfin aussitôt après la mort du chancelier
Voysin, et, sans plus de prétentions, rapportent debout au conseil de
régence.}} {\textsc{- Les conseillers d'État emportent d'y précéder tout
ce qui n'est pas duc ou officier de la couronne, lorsqu'ils y viennent
extraordinairement.}}

~

Tous ces conseils choisis, il fallut enfin en venir à celui de régence,
dont la formation était la plus difficile. Il devait être composé
d'assez peu de membres pour le rendre plus auguste, et il y avait
plusieurs personnages ennemis de M. le duc d'Orléans, ou fort suspects,
que leur état ne permettait pas d'en exclure. Tels étaient le duc du
Maine, le comte de Toulouse, le maréchal de Villeroy, le maréchal
d'Harcourt dès qu'il avait refusé la place de chef du conseil des
affaires du royaume, le chancelier Voysin dès que M. le duc d'Orléans
avait fait la faute énorme de se laisser engager à lui laisser les
sceaux. Toulouse et Harcourt n'étaient que suspects\,: ils l'étaient
beaucoup, l'un par son être et par son frère, quelque différent qu'il
fût de lui\,; l'autre par son ancienne intimité avec
M\textsuperscript{me} de Maintenon et la princesse des Ursins. Tous les
autres étaient ennemis. Il fallait donc les contre-balancer par des gens
sûrs pour M. le duc d'Orléans, et qui fussent en état de se faire
écouter dans le conseil, où toutes les affaires du dehors et du dedans
étaient rapportées des autres conseils, et décidées en dernier ressort
en celui-ci à la pluralité des voix. Il fallut de plus considérer que
celle de M. le duc ne pouvait encore être d'aucun poids, et que ce
poids, venu avec l'âge, se pouvait, par les intérêts et les cabales,
détourner aussi aisément contre que pour M. le duc d'Orléans.

La facilité de ce prince fut telle eu chose de cette importance, qu'il
se laissa aller aux instances du maréchal de Besons, appuyé d'Effiat,
pour le changer du conseil de guerre, où il était destiné, et où il n'y
avait que la bienveillance du régent qui l'y pût faire préférer à
d'autres, pour le placer dans le conseil de régence. C'était un rustre
brutal qui s'était échappé tout jeune de la maison de son père, qui le
voulait faire d'église, s'était enrôlé dans les troupes qui passaient
clandestinement en Portugal, et y porta le mousquet. Y étant reconnu par
les perquisitions de son père, il fut bientôt fait officier, et servit
avec application. C'est avec le latin qu'il savait avant que de
s'enrôler, toute l'éducation qu'il avait eue. Il était bon officier
général, entendait bien à mener une aile de cavalerie, et de certains
détails, encore ses brusqueries et son emportement l'empêchaient-ils
souvent de voir et d'entendre. Ce qui était au delà surpassait fort sa
portée, comme il a paru quand il a eu quelquefois des armées à
commander, par accident. Avec une humeur insupportable et fort peu
d'entendement, c'était un homme brave de sa personne, et qui savait ce
que c'était que l'honneur, mais embarrassé de tout, infiniment timide,
qui ménageait tout, avait grande passion d'être et d'avoir, fort bas et
fort plat, qui ne manquait pas de sens ni d'un certain petit esprit de
courte intrigue, avec assez de jugement. Une tête de lion et fort
grosse, lippu, dans une grosse perruque qui eût fait une bonne tête de
Rembrandt, et qui, paraissant tout d'une pièce, comme tout son corps,
passait parmi les sots pour une bonne tête.

Son père était conseiller d'État\,; et son frère aîné, qui était mort,
l'avait été aussi, tous deux avec réputation. Leur nom est Bazin, de la
plus courte bourgeoisie, et Besons, dont ils portaient tous le nom, est
ce village sur la Seine, près de Paris, si connu par la foire qui s'y
tient tous les ans, dont le père avait acquis la seigneurie. Ce n'était
pas là un personnage à opposer à personne dans un conseil de régence. M.
le duc d'Orléans fut honteux avec moi de s'y être laissé engager\,; et
moi, dont la destination n'avait point changé, fort fâché de me trouver
si mal attelé.

Un autre homme que le régent mit dans le conseil de régence, dont il fut
très embarrassé avec moi, et qu'il ne me laissa entendre que par degrés,
fut Torcy, à la surprise de toute la France. Il était lié de tout temps
à la cour avec tout ce qui était le plus opposé à M. le duc d'Orléans,
si on en excepte ses deux plus funestes ennemis, M\textsuperscript{me}
de Maintenon et M. du Maine. M. le duc d'Orléans avait eu souvent des
raisons de n'en être pas content, et jusqu'après la mort du roi, jamais
lui ni sa femme n'avaient fait aucun pas pour s'en rapprocher. Ils
étaient amis intimes de M. et de M\textsuperscript{me} de Castries et de
l'abbé de Castries, qui était une voie bien naturelle qu'ils pouvaient
prendre. Castries était chevalier d'honneur, et sa femme, dame d'atours
de M\textsuperscript{me} la duchesse d'Orléans, et fille de M. de
Vivonne, frère de M\textsuperscript{me} de Montespan, et très bien avec
M. {[}le duc{]} et M\textsuperscript{me} la duchesse d'Orléans. Ils
étaient si persuadés que Torcy leur était opposé, qu'ils étaient peinés
contre les Castries de leur liaison avec lui, et je me souviens que
longtemps après que M\textsuperscript{me} la duchesse d'Orléans eut
commencé d'avoir une tablé à Marly, et que les dames se furent
accoutumées à y aller, ce fut une manière de négociation de
M\textsuperscript{me} de Castries pour y faire manger
M\textsuperscript{me} de Torcy. Elle n'y avait point encore été conviée,
c'était une singularité peu agréable, et néanmoins elle ne s'en
empressait pas. Surtout elle ne pouvait se résoudre à la présence de M.
le duc d'Orléans, et M\textsuperscript{me} de Castries prit si bien son
temps, qu'elle lui procura d'y dîner pendant que ce prince était allé
faire un tour à Paris.

J'étais aussi fort persuadé de l'opposition de Torcy à M. le duc
d'Orléans\,; j'étais gîté sur lui, je l'avoue franchement, par les
sentiments que les ducs de Chevreuse et de Beauvilliers avaient pris
pour lui, quoique leurs raisons d'éloignement ne fussent guère que par
rapport aux matières de Rome. Jamais je n'avais eu avec eux, non pas de
liaison, mais de connaissance la plus légère, et si la vérité veut qu'on
ne cache rien, ils n'avaient chez eux que la meilleure compagnie et la
plus trayée, et mon amour-propre n'était pas content de n'avoir jamais
reçu la moindre avance de leur part. C'était de plus un homme de
l'ancien ministère, et dans mon dessein d'anéantir les secrétaires
d'État et leur puissance, Torcy, qui l'était après son père et son
beau-père, ne pouvait être à mon gré. J'avais souvent pressé M. le duc
d'Orléans de l'exclure\,; quoiqu'il ne m'eût jamais répondu là-dessus
aussi net que je le désirais, j'espérais pourtant son exclusion, et j'y
travaillais encore, lorsque le régent me laissa entrevoir que je n'y
devais pas compter. Je redoublai mes efforts\,; à la fin il m'avoua avec
grand embarras qu'il se le croyait nécessaire par avoir le secret de
toutes les affaires étrangères depuis tant d'années qu'il en était le
ministre, et par le secret des postes dont lui ne pouvait se passer. Ce
fut en effet ce qui conserva Torcy.

Pour se l'acquérir entièrement, M. le duc d'Orléans le combla de
caresses, de confiance et de choses. Il avait six cent cinquante mille
livres de brevet de retenue sur sa charge de secrétaire d'État\,; il en
eut cent cinquante mille de plus et tout payé en en donnant sa
démission. Sa pension de vingt mille livres de ministre d'État lui tut
conservée, et il en eut encore une autre de soixante mille livres sur
les postes, dont il conserva la direction, l'autorité et la confiance.

On ne peut exprimer l'étonnement public de ce traitement. Torcy y
passait, pour le moins, et avec raison, pour n'avoir jamais eu de
liaison avec M. le duc d'Orléans, même pour lui avoir été contraire. On
ne lui avait découvert aucun mouvement vers ce prince\,; les Castries
étaient trop faibles et trop suspects par rapport à
M\textsuperscript{me} la duchesse d'Orléans, pour y avoir été utilement
employés. Nancré le fut peut-être\,; mais je l'ai toujours ignoré, et
tout ce que j'ai tâché de pénétrer là-dessus ne m'a rien rapporté, sinon
à me confirmer que le secret des postes avait seul opéré ce traitement
si peu attendu. On verra dans la suite combien je reconnus mon erreur,
et la liaison étroite que l'estime, que j'ose dire réciproque, fit entre
Torcy et moi, qui a duré jusqu'aujourd'hui que nous sommes en mars 1746.

M. le duc d'Orléans avait toujours compté de mettre un évêque dans le
conseil de régence. Je croyais qu'il pouvait s'en passer. Je pensais
là-dessus comme le feu roi, et je crois comme tout homme sage, surtout
dans le feu des affaires de la constitution. L'intérêt du feu archevêque
de Cambrai, par le poids immense du feu duc de Beauvilliers sur moi,
m'avait empêché de combattre ce sentiment, de sorte qu'il n'était plus
temps de s'y opposer avec fruit depuis la mort de ces deux personnages.
Je pensai donc alors au moins mauvais et au plus approuvé qu'on pourrait
choisir, et je proposai à M. le duc d'Orléans l'ancien évêque de Troyes.

On a vu qui et quel il était, au commencement de ces Mémoires où je me
suis étendu sur lui à l'occasion de sa retraite. Elle arriva tout au
commencement de mon mariage. À l'âge que j'avais lors, j'avais vu son
visage tout au plus, et je ne l'avais jamais connu. Mais à ce que j'en
savais, il me parut fait exprès pour entrer dans le conseil de régence.
Sans répéter ce que j'en ai dit lors de sa retraite, j'y trouvais un
prélat consommé dans les affaires temporelles du clergé, versé dans les
matières de Rome, et avec cela Français\,; assez de savoir
ecclésiastique. Voilà quelle était sa réputation. Il avait de plus passé
sa vie jusqu'à la retraite dans le plus grand monde de la cour et de la
ville, recherché des meilleures et des plus importantes compagnies, ami
de la plupart des personnages et des principales femmes de son temps, où
il s'était mêlé de beaucoup de choses. Cette grande connaissance du
monde était un grand point.

C'était un évêque sans diocèse, et un évêque qui ne pensait à rien moins
qu'à revenir sur l'eau. Il y avait quinze ou seize ans qu'il vivait dans
la plus exacte retraite et la plus soutenue. Il ne l'avait interrompue
que depuis quatre ou cinq ans par respect pour cette fantaisie du roi de
voir les gens retirés, et qui lui fit dire qu'il voulait le voir une
fois l'année. Il venait passer quatre jours à Fontainebleau, où le roi
lui faisait merveilles, et où, dans ce qu'il y avait de plus grand et de
meilleur, c'était à qui l'aurait. Il allait de là passer deux jours à
Paris, revenait pour un jour ou deux à Fontainebleau, et s'en retournait
dans sa retraite, sans avoir paru ni rouillé, ni béat, ni déplacé, ni
gâté. À Troyes il ne voyait pas même les passants. Il y vivait avec son
neveu dans l'évêché. Dès que son neveu était en visites ou à Paris, il
occupait un appartement qu'il s'était accommodé dans la Chartreuse de
Troyes, où il ne voyait que les chartreux, et se rendait assidu à leurs
offices\,: il y passait de plus les avents et les carêmes. Une telle
vie, entée sur celle du plus grand monde, uniquement par choix, et si
bien soutenue, me parut devoir être d'un grand poids pour retenir la
licence de la vie de M. le duc d'Orléans. Cet évêque n'avait rapport à
aucune cabale\,; il était frère de la maréchale de Clérembault, en
amitié avec elle, qui était dans l'intimité de Madame, laquelle avait
beaucoup d'amitié aussi et de confiance en lui. Tout me persuada donc
qu'il était fait exprès pour cette place, dès qu'il y fallait un évêque.
M. le duc d'Orléans l'approuva et l'exécuta.

Rien ne fut plus applaudi que ce choix. Il le manda\,; il arriva, il
accepta sans simagrée. Le monde, qui exige presque toujours des gens de
bien fort au delà du but, aurait voulu une défense, ou même un refus.
Les commencements furent admirables. On ne le voyait que pour des
devoirs indispensables. Je me félicitais d'avoir si bien rencontré. Ces
merveilles furent de médiocre durée\,; je me trompai sur lui comme
j'avais fait sur Torcy, mais d'une manière tout opposée\,; il n'est pas
encore temps d'en parler. Le régent lui fit la galanterie de ne faire
entrer Torcy au conseil de régence qu'après que ce prélat y eut assisté
une lois, afin de lui assurer sans dispute la préséance sur Torcy qui
avait été jusqu'à la fin ministre d'État sous le feu roi.

La Vrillière me dut tout ce qu'il fut, et, comme je l'ai dit ailleurs,
ce ne fut pas sans peine, mais le travail opiniâtre de plus d'une année.
Il conserva sa charge de secrétaire d'État, fut établi secrétaire du
conseil de régence pour en tenir le registre, signer les grâces des
départements des autres secrétaires d'État, et tout ce qui avait besoin
de la signature d'un secrétaire d'État\,; avec le temps celle des
expéditions et des ordres secrets, l'autorité sur la police de Paris\,;
enfin en très peu de temps, il fut l'unique secrétaire d'État en
fonction. Lui et Pontchartrain entrèrent au conseil de régence, tous
deux sans voix\,; Pontchartrain sans nulle fonction. Je me plaignis à M.
le duc d'Orléans de la conservation de celui-là. Il balbutia\,; il fut
embarrassé\,; je jugeai donc qu'il fallait attendre\,; j'ignorais alors
la visite du chancelier de Pontchartrain. J'attendis donc\,; mais je
n'attendis pas longtemps

La Vrillière était un petit homme vif, actif, qui élevé dans les bureaux
de son père en possédait la routine, obligeant, très serviable, fort
poli, intérieurement glorieux, capable d'expédient et de mécanique,
liant et rompu au monde, homme d'honneur. Il n'était pas heureux en
femme, qui le gâta à la fin, au point qu'il n'était plus reconnaissable.
Cela se trouvera en son temps.

J'ai, ce me semble, assez fait connaître le caractère et les liaisons de
ce qui composait la cour du feu roi, et des personnages qui entrèrent
dans ces divers conseils, pour n'avoir pas besoin de retoucher cette
matière. Mais il faut encore faire voir quel fut le tout ensemble de cet
important conseil de régence qui devait décider de tout à la pluralité
des voix, et qui fut en effet un vrai conseil pendant près de trois
années. J'y ajouterai les chefs ou autres des autres conseils qui y
venaient rapporter leurs affaires, et qui, pour de certaines, y furent
quelquefois appelés, tandis que les conseils demeurèrent dans leur
premier établissement.

La régence était donc, pour le répéter de suite, ainsi composée. M. le
duc d'Orléans, M. le Duc, le duc du Maine, le comte de Toulouse, Voysin
chancelier, moi, puisqu'il faut que je me nomme, les maréchaux de
Villeroy, d'Harcourt, de Besons, l'ancien évêque de Troyes, et Torcy
opinants, et La Vrillière tenant le registre, et Pontchartrain, tous
deux sans voix.

Ceux qui y venaient rapporter étaient l'archevêque de Bordeaux, les
maréchaux de Villars, d'Estrées et d'Huxelles, les ducs de Noailles et
d'Antin.

On voit ainsi sur quels et sur combien le régent pouvait compter pour
amis, pour ennemis ou pour assez indifférents. Il arriva pourtant
presque toujours que le conseil fut tranquille et que le régent y fut
maître de tout. Le personnage que chacun de ceux-là y fit se verra avec
le temps.

De cette façon Desmarets fut le seul des ministres du feu roi congédié
alors par une courte lettre que M. le duc d'Orléans lui écrivit, et les
six conseils furent enregistrés au parlement, c'est-à-dire leur
établissement, non pas les noms ni le nombre de leurs membres. Il n'y
fut pas mention du conseil de régence, comme étant le conseil du roi, et
le gouvernement même.

Tallard fut aussi le seul qui ne fut point employé de tous ceux que le
roi avait nommés dans son testament. Ce n'est point trop dire qu'il
pensa en devenir fou, et qu'il fit plusieurs extravagances. Il alla
disant partout qu'il se ferait écrire le testament du roi sur le dos\,;
il cria, clabauda, lâcha au régent le maréchal de Villeroy et les
Rohan\,; plaintes, clameurs, dépits, bassesses, prostitutions, tout fut
mis inutilement en usage. Jamais le régent, si ordinairement facile, ne
put être entamé. En général il le regardait comme contraire à lui, avec
raison, mais il fallait qu'il y eût quelque autre cause que je n'ai
point démêlée, qui le soutint le même contre tant d'efforts. Tallard,
les voyant enfin inutiles, déclara qu'il n'avait plus qu'à s'enterrer.
Il acheta la Planchette, vilaine petite maison près de Paris, et s'y
confina en effet sans presque en sortir ni y recevoir personne. Nous
verrons sa résurrection dans son temps.

Le régent vécut en amitié avec M. le Duc, en mesure froide et polie avec
le duc du Maine, avec plus d'onction, mais en réserve avec le comte de
Toulouse. Il crut gagner le maréchal de Villeroy à force de marques
d'estime, de considération, de distinction, même de confiance fort
hasardée\,; le ramener, au moins émousser ses pointes et ses écarts par
d'Effiat, son ami de tous les temps, et par M. de Troyes qui l'était
aussi. Le premier était vendu au duc du Maine\,; l'autre, marchant sur
des oeufs, n'osait être que complaisant. Le maréchal reçut toutes sortes
de faveurs et se piqua de ne s'en pas laisser ébranler. Il fallait
exposer cela d'abord. C'est une matière qui se présentera plus d'une
fois. Pour Harcourt, sa malheureuse santé ne lui permit pas de faire
aucun personnage, ni à Voysin le dégoût et le mépris dans lequel il
était tombé. Villars en fit toujours un fort misérable\,; Huxelles aussi
avec toutefois beaucoup d'importance\,; Estrées comme point\,; d'Antin
aussi peu. Le cardinal de Noailles ne se haussa ni baissa\,; il eut
assez d'affaires à se défendre des insidieux chefs de la constitution.
Le duc de Noailles joua le grand personnage. M. le Duc encore trop
jeune, le duc du Maine silencieux, ténébreux, solitaire, profondément
caché, poli jusqu'au respectueux, et attentif au dernier point à tout le
monde, quand il était forcé d'en voir\,; le comte de Toulouse froid,
tranquille, et menant sa vie ordinaire autant qu'il la put accommoder à
ses nouvelles fonctions.

Desmarets tomba dans une surprise incroyable. Sa suffisance extrême lui
avait persuadé qu'il était impossible de se passer de lui à la tête des
finances. Il était de tout temps ami intime du maréchal de Villeroy\,;
il l'était demeuré d'Effiat, qui l'avait toujours été au Palais-Royal de
Bechameil, son beau-père. II comptait donc entièrement sur ces deux
appuis\,; mais ce qui combla son étonnement et son indignation fut de
voir le duc de Noailles à sa place, lui qui l'avait recueilli, lorsqu'à
son retour d'Espagne il ne sut, comme on l'a vu dans son temps, où
donner de la tête\,; qui en avait fait son disciple et son élève dans
les finances, et pour qui il avait contraint toute sa féroce humeur.
Noailles ne songea pas seulement à garder avec lui aucunes mesures, et
on verra bientôt jusqu'où il poussa l'ingratitude à son égard. M. le duc
d'Orléans néanmoins, pressé par Effiat et par le maréchal de Villeroy,
lui fit donner trois cent cinquante mille livres au renouvellement des
fermes, sur ce qu'ils lui représentèrent que c'était un droit des
contrôleurs généraux, que Desmarets n'avait pas voulu toucher au dernier
renouvellement, dans l'extrémité où étaient les besoins de l'État.

Une si forte grâce, et faite si fort à contre-temps, à la suite de
plusieurs autres facilités du régent, dont j'ai parlé, et d'autres
moindres que j'ai omises, firent augurer en lui une faiblesse fort
nuisible à l'État et aux honnêtes gens, et fort utile aux impudents et
aux effrontés. Malheureusement l'augure ne s'en est trouvé que trop
véritable.

M\textsuperscript{me} Desmarets qui, sous l'ombre de la place de son
mari, faisait à part pour elle quantité d'affaires, culbuta avec lui. Un
nommé La Fontaine, longtemps receveur de M. le Prince à Senonches, près
de la Ferté, où je l'avais vu, et qui de là, qui est aussi auprès de
Maillebois, avait été leur complaisant pendant leur exil, avait aussi
fait fortune avec eux, et s'était fait trésorier du régiment des gardes.
C'était l'homme de confiance de M\textsuperscript{me} Desmarets, pour
lui faire faire tous les jours des affaires, et pour placer et gouverner
l'argent qu'elle en tirait. Tout cela se renversa à la chute de la
place. Elle prétendit avoir été volée. Elle en fut étrangement troublée.
Dans cet état la petite vérole la prit\,; elle en releva folle\,; et
personne même ne l'a jamais vue depuis, quoiqu'elle ait encore vécu
quelques années. Ainsi les deux rivales des bonnes grâces de
M\textsuperscript{me} de Maintenon, M\textsuperscript{me} Voysin et
M\textsuperscript{me} Desmarets sont mortes, l'une de désespoir de les
avoir perdues et d'être supplantée par sa rivale\,; celle-ci folle de la
perte de sa place et de son magot particulier. Bercy, intendant des
finances et gendre de Desmarets, qui faisait tout sous lui, fut chassé
en même temps sans retour, avec l'acclamation publique.

Il fut réglé que le conseil de conscience se tiendrait à l'archevêché,
et tous les autres en divers appartements du vieux Louvre, qu'on fit
accommoder et meubler. Mais peu à peu le maréchal de Villars usurpa de
tenir celui de guerre fort souvent chez lui, et à son exemple le
maréchal d'Huxelles, que les autres chefs ne suivirent pas.

Je ne m'arrêterai pas aux prétentions, aux entreprises, aux usurpations,
aux tracasseries du duc de Noailles entre le conseil des finances et les
autres conseils, des conseils des uns aux autres, et des membres de
chacun entre eux, pour lasser et eux et M. le duc d'Orléans, fatiguer le
public, les rende incommodes et ridicules, et les faire tomber dans les
vues qui ont été expliquées\,; cela serait trop long et ennuyeux. Mais
il faut parler du général.

M. le Duc, M. le duc du Maine et le comte de Toulouse ne voulurent point
d'appointements. Le chancelier, le maréchal de Villeroy, Torcy, La
Vrillière, Pontchartrain, conservèrent les leurs sans innovation, et on
ne donna rien au cardinal de Noailles, au procureur général ni à
l'avocat général. Harcourt, Besons, l'évêque de Troyes et moi, pour la
régence, les chefs des conseils, les ducs de Noailles, de Guiche et le
maréchal d'Estrées, eûmes vingt mille livres d'appointements, et les
membres des conseils dix mille livres, les secrétaires six mille livres.

Il fut réglé que les conseils tiendraient aussi souvent qu'il serait
nécessaire, à la discrétion des chefs, et que les chefs auraient chacun
un jour de chaque semaine, ou davantage quand il serait nécessaire, pour
venir rapporter les affaires de son conseil en celui de régence, où il
ne rapporterait pas son avis particulier, mais celui de la pluralité des
voix de chaque délibération de son conseil, et leurs jours aussi pour
travailler seuls avec le régent. Il fut décidé que les chefs ou
présidents des conseils ne seraient dans le conseil de régence que pour
les affaires de leurs conseils, et qu'ils en sortiraient dès qu'elles
seraient finies, où ils auraient leurs voix, quoique le conseil ne levât
pas, et qu'ils couperaient les membres de la régence, quant à la séance,
suivant leur rang entre eux\,; mais qu'ils s'y mettraient en la dernière
place, s'ils n'étaient point ducs ou officiers de la couronne\,; et à
l'égard de l'opinion, qu'en quelque place qu'ils fussent ils opineraient
les premiers de tous à la suite de leur rapport. Les dues, comme
partout, eurent la préséance, et les officiers de la couronne après eux,
les uns et les autres suivant leur ancienneté de dignité\,; et entre les
ducs, que la pairie y aurait la préséance, parce que cette séance tenait
plus des fonctions d'État et de la couronne que des cérémonies de cour.

Ils ne disputèrent pas, pour ne rien innover, la préséance usurpée du
chancelier au conseil, en sorte que Voysin y fut toujours au-dessous
immédiatement, et sans intervalle, du duc du Maine d'un côté, et moi
pareillement de l'autre du comte de Toulouse. Chacun était ainsi par
rang, à droite ou à gauche, et on opinait comme on était assis, le
dernier du conseil opinant après le rapporteur, et tous les autres l'un
après l'autre, en remontant, et M. le duc d'Orléans le dernier.

Les sièges furent égaux pour tout le monde dans tous les conseils. Celui
de régence n'eut que des ployants, le régent comme les autres, parce que
le roi était censé y être, et que son fauteuil vide était au bout de la
table longue, seul. Le régent à droite, en retour à la première place,
M. le Duc vis-à-vis de lui. Au bas bout, vis-à-vis le fauteuil du roi,
étaient Pontchartrain et La Vrillière.

Aucun de tous les conseils ne prêta de serment, sur le fondement que les
ministres d'État n'en prêtaient point, et aucun de ceux du conseil de
régence n'eut de patente ni de lettre du roi ou du régent pour y entrer,
parce que les ministres d'État n'en ont point. Mais comme ils ne se
peuvent présenter au conseil qu'ils ne soient avertis à chaque fois d'y
venir de la part du roi, par l'huissier de son cabinet, les membres de
la régence le furent ainsi la première fois\,; et au premier conseil de
régence, M. le duc d'Orléans intima celui d'après, et ainsi de l'un à
l'autre, et on n'avertit plus j parce qu'il y aurait eu trop à courir,
sinon pour des conseils extraordinaires et imprévus auxquels on ne
pouvait s'attendre.

Le régent arrivé, on n'attendait personne sans exception\,; si on
arrivait le conseil commencé, ce qui était rare, on entrait et on
s'approchait de la table derrière\,; le régent vous disait de prendre
place, qui dans ces cas était laissée vide, et on la prenait avec un mot
d'excuse.

Aucun conseil ne s'était encore assemblé qu'il y eut une rare difficulté
pour celui des finances, tant les prétentions, pour ridicules qu'elles
puissent être, prennent de force du mépris qu'on en fait, quand on se
contente du mépris, sans les proscrire, comme fit le roi, qui se
contenta de se moquer de la chimère des conseillers d'État, mise pour la
première fois en avant, de ne céder qu'aux gens titrés, lors de la
signature du traité de Bade, et de châtier La Houssaye, nommé troisième
ambassadeur, avec le maréchal de Villars\,: et le comte du Luc, en y
envoyant Saint-Contest au lieu de lui.

Sur ce bel exemple, qui n'en fut jamais un, mais une dérision, comme le
roi s'en expliqua alors, les conseillers d'État qui étaient du conseil
des finances, et il n'y en avait point dans les autres conseils,
prétendirent y précéder le marquis d'Effiat, qui était de leur étoffe à
la vérité, mais dont le grand-père était mort chevalier de l'ordre,
ambassadeur, surintendant des finances, et par commission de
l'artillerie, et maréchal de France. Il était fils du frère aîné de
Cinq-Mars, grand écuyer de France, et lui-même était chevalier de
l'ordre de la promotion de 1688. Ces messieurs alléguaient qu'aux
conseils de Charles IX et d'Henri III, et aux états généraux du règne de
ce dernier roi, les conseillers d'État de robe avaient eu la droite sur
ceux d'épée qui n'étaient pas ducs ou officiers de la couronne\,; et ils
disaient vrai. Mais ils se gardaient bien d'ajouter que c'était une
innovation jusqu'alors inouïe et abrogée par Henri IV, et qui n'a jamais
eu lieu depuis, innovation faite par les Guise dans le même esprit qui
les engagea à faire établir les charges de l'ordre du Saint-Esprit comme
elles le furent, pour favoriser et s'attacher la bourgeoisie qu'ils
avaient séduite, ainsi que le clergé, et abattre, en tout ce qu'ils
purent, la noblesse qu'ils craignaient et qu'ils haïssaient, comme étant
trop attachée au roi et à la couronne, ainsi qu'il y a bien paru par
tout le secours qu'en reçut Henri IV, qui lui affermit la couronne sur
la tête et qui l'arracha à ces perfides étrangers.

J'arrivai une après-dînée chez le régent, comme il se promenait dans sa
grande galerie, entre Canillac et le duc de Noailles, qui discutaient
cette belle difficulté de préséance. C'étaient les deux champions de ce
qu'ils avaient appelé la noblesse à l'occasion de l'insigne calomnie du
duc de Noailles contre moi. Ma surprise fut donc extrême lorsque,
m'étant joint à cette promenade, je les entendis tous deux plaider avec
chaleur la cause des conseillers d'État contre les gens de qualité non
titrés.

Après les avoir écoutés quelque temps, le régent me demanda ce que je
disais à cela. Je souris, et répondis que je ne me serait pas attendu à
la prétention, moins encore aux avocats que je venais d'entendre. Je
remis le fait des Guise que je viens de rapporter, celui du comte du
Luc, et je suppliai le régent de se souvenir comment le feu roi et
l'universalité du monde avaient pris cette prétention des conseillers
d'État. De là je vins au fond de la chose, et je dis qu'en France il n'y
avait que trois états\,; que tous les trois avaient toujours été
précédés par les pairs, les ducs et les officiers de la couronne sans
nulle difficulté partout, et qui aux états généraux étaient avec le roi
sur le théâtre\,; et en bas les trois états\,; qu'entre personnes de
même état il se pouvait qu'il y eût des prétentions de préséance, mais
que d'état à état il n'y en eut jamais en aucun temps\,; que l'église et
la noblesse, la première à droite, l'autre à gauche, étaient assis et
couverts, et parlaient en cette sorte en égalité parfaite de l'une à
l'autre\,; qu'au fond de la salle, vis-à-vis du théâtre, était le tiers
état, assis, mais découvert, et qui pour parler se mettait à genoux\,;
posture qui en est restée à tout le parlement, et au premier président
comme aux autres membres, parlant aux lits de justice, parce que tout
magistrat, quel qu'il soit de naissance, est du tiers état par sa
magistrature\,; que les conseillers d'État étaient de robe et
magistrats, par conséquent aussi du tiers état, d'où il résultait
qu'entre conseillers d'un même conseil, le tiers état devait céder aux
deux premiers\,; d'où il était clair que la prétention des conseillers
d'État de robe était sans aucun fondement contre le marquis d'Effiat. Ce
raisonnement, auquel Noailles et Canillac ne s'étaient pas apparemment
attendus, leur ferma la bouche, et à M. le duc d'Orléans aussi.

J'ajoutai, après un moment de silence, que je parlais contre mon
intérêt, puisque la prétention que je venais de combattre allait à
mettre un étage de gens dans la personne des conseillers d'État de robe,
entre les ducs et officiers de la couronne et les gens de qualité, mais
que la vérité devait toujours être la plus forte, et que je ne
comprenais pas la patience de Son Altesse Royale de souffrir des
disputes aussi ineptes, et dont la tolérance et le délai à les finir,
comme elles le doivent être, donnerait lieu à cent autres, dont
l'impertinence ferait honte et troublerait tout. Noailles et Canillac
n'osèrent en attendre davantage, ne répondirent pas un mot, et s'en
allèrent.

Le rare est que les gens de qualité ignorèrent leur conduite à cet
égard, ou la voulurent ignorer ainsi que la mienne, et que la robe leur
sut et à moi tout le divers gré que nous méritâmes d'elle là-dessus.

Resté seul avec le régent, je le pressai de décider. Ces deux hommes qui
avaient peur de tout, et lui aussi, l'avaient effarouché sur la robe. Il
me proposa l'expédient de faire d'Effiat vice-président pour précéder à
ce titre. Je lui représentai, en général, les inconvénients des
\emph{mezzo termine}, qui sont les pères des plus folles prétentions et
qui ne sont jamais qu'en faveur de ceux qui ont tort et contre ceux qui
ne peuvent perdre en jugement définitif, et en particulier, l'indécence
et le danger de tolérer une prétention absurde, dont le succès en ferait
naître de toutes les couleurs. Je le laissai dans sa bonne amie
l'irrésolution et l'indécision, après avoir parlé d'autres affaires.

Deux jours après, qui se passèrent en ridicules négociations, les
conseillers d'État, qui ne demandaient pas mieux que d'en sortir avec un
titre qui réalisât leur prétention, eurent la bonté de consentir de
céder au titre de vice-président\,; ce qui était s'assurer la préséance
sur tout autre homme de qualité qui pourrait entrer au conseil de
finances, etc. Le régent reçut cette complaisance avec gratitude, et
d'Effiat fut déclaré vice-président.

Ce que j'avais prédit au régent arriva, et il vaut mieux le raconter
tout de suite que d'en interrompre des matières plus importantes. Il fut
réglé que les procès évoqués au roi, qui se voient dans un bureau du
conseil des parties, les affaires des prises qui se voient au conseil
des prises, et maintenant de marine, quelques-unes de finances qui
étaient contentieuses ou qui demandaient un règlement, toutes choses
usitées sous le feu roi, se rapporteraient comme de son temps, devant
lui, c'est-à-dire alors au conseil de régence, à quoi on ajouta
certaines affaires du conseil de guerre, comme étapes, etc., et autres
genres de règlements concernant les troupes.

Sous le feu roi, le bureau du conseil des parties, qui avait vu une
affaire évoquée devant lui, entrait tout entier au conseil où étaient le
roi et ses ministres, et le maître des requêtes, qui avait rapporté
l'affaire au bureau du conseil des parties, la rapportait devant le roi.
Les conseillers d'État de ce bureau opinaient tous quatre ou cinq après
lui, puis les ministres, et le roi jugeait en se rendant toujours ou
presque toujours à la pluralité des voix. Pour les affaires des prises,
il y avait sous le feu roi un conseil des prises, composé de quelques
conseillers d'État, qui se tenait chez M. le comte de Toulouse quand il
y avait matière, lequel entrait après au conseil du roi seul, avec le
maître des requêtes qui avait rapporté chez lui, et qui rapportait
devant le roi et ses ministres, le comte de Toulouse présent et opinant,
et se retirant avec le rapporteur dès que l'affaire était jugée. À
l'égard de celles de finances dont on vient de parler, le contrôleur
général en chargeait un maître des requêtes à son choix, qui entrait
seul au conseil du roi un jour de conseil des finances, et qui
rapportait l'affaire. Dans tous ces conseils, tout ce qui y entrait y
était assis, excepté le maître des requêtes rapporteur qui rapportait
debout. Il fut donc réglé que cela se passerait de même à la régence, et
qu'à l'égard des affaires du détail de la guerre, dont on vient de
parler, elles seraient rapportées au conseil de régence par l'un des
deux maîtres des requêtes de ce conseil, Le Blanc et Saint-Contest.

Pour ne rien laisser en arrière sur les conseils du feu roi, il faut
ajouter que le seul conseil des dépêches était tout différent des
autres. La matière en était les disputes ou les règlements à faire dans
les provinces et dans les villes, qui était proprement celle des
départements des provinces des secrétaires d'État, qui, étant bien aises
de s'en rendre les maîtres, en disaient un mot le matin au roi à l'issue
de son lever, puis expédiaient comme ils voulaient\,; ce qui rendait ces
conseils plus rares, sous prétexte de soulager le roi. Mais il y avait
aussi telle nature de ces affaires, ou telles personnes qui s'y
trouvaient intéressées, que les secrétaires d'État ne pouvaient crosser
de la sorte, et qui se rapportaient au conseil des dépêches. Il y avait
aussi des natures d'affaires contentieuses qui s'y rapportaient aussi
par le secrétaire d'État du département duquel elle venait, ou, si elle
n'était d'aucun plus que d'un autre, par un des secrétaires d'État nommé
pour cela par le roi, très rarement par un maître des requêtes nommé par
le chancelier, lequel seul d'extraordinaire entrait un jour de conseil
de dépêches\,; et il y en avait un de règle tous les quinze jours. En
ces conseils des dépêches, il n'y avait d'assis que les fils de France,
le chancelier et le duc de Beauvilliers. Les quatre secrétaires d'État y
demeuraient toujours debout, même M. de Croissy, tout goutteux et tout
président à mortier au parlement de Paris qu'il était, et ils y
rapportaient tout de suite chacun leurs affaires, suivant entre eux leur
ancienneté de secrétaires d'État. S'il y avait un maître des requêtes
rapporteur, les quatre secrétaires d'État y demeuraient également
debout, et y opinaient. Le contrôleur général n'y entrait point s'il
n'était aussi secrétaire d'État, et alors debout comme ses trois autres
confrères. Ce conseil des dépêches devint proprement celui des affaires
du dedans du royaume, que d'Antin duc et pair venait seul rapporter, ou,
si c'était un procès évoqué, un maître des requêtes de ce conseil qui
l'y avait rapporté\,; ainsi la forme unique de ce conseil des dépêches
ne put avoir lieu depuis l'établissement du conseil de régence et des
autres conseils.

On fut bien étonné la première fois qu'un maître des requêtes eut à
rapporter au conseil de régence, qu'il déclara au chancelier qu'il
prétendait rapporter assis, ou que tout ce qui n'était ni duc, ni
officier de la couronne ou conseiller d'État, se tint debout tant qu'il
serait lui-même debout. Ce fut une suite de la mollesse du régent dans
la prétention des conseillers d'État de précéder Effiat. On se récria\,;
on hua\,; mais il n'en fut autre chose\,; le régent n'eut pas la force
de commander. On eut recours aux conseillers du parlement qui étaient
dans les conseils\,; ils répondirent qu'ils ne prétendaient pas moins
que les maîtres des requêtes. On fut donc réduit à faire tout rapporter
par les chefs ou les présidents des conseils, qui, excepté d'Antin, qui
y excella, n'y étaient pas propres. Je raconterai là-dessus deux
aventures qui montreront combien les affaires en souffrirent.

Le maréchal de Villars, qui griffonnait à ne pouvoir être lu de
personne, vint au conseil de régence avec un règlement de quarante ou
cinquante articles que le conseil de guerre avait fait sur les étapes,
les magasins, la marche des troupes par le royaume, et divers détails
qui les concernaient. Il en fit la lecture par articles, sur chacun
desquels on opina à mesure qu'il les lisait, et on fit divers
changements à plusieurs qu'il écrivit aussi à mesure à la marge. Quand
tout fut achevé, M. le duc d'Orléans dit au maréchal de Villars de
relire le tout par article, avec chacun la note qu'il y venait de
mettre, pour qu'on vit si tout était bien, et s'il n'y avait plus rien à
changer ou à y ajouter. Le maréchal, qui était auprès de moi, prit donc
son papier, lut un article, mais quand ce fut à la note, le voilà à
regarder de près, à se tourner au jour d'un côté, puis de l'autre, enfin
à me prier de voir si je pourrais la lire. Je me mis à rire, et à lui
demander s'il croyait que j'en pusse venir à bout quand lui-même ne
pouvait lire sa propre écriture, et qu'il venait d'écrire tout
présentement. Tout le monde en rit sans qu'il en fût le moins du monde
embarrassé. Il proposa de faire entrer son secrétaire qui était,
disait-il, dans l'antichambre, et qui savait lire son écriture, parce
qu'il y était accoutumé. Le régent dit que cela ne se pouvait pas, et
chacun se regarda en riant, sans savoir par où on en sortirait. À la fin
le régent dit qu'il n'y avait qu'à recommencer, comme si on n'avait rien
fait, et m'ordonna de prendre la plume pour écrire les notes à mesure
qu'on opinerait de nouveau sur chaque article, ce qui doubla la longueur
de cette affaire. Il est vrai que ce ne fut que du temps ridiculement
perdu. Mais l'inconvénient était bien plus fâcheux quand, par de mauvais
rapports d'affaires longues et embarrassées, on n'était pas mis en état
de les bien entendre, par conséquent de les bien décider.

L'autre histoire y a plus de rapport, et la voici\,: le maréchal
d'Estrées rapportait au conseil de régence tout ce qui y passait du
conseil de marine, et La Vrillière le comparait plaisamment, mais trop
justement, à une bouteille d'encre fort pleine, qu'on verse tout à coup,
et qui tantôt ne fait que dégoutter, tantôt ne jette rien, tantôt vomit
des flaques et de gros bourbillons épais. Comme il commençait un jour le
rapport d'une affaire de prise fort embarrassée, le comte de Toulouse
qui s'était fort appliqué aux affaires de sa charge, et dont l'esprit
était juste, exact, concis, et lui-même fort judicieux, me dit que je
n'entendrais rien au rapport du maréchal d'Estrées, que cependant
l'affaire était importante, et méritait d'être bien entendue, et qu'il
me l'allait rapporter à l'oreille tandis que le maréchal parlerait. Je
l'entendis donc assez clairement pour être en connaissance de cause de
l'avis du comte de Toulouse, mais non avec assez d'instruction pour bien
appuyer mon opinion, d'autant que le comte de Toulouse me parlait
encore, lorsque ce fut à mon autre voisin à opiner. Quand ce fut à moi
je dis au régent que M. le comte de Toulouse me venait d'expliquer si
clairement l'affaire tandis qu'on la rapportait, que je l'entendais
assez distinctement pour être de l'avis dont serait M. le comte de
Toulouse, mais non assez pour m'en assez expliquer. Le régent se mit à
rire, et à dire qu'on n'avait jamais opiné de la sorte\,; je répondis,
en riant aussi, que s'il ne voulait pas prendre mon avis ainsi, il eût
la bonté de compter pour deux celui de M. le comte de Toulouse, et la
chose passa ainsi. On sut bientôt quel il était, car il n'y avait jamais
que le chancelier à opiner entre lui et moi.

Je pris cette occasion le lendemain pour remontrer à M. le duc d'Orléans
le préjudice essentiel qui arrivait aux affaires de l'opiniâtreté des
maîtres de requêtes, et de sa mollesse à la souffrir. Je n'y gagnai
rien.

Je crois que le chancelier soutenait sourdement cette prétention par
malice, et ce qui m'en persuada mieux, c'est que dès qu'il fut mort, et
que d'Aguesseau fut chancelier, tout idolâtre qu'il fût de la robe, il
la fit cesser, et les maîtres des requêtes vinrent rapporter debout tout
ce qu'on voulut au conseil de régence, sans plus parler d'y être assis
ni d'y faire lever personne. Mais à l'égard des conseillers d'État,
lorsque pour un procès évoqué devant le roi, c'est-à-dire au conseil de
régence, le bureau du conseil des parties, qui avait vu l'affaire,
venait au conseil de régence avec le rapporteur. Ces conseillers d'État
s'y mettaient après les maréchaux de France, et au-dessus des autres de
la régence, le rapporteur maître des requêtes rapportant debout.

\hypertarget{chapitre-ix.}{%
\chapter{CHAPITRE IX.}\label{chapitre-ix.}}

1715

~

{\textsc{Éclat des princes du sang sur la qualité de prince du sang
prise par le duc du Maine avec eux.}} {\textsc{- Protestation de MM. de
Courtenay pour la conservation de leur état et droits, présentée au
régent.}} {\textsc{- Malheur et extinction de cette branche de la maison
royale.}} {\textsc{- Béthune épouse la fille du duc de Tresmes.}}
{\textsc{- Nangis obtient de vendre le régiment d'infanterie du roi.}}
{\textsc{- Poirier premier médecin du roi.}} {\textsc{-
M\textsuperscript{me} la duchesse de Berry logée à Luxembourg avec sa
cour, où M\textsuperscript{me} de Saint-Simon et moi ne voulûmes point
habiter.}} {\textsc{- Villequier obtient les survivances du duc
d'Aumont, son père.}} {\textsc{- Deux nouveaux premiers valets de
chambre.}} {\textsc{- Le cardinal de Polignac vend sa charge de maître
de la chapelle à l'abbé de Breteuil, depuis évêque de Rennes\,; et le
baron de Breteuil la sienne d'introducteur des ambassadeurs, à Magny.}}
{\textsc{- Le marquis de Simiane lieutenant général de Provence\,; et
Fervaques gouverneur du Perche et du Maine, sur la démission de Bullion,
son père.}} {\textsc{- Le prince Charles de Lorraine obtient un million
de brevet de retenue sur sa charge de grand écuyer, et peu après la
survivance du gouvernement de Picardie du duc d'Elboeuf.}} {\textsc{-
J'eus aussi la survivance de mes deux gouvernements pour mes deux fils,
et l'abbaye de Jumièges pour l'abbé de Saint-Simon.}} {\textsc{-
Réflexion sur les coadjutereries régulières.}} {\textsc{- Grand et fort
étrange présent du régent au duc de La Rochefoucauld.}} {\textsc{-
Dépouille de l'appartement du feu roi au duc de Tresmes.}} {\textsc{-
Noailles et Rouillé maîtres des finances, dont le conseil prend forme,
et les autres conseils aussi.}} {\textsc{- Premier conseil de régence.}}
{\textsc{- Je me raccommode avec le maréchal de Villeroy.}} {\textsc{-
Placets dits à l'ordinaire.}} {\textsc{- Tentative échouée de Besons,
qui s'éloigne de moi de plus en plus.}} {\textsc{- Amelot arrive de
Rome, qui me conte un rare entretien entre le pape et lui sur la
constitution.}} {\textsc{- Amelot exclu de tout, et pourquoi\,; mis
enfin à la tête d'un conseil de commerce.}} {\textsc{- Spectacles
recommencés.}} {\textsc{- Don à Canillac.}} {\textsc{- Garde-robe et
cassette du roi.}} {\textsc{- Le grand prieur est rappelé.}} {\textsc{-
Belle-Ile obtient quatre cent mille livres comptant sur les états de
Bretagne.}} {\textsc{- Quel fut Belle-Ile.}} {\textsc{- Sa famille.}}
{\textsc{- Quels sont les Castille, dits Jeannin.}} {\textsc{- Caractère
des deux frères Belle-Ile.}}

~

À peine M. le duc d'Orléans fut-il sorti de l'embarras, où il s'était
bien voulu laisser mettre, de la prétention des conseillers d'État, par
la vice-présidence d'Effiat, qu'il s'en éleva un autre d'une autre
importance. Je ne ferai ici qu'en marquer l'époque, parce que les suites
n'en sont pas de ce moment-ci. Le procès de la succession de M. le
Prince allait son train. Dans une signification que M. le duc du Maine y
fit, il prit la qualité de prince du sang, comme autorisé par la
déclaration du feu roi enregistrée au parlement, qui la lui donnait, et
lui permettait de la prendre en tous actes et partout, tant à lui et à
ses enfants qu'au comte de Toulouse. M\textsuperscript{me} la Duchesse
et M. le Duc, qui n'avaient osé souffler sous le feu roi, firent grand
bruit et prétendirent que, quelque protection que le duc du Maine
prétendît tirer de cette déclaration, elle ne lui donnait pas droit de
se qualifier prince du sang avec les princes du sang véritables, ni dans
les significations juridiques dans un procès avec eux. Ils attirèrent
M\textsuperscript{me} la princesse de Conti et M. son fils dans cet
intérêt commun de princes du sang, quoique unis avec M. et
M\textsuperscript{me} du Maine par communauté d'intérêt dans le procès
contre M. le Duc pour la succession de M. le Prince. L'éclat fut grand,
le régent chercha à l'apaiser. On en verra ailleurs les suites.

Le prince de Courtenay, l'abbé son frère, et le fils unique du premier
auxquels cette branche se trouvait réduite, présentèrent au régent une
parfaitement belle protestation, forte, prouvée, mais respectueuse et
bien écrite, pour la conservation de leur état et droits, comme ils ont
toujours fait aux occasions qui s'en sont présentées, et à chaque
renouvellement de règne. Elle fut reçue poliment et n'eut pas plus de
succès que toutes les précédentes. L'injustice constante faite à cette
branche de la maison royale légitimement issue du roi Louis le Gros est
une chose qui a dû surprendre tous les temps qu'elle a duré, et montrer
en même temps la funeste merveille de cette maison, qui dans un si long
espace n'a pu produire un seul sujet dont le mérite ait forcé la
fortune, d'autant plus que nos rois ni personne n'a jamais douté de la
vérité de sa royale et légitime extraction, et le feu roi lui-même. J'en
ai parlé t. Ier, p.~113, 114, et t. IX, p.~23.

Ce prince de Courtenay-ci était un homme dont la figure corporelle
marquait bien ce qu'il était. Le cardinal Mazarin eut envie de voir s'il
en pourrait faire quelque chose, et s'il le trouvait un sujet de le
faire reconnaître pour ce qu'il était, en lui donnant une de ses nièces.
Pour l'éprouver à loisir par soi-même, il le mena dans son carrosse de
Paris à Saint-Jean de Luz pour les conférences de la paix des Pyrénées.
Le voyage était à journées, et il fut plein de séjours. Courtenay était
né en mai 1640\,; il avait donc près de vingt ans. Il n'eut ni l'esprit
ni le sens de cultiver une si grande fortune. Il passa tout le voyage
avec les pages du cardinal, qui ne le vit jamais qu'en carrosse, et qui
désespéra d'en pouvoir faire quoi que ce soit. Aussi l'abandonna-t-il en
arrivant à la frontière, où il devint et d'où il revint comme il put. Il
n'a pas laissé de servir volontaire avec valeur en toutes les campagnes
du feu roi, et je l'ai vu souvent à la cour chez M. de La Rochefoucauld
sans qu'il ait jamais été de rien.

Pendant le fort du Mississipi\footnote{{[}17{]}}, le cardinal Dubois se
piqua, je ne sais comment, de le tirer de l'affreuse pauvreté où il
avait vécu, et lui fit donner de quoi payer ses dettes, et vivre fort à
son aise. Il mourut en 1723. Il avait perdu son fils aîné, tué
mousquetaire au siège de Mons que faisait le roi, qui l'alla voir sur
cette perte, ce qui fut extrêmement remarqué, parce qu'il ne faisait
plus depuis longtemps cet honneur h personne, et que M. de Courtenay
n'avait ni distinction ni familiarité auprès de lui.

Son autre fils servit peu, et fut un très pauvre homme, et fort obscur.
Il épousa une soeur de M. de Vertus-Avaugour des bâtards de Bretagne,
revenue de Portugal veuve de Gonzalès-Joseph Carvalho Patalin,
surintendant des bâtiments du roi de Portugal. C'était une femme de
mérite qui n'eut point d'enfants de ses deux maris.

M. de Courtenay vécut très bien avec elle. Il était riche, se portait
bien, et sa tête et son maintien faisaient plus craindre l'imbécillité
que la folie. Cependant un matin étant à Paris, et sa femme à la messe
aux Petits-Jacobins, sur les neuf heures du matin, ses gens accoururent
dans sa chambre au bruit de deux coups de pistolet tirés sans intervalle
qu'il se tira dans son lit, et l'y trouvèrent mort, ayant été encore la
veille fort gai, tout le jour et tout le soir, et sans qu'il eût aucune
cause de chagrin. On étouffa ce malheur qui éteignit enfin la
malheureuse branche légitime de Courtenay, car il n'en resta que le
frère de son père, qui était un prêtre de sainte vie, dans la retraite
et les bonnes œuvres, quoiqu'il sentît fort la grandeur de sa naissance.
Il avait les abbayes des Eschallis et de Saint-Pierre d'Auxerre, et le
prieuré de Choisy en Brie, et mourut dans une grande vieillesse, le
dernier de tous les Courtenay. C'était un grand homme, bien fait, et
dont l'air et les manières sentaient parfaitement ce qu'il était. Il
n'en reste plus que la fille de son frère mariée au marquis de
Bauffremont. L'extinction de cette infortunée branche méritait d'être
marquée, puisque l'occasion s'en est trouvée si naturellement.

Béthune, fils de la soeur de la reine de Pologne, et veuve d'une sœur du
maréchal d'Harcourt, dont il a eu la maréchale de Belle-Ile, se remaria
à la fille du duc de Tresmes, qui en fit la noce chez lui, à Saint-Ouen,
près Paris.

Nangis, mort longtemps depuis maréchal de France et chevalier d'honneur
de la reine, voyant que le régiment du roi ne lui était plus d'aucun
usage depuis la mort du feu roi, qui entrait dans tous les détails de ce
corps, comme on l'a dit ailleurs, demanda la liberté de le vendre. Il ne
s'achetait ni se vendait. Le régent, facile, le lui permit. Il en traita
avec le duc de Richelieu pour trente mille écus. Mais le marché se
rompit, dont on verra la suite.

La charge de premier médecin étant l'unique qui se perde par la mort du
roi, il en fallut choisir un. Chirac, qui avait la première réputation
en ce genre, était au régent, et dès là exclus. Boudin, médecin
ordinaire, et qui avait été premier médecin de Monseigneur, puis de la
dernière Dauphine, y avait plus de droit que personne, et il était porté
par toute l'ancienne cour. Mais c'était un compagnon d'esprit,
d'intrigue, hardi, lié à tout ce qui était le plus opposé à M. le duc
d'Orléans. Il avait de plus crié sans mesure, et sur le ton de
M\textsuperscript{me} de Maintenon et du duc du Maine, sur les poisons,
en sorte qu'il ne fut pas seulement question de lui. Faute de mieux
parmi les médecins de la cour, Poirier fut choisi, parce qu'il avait été
le médecin de Saint-Cyr, et en dernier lieu des enfants de France. Les
amis de Boudin crièrent, et on les laissa crier.

M\textsuperscript{me} la duchesse de Berry vint s'établir à Luxembourg
avec sa petite cour. On y chercha de quoi nous loger commodément,
M\textsuperscript{me} de Saint-Simon et moi\,; mais
M\textsuperscript{me} de Saint-Simon, ne pouvant honnêtement la quitter,
prit cette occasion pour en vivre la plus séparée qu'il lui fut
possible. Il ne se trouva donc rien qui nous pût loger tous deux, et
nous continuâmes de loger à Paris dans notre maison ensemble.
M\textsuperscript{me} la duchesse de Berry voulut pourtant qu'elle prît
un logement à Luxembourg, mais elle ne le meubla point, et n'y mit
jamais le pied. Elle n'alla chez M\textsuperscript{me} la duchesse de
Berry les matins que lorsqu'il y avait des audiences ou quelque
cérémonie, mais presque tous les soirs, à l'heure du jeu public, où les
dames eurent permission d'aller sans être en grand habit, et où
plusieurs étaient retenues à souper avec M\textsuperscript{me} la
duchesse de Berry. M\textsuperscript{me} de Saint-Simon n'y soupait
presque jamais. Nous avions tous les jours du monde à dîner et à souper,
comme nous avions eu toujours, et très rarement aussi la suivait-elle
aux promenades, aux visites, excepté chez le roi, et aux spectacles, et
se tint ferme en cette liberté avec grande et juste raison, mais
toujours traitée avec la plus grande considération. Elle avait toujours
demeuré à Saint-Cloud avec elle, parce il n'y avait pas eu moyen de
faire autrement. Pour moi j'en usai à mon ordinaire. Je n'avais qu'une
ou deux fois l'an chez M\textsuperscript{me} la duchesse de Berry, un
moment chaque fois, toujours très bien reçu\,; on a vu ailleurs les
raisons de cette conduite.

Le duc d'Aumont obtint du régent la survivance de ses charges de premier
gentilhomme de la chambre et de gouverneur de Boulogne et pays
boulonnais pour le marquis de Villequier, son fils unique. Bachelier,
fils de celui dont j'ai parlé, acheta en même temps de Bloin sa charge
de premier valet de chambre, et je fis donner au fils de Bontems la
survivance de la sienne, qui m'en avait prié. Oncques depuis n'ai ouï
parler du père ni du fils. J'ai bien trouvé de leurs semblables.

Le cardinal de Polignac, qui ne se souciait plus, depuis la mort du roi,
de sa charge de maître de la chapelle, obtint permission de la vendre,
et il en eut gros du frère de Breteuil. L'un fut depuis évêque de
Rennes, l'autre secrétaire d'État. Leur oncle, le vieux baron de
Breteuil, vendit aussi sa charge d'introducteur des ambassadeurs à
Magny, fils de Foucauld, conseiller d'État, à qui il avait succédé dans
l'intendance de Caen, où il fit tant de sottises qu'il en fut rappelé à
la fin du dernier règne, après quoi il se défit de sa charge de maître
des requêtes. Il y aura plus d'une occasion de parler de cette bonne
tête.

Simiane, premier gentilhomme de la chambre de M. le duc d'Orléans, eut
la lieutenance générale de Provence, demeurée vacante depuis la mort du
comte de Grignan, chevalier de l'ordre, son beau-père, et Fervaques,
fils de Bullion, eut, sur sa démission, le gouvernement du Perche et du
Maine. C'est ainsi que M. le duc d'Orléans donnait à toutes mains à qui
voulait avoir, et qu'il profita si peu du conseil qu'on a vu que je lui
avais donné là-dessus. M. le Grand, au père duquel la charge de grand
écuyer n'avait coûté que le vol qu'il en fit, comme on l'a vu, à mon
père, fit donner au prince Charles\footnote{{[}18{]}}, son fils, qui en
avait la survivance, un million de brevet de retenue dessus\,; ce qui
était la rendre héréditaire, et {[}ils{]} cajolèrent si bien le duc
d'Elboeuf, qui n'avait point d'enfants, que peu après ils obtinrent pour
le même prince Charles la survivance du gouvernement de Picardie du duc
d'Elbœuf. Jusque-là j'avais eu patience, mais cela me piqua. J'en dis
mon avis à M. le duc d'Orléans, et j'ajoutai que puisqu'il donnait tout
indifféremment à tout le monde, je voulais aussi la survivance de mes
deux gouvernements pour mes deux fils, de Blaye pour l'aîné, de Senlis
pour le cadet, qu'il me donna sur-le-champ. Torcy donna la démission de
sa charge de secrétaire d'État qui fut supprimée, comme celle qu'avait
Voysin, et prêta serment entre les mains du roi de sa nouvelle charge de
grand maître des postes.

J'avais représenté à M. le duc d'Orléans la triste situation de la
branche aînée de ma maison, et je l'avais supplié de donner au jeune
abbé de Saint-Simon, qui avait près de vingt ans, une abbaye dont il put
aider ses frères, parce que je n'aime pas la pluralité des bénéfices. Il
lui donna Jumièges, en même temps qu'Anchin au cardinal de Polignac, et
Saint-Waast d'Arras au cardinal de Rohan. Mais il souffrit qu'ils
eussent des coadjuteurs religieux de ces abbayes, qui, étant régulières,
pouvaient être possédées en commende par des cardinaux, dont un des
principaux privilèges est de pouvoir tout engloutir. Mais les moines
surent si bien représenter à Rome la lésion de leur droit de s'élire des
abbés réguliers par la nomination successive de cardinaux à leurs
abbayes, que le pape insista pour ces coadjutoreries, et que le régent
eut la faiblesse d'y consentir. Je dis la faiblesse, parce que jamais
Rome ne se serait opiniâtrée à une chose de cette qualité, et que,
puisqu'on a le peu de sens de vouloir des cardinaux en France, et la
manie de se persuader qu'il leur faut cent mille écus de rente à chacun,
il vaut mieux les prendre sur de riches abbayes régulières qu'autres que
des cardinaux ne peuvent posséder, que laisser cent mille livres de
rente à un abbé moine, et donner aux cardinaux de grosses abbayes
qu'autres qu'eux pourraient posséder.

M. le duc d'Orléans fit un prodigieux présent au duc de La
Rochefoucauld, qui n'avait jamais marqué que de l'éloignement pour lui,
et qui n'en montra pas moins après. Ce fut de toutes les pierreries de
la garde-robe qui n'étaient pas de la couronne. Ce don monta fort haut
et reçut peu l'approbation du public. M. de La Rochefoucauld n'avait
droit que sur les habits, étoffes et autres choses pareilles de la
garde-robe, et aucun sur pas une des pierreries, qui devaient demeurer
au roi. Il était d'ailleurs extrêmement riche. Le duc de Tresmes,
premier gentilhomme de la chambre en année, quand le roi mourut, eut
gros aussi, parce que l'ameublement dans lequel le roi mourut était fort
beau, mais M. de Tresmes n'eut que ce qui appartenait de droit et
d'usage à sa charge.

Le conseil de finances commença à prendre forme. M. le duc d'Orléans y
assista quelquefois, mais rarement\,; le maréchal de Villeroy presque
jamais. Toute l'autorité en fut dévolue au duc de Noailles, qui prit
Rouillé du Coudray pour son mentor, et qui fit tout dans ce conseil avec
sa férocité accoutumée, qui n'était plus contrainte comme lorsqu'il
n'était que directeur des finances avec Armenonville sous Chamillart. Sa
débauche, bien plus cachée alors, n'eut plus de frein ni de secret, et
le duc de Noailles toujours réglé sur le ton du maître\,; et qui depuis
son retour d'Espagne avait été dévot jusqu'à la mort du roi, prit en ce
temps-ci et entretint publiquement une fille de l'Opéra. Fagon fut fait
conseiller d'État surnuméraire, sur l'exemple de ce même Rouillé qui
était unique, et que le roi avait fait ainsi, lorsqu'il supprima les
deux directeurs des finances, après que Desmarets fut contrôleur
général. Des Forts et Fagon eurent les mêmes départements qu'ils avaient
étant intendants des finances\,; Ormesson, Gilbert, Gaumont, Baudry et
Dodun eurent les autres départements. On en garda un pour La Houssaye
qu'on fit revenir de Strasbourg, où on envoya Angervilliers intendant à
sa place, qui l'était de Dauphiné. Les quatre premiers étaient maîtres
des requêtes et devinrent conseiller, d'État. Dodun était président
d'une chambre des enquêtes, qui vendit sa charge. Nous verrons enfin La
Houssaye et lui successivement contrôleurs généraux. Rouillé eut cent
quatre-vingt mille livres d'appointements, et régenta ouvertement les
finances. Il devint à la mode d'admirer ses brutalités et ses débauches.
Les conseils de guerre et de marine furent aussi partagés en
départements, et en différents détails entre les membres de ces
conseils. M. le duc d'Orléans alla quelquefois aussi au conseil de
guerre, mais fort rarement. Il travailla particulièrement aux finances
et aux affaires étrangères. Il entendait très bien ces dernières et se
piquait de capacité en finance.

Le lundi 28 septembre, après dîner, se tint à Vincennes, dans le grand
cabinet du roi, le premier conseil de régence, auquel pour cette fois
les chefs et présidents des autres conseils furent admis, excepté le
cardinal de Noailles, à cause de sa prétention de préséance. Il y fut
réglé qu'il y en aurait quatre par semaine, savoir\,: le samedi après
dîner, le dimanche matin, le mardi après dîner, et le mercredi matin\,;
qu'on se tiendrait averti une fois pour toutes de ces quatre conseils\,;
mais qu'on le serait des extraordinaires, outre ceux-ci, si le régent en
assemblait. Il fut réglé aussi quels jours chaque chef ou président du
conseil viendrait y rapporter les affaires de son conseil\,; qu'il
sortirait lorsqu'elles seraient finies, quoique le conseil ne le fût
pas\,; que tous les chef et présidents des conseils y seraient mandés
quelquefois pour des affaires extraordinaires, lorsque le régent le
jugerait à propos. Ce premier conseil se passa en ballottages\,; ce ne
fut que le suivant qui commença en être un sérieux, qui ne fut que
d'affaires d'État.

En ce premier, comme on fut sur le point de se mettre en place, le
maréchal de Villeroy, à qui je ne parlais point, et que je saluais fort
médiocrement depuis l'affaire du duc d'Estrées et du comte d'Harcourt
dont j'ai parlé en son temps, vint à moi me dire qu'étant ministre
d'État sous le feu roi, et moi ne faisant qu'entrer ce jour-là dans le
conseil, il pourrait être fondé à me disputer la préséance, mais qu'il
ne voulait point former de difficulté. Je lui répondis crûment et
nettement que je le précéderais au conseil, comme je le précédais
partout ailleurs\,; puis, me radoucissant, j'ajoutai qu'il savait trop
ce qu'il se devait à lui-même et à sa dignité permanente pour en faire
la moindre difficulté. Que c'était aussi par cette même raison que je
conservais ce qui m'était dû, honteux d'ailleurs de précéder un homme de
son âge et de son mérite. Cela fut bien reçu, et les compliments
finirent par nous mettre en place.

Pendant le conseil, je songeai que {[}vu{]} la considération où les
emplois du maréchal de Villeroy le mettaient, je pouvais, après ce qui
venait de se passer entre nous, finir galamment une vieille brouillerie
qui n'avait rien de personnel, et où ses prétentions avaient eu
pleinement le dessous, qu'il se présenterait des affaires que nous
aurions à traiter ensemble, outre la fréquence des conseils de régence
où nous nous trouverions tous deux\,; et que ce serait même ôter à M. le
duc d'Orléans une brassière qui, fait comme il était, l'importunerait.
Je m'amusai donc assez exprès après le conseil des finances pour laisser
retourner le maréchal de Villeroy dans sa chambre, car il logeait à
Vincennes depuis que le roi y était, et j'allai lui faire une visite.
Cet homme, également fastueux et bas, fut bien surpris de me voir entrer
dans sa chambre. Il se peignit sur son visage une joie singulière. Les
compliments de part et d'autre furent merveilleux, et nous nous
séparâmes les meilleurs amis du monde. Le lendemain au conseil il m'en
fit encore quantité, et il chercha depuis à me parler d'affaires, et
même fort librement, et à avoir liaison avec moi. Je dis à M. le duc
d'Orléans le lendemain matin la visite que j'avais faite la veille. Il
en fut aise jusqu'à m'en remercier.

Il régla le même jour que les placets du commun, dits \emph{à
l'ordinaire}, que du temps du roi chacun qui voulait venait jeter deux
fois la semaine sur une table dressée pour cela dans l'antichambre où le
roi soupait, s'y jetteraient les mêmes jours et de la même manière\,;
mais qu'au lieu du secrétaire d'État de la guerre qui s'y trouvait
debout derrière le fauteuil vide qui était contre cette table, et qui
emportait tous ces placets chez lui pour en rendre compte au roi, ce
serait un des membres de la régence qui y ferait la même fonction\,;
qu'il y aurait deux maîtres des requêtes qui emporteraient les placets,
qui viendraient les rapporter chez lui, après quoi il les viendrait
rapporter au Palais-Royal au régent seul, accompagné des deux mêmes
maîtres des requêtes avec qui il en aurait fait les envois et les
triages, pour ne rapporter au régent que ceux en petite quantité qui
paraîtraient le mériter. Le régent régla aussi que les derniers de la
régence commenceraient les premiers en remontant jusqu'au chancelier
exclusivement, et non plus, puis coulés à fond recommenceraient, et que
chacun ferait cette fonction pendant un mois de suite.

Les membres du conseil de régence n'avaient point de département, parce
que tout se portait devant eux. J'appris que le maréchal de Besons s'en
voulut faire un de ces placets, et qu'il avait demandé de les recevoir
toujours. Cette impudence me choqua\,; j'en parlai vivement au régent,
qui était déjà ébranlé, et à qui je fis sentir la conséquence d'un
ministère direct et continuel qui embrasserait bientôt autre chose que
ces placets du commun, et qui se rendrait bientôt maître dans une
matière qu'il lui serait aisé d'étendre. J'ajoutai qu'un homme de sa
sorte se méconnaissait étrangement de n'être pas content d'être du
conseil de régence, et de ne vouloir pas en partager les fonctions avec
des gens en tout genre si supérieurs à lui. Le maréchal échoua
vilainement dans ce projet, avec la honte qu'il ne fut pas ignoré. Il
n'ignora pas aussi que c'était à moi qu'il devait ce mauvais succès.

La liaison entre lui et moi n'avait pas pris après l'éloignement de
M\textsuperscript{me} d'Argenton. C'était un homme entre deux terres qui
craignait le grand jour. D'Effiat, à qui il s'était livré depuis,
l'avait aussi éloigné de moi, quoiqu'il ne me connût point, mais il
voulait gouverner son maître, et le mener noyer à son plaisir sans
obstacle, et j'en étais un grand à ses desseins dictés par le duc du
Maine, auquel il était vendu de longue main, lequel sûrement ne lui
avait pas inspiré d'affection pour moi. La partialité encore pour et
contre Pontchartrain formait une autre sorte d'éloignement. Cette
dernière affaire l'acheva, en sorte qu'il n'y eut plus de commerce que
de la plus simple civilité entre Besons et moi, que déjà je ne voyais
plus guère depuis longtemps. Ce fut pour moi une perte des plus légères,
d'autant même que son frère l'archevêque et moi demeurâmes comme nous
avions toujours été. J'eus loisir de voir comme les autres faisaient
pour ces placets, parce que je fus le dernier qui les reçus.

M. Amelot arriva de Rome sans avoir pu obtenir le concile national, ni
aucune chose raisonnable de cette cour, ou le nonce Bentivoglio, les
cardinaux de Rohan et de Bissy, les jésuites et maints autres ambitieux
et brouillons soufflaient sans cesse le feu. Quelque temps après son
retour, Amelot me vint voir, et nous parlâmes beaucoup de Rome. Il me
conta un fait bien remarquable, et qui mérite placé ici.

Il me dit que le pape l'avait pris en goût, et lui parlait souvent avec
confiance, gémissant d'être en brassière, et de ne pouvoir ce qu'il
voudrait. Dans une de ces conversations, le pape se répandit avec lui en
regrets de s'être laissé aller à donner sa constitution, que les lettres
du roi lui avaient arrachée, dans la persuasion où elles l'avaient mis,
et toutes celles au P. Tellier, que le roi était si absolu en France, et
tellement maître des évêques, du reste du clergé, et des parlements, que
sa bulle serait reçue de tous unanimement, enregistrée et publiée
partout sans la moindre difficulté\,; et que s'il eût pu penser en
trouver la centième partie de ce qu'il en rencontrait, jamais il ne
l'aurait donnée. Là-dessus Amelot lui demanda avec liberté pourquoi
aussi, voulant donner sa bulle, il ne s'était pas contenté de la censure
de quelques propositions du livre du P. Quesnel, au lieu d'en faire une
baroque de cent une propositions\,; que là-dessus le pape s'était écrié,
s'était mis à pleurer, et, lui saisissant le bras, lui avait répondu en
propres termes italiens, répondant à ceux qu'il me dit en français, que
voici\,: «\,Eh\,! monsieur Amelot, monsieur Amelot, que voulez-vous que
je fisse\,! je me suis battu à la perche pour en retrancher\,; mais le
P. Tellier avait dit au roi qu'il y avait dans ce livre plus de cent
propositions censurables\,; il n'a pas voulu passer pour menteur, et on
m'a tenu le pied sur la gorge pour en mettre plus de cent, pour montrer
qu'il avait dit vrai, et je n'en ai mis qu'une de plus. Voyez, voyez,
monsieur Amelot, comment j'aurais pu faire autrement\,!»

On peut juger que ce récit ne se passa pas en commentaire. Rien ne
prouve plus solidement ni plus évidemment que ce discours du pape le cas
qu'il faisait lui-même de sa constitution, de la nécessité de le faire,
et de la manière dont on la lui a fait donner, par conséquent du respect
qui peut être dû à ce fruit de tant de machines infernales, et qui a en
effet allumé un feu d'enfer, suivant la louable intention de ceux qui
l'ont extorquée et fabriquée, et quelle est cette pièce qui a fait
depuis la fortune d'être érigée et présentée en article de foi par ses
créateurs. Personne ne révoquera en doute la probité et la vérité
d'Amelot dans ce récit, et j'ose dire sans insolence que la même foi est
due à celui que j'en fais ici, qui n'en est que le rapport mot pour mot.

Amelot fut bien reçu, mais sa réputation trop justement établie blessa
la jalousie du maréchal d'Huxelles, qui l'accable de louanges et
d'honnêtetés. Elle n'inquiéta pas moins Noailles et Rouillé. Ils
n'eurent pas peine à l'exclure. Sa place de conseiller d'État leur y
donna beau jeu par les prétentions dont on vient de parler.

D'ailleurs M. le duc d'Orléans le craignait par l'union avec laquelle il
avait vécu avec la princesse des Ursins en Espagne, ou sous le nom
d'ambassadeur il avait fait la fonction de premier ministre, y avait
réparé les finances et les troupes, mis l'ordre partout, et avait en
même temps gagné tous les cœurs. C'était dans ces temps de désastre le
comble de la capacité, et en même temps celui de l'esprit, de l'adresse
et du liant, d'avoir si longtemps tout fait sans donner de jalousie à
une femme qui en était si susceptible, et avec qui, de son su à elle, il
avait les ordres du feu roi {]}es plus exprès et les plus réitérés de
n'agir que de concert, et avec dépendance.

Il ne put donc entrer dans le conseil des affaires étrangères, ni dans
celui des finances, lui qui aurait été si utilement et si convenablement
placé dans celui de régence, et jamais il ne fut consulté sur rien.
Néanmoins on fut honteux de le laisser dans les uniques fonctions
judiciaires de sa place de conseiller d'État, qu'il reprit toutes avec
la dernière modestie, sans chercher rien. On établit un conseil du
commerce, dont on le fit président. Il était composé des députés des
principales villes marchandes du royaume, de quelques conseillers d'État
et maîtres des requêtes, et le maréchal de Villeroy et le duc de
Noailles y pouvaient aller présider quand ils voulaient\,; ils n'y
furent le premier presque jamais, l'autre rarement. Il se fit en même
temps un grand changement d'intendants des provinces.

Les spectacles interrompus à Paris, depuis l'extrémité du feu roi,
recommencèrent le 1er octobre.

Canillac obtint un don fort considérable de marais en Flandre, dont une
partie à dessécher.

Le régent régla dix mille francs par mois pour la cassette du roi, et
mille écus pour sa garde-robe, tellement que la duchesse de Ventadour
eut ainsi la disposition de cinquante-cinq mille écus, et le maréchal de
Villeroy après elle.

Le grand prieur, qui se tenait à Lyon exilé par le roi, eut permission
de revenir à Paris, et de voir le roi et d'y demeurer.

Une des premières affaires particulières qui se présentèrent au conseil
de régence fut une prétention de Belle-Ile contre la province de
Bretagne, pour un dédommagement des choses prises par le feu roi sur le
domaine de Belle-Ile. Il la gagna fort lestement, à la fin d'un conseil,
par la faveur de M. le Duc, en quoi je l'aidai fort. L'affaire avait été
instruite\,; le feu roi était persuadé de la justice de la prétention,
en sorte qu'il lui fut adjugé quatre cent mille livres payables comptant
par les états de Bretagne, qu'il toucha bientôt après. Ce personnage a
fait une si surprenante fortune, par des routes si singulières et à
travers de si puissants revers, il est même encore aujourd'hui si
considérable, après avoir toujours été personnage, de quelque façon que
ç'ait été, qu'il est nécessaire de le faire connaître, et pour cela de
remonter à son grand-père, M. Fouquet, célèbre par sa haute fortune et
par ses profonds malheurs.

Ces Fouquet sont de Bretagne, originairement de robe, et ont été
conseillers et présidents au parlement de Bretagne, jusqu'au père du
surintendant. Je fus commissaire de Belle-Ile avec le maréchal de
Berwick, quand il lut chevalier de l'ordre, 1er janvier 1735\,; il ne
farda rien, et ne se donna point pour meilleur qu'il n'est. Le père du
surintendant se fit maître des requêtes, épousa une fille de Maupeou
d'Ableiges, maître des requêtes et intendant des finances. Ce premier
Fouquet, établi à Paris, devint conseiller d'État, et il acquit
tellement l'estime de Louis XIII et du cardinal de Richelieu par sa
probité et sa capacité, qu'ils le voulurent faire surintendant des
finances, qu'il refusa par délicatesse de conscience. Sa femme est
encore célèbre à Paris par sa piété et ses bonnes œuvres, et par le
courage et la résignation avec laquelle elle supporta la chute du
surintendant son fils, et la disgrâce de toute sa famille. Elle faisait
des remèdes, pansait les pauvres, et on a encore des onguents très
utiles de son invention, et qui portent son nom. Elle mourut, en 1681, à
quatre-vingt-onze ans, dans les dehors du Val-de-Grâce où elle était
retirée, aimée et respectée généralement. Elle eut cinq fils et six
filles, toutes six religieuses. Des fils, l'aîné fut surintendant des
finances, auquel je reviendrai\,; le second, archevêque de Narbonne,
exilé bien des années hors de son diocèse à la chute de son frère, mort
en 1673. L'abbé Fouquet fut le troisième, grand important, galant,
dépensier, extravagant, qui de jalousie de femme contribua le plus à la
perte de son frère, et en fut perdu lui-même. Il avait été chancelier de
l'ordre, après M. Servien en 1656. Il était conseiller d'État, et avait
des abbayes. Il mourut à cinquante-huit ans, tout au commencement de
1680. {[}Les autre furent{]} un conseiller au parlement, mort jeune sans
alliance\,; l'évêque d'Agde, chancelier de l'ordre sur la démission de
son frère en 1659\,; il fut exilé à la chute du surintendant en 1661. M.
de Péréfixe, un an après archevêque de Paris, eut sa charge de l'ordre.
L'abbé Fouquet et l'évêque d'Agde perdirent le cordon bleu, et le
dernier sa charge de maître de l'Oratoire. Il est mort à Agde au
commencement de 1708, à soixante-quinze ans. Le dernier des frères était
premier écuyer de la grande écurie, et perdit aussi sa charge et fut
chassé. Il avait épousé la fille du marquis d'Aumont, dont il n'eut
point d'enfants. Les sœurs de sa femme furent religieuses, et ses frères
moururent jeunes. Lui est mort en 1694.

Le surintendant qui causa leur fortune et leur perte fut vingt ans
maître des requêtes, et, à trente-cinq ans procureur général au
parlement de Paris. Au commencement de 1653, le cardinal Mazarin le fit
surintendant des finances. Sa fortune, sa conduite, sa catastrophe ne
sont pas de mon sujet, et sont connues de tout le monde. Il fut arrêté à
Nantes en 1661\footnote{{[}19{]}}, où le roi était allé exprès\,;
conduit à Paris à la Bastille\footnote{{[}20{]}}, trois ans après dans
le château de Pignerol, où il demeura prisonnier le reste de ses jours,
qu'il employa pieusement, et qui finirent en mars 1680, ayant
soixante-trois ans. De Marie Fourché\footnote{}, sa première femme, il
n'eut qu'une fille, mariée au comte depuis duc de Charost, de laquelle
j'ai parlé ailleurs, qui fut mère du duc de Charost, lequel fut fait
gouverneur de la personne de Louis XV, lorsque le maréchal de Villeroy
fut chassé. Le surintendant épousa en secondes noces la fille de Pierre
de Castille, intendant des finances, et de la fille du célèbre président
Jeannin, d'où leur fils s'appela Nicolas Jeannin de Castille, qui fut
greffier de l'ordre, en 1657, sur la démission de Novion, depuis premier
président, qui en fut chassé pour ses friponneries et ses injustices
hardies, comme je l'ai dit ailleurs. Castille fut arrêté à la chute de
son beau-frère, sous lequel il travaillait, puis exilé chez lui à Monjeu
en Bourgogne. C'est lui dont ces fades lettres de Bussy-Rabutin parlent
tant. Il avait eu ordre en prison de donner la démission de sa charge de
l'ordre\,; ce qu'il refusa sous ce prétexte de ne le pouvoir étant
prisonnier. Il eut le même commandement lorsqu'il fut élargi et exilé\,;
il persista dans son refus. On lui ôta le cordon bleu nonobstant sa
charge\,; et, comme son opiniâtreté durait toujours, la charge de
greffier de l'ordre fut donnée par commission à Châteauneuf, secrétaire
d'État, fils et père de La Vrillière, en 1671, enfin en titre, en 1683.
Ce Jeannin de Castille épousa une Dauvet, fille de Desmarets, grand
fauconnier de France, dont il eut une fille unique, que nous avons vue
épouser le comte d'Harcourt-Lorraine, fils unique du prince et de la
princesse d'Harcourt, desquels j'ai parlé quelquefois, lequel comte
d'Harcourt obtint une terre en Lorraine, à qui il fit donner le nom de
Guise par le duc Léopold de Lorraine. Il en prit le nom, que le fils
unique de ce mariage porte encore aujourd'hui. Je n'ai pu me défendre de
cette petite parenthèse des Castille qui sont gens de rien, dont
l'occasion s'est offerte d'elle-même. Revenons maintenant aux enfants
que le surintendant Fouquet a eus de cette Castille sa seconde femme.

Il eut trois fils, et une fille qui épousa, en 1683, le marquis de
Crussol, fils du chef de la branche de Montsalez, lequel était frère du
troisième duc d'Uzès, bisaïeul du duc d'Uzès d'aujourd'hui. Il y a
postérité de ce mariage. Les trois fils, frères de cette dame de
Montsalez, furent M. de Vaux, fort honnête et brave homme, qui a servi
volontaire, à qui le roi permettait d'aller à la cour, mais qui jamais
n'a pu être admis à aucune sorte d'emploi. Je l'ai vu estimé et
considéré dans le monde. Il avait épousé la fille de la célèbre
M\textsuperscript{me} Guyon, et mourut sans enfants en 1705. Le
chevalier de Sully, devenu duc et pair par la mort de son frère,
l'épousa par amour, et ne déclara son mariage que fort tard, à cause de
la duchesse du Lude, sa tante, qui en fut outrée, principalement parce
qu'elle n'était pas en état d'avoir des enfants. Elle était fort belle,
vertueuse, et avait beaucoup d'esprit et d'amis. Le second fils fut le
P. Fouquet, grand directeur et célèbre prêtre de l'Oratoire\,; le
troisième, M. de Belle-Ile qui, non plus que son frère, n'a jamais pu
obtenir aucune sorte d'emploi, qui n'a jamais paru à la cour, et presque
aussi peu dans le monde, fort honnête homme aussi avec beaucoup d'esprit
et de savoir. Je l'ai fort connu à cause de son fils. Il était sauvage
au dernier point, et néanmoins de bonne compagnie, mais battu de ses
malheurs.

Je ne sais ou il vit une fille de M. de Charlus, père du duc de Lévi.
Ils se plurent peut-être un peu trop\,; on les fit marier\,; on ne leur
donna rien\,; on ne les voulut point voir. Ils s'en allèrent vivre à
Agde, où ils ont passé nombre d'années au pain et au pot de l'évêque,
leur oncle. Ils revinrent enfin à Paris chez M\textsuperscript{me}
Fouquet, leur mère, dans ces mêmes dehors du Val-de-Grâce, qui les
nourrit tant qu'elle vécut\,; après quoi ils eurent quelque peu de bien.
Longtemps après ils recueillirent Belle-Ile, et tout ce qui avait été
sauvé des débris du surintendant, par la mort de M. de Vaux, l'aîné des
trois, et du P. Fouquet, le second. Ils eurent deux fils, et une fille
qui, après l'avoir été longtemps, épousa enfin le fils aîné de M. de La
Vieuville et de la soeur du comte de La Mothe-Houdancourt, sa première
femme. Ce La Vieuville était un néant obscur, qui bientôt après la
laissa veuve avec deux fils.

Les deux fils, frères de cette dame de La Vieuville, portèrent le nom de
comte et de chevalier de Belle-Ile. Jamais le concours ensemble de tant
d'ambition, d'esprit, d'art, de souplesse, de moyens de s'instruire,
d'application de travail, d'industrie, d'expédients, d'insinuation, de
suite, de projets, d'indomptable courage d'esprit et de cœur, ne s'est
si complètement rencontré que dans ces deux frères, avec une union de
sentiments et de volontés, c'est trop peu dire, une identité entre eux
inébranlable\,: voilà ce qu'ils eurent de commun. L'aîné, de la douceur,
de la figure, toutes sortes de langages, de la grâce à tout, un
entregent, une facilité, une liberté à se retourner, un air naturel à
tout, de la gaieté, de la légèreté, aimable avec les dames et en
bagatelles, prenant l'unisson avec hommes et femmes, et le découvrant
d'abord. Le cadet plus froid, plus sec, plus sérieux, beaucoup moins
agréable, se permettant plus, se contraignant moins, et paraissant moins
aussi, peut-être plus d'esprit et de vue, mais moins juste, peut-être
encore plus capable d'affaires et de détails domestiques, qu'il prit
plus particulièrement, tandis que l'aîné se jeta plus au dehors\,:
haineux en dessous et implacable, l'aîné glissant aisément et pardonnant
par tempérament\,; tous deux solides en tout, marchant d'un pas égal à
la grandeur, au commandement, à la pleine domination, aux richesses, à
surmonter tout obstacle, en un mot, à régner sur le plus de créatures
qu'ils s'appliquèrent sans relâche à se dévouer, et à dominer
despotiquement sur gens, choses et pays que leurs emplois leur
soumirent, et à gouverner généraux, seigneurs, magistrats, ministres
dont ils pouvaient avoir besoin, toutes parties en quoi ils réussirent
et excellèrent jusqu'à arriver à leurs fins par les puissances qui les
craignaient et qui même les haïssaient. C'est ce qui se verra par la
suite, et qui s'est vu encore mieux au delà du temps de l'étendue que je
puis donner à ces Mémoires.

Ils se trouvaient cousins germains des ducs de Charost et de Lévi, issus
de germains de la comtesse d'Harcourt, mère de M. de Guise et des
duchesses de Bouillon et de Richelieu, cousins germains de MM. de
Crussol-Montsalez. Leur mère était une femme qui avait plus d'esprit
qu'elle n'en paraissait, et encore plus de sens, avec beaucoup de
douceur et de modestie. Elle et son mari vécurent toujours intimement\,;
et leurs enfants leur furent toujours entièrement attachés. M. de Lévi,
qui au fond était bon homme, eut pitié de sa tante\,;
M\textsuperscript{me} de Lévi encore plus. L'un et l'autre la prirent en
amitié, et par elle sa famille. Cette affection alla toujours croissant,
en sorte que M\textsuperscript{me} de Lévi, qui était vive et ardente,
se serait mise au feu pour eux. Le duc de Charost ne fut pas moins
échauffé pour eux. On a vu souvent dans quelle liaison
M\textsuperscript{me} de Saint-Simon et moi vivions avec lui et avec
M\textsuperscript{me} de Lévi, et c'est ce qui la forma entre les
Belle-Ile et nous, qui de là devint après directe. L'aîné avait épousé
une Durfort-Sivrac, avec qui ils vécurent tous à merveilles et avec une
patience surprenante. C'était une manière de folle, qui mourut,
heureusement pour eux, et n'eut point d'enfants.

Il servit quelque temps capitaine en Italie. Là et partout où il servit
depuis, il s'appliqua à connaître ce qui valait le mieux en chaque
partie militaire\,: troupes, partisans, officiers généraux, artillerie,
génie, jusqu'aux vivres, aux dépôts, aux munitions, à faire sa cour à
ces meilleurs-là de chaque espèce, et à les suivre pour s'en faire aimer
et instruire. Le roi qui connaissait encore quelque mesure entre les
gens, ne put refuser enfin un régiment à Belle-Ile\,; mais il lui en
refusa d'infanterie et de cavalerie. Il lui permit d'en acheter un de
dragons, où les gens d'une certaine qualité ne voulaient pas entrer
alors, si ce n'était tout à coup dans les deux charges supérieures.
Belle-Ile, qui avait déjà capté des généraux, non content de faire les
campagnes en homme qui ne ménage rien pour voir tout et apprendre,
passait après les hivers à visiter les différentes frontières, ceux qui
y commandaient, à s'y instruire de tout ce qu'il pouvait\,; et s'il y
avait en Italie ou ailleurs un reste de campagne plus longue, il y
allait l'achever, volontaire, toujours cherchant à apprendre tout et de
tous. Cette volonté l'instruisit en effet beaucoup, le fit connaître à
toutes les troupes, et lui donna de la réputation. On a vu qu'il en
acquit beaucoup à la défense de Lille, sous le maréchal de Boufflers qui
le vanta fort, et qu'il en sortit brigadier, fort dangereusement blessé.
Sa blessure se rouvrit la campagne suivante en Allemagne. Il fut porté à
Saverne. Il y fut longtemps, il sut en profiter, et il devint intime du
cardinal de Rohan et de tous les Rohan, et l'est toujours demeuré
depuis. Son frère en sa manière se conduisit et s'instruisit avec le
même soin, et eut à la fin un brevet de colonel de dragons. L'aîné fit
pourtant si bien qu'il obtint l'agrément du feu roi d'acheter, en 1709,
d'Hautefeuille, la charge de mestre de camp des dragons, qui a été le
premier pas de sa fortune, où nous le laisserons présentement.

\hypertarget{chapitre-x.}{%
\chapter{CHAPITRE X.}\label{chapitre-x.}}

1715

~

{\textsc{Pontchartrain reçoit en face les plus cruels affronts en plein
conseil de régence.}} {\textsc{- Bassesse et avarice de Pontchartrain.}}
{\textsc{- Désordre des finances.}} {\textsc{- Frayeur des partisans.}}
{\textsc{- Plénoeuf en fuite.}} {\textsc{- Suite et détail des finances,
trop fort et trop vaste pour moi à le raconter.}} {\textsc{- Replâtrage
entre M. le Duc et le duc du Maine sur la qualité de prince du sang.}}
{\textsc{- M. le Grand prétend toute supériorité et autorité sur la
petite écurie et sur le premier écuyer du roi, et d'avoir la dépouille
de la petite écurie.}} {\textsc{- Caractère de M. le Grand.}} {\textsc{-
Faiblesse du conseil de régence.}} {\textsc{- Raisons de M. le Grand.}}
{\textsc{- Raisons de M. le Premier.}} {\textsc{- M. de Troyes s'enfuit
à Troyes, de peur de juger l'affaire de M. le Grand et de M. le
Premier.}} {\textsc{- Conseil de régence où les prétentions du grand et
du premier écuyer sont jugées toutes en faveur du premier écuyer.}}
{\textsc{- Le premier écuyer me parle en faveur de sa femme et me presse
de la recevoir.}} {\textsc{- Caractère de M\textsuperscript{me} de
Beringhen.}} {\textsc{- Je reçois enfin sa visite.}} {\textsc{- Le
régent permet au grand écuyer de protester, qui en abuse et tient
l'affaire comme non jugée.}} {\textsc{- Continuation des mêmes démêlés,
qui, après la mort de M. le Grand, tuent M. le Premier, et qui
continuent entre leurs fils jusqu'à ce que le roi majeur décida comme
avait fait le conseil de régence.}} {\textsc{- Le prince Charles refuse
de signer les dépenses de la petite écurie à l'ordinaire, sans examen.}}
{\textsc{- M. le Duc, sur ce refus, les signes comme grand maître de
France, et le grand écuyer en perd le droit.}}

~

J'avais bien résolu, dès que je verrais le conseil de régence prendre
forme, d'y faire révoquer l'édit de création des gardes-côtes qui
m'avait brouillé, comme on l'a vu, avec Pontchartrain, et de me donner
le plaisir de le faire en sa présence. J'en parlai au comte de Toulouse,
qui abhorrait Pontchartrain, comme on l'a vu aussi, et qui la lui
gardait bonne, ainsi que le maréchal d'Estrées. Nous convînmes que cela
serait proposé au conseil du mardi 1er octobre, qui devait être occupé
des affaires de marine, et où le comte me dit que je verrais de belles
choses sur Pontchartrain. En effet, ce jour-là, dès que nous fûmes
assis, il proposa cet édit à casser comme inutile, et même préjudiciable
au service et au repos des peuples, qu'on harcelait à trente lieues de
la mer, le long des rivières, comme il plaisait à Pontchartrain et aux
valets à qui il donnait les emplois de gardes-côtes, ou qui les
achetaient pour s'en récompenser au décuple aux dépens des peuples de
leur département. Je regardais cependant Pontchartrain de ma place d'un
bout de la table à l'autre, avec tout le plaisir que je m'en étais
promis depuis longtemps. Chacun approuva en deux mots. Ce que je dis à
mon tour fut très court, mais très amer, et l'édit fut supprimé, ainsi
que tous ceux qu'il avait établis, et sur-le-champ destitués de toute
sorte de fonction. Pontchartrain rageait, et je le regardais à le
pénétrer. Il n'était pas au bout.

Les mémoires pleuvaient contre lui\,; il ne passait pas pour avoir les
mains nettes. La marine entière, qu'il s'était complu à désespérer,
criait alors sans crainte et sans ménagement. Il fallait voir clair à
des accusations qui n'allaient à rien moins qu'à le charger d'avoir
immensément profité de la vente qu'il avait fait faire de tous les
magasins des ports pour anéantir la marine, et ôter tout moyen au comte
de Toulouse et au maréchal d'Estrées de retourner à la mer. Tous les
magasins partout se trouvèrent en effet vides, et le comte de Toulouse
ne voulut pas se commettre à rien avancer sans le bien prouver. Il en
trouva les preuves parfaites, et en sut faire usage sans que
Pontchartrain s'en doutât le moins du monde. Dès que l'affaire de la
révocation de l'édit de création des officiers gardes-côtes fut finie,
le maréchal d'Estrées, qui de concert avec le comte de Toulouse en avait
apporté un mémoire, le tira de sa poche et demanda permission de lire,
pour mettre le conseil au fait de l'état où se trouvait la marine, et se
mit à en faire la lecture. C'était un mémoire fort détaillé, et bien
exactement prouvé, sur la déprédation des bois de la marine de
Rochefort, où les accusations étaient directes et personnelles sans nul
ménagement. De temps en temps le comte de Toulouse interrompait pour
appuyer certains endroits, en faire remarquer d'autres, en commenter
quelques-uns avec un air froid et modeste, mais avec la plus grande
force, et sans le plus petit égard pour Pontchartrain présent. Il voulut
dire quelque chose, mais au premier mot le maréchal d'Estrées lui dit
qu'il n'avait pas droit de parler au conseil, et le fit taire comme un
petit garçon, avec toute la hauteur et le mépris possible. Il continua
sa lecture tout de suite, et le comte de Toulouse par-ci par-là ses
fâcheuses annotations.

Surpris au dernier point d'une telle ignominie en face, j'en dis ma
pensée au comte de Toulouse, qui me répondit tout bas aussi, en
souriant, que je verrais bien autre chose le lendemain matin. Il tint
parole.

Sitôt que nous eûmes pris nos places, le comte de Toulouse tira de sa
poche un mémoire dont il fit la lecture, le plus amer, le plus cruel qui
fut jamais. Il traitait la même matière de la veille, et bien d'autres
déprédations, les commenta toutes à mesure, insista sur les ordres que
Pontchartrain avait donnés, et qu'il ne pouvait nier, montra que de
propos délibéré il avait ruiné la marine, et très nettement qu'il ne s'y
était rien moins que ruiné lui-même.

L'étonnement de chacun fut sans pareil, non du contenu du mémoire qui ne
surprit personne pour le fond, mais de ces pointes cruellement acérées à
chaque mot, mais du poids qu'y donnait le lecteur par le sien, et par
les réflexions qu'il y faisait très fréquemment plus dures encore que le
texte du mémoire, mais de la présence de Pontchartrain si outrageusement
attaqué en face, en sa propre personne, qui paraissait là pis que sur la
sellette, et qui, instruit de la veille par le maréchal d'Estrées, n'osa
jamais souffler. La lecture fut terminée par l'aveu que fit le comte de
Toulouse d'avoir fait lui-même ce mémoire. Ce fut le comble de
l'ignominie, d'autant que le comte ajouta qu'il avait adouci ce qu'il
avait pu, et supprimé même beaucoup de vérités très fâcheuses.

Il est incroyable comment une telle infamie put être supportée par un
homme de l'insolence, de la tyrannie et de la pédanterie gauche,
austère, insupportable avec tout le monde de Pontchartrain, et qui
ajoutaient encore à sa malignité et à sa méchanceté naturelle\,; car il
avait le bien de les posséder suprêmement toutes deux. Cependant il ne
sourcilla pas, et fut assez impudent, ou assez prodigieusement
insensible, pour sortir du conseil comme si rien ne s'y fût passé à son
égard. Il ne s'en fallut rien pourtant qu'il ne fût juridiquement
attaqué et recherché, et il y aurait sûrement succombé, mais il fut
encore sauvé de ce gouffre par la considération de son père.

Je fus bien étonné chez moi, le lendemain, de me l'entendre annoncer.
J'étais alors avec La Chapelle, ce premier commis si fort autrefois de
sa confiance, et qu'une basse jalousie lui avait fait chasser, comme on
l'a vu en son temps, et pour qui j'avais obtenu la place de secrétaire
du conseil de marine, parce que le comte de Toulouse et le maréchal
d'Estrées l'avaient toujours estimé. Je le fis passer dans un
arrière-cabinet, et je reçus Pontchartrain, que je ne me souvenais guère
d'avoir vu chez moi. Ma surprise fut encore plus grande quand cet homme,
à qui je n'avais pas parlé depuis la mort du roi, et fort rarement
longtemps auparavant, me dit qu'il venait à moi pour me parler de sa
douleur de la scène de la veille, me demander conseil sur ce qu'il
ferait, et protection auprès du régent. Il ajouta quelques plaintes
modestes, bien différent de son ton sous le feu roi, et me dit qu'il
avait pensé plus d'une fois interrompre et répondre. Je lui dis que,
pour ce dernier article, il avait bien fait de se contenir\,; qu'encore
qu'il y ait grande différence entre se défendre quand on est
personnellement attaqué, et opiner dans un conseil, il devait savoir
qu'il n'y avait point de voix, et sentir qu'on l'eût fait taire, et
qu'on n'eût pas souffert, sur des matières si intéressantes, une dispute
entre le maréchal d'Estrées et lui, beaucoup moins entre lui et le comte
de Toulouse, qui si aisément aurait pu aller trop loin.

Il me demanda après ce qu'il avait donc à faire\,: «\,Démentir, lui
dis-je, les deux mémoires et leurs preuves par un mémoire, et des
preuves contraires bien claires et bien évidentes, où jusqu'aux moindres
faits soient si nettement articulés qu'il ne soit pas possible de se
refuser à la démonstration, le présenter au régent, le distribuer à tous
les conseils, et en inonder Paris et les ports de mer. Si, au contraire,
il n'était pas en état de présenter un mémoire de cette transcendance,
se taire, et tendre le dos en silence sous la gouttière\,; sur quoi
c'était à lui à se juger.\,» Ce conseil, le seul pourtant qu'il pût
prendre, me parut ne lui pas plaire. Il barbouilla à son ordinaire avec
sa division en trois points, dont il usait en toute espèce de
raisonnement et de choses. Le fait est qu'il n'avait rien à opposer aux
faits et aux preuves qu'il venait d'essuyer en face, et que le pot aux
roses était pleinement découvert.

Il se rabattit à vanter ses services, à regretter le feu roi, à se
plaindre qu'au lieu des récompenses qu'il avait droit d'attendre, on
l'eût réduit à n'être plus rien\,; qu'on le faisait passer pour fort
riche\,; qu'il n'était rien moins (c'est-à-dire qu'il l'était à
millions)\,; que ce serait bien le moins qu'on pût faire que de lui
donner quelque marque de considération publique, et il finit tout ce
jargon par me prier de demander pour lui au régent une pension de vingt
mille livres. Cette bassesse d'avoir recours à moi, au point où nous en
étions ensemble, me fit envie de vomir, et j'en admirai l'avarice, le
contre-temps et l'impudence. Je lui répondis doucement que ce serait mal
prendre son temps avant d'avoir pleinement détruit les accusations
personnelles, qu'il ne pouvait avoir oubliées depuis vingt-quatre heures
qu'il les avait ouï lire et appuyer en si bonne compagnie, et qu'à
l'égard de son indigence, indépendamment de ces accusations, et des
preuves qu'il en avait ouïes, indépendamment encore de ses biens et de
ses acquisitions connues, il avait plus de cent mille livres de rente de
sa charge de secrétaire d'État, et vingt mille livres de pension de
ministre par-dessus, quoiqu'il ne l'eût jamais été. Je ne lui dissimulai
pas qu'il se ferait moquer de lui, et que ce serait tout le succès d'une
demande si déplacée. Nous nous séparâmes de la sorte, et je ne l'ai vu
qu'une fois ou deux depuis, chez lui ou chez moi.

Dès qu'il fut sorti, je rappelai La Chapelle, et lui montrant une pièce
de tamiserie de l'histoire d'Esther, tendue où nous étions, je lui
présentai Aman et Mardochée, et lui dis\,: «\,Vous voilà et
Pontchartrain.\,» Ce hasard nous divertit, et plus encore la proposition
qu'il venait de me faire.

Il était aussi rampant avec tout le monde qu'il avait été insolent,
gauche et brutal, sans exception de personne, et n'y gagna qu'un parfait
mépris. Il mourait de peur d'être chassé, et de rage de ne pouvoir plus
mal faire\,; le néant et l'oisiveté le rongeaient. Il tenait encore à un
filet par le vain titre de sa charge, dont le conseil de marine ne lui
laissait pas la moindre fonction, et par cette entrée sans voix au
conseil de régence. Il s'attachait néanmoins à ce filet\,; dans
l'espérance qu'il lui servirait enfin à remonter, et pour passer
cependant pour être encore quelque chose.

Nous ne nous parlâmes point de son édit révoqué des gardes-côtes. Il
devait avoir vu que je commençais à lui tenir la parole qu'on a vu que
j'avais donnée, et comprendre par là que mon dessein était de la tenir
tout entière, conséquemment à ne me pas choisir pour son conseil et son
protecteur. Je crois qu'il en fut désabusé par cette visite. Laissons-le
végéter dans son humiliation encore quelque temps\,; car il était sur un
pied et sur un autre tandis que le conseil de régence s'assemblait ou
sortait, sans que qui que ce soit lui dît une parole, ou lui répondit
plus d'un seul mot, s'il s'avisait de parler à quelqu'un, excepté La
Vrillière, encore fort peu par honneur, et beaucoup moins le maréchal de
Besons.

Le conseil des finances les avait trouvées dans un étrange état. Il
était dû seize cent mille francs à nos ambassadeurs, et à ceux que le
roi tenait auprès des princes étrangers, dont la plupart, à la lettre,
n'avaient pas de quoi payer le port de leurs lettres, ayant mangé tout
le leur\,; ce qui faisait un cruel discrédit par toute l'Europe. Les
financiers cependant avaient profité du temps qu'on avait eu besoin
d'eux, jusqu'à passer tout ce qu'ils voulaient. Noailles et Rouillé
voulurent les ressasser. L'épouvante se mit parmi eux, et Plénœuf
disparut et se sauva en Italie. J'aurai à parler de lui ailleurs. Il
faudrait une grande connaissance des finances, une vaste et juste
mémoire, et de gros volumes uniquement sur cette matière, à qui voudrait
exposer tout ce qui fut tenté, manqué, exécuté là-dessus. Ce travail est
au-dessus de mes forces et de mon goût. Je me contenterai donc de
marquer les événements principaux en ce genre, que je laisserai traiter
à fond par qui en sera plus capable que je ne le suis.

L'affaire, dont j'ai fait mention, de la qualité de prince du sang prise
par le duc du Maine dans une signification de lui à M. le Duc, dans leur
procès de la succession de M. le Prince, fut, après bien des allées et
venues, replâtrée chez M\textsuperscript{me} la Princesse, où les
parties se trouvèrent avec l'abbé Menguy, conseiller de la
grand'chambre, qui avait été chargé de ce détail. M. le Duc retira
toutes les protestations qu'il avait faites contre tous les actes où le
duc du Maine avait pris la qualité de prince du sang, s'engagea de
promettre à M. le duc d'Orléans de ne les point renouveler sans son
consentement, et ne voulut donner aucune parole à M. ni à
M\textsuperscript{me} du Maine, consentit de ne point prendre lui-même
la qualité de prince du sang dans les actes qui se feraient avec le duc
du Maine, pour que celui-ci ne la prît pas non plus avec lui, et trouva
bon que lui et le comte de Toulouse la prissent avec tout ce qui n'est
point prince du sang. Ainsi M. le Duc recula sur tout, et le duc du
Maine gagna tout, puis {[}que{]} M. le Duc et lui demeuraient égaux en
ne prenant ni l'un ni l'autre ensemble la qualité de princes du sang, et
M. du Maine demeurant autorisé par M. le Duc à la prendre, lui et son
frère, avec toutes autres personnes. Il n'y avait que le
\emph{retentum}\footnote{} de ne renouveler ses protestations contre
cette qualité que du consentement de M. le duc d'Orléans. Aussi ne
fut-ce qu'un replâtrage, qui n'eut pas même loisir de sécher. Tout se
passa entre eux d'une manière fort aride, et qui promettait ce qui
arriva depuis. Je passe sous silence ce qui fut convenu sur l'intérêt
pécuniaire, comme n'étant intéressant qu'en tant que ce fut cet intérêt
qui porta celui de la qualité du rang, etc., jusqu'où les choses furent
portées dans la suite.

Une autre affaire se présenta à juger au conseil de régence, parce que
M. le duc d'Orléans ne sut pas imposer et ordonner que les choses
demeureraient sur le même pied qu'elles avaient été sous le feu roi et
sous ses derniers prédécesseurs. Nous étions encore à Versailles après
la mort du roi, que Beringhen, premier écuyer, me dit que M. le Grand
voulait prétendre toute la dépouille de la petite écurie, et toute
supériorité de sa charge sur la sienne. J'en fus d'autant plus surpris
que le comte d'Harcourt et M. le Grand, son fils, d'une part, et les
deux Beringhen, père et fils, d'autre part, avaient passé leur vie et
toute celle du feu roi dans ces deux charges, sans prétention d'une
part, sans dépendance de l'autre, nonobstant toute la supériorité
personnelle et tout le crédit constant des deux grands écuyers, dont le
dernier n'avait qu'à ouvrir la bouche pour obtenir sur-le-champ du roi
tout ce qui lui plaisait. Je n'ai point su qui mit cela si tard dans la
tête de M. le Grand, mais il l'entreprit tout d'un coup, et en fit une
affaire majeure, de l'instruction et du rapport de laquelle au conseil
de régence M. de Torcy fut chargé par M. le duc d'Orléans. Il se donna
des mémoires de part et d'autre, et cette affaire partagea toute la
cour. Le rare fut que ceux qui en devaient être juges prirent
l'épouvante. Ils mourraient de peur de ce reste inanimé de la maison de
Lorraine, surtout ils redoutaient M. le Grand, que le superbe état qu'il
avait tenu toute sa vie, son crédit prodigieux et constant auprès du
roi, les manières si supérieures auxquelles il avait accoutumé tout le
monde, rendaient très autorisé.

C'était un homme sans aucun autre esprit qu'un long usage de la cour et
du plus grand monde, gâté par sa faveur et par la sottise du monde, très
bon homme, très noble, très désintéressé, fort poli avec discernement,
encore plus haut, et le dernier de sa maison qui ait porté jusqu'à la
fin de sa vie la grandeur dans toutes ses prétentions, qu'on lui passait
à la faveur de sa maison toujours ouverte, avec le plus grand jeu et la
plus grande chère soir et matin. Fort brutal et alors sans ménagement en
face, même aux femmes, quand il s'y mettait, et d'une gourmandise
singulière\,; son âge et sa goutte presque continuelle l'avaient
affranchi de tout devoir\,; mais en aucun temps il n'avait fait sa cour
qu'au roi, à la vérité avec la plus grande bassesse, et des flatteries
dont l'excès et la fadeur faisaient mal au cœur. Jamais il n'avait mis
le pied chez aucun ministre\,; il conservait avec eux toute sa grandeur,
en était craint et ménagé, et ne se contraignait pour personne. C'était
donc un homme que, sur ce qu'il s'était une fois mis en tête, on
craignait de choquer.

D'autre part Beringhen, premier écuyer, était aimé, estimé, considéré de
tout temps, et avait beaucoup d'amis. Il n'avait d'existence que par sa
charge, que M. le Grand prétendait nettement mettre au niveau de celle
du premier écuyer de la grande écurie qui la commande sous lui, qui lui
est soumis et subordonné en tout, et qui n'est proprement qu'un écuyer
renforcé. Les juges avaient donc peine à réduire Beringhen à ce néant si
distant d'une des plus belles charges de la cour que son père et lui
avaient exercée toute leur vie.

Dans cet embarras chacun des juges eût fort désiré ne l'être point, mais
l'affaire était engagée. Ils imaginèrent de s'en tirer en proposant à M.
le duc d'Orléans de renvoyer le jugement à la majorité du roi. Le régent
goûta cet expédient, et sans rien déclarer tira de longue. M. le Grand,
qui par ce délai perdait de fait, puisque les choses demeuraient comme
elles étaient, se mit à usurper tout sur le service de la petite écurie.
Tous les jours c'étaient des voies de fait. Les écuyers, les pages, les
valets de pied étaient aux prises jusque dans la cour et dans les
antichambres du roi. C'étaient des mainmises continuelles\,; chaque
écurie ne s'y pré sentait qu'en force, prêtes toutes deux à
s'entr'égorger. Le premier écuyer contenait ses gens et se plaignait, et
criait de toute sa force\,; le grand écuyer avouait les siens tout haut,
et ne se cachait pas d'usurper à force ouverte tout ce qu'il prétendait,
en sorte que cela pouvait aller bien loin entre les écuyers et les pages
des deux parties, dans des occasions journelles d'un service continuel
impossible à éviter, et une indécence et un manque de respect au roi
extrême.

Ce désordre me toucha. J'en parlai au régent, et je lui remontrai
combien il y allait du sien à le souffrir\,; qu'à la fin il arriverait
quelque catastrophe peut-être sous les yeux du roi\,; qu'il verrait la
cour se partialiser, et les choses très aisément à un point qu'il y
serait fort empêché, et aurait à se bien repentir de sa tolérance.
J'ajoutai qu'il était honteux aux membres de la régence de montrer une
telle timidité qui les ferait mépriser de tout le monde\,; que j'étais
celui de tous qui avait le plus d'intérêt à ne point juger ce procès,
parce que de quelque côté que je décidasse, je ne manquerais pas d'être
blâmé\,: si en faveur du premier écuyer, on dirait que c'est parce que
mon père l'avait été\,; qu'il n'était pas en moi de n'être pas contre la
maison de Lorraine en quoi que ce pût être, et que sur M. le Grand en
particulier, dont le père avait volé sa charge au mien, et après les
démêlés publics que lui et moi avions ensemble, et qui plus d'une fois
avaient été jusqu'au feu roi, je n'étais pas homme à les oublier. Si, au
contraire, j'étais pour M. le Grand, je pouvais m'attendre qu'on dirait
qu'une passion cédait à une autre, et les anciennes querelles aux
nouvelles\,; que le premier écuyer était l'ami intime du premier
président\,; que M\textsuperscript{me} de Beringhen, comme il était
vrai, s'était répandue contre moi sans mesure\,; que c'était souvent
chez elle où le premier président avait tenu ses conseils dans l'affaire
du bonnet, et les y tenait encore contre nos poursuites\,; qu'en ces
circonstances on pouvait bien s'attendre de quel avis je serait, et avec
quel plaisir je saisirais l'occasion d'anéantir la charge de l'ami
intime du premier président, et de me venger de M\textsuperscript{me} de
Beringhen. La chose était ainsi, et je ne disais que trop vrai. Le
régent sentit le poids de l'indécence et les suites des mainmises, et de
ce moment, se résolut à juger incessamment. L'affaire est courte et
curieuse, et mérite bien d'être exposée ici.

M. le Grand produisait ses provisions de grand écuyer de France, qui lui
donnaient égale et entière autorité sur la grande et sur la petite
écurie, et sur tous leurs officiers. Il y prouvait qu'à son égard il n'y
avait ni distinction ni différence entre les deux premiers écuyers de la
grande et de la petite écurie\,; que le titre de premier écuyer du roi
n'est qu'un nom, qu'un usage sans fondement a établi, et que son unique
titre est celui de premier écuyer de la petite écurie du roi, comme le
titre de l'autre est de premier écuyer de la grande écurie du roi,
lequel est demeuré jusqu'alors dans son entière dépendance en tout et
pour tout. Il ajoutait que, encore que tous les carrosses et tous les
attelages du roi soient de sa petite écurie, c'était de tout temps, sans
interruption jusqu'alors, au grand écuyer seul à ordonner le deuil des
carrosses et des harnais des attelages toutes les fois que le roi
drapait, et celui de toute la livrée de la petite écurie sans aucune
exception. Enfin il montrait qu'il était seul et unique ordonnateur de
la petite écurie comme de la grande\,; que la chambre des comptes ne
connaît que sa seule signature pour la petite comme pour la grande
écurie, et que bien qu'il laissât faire au premier écuyer toutes les
dépenses de la petite écurie, c'était au grand écuyer que ces dépenses
étaient apportées lorsqu'il en fallait compter, pour qu'il fît, comme il
l'y faisait toujours, la même fonction d'ordonnateur qu'il faisait pour
les dépenses de la grande écurie, avec quoi elles étaient allouées à la
chambre des comptes, sans que le nom du premier écuyer y parût jamais en
rien. Sa conclusion était l'entière dépendance de lui de toute la petite
écurie et de son premier écuyer, à quoi ne pouvait préjudicier la
complaisance qu'il avait eue de ne la pas faire sentir, et conséquemment
qu'à lui, privativement au premier écuyer, appartenait toute la
dépouille de la petite écurie.

M. le Premier convenait de tous ces faits, et en niait les conséquences.
Il prétendit que les provisions de l'office de grand écuyer, toutes
copiées sur l'ancien style, ne prouvaient rien contre l'état présent des
choses\,; que la plupart des charges se sont faites et accrues aux
dépens les unes des autres. Il disait qu'on serait bien étonné de voir
le grand chambellan prétendre se soumettre aujourd'hui les quatre
premiers gentilshommes de la chambre, le grand maître et les maîtres de
la garde-robe et tous les officiers qui dépendent d'eux, vouloir
commander seul dans la chambre et les appartements du roi, y ordonner et
payer les fêtes et les cérémonies, ôter aux premiers valets de chambre
la cassette du roi, s'arroger un petit sceau du roi, et en sceller comme
autrefois une infinité de choses, à l'insu du chancelier, et recevoir,
privativement à lui et à la chambre des comptes, un grand nombre de foi
et hommages\,: toutes fonctions qu'il n'est pas contesté qu'il n'ait
eues autrefois, et qui peu à peu ont été démembrées de son office\,; et
qu'il en est ainsi de beaucoup d'autres charges.

Sur le deuil des carrosses, harnais, livrée, etc., de la petite écurie,
lorsque le roi drape, ordonné par le grand écuyer, Beringhen
représentait que le grand écuyer dans les grands deuils de la cour
envoyait son propre tailleur prendre la mesure des quatre capitaines des
gardes du corps, dont il ordonnait et payait les habits de deuil qu'il
leur envoyait tout faits, et qui passaient sur son ordonnance\,; que,
plus encore, il faisait faire les étendards des quatre compagnies des
gardes du corps, les leur envoyait, le payait, et les faisait allouer
sur son ordonnance\,; que néanmoins on n'avait pas vu que le grand
écuyer eût à prétendre ni autorité, ni détail, ni subordination
quelconque, sur pas une des quatre compagnies des gardes du corps, ni
sur leurs capitaines, ni sur les officiers de ces troupes\,; d'où il
concluait qu'il n'avait pas plus de droit sur la petite écurie, ses
officiers et le premier écuyer, par la raison du deuil qu'il y
ordonnait.

Quant à ce qui regarde la chambre des comptes, qui ne connaît que la
signature du grand écuyer pour les dépenses de la petite écurie, que le
premier écuyer lui envoie à signer comme ordonnateur, ce n'est que pour
diminuer le nombre des différentes signatures, et entretenir un meilleur
ordre dans la chambre des comptes, qui ne lui donne pas plus
d'inspection sur la petite écurie que les étendards des quatre
compagnies des gardes du corps et les habits de deuil de leurs quatre
capitaines, dont il est le seul ordonnateur, ne lui en donnent sur eux
et sur leurs troupes\,; enfin que M. le Grand ne peut disconvenir qu'il
signe, et a toujours signé ainsi que M. son père, les états de dépense
de la petite écurie sans les voir, sans le plus léger examen, et
uniquement sur la signature du premier écuyer qu'il y trouve.

À ces raisons générales, le premier écuyer ajoutait des faits constants.
Il disait qu'il était vrai qu'anciennement le premier écuyer et la
petite écurie étaient dans l'entière dépendance du grand écuyer, mais
qu'Henri III l'en avait totalement séparée et rendue indépendante en
tout et pour tout, et le premier écuyer et tous les officiers de la
petite écurie exempts de toute subordination au grand écuyer\,; qu'en un
mot, ce prince en avait fait deux choses entièrement distinctes et
séparées, en sorte que le premier écuyer était devenu dans la petite
écurie semblable en autorité au grand écuyer dans la grande écurie. Que
cela s'était fait en faveur de M. de Liancourt, mari de la célèbre
marquise de Guercheville, qui fut depuis dame d'honneur de Marie de
Médicis, lorsqu'Henri IV l'épousa, père et mère du duc de Liancourt\,;
que les choses sont toujours depuis restées de la sorte\,; qu'Henri III
et tous ses successeurs avaient toujours depuis donné l'ordre au grand
et au premier écuyer distinctement et séparément, en présence l'un de
l'autre, et qui plus est, à un simple écuyer de la petite écurie, en
présence du grand écuyer, toutes les fois qu'il était présent e que le
grand ne l'était pas\,; que M. le Grand ne pouvait nier que la même
chose ne lui fût arrivée tant que le roi avait vécu, et qu'il en avait
pris l'ordre, sans que jamais il en eût fait la moindre représentation,
beaucoup moins de plainte\,; enfin que M. de Liancourt avait eu toute la
dépouille de la petite écurie par deux fois, à la mort d'Henri III et à
celle d'Henri IV, et mon père à celle de Louis XIII, tous trois sans la
moindre difficulté ni opposition.

Sa conclusion était qu'il devait continuer à vivre avec M. le Grand
comme il y avait toujours vécu, c'est-à-dire que le premier écuyer, la
petite écurie et tout ce qui y appartenait, demeurassent, à l'égard du
grand écuyer et de la grande écurie, sur le pied de séparation entière
et de totale indépendance où Henri III l'avait mise, et où les rois ses
successeurs l'avaient maintenue jusqu'alors, sans que, depuis près de
cent quarante ans, il y eût jamais eu de prétention ni de plainte au
contraire.

Les mémoires de part et d'autre, redoublés et imprimés, furent
distribués aux juges et au public. M. le Grand et les siens agissaient
comme dans une affaire dont son honneur dépendait, M. le Premier et les
siens comme dans une affaire où il y allait de tout son état et de toute
sa fortune. Une attaque et une défense si vive et si sérieuse, et le
grand nombre de personnes considérables qui s'intéressaient pour l'une
ou pour l'autre partie, acheva de déconcerter les juges, tellement que
M. de Troyes, sur le point du jugement, s'enfuit à Troyes sous prétexte
d'un reste de déménagement, et ne revint qu'après que l'affaire fut
jugée.

Le régent lui-même ne se trouva pas peu embarrassé. Il voyait trop clair
pour ne pas comparer intérieurement le procédé de M. le Grand à la fable
du loup et de l'agneau\,; mais il avait un faible héréditaire pour les
Lorrains, qui, par Monsieur et par le chevalier de Lorraine, lui avaient
imposé dans sa première jeunesse, et ce faible était soutenu par
M\textsuperscript{me} la duchesse de Lorraine, sa sœur, qu'il aimait
fort, et par le ton haut de Madame, tout Allemande. Les entreprises de
la grande écurie sur la petite ne faiblissaient point, soit que le
régent ne voulût ou ne crût pas pouvoir imposer assez à M. le Grand
là-dessus. Il se repentait de ne l'avoir pas fait d'abord, et ordonné
provisoirement, jusqu'à la majorité, que les choses demeurassent comme
le feu roi les avait laissées. Mais il n'était plus temps, et pour
arrêter cette petite guerre, également indécente, dangereuse et
journalière, rien n'était plus pressé que de juger. C'est aussi à quoi
enfin M. le duc d'Orléans se résolut.

Je savais bien à quoi m'en tenir sur cette affaire, mais je m'y défiai
de moi-même, et je voulus me mettre au large et à mon aise avec moi-même
là-dessus. Je priai l'abbé Pucelle, habile et intègre conseiller clerc
de la grand'chambre, qui depuis est justement devenu célèbre et qui a
toujours joui en ces deux genres de la première réputation, de me donner
une après-dînée de son temps. Il vint chez moi. Nous y lûmes ensemble
tous les mémoires de part et d'autre, nous les discutâmes exactement
pour et contre\,; je lui expliquai cet usage de donner l'ordre qu'il ne
pouvait savoir. Je ne m'ouvris en aucune sorte\,; j'appuyai même autant
que je le pus les raisons de M. le Grand, parce que je ne les trouvais
pas bonnes. J'eus la satisfaction que l'abbé Pucelle fut de mon avis
avant que d'avoir su quel il était, et qu'il me dit nettement que cela
ne faisait pas de question. Je disputai encore contre lui. À la fin, je
lui avouai que j'avais toujours été du même avis que lui\,; mais que
j'avais voulu le lui cacher jusqu'au bout, pour rendre sa décision plus
libre.

Plus je me sentis fixé dans mon avis, plus j'étais en garde et serré
avec le premier écuyer qui venait souvent me faire ses plaintes, et
chercher à me pénétrer. Il me pria avec les dernières instances de lui
prêter le compte rendu à mon père par son intendant de l'année 1643.
Comme ces pièces se conservent toujours dans les maisons qui ont quelque
ordre, je ne pus nier que je ne l'eusse, mais je lui dis qu'étant juge
de son affaire, je me garderais bien de lui rien administrer. Il avait
avancé que mon père avait eu la dépouille de la petite écurie\,; c'était
le dernier exemple et le dernier état. Il fallait le prouver et le
trouver dans ce compte\,; c'était pour lui une preuve transcendante. Il
me pressa beaucoup sans succès, puis me tourna tant qu'il put pour
apprendre si en effet mon père avait eu cette dépouille. Je ne le
satisfis pas plus sur cela que sur les instances de lui montrer au moins
ce compte.

Quatre jours après, le prince Charles vint chez moi avec force excuses
de M. le Grand, que la goutte empêchait d'y venir, qui l'avait chargé de
me prier de vouloir bien lui prêter ce même compte. Sur les difficultés
que je lui en fis, il redoubla ses instances. Je lui dis que M. le
Premier m'avait fait les mêmes, et que je l'avais refusé, mais que si M.
son père et lui le voulaient absolument, je le lui prêterais à deux
conditions\,: l'une, qu'il ne le garderait que trois jours\,; l'autre,
que M. le Grand et lui trouveraient bon que je tinsse la balance égale,
et que je l'envoyasse à M. le Premier dès qu'ils me l'auraient rendu, et
que je le lui laissasse aussi trois jours. Le prince Charles accepta
pour M. son père et pour lui les deux conditions, et il emporta mon
compte. Il fut fidèle à me le rendre au bout de trois jours, et moi à
l'envoyer sur-le-champ à M. le Premier qui en fut bien étonné, et qui
n'avait pas lieu de s'y attendre. Il me le rapporta au bout des trois
jours, bien satisfait d'y avoir trouvé ce qu'il désirait, c'est-à-dire
le compte entier de toute la dépouille de la petite écurie dans ce
compte.

Lorsqu'on fut sur le point de juger, M. le Premier me vint prier de
porter ce compte au conseil de régence. Je le refusai, et lui dis que ce
n'était qu'au rapporteur à porter des pièces, que je ne savais à qui
celle-là pouvait être favorable, contraire ou indifférente, mais que ce
n'était pas à moi à la porter, et que très certainement je ne la
porterais pas. La dispute avait duré\,: le Premier, qui sentait le poids
de la pièce, s'était échauffé, et me dit\,: «\,Mais si M. le duc
d'Orléans vous l'ordonne\,?» Alors j'avoue que je le regardai fixement,
et lui dis d'un ton brusque, mais bien articulé\,: «\,S'il me l'ordonne
verbalement, je n'en ferai rien.\,» Le Premier comprit la réponse, et ne
répliqua pas. Mais je fus surpris que la veille du jugement je reçus un
billet de la main de M. le duc d'Orléans, qui m'ordonnait d'apporter le
lendemain matin le compte rendu à mon père, de l'année 1643, au conseil
de régence, à quoi j'obéis.

J'arrivai le mardi matin, 22 octobre, à Vincennes, pour le conseil
extraordinaire de régence destiné au jugement de ce procès. M. le Grand,
M. le prince Charles ni M. le Premier n'y parurent. Les chefs et les
présidents des conseils y étaient mandés. Le maréchal de Villeroy
parlait à chacun pour M. le Grand son beau-frère\,; le maréchal
d'Huxelles pour M. le Premier, son cousin germain et son ami intime\,;
et tous deux sortirent quand on se mit à prendre place. Comme Torcy,
rapporteur, ouvrit son sac, je tirai de ma poche ce compte de mon père
et le billet de M. le duc d'Orléans, et je dis\,: «\,Messieurs, voilà un
compte de l'année 1643, rendu à mon père par son intendant, et voici un
billet de la main de M. le duc d'Orléans, que je reçus hier, par lequel
il m'ordonne d'apporter aujourd'hui ce compte au conseil.\,» Et en même
temps je mis l'un et l'autre sur la table, au milieu de sa largeur
devant moi. Tous regardèrent sans y toucher, personne ne répondit\,;
jamais je ne vis des visages si embarrassés. Après, Torcy commença son
rapport.

Il le fit nettement, correctement, exactement, n'oublia rien de part et
d'autre, compara les raisons, les commenta, et conclut en tout et
partout en faveur de M. le Premier. Ses termes furent bons et justes,
mais la voix basse, souvent coupée, et faiblit sensiblement aux
conclusions.

Nous étions treize juges ainsi opinants\,: Torcy, rapporteur, les
maréchaux de Besons et d'Estrées, le duc d'Antin, les maréchaux
d'Harcourt et de Villars, le duc de Noailles, moi, Voysin chancelier, le
comte de Toulouse et le duc du Maine, M. le Duc, M. le duc d'Orléans.
Ainsi j'étais à l'ordinaire vis-à-vis du chancelier, auprès du comte de
Toulouse, et le maréchal de Villars auprès de moi ce jour-là.

Le rapport fait, M. le duc d'Orléans ordonna à Torcy de lire l'endroit
du compte de mon père où celui de la dépouille de la petite écurie lui
devait être rendu, en cas qu'il l'eût eue. Je poussai le compte à Torcy,
je repris le billet de M. le duc d'Orléans, je le montrai bien à mes
deux voisins, et je le remis devant moi sur la table. Torcy trouva
l'endroit du compte dont il s'agissait, et le lut. Le régent ensuite
demanda l'avis à Besons, qui barbouilla, et qui proposa une cote mal
taillée. Estrées saisit cet expédient, parla longtemps sans rien dire,
et ne put conclure.

Ce début me parut si misérable pour des juges de cette suprême sorte, et
en tout pour des juges, que je pris la parole. Je dis au maréchal
d'Estrées que nous étions tous là pour dire, non ce qui serait à
souhaiter, et faire des raisonnements étrangers à la question, mais pour
dire nos avis nettement, en conscience\,; qu'il avait parlé, mais point
opiné ni conclu\,; qu'il s'agissait de savoir s'il était pour M. le
Grand ou M. le Premier, en tout ou en partie, et au dernier cas en
quelles parties. Le maréchal fut étourdi. Il barbouilla encore je ne
sais quoi d'indécis\,; je me tournai au régent à qui je dis\,: «\,
Monsieur, il faudrait opiner, et cela ce n'est pas avoir un avis.\,»
Alors le régent dit au maréchal d'Estrées\,: «\,Monsieur le maréchal,
opinez donc, s'il vous plaît, et que nous sachions votre avis, car nous
n'en savons rien encore. » Tout le conseil baissa les yeux, et je ne vis
jamais gens si consternés. Le maréchal d'Estrées, dans un embarras
extrême, se mit à reprendre les points de prétention sans pouvoir se
résoudre à décider. Le régent le pressa encore\,; il décida enfin partie
pour l'un, partie pour l'autre, sans en apporter aucune raison.

Le régent, qui vit qu'il n'en tirerait pas davantage, dit à d'Antin
d'opiner. L'aventure du maréchal d'Estrées lui fut une leçon. Il fit une
préface de compliments pour les deux parties, et sur le malheur de ce
procès\,; il bégaya plus qu'à l'ordinaire, mais il fut pour M. le
Premier sur tous les chefs. Harcourt, qui parla après, et qui déjà
s'énonçait avec difficulté, fut court et de même avis. Villars pouffa,
verbiagea, complimenta les parties, se plaignit du procès, désira des
cotes mal taillées, mais conclut pour M. le Premier. Noailles parut
comme chat sur braise. Il craignit quelque chose de plus fort que ce que
j'avais dit à son beau-frère, car je ne le ménageais pas en plein
conseil. Il eût bien voulu aussi ne point décider, mais il n'osait s'en
dispenser. Cela produisit un long verbiage, mais à la fin il fallut
conclure. Il tenta un avis équivoque de cote mal taillée\,; il se
reprit, il y revint, en sorte qu'on put moins dire ce qu'il avait opiné,
que dire qu'il n'avait pas opiné.

L'impatience où me mit une si méprisable misère fit que je repris
l'affaire d'un bout à l'autre. Je discutai tous les points des
prétentions et des réponses\,; j'exposai plusieurs changements arrivés
dans les grandes et les moindres charges, et les formations d'où et
comment faites, aux dépens de quelles charges, dont je fis l'application
aux questions particulières à juger\,; je m'étendis sur la séparation et
l'indépendance des deux écuries, et du premier et du grand écuyer faite
par Henri III, en faveur de M. de Liancourt, maintenue en entier par ses
successeurs jusqu'alors, en conséquence sur l'ordre donné chaque jour
distinctement et séparément pour les deux écuries, même à un simple
écuyer de la petite en absence du premier écuyer, et en présence du
grand écuyer, sans plainte ni réclamation de sa part, jusqu'après la
mort du roi, sans que cette retenue pût être attribuée à timidité ni à
défiance de considération et de crédit de la part de M. le Grand. Enfin
je montrai toute la force que la cause de M. le Premier tirait du compte
rendu à mon père de la dépouille de la petite écurie, et je conclus
distinctement après sur tous les points l'un après l'autre, en faveur de
M. le Premier. Je remarquai qu'on me prêta grande et silencieuse
attention, et qu'encore que je parlasse longtemps, on ne s'ennuya pas,
peut-être à cause de l'historique qui fut nouveau presqu'à tous.

Le chancelier barbouilla à son ordinaire, s'affligea de la naissance et
du progrès de la contestation, plus encore de la difficulté d'une cote
mal taillée, et finit enfin par être de mon avis. Les deux bâtards, qui
aimaient bien mieux le premier écuyer, qui sourdement et cauteleusement
était attaché au duc du Maine, firent l'un après l'autre un petit
compliment pour M. le Grand, et opinèrent nettement et entièrement
contre lui. M. le Duc, sans compliment ni remarque, dit en deux mots
qu'il était d'avis sur tous lés points que M. le Premier était fondé et
y devait être maintenu.

Alors ce fut au régent à parler et à prononcer. Par l'exposé que je
viens de faire, auquel la singularité de l'embarras des juges m'a
engagé, on voit que l'arrêt était fait dès lors, et que le premier
écuyer avait pleinement et entièrement gagné tout. Il n'y avait donc
plus qu'à prononcer. Néanmoins le régent, aussi embarrassé que les
autres juges, dit qu'il paraissait qu'on n'était pas bien d'accord sur
la dépouille, et même sur d'autres articles, dont quelques-uns ne
s'étaient pas bien expliqués. Par ce qu'il ajouta, il montra qu'il
tendait lui-même à une cote mal taillée, qu'il voulait sauver la charge
de premier écuyer, et ne la pas soumettre au grand écuyer, mais qu'il
désirait en même temps compenser cela par quelque extension de
l'autorité du grand écuyer sur la petite écurie, au delà du deuil,
surtout apaiser M. le Grand en lui en adjugeant la dépouille.

Je pris la parole dès qu'il eut fini. Je lui dis que les prétentions de
M. le Grand n'étaient pas de nature à pouvoir être séparées, qu'elles
étaient toutes fondées sur celle de l'entière dépendance, comme les
défenses du premier écuyer sur chaque article n'avaient d'appui que dans
celle de son indépendance et de la séparation et soustraction de la
petite écurie de toute autorité et inspection du grand écuyer faite par
Henri III pour M. de Liancourt, qu'on ne pouvait se dissimuler, ni M. le
Grand lui-même, avoir duré entière et sans atteinte jusqu'alors\,; que
le titre y était donc par le fait d'Henri III\,; que l'usage et la
possession constante y était de même jusqu'alors par l'usage non
interrompu et non contesté par aucun des grands écuyers sous Henri IV,
Louis XIII et le feu roi\,; que rien n'y manquait donc pour former un
droit certain, constant et stable, ou que rien ne pouvait être assuré\,;
qu'enfin pour la dépouille, elle avait le même fondement, le même titre,
la même possession, puisque MM. de Liancourt père et fils l'avaient eue
sans réclamation ni plainte des grands écuyers à la mort d'Henri III et
d'Henri IV, et que le compte rendu à mon père, qui venait d'être lu par
son ordre, faisait foi que mon père l'avait eue pareillement à la mort
de Louis XIII. L'attention du conseil fut encore plus grande à cette
réplique, et il parut à l'air du régent, non à aucune parole, qu'il s'en
serait passé. Mais moi, voyant un arrêt fait et juste, j'eus peur que
faiblesse, crainte, complaisance n'y donnassent atteinte, et je crus
devoir à l'équité d'aller à temps au-devant.

Le régent, quand j'eus fini, dit qu'il suffirait de reprendre les voix
en deux mots de chacun, sans opiner de nouveau. Il n'y eut que Besons
qui balbutia encore, Noailles moins, mais encore un peu. Tous les autres
parlèrent net en deux mots en faveur du premier écuyer, excepté le
maréchal d'Estrées qui tâcha de faire une différence de la dépouille, et
qui s'y barbouilla.

Quand tous eurent dit cette seconde fois leur avis en deux mots, je ne
doutai plus que le régent n'allât prononcer. Point du tout. Il dit qu'il
voyait bien que tous les suffrages décidaient pour l'entière séparation
et la totale indépendance, et pour laisser les choses sur le pied où
elles avaient été sous le feu roi\,; que c'était aussi son sentiment,
mais qu'il ne voyait pas la même uniformité sur la dépouille\,; que
lui-même y trouvait quelque difficulté\,; qu'il serait bon qu'omettant
le reste comme jugé, chacun s'expliquât encore nettement sur la
dépouille. «\,Et le compte de mon père, monsieur, repris-je tout haut,
que vous m'avez commandé d'apporter ici par votre billet que voilà\,!
N'est-il pas décisif là-dessus, à la suite du même exemple de MM. de
Liancourt père et fils, indépendamment que la dépouille coule du même
principe que tous les autres articles tenus pour jugés\,?» Ce mot, dit
un peu ferme, frappa tout le monde. Les balbutieurs ne surent qu'y
opposer. Ils haussèrent les épaules, et d'une voix assez basse
convinrent que la dépouille devait appartenir au premier écuyer. Tous
les autres furent du même avis, et le dirent très ferme. Le régent
baissa la tête, ce que je remarquai bien, et enfin prononça.

Alors, craignant par ce que j'avais vu de penchant et de faiblesse, que
les cris, l'impétuosité et les appuis de M. le Grand n'obtinssent des
choses contraires à ce qui venait d'être jugé, je proposai au régent
l'importance que Torcy écrivît le détail des choses jugées, c'est-à-dire
le fond inaltérable de l'arrêt, et le lût avant que le conseil levât. Le
régent le trouva bon, et l'ordonna à Torcy. Il se mit donc à écrire,
puis il dit tout haut chaque chef comme il l'allait écrire avant de le
mettre sur le papier. J'eus soin sur chacun de dire tout haut comme il
avait passé quand Torcy paraissait douter, comme il lui arriva souvent,
apparemment pour être plus assuré de ce qu'il écrirait. Personne ne dit
mot, même le régent, tellement que plusieurs du conseil dirent que
j'avais fait et dicté l'arrêt. Torcy, après avoir achevé, lut tout haut
ce qu'il venait d'écrire, qui fut approuvé de tous à la fois sans ordre
d'opinions\,; et cependant La Vrillière, ami intime du premier écuyer,
écrivait aussi sur le registre du conseil, qui leva aussitôt après que
Torcy eut achevé de lire, et eut signé ce qu'il avait écrit.

Je sortis du conseil avec le comte de Toulouse, causant de ce qui venait
de se passer, et de ce qu'eût pu devenir Beringhen à son âge, s'il eût
perdu son procès, c'est-à-dire sa charge, et avec elle sa fortune et son
être. Tournant sur le grand degré pour le descendre, {[}nous
trouvâmes{]} des Épinay, vieil écuyer de la petite écurie, et fort
attaché de tout temps à Beringhen, qui était là plus mort que vif,
embusqué dans un coin pour apprendre le sort de l'affaire, qui nous la
demanda véritablement comme un homme demi-mort. Le comte de Toulouse
avec son froid lui répondit que M. de Torcy le lui apprendrait. Des
Épinay insista comme un mendiant. La pitié m'en prit, et du premier
écuyer qui l'avait envoyé. Je dis au comte de Toulouse\,: «\,Pourquoi le
faire languir pour un secret qui va être public dans quatre ou cinq
minutes\,? » Tout de suite je me tournai à des Épinay et lui dis\,:
«\,Allez, monsieur des Épinay, M. le Premier a gagné en plein\,:
indépendance, dépouille, en un mot, tout sans exception.\,» Cet homme,
qui était vieux, et le même qui du temps du roi était attaché au
carrosse de M\textsuperscript{me} de Maintenon, se jeta à mes genoux, me
dit d'une voix faible et entrecoupée que je lui rendais la vie, qu'il
l'allait rendre à M. le Premier, et vola à l'instant par le degré, {[}de
sorte{]} que nous le perdîmes de vue que nous n'étions qu'à la troisième
marche. J'allai dîner chez le marquis du Châtelet, où j'appris que le
premier écuyer, sa femme et quelque peu de leurs plus intimes amis,
étaient cachés dans le premier pavillon d'entrée, tout près de la porte
de la basse-cour du château qui mène au village\,; qu'ils ne voulaient
pas qu'on les y sût\,; et qu'ils avaient leurs carrosses cachés aussi,
et tout attelés, pour s'en aller de là droit chez eux à Armainvilliers,
s'ils perdaient ce procès, à l'instant qu'ils en auraient la nouvelle.
Elle fut bien différente pour eux. Des Épinay arriva à toute course qui
ne pouvait plus parler, et qui enfin les mit au large et dans la joie.

Le premier écuyer ne tarda pas à me venir remercier dès que je fus à
Paris. Je ne sais par qui il avait su jusqu'au dernier détail de tout ce
qui s'était passé au jugement de son affaire\,; j'imaginai que ce fut
par La Vrillière. Beringhen en transissait encore, et me répéta bien des
fois que je lui avais sauvé sa charge et sa fortune, et plus que cela,
l'honneur et la vie\,; qu'il me devait tout cela, et que lui et les
siens ne l'oublieraient jamais.

Je dois cette justice à M. le Grand, et à M. le prince Charles, son
fils, qu'ils ne me surent pas le moindre mauvais gré\,; qu'il ne leur
est jamais depuis rien échappé à mon égard\,; et qu'ils ne m'ont jamais
donné le plus léger soupçon qu'ils n'aient pas été satisfaits de toute
ma conduite\,; et que tout ce qui tenait à eux les a imités en cela.

Le premier écuyer ne fut pas longtemps sans me parler de l'extrême désir
de sa femme de me venir témoigner la reconnaissance dont elle était
pénétrée, et leur douleur commune de n'oser l'entreprendre dans les
dispositions où tous deux me savaient pour elle, dont il est vrai que je
ne m'étais pas tu, et sans ménagement. Je lui dis que c'était une peine
que je le priais de l'empêcher de se donner, parce que ma porte lui
serait exactement fermée. Il voulut entrer en justification pour elle,
non en tout, mais en partie, et insister sur son repentir et sa douleur.
Je répondis que j'étais trop bien informé pour que les justifications et
les explications eussent sur moi aucune prise, que je savais très bien à
quoi m'en tenir avec elle, et que je le priais de ne m'en pas parler
davantage.

M\textsuperscript{me} de Beringhen était parfaitement fausse, basse,
intrigante, non seulement dangereuse, mais fort méchante, avec l'air
humble et modeste, les propos les plus doux et les plus séduisants,
toujours dans les intérêts et dans les sentiments des gens à qui elle
parlait\,; jamais rien sans vues et sans desseins, avide d'argent et
d'affaires les plus sales, avec un air d'aisance, de dépense, de
désintéressement\,; toujours merveilleusement parée, quoique très laide,
et rien moins que jeune, fort glorieuse en dessous, tant qu'elle pouvait
dans les cabales, ayant été toujours fort avant dans celle de Meudon,
désolée de ce qu'ils n'avaient pu parvenir au duché, quoiqu'elle ne pût
ignorer qui était son mari. Elle avait plus d'esprit encore que le duc
d'Aumont, et infiniment liant. C'était son bon et cher frère, aussi
étaient-ils en tout parfaitement homogènes. Elle avait été longtemps
toujours à la cour, à Marly, de tous les voyages, de toutes les fêtes.
On n'a jamais découvert la cause de sa disgrâce, que toute la bonté du
roi pour son mari, et la familiarité qu'il eut toute sa vie, ni la
considération de la nécessité où il était de ne bouger d'où était le
roi, ne put jamais diminuer. Les quinze dernières années du feu roi au
moins elle n'était plus de rien, et n'allait à la cour que deux ou trois
fois l'année passer au plus deux jours, mais quelquefois à Meudon, quand
il y avait des dames et que le roi n'y était pas\,; jamais même à
Fontainebleau. Cela était fort remarqué\,; mais ils étaient si sages et
si cachés qu'on n'en fut pas plus instruit. Le Premier, qui aimait fort
sa femme, et à être avec cette flatteuse, en était secrètement,
amèrement affligé, mais il ne put rien changer à cette disgrâce, qui
dans les premiers temps bannit sa femme de la cour, sans y oser paraître
du tout pendant quelques années.

Il me poursuivit plus de six semaines pour voir sa femme, avec une
assiduité qui me désolait et qui enfin me vainquit. Elle vint donc un
matin seule avec son langage composé où elle mit toute l'éloquence qui
lui fut possible, qu'elle accompagna de beaucoup de larmes. Je la reçus
avec toute la civilité, mais avec toute la froideur possible. Je lui dis
qu'il ne s'agissait point de s'expliquer sur ce qui s'était passé chez
elle à mon égard, que je n'en ignorais rien, que je savais à quoi m'en
tenir, que je voulais bien croire qu'elle en était fâchée, que cela ne
m'avait pas empêché de rendre justice à M. le Premier. Du reste, je la
payai de compliments secs, sans me rendre à ses protestations, ni à tous
ses empressements pour obtenir oubli et mon amitié. Il n'y eut rien
qu'elle ne me dît pour m'assurer que, quelque rigueur que je lui tinsse,
rien n'égalerait à jamais sa reconnaissance, son attachement, son
respect pour moi, car elle ne ménagea aucun terme, et pour me les
témoigner par toute sa conduite. Tous ces verbiages durèrent une bonne
heure tête-à-tête, et quoique de ma part la sécheresse se fît soutenue
jusqu'au bout à travers toute la politesse dont je la pus tempérer, son
mari vint me remercier le lendemain de l'avoir reçue, et me dit encore
merveilles pour elle.

Elle m'est depuis revenue voir quelquefois du vivant de M. le Premier,
jamais depuis. Je la voyais chez son mari quelquefois\,; jamais je ne
lui ai rendu de visite. Le Premier me dit bien des fois depuis le
jugement que je l'avais étrangement mis en peine par le serré et le
concis dont je lui parlais, qui lui avait fait tout craindre de ma part
pour la décision de son affaire, laquelle fut fort approuvée du public.

J'eus lieu de me savoir gré d'avoir fait dresser l'arrêt tout de suite
dès qu'on l'eut prononcé. M. le Grand vint au Palais-Royal, criant qu'on
l'avait égorgé, et tempêta tant que le régent lui permit de faire telles
protestations qu'il voudrait contre le jugement que le conseil de
régence, c'est-à-dire que le roi même venait de rendre (car il était de
pareille force ainsi que tout ce qui émanait de ce conseil) et lui signa
un ordre à tout notaire qu'il voudrait choisir de recevoir ses
protestations et de lui en donner acte. Outre la misère d'une faiblesse
si honteuse qui allait à saper l'autorité et la stabilité de tout ce que
le conseil de régence pouvait ordonner, le régent n'en prévit pas les
autres conséquences. M. le Grand fit donc ses protestations, publia
qu'il ne se tenait pas pour battu, et qu'à la majorité il espérait avoir
justice.

Des paroles il passa tôt aux effets. La guerre recommença par les
usurpations et les attaques de la grande écurie contre la petite, avec
la même indécence, la même fréquence, le même danger qu'avant le
jugement, que M. le Grand traita toujours de nul, fondé sur la
permission qu'il avait obtenue de protester contre, en sorte que, dans
le fait et à la dépouille de la petite écurie près, que le premier
écuyer eut, ce dernier ne se trouva ni mieux ni plus en sûreté qu'avant
le jugement. Les plaintes qu'il en porta au régent furent écoutées\,;
mais ce fut tout. Ce prince n'imposa point\,; et les embûches, les
entreprises et les combats furent journaliers.

Achevons cette matière, puisqu'elle se présente si naturellement,
quoiqu'elle dépasse la mesure du temps que j'ai compté de donner, si je
vis, à mes Mémoires. Le prince Charles continua les mêmes entreprises
journalières, à force ouverte, après la mort de M. le Grand, arrivée en
1718. Le premier écuyer n'opposait que sagesse et plaintes inutiles,
dont le chagrin, qui se renouvelait tous les jours, le conduisit enfin
amèrement au tombeau en 1723, et le lui avança. Il n'est pas de ce temps
d'expliquer par qu'elle fortune son fils obtint enfin sa charge, que M.
le duc d'Orléans assurément ne lui destinait pas, et qu'il n'eut que par
la mort de ce prince, arrivée bien à propos pour lui, sans qu'il eût
disposé de la charge, pendant plus de sept mois qu'il l'aurait pu.

Par autre fortune M. de Fréjus avait été fort des amis de Beringhen et
de sa femme. Il venait de faire M. le Duc premier ministre, qui était
obligé de compter fort avec lui. Fréjus fit sa propre affaire de celle
du premier écuyer. Il la fit décider de nouveau, mais sans forme de
jugement, suivant en tout celui qui avait été rendu par le conseil de
régence. Le roi était majeur\,; ainsi les protestations du grand écuyer
tombèrent, et il n'y eut plus pour lui à en revenir. M. le Duc et M. de
Fréjus lui parlèrent si ferme qu'il n'osa plus rien entreprendre sur la
petite écurie, ni tenter les voies de fait. Ainsi le nouveau premier
écuyer jouit, en entrant en charge, d'une paix et d'un repos auxquels
son père n'avait pu parvenir depuis la mort du feu roi.

Le prince Charles, piqué de voir ses prétentions condamnées sans retour,
refusa de signer à l'ordinaire, sans examen, les dépenses de la petite
écurie, lorsqu'elles lui furent portées avec la signature du premier
écuyer. Celui-ci, son nouvel arrêt en main, refusa de s'y soumettre, et
prétendit que le prince Charles devait, comme son père, son grand-père
et tous les autres grands écuyers, depuis Henri III, signer sans voir,
sur la simple inspection de la signature du premier écuyer. Les choses
demeurèrent assez longtemps ainsi. Cependant il fallait les finir pour
porter ces dépenses à la chambre des comptes. On tâcha de vaincre
l'opiniâtreté du prince Charles, et par raison et par exemples\,; on ne
put le persuader. À la fin, M. le Duc, qui était premier ministre,
déclara au prince Charles que, s'il persistait au refus, lui, M. le Duc,
comme grand maître de la maison du roi, signerait les dépenses de la
petite écurie, et les enverrait ainsi à la chambre des comptes. Le
prince Charles lui répondit qu'il ferait tout ce qui lui plairait, mais
qu'il ne les signerait pas sans les examiner. M. le Duc les signa donc
comme grand maître de France\,; et de cette manière le grand écuyer
perdit le droit de les signer, ou plutôt l'usage, qui était un des plus
beaux restes de son ancienne supériorité sur la petite écurie et sur le
premier écuyer du roi.

\hypertarget{chapitre-xi.}{%
\chapter{CHAPITRE XI.}\label{chapitre-xi.}}

1715

~

{\textsc{Mariage de Sandricourt qui me brouille pour toujours avec
lui.}} {\textsc{- Obsèques du roi à Saint-Denis.}} {\textsc{- Caractère
de Dreux.}} {\textsc{- Le régent veut la confusion et la division.}}
{\textsc{- Je veux me retirer de tout à la mort du roi, et je me laisse
raccrocher malgré moi par M. le duc d'Orléans.}} {\textsc{- Conduite de
ce prince à l'égard des ducs.}} {\textsc{- Courte comparaison des
assemblées de la noblesse en 1649 et en 1715.}} {\textsc{- Ressorts et
fanatisme de celle-ci.}} {\textsc{- Le régent trompé sur cette prétendue
noblesse.}} {\textsc{- Étrange personnage du duc de Noailles.}}
{\textsc{- Le régent trompé sur le parlement.}} {\textsc{- Menées du duc
de Noailles pour diviser les ducs, et faire tomber leurs poursuites
contre les usurpations du parlement à leur égard\,; à quoi enfin il
réussit.}}

~

On a pu voir quelque part, au commencement de ces Mémoires, que j'avais
pris le même soin du marquis de Sandricourt que s'il eût été mon fils.
Nous sommes de même maison, quoique de branche séparée depuis plus de
trois cents ans. J'ai toujours aimé mon nom\,; je n'ai rien oublié pour
élever tous ceux qui l'ont porté de mon temps\,; je n'y ai pas été
heureux. Son père et sa mère, gens de beaucoup d'esprit, mais avares,
obscurs, fort retirés, n'avaient point d'autres enfants. Ils étaient
riches en belles terres en Picardie\,; ils ne bougeaient de chez mon
père, et après de chez moi.

Je procurai une compagnie de cavalerie à leur fils de fort bonne heure,
et le premier usage que je fis de l'amitié de Chamillart fut de faire
donner fort tôt après à ce jeune homme l'agrément au régiment de Berry
cavalerie, que Yolet, très bon officier, vendit de dépit de n'être pas
maréchal de camp. La cherté effraya le père\,; je m'obligeai à le payer,
et priai Yolet de faire le marché au mot du père, et que je donnerais le
surplus. Le père, étonné d'un si grand et si prompt rabais, se douta de
ce que j'avais fait, se piqua, et conclut, à peu de chose près, qui
demeura sur mon compte, et qu'ils m'ont rendu depuis. Ce régiment alla
bientôt en Espagne. M\textsuperscript{me} des Ursins y régnait, et je
pouvais compter sur elle\,; M. le duc d'Orléans y commanda l'armée
bientôt après\,; il eut toutes les bontés les plus marquées pour
Sandricourt, et M\textsuperscript{me} des Ursins lui donna une
protection distinguée. Je le recommandai aussi à tout ce que je connus
qui le pouvait servir et même conduire. Il avait de la valeur et de la
volonté\,; en trois ans Chamillart le fit brigadier, aux cris de la
foule de ses cadets d'Italie, d'Allemagne et de Flandre. Il fit un tour
à Paris l'hiver d'après le mariage de M. le duc de Berry. Je l'eus chez
moi à la cour, le présentai partout, et lui fis donner les entrées chez
ce prince, sous prétexte qu'il commandait son régiment. À son retour, à
la paix, j'en usai de la même manière, et je crus pouvoir le former au
monde après l'avoir vu plusieurs campagnes à la guerre, où il s'était
acquis de la réputation.

Il y avait déjà longtemps que son père et sa mère le voulaient marier.
Je les en avais toujours détournés comme d'une chose prématurée à l'âge
et au grade militaire de leur fils qui, en avançant en âge et en
fortune, ne pouvait que trouver des partis plus avantageux, et propres à
avancer sa fortune. Surtout je les exhortais à profiter de leur
situation heureuse sans dettes, avec près de cinquante mille livres de
rente en belles terres depuis Paris jusqu'à Abbeville, pour ne pas faire
de mésalliance, dont leur fils m'avait toujours paru infiniment éloigné.

Voyant leur empressement de le marier devenu incapable de raison, nous
pensâmes, M\textsuperscript{me} de Saint-Simon et moi, à chercher à les
satisfaire d'une manière convenable, et nous crûmes trouver tout dans
M\textsuperscript{lle} de Risbourg. Le marquis de Risbourg, son père,
était petit-fils du frère du prince d'Espinoy, du fils duquel prince
d'Espinoy il a été parlé ici plus d'une fois, qui était mort il y avait
déjà quelques années, et de la veuve duquel, sœur de
M\textsuperscript{lle} de Lislebonne, il a encore été plus souvent
mention dans ces Mémoires. Ce marquis de Risbourg, dont il s'agit ici,
avait suivi en Espagne la fortune de son père et de son grand-père, qui
s'y étaient attachés, et il y était demeuré au service de Philippe V. Il
était alors grand d'Espagne, chevalier de la Toison-d'Or, colonel du
régiment des gardes wallonnes, vice-roi de Catalogne, et résidait à
Barcelone. Il était veuf, riche, et n'avait que deux filles, dont
l'aînée, fort dévote, avait renoncé au mariage, et qui toutes deux
vivaient ensemble dans leurs terres en Flandre, ou dans nos villes qui
en étaient voisines, avec une grande bienséance et beaucoup de
réputation de vertu. Leur père ne voulait point se remarier, était assez
singulier. Tous ses biens de Flandre et tout ce qu'il avait amassé en
Espagne, qui allait à beaucoup, revenait donc après lui à ses filles, et
plus que tout cela sa grandesse après lui. Il avait depuis longtemps mis
toute sa confiance en la princesse d'Espinoy, dont je viens de parler\,;
elle avait sa procuration pour gouverner ses biens de Flandre, et pour
la conduite personnelle de ses filles, et leur commerce de lettres et
d'amitié était continuel.

Personne de distingué n'avait pensé à un si grand parti, mais peu connu
et relégué, et plus douteux encore par l'âge et la situation du père, à
qui il pouvait prendre envie de se remarier. Nous en parlâmes à
Sandricourt, et à son père et à sa mère, qui regardèrent cette affaire
comme la plus grande qu'ils pussent faire, et telle qu'ils ne l'osaient
espérer. En effet, tout y était\,: biens, alliance, la plus grande
naissance, un père dans les premiers honneurs et emplois, et par ce que
nous savions de son éloignement pour un second mariage, certitude de sa
grandesse après lui. Les Sandricourt nous pressèrent de voir ce qu'ils
en pourraient espérer.

M\textsuperscript{me} de Saint-Simon en parla à M\textsuperscript{me}
d'Espinoy, qui reçut la proposition avec toute sorte d'agrément. Elle
convint de tout l'éloignement du marquis de Risbourg de se remarier,
parla franchement sur la confiance qu'il avait en elle, et promit de lui
en écrire au plus favorablement.

À peine sa lettre était-elle partie, que les Sandricourt nous vinrent
dire que cette affaire ne réussirait jamais\,; qu'ils étaient pressés de
marier leur fils\,; qu'il n'y avait rien de meilleur que de s'allier à
la robe pour la conservation des droits des terres, et pour les procès
qui pouvaient survenir\,; et qu'ils étaient résolus à le faire. Le fils
vint me trouver, fit le désolé, me conjura de ne le point abandonner à
la fantaisie de son père et de sa mère. Il en dit autant à ma mère et à
M\textsuperscript{me} de Saint-Simon, et nous le crûmes de bonne foi.

Il est aisé d'imaginer ce que nous dîmes au père et à la mère, surtout
la lettre de la princesse d'Espinoy au marquis de Risbourg étant partie.
Leur embarras fut grand, mais leur opiniâtreté la fut davantage. Ils ne
parlaient qu'en général, et nous espérions qu'avant qu'ils eussent
trouvé, et le jeune homme persistant dans les sentiments qu'il ne
cessait de nous témoigner, l'affaire s'engagerait avec le marquis de
Risbourg, et que nous ferions le mariage. Cette espérance ne dura pas
longtemps.

Deux jours après, le jeune homme bien empêtré me vint dire que son
mariage était fait avec M\textsuperscript{lle} de Gourgues. Je m'écriai,
et lui demandai s'il y consentait. Il répondit qu'il n'osait résister à
son père et à sa mère qui voulaient la robe absolument. Je le menai à ma
mère et à M\textsuperscript{me} de Saint-Simon, qui lui représentèrent
tout ce qu'il était possible. À la fin je lui dis que s'ils avaient la
rage de la robe au point de la préférer à une fille fort riche de là
maison de Melun, qui ferait avec certitude son mari grand d'Espagne, et
au point encore de ne pas attendre la réponse du marquis de Risbourg à
M\textsuperscript{me} d'Espinoy, après nous avoir engagés à lui en faire
écrire par elle, il fallait du moins choisir une famille honnête et qui
pût lui être de quelque utilité\,; que le père de celle qu'il voulait
épouser était un maître des requêtes si étrangement déshonoré, que le
chancelier de Pontchartrain m'avait dit avoir reçu une députation en
forme des maîtres des requêtes, pour lui demander de faire défaire
Gourgues de sa charge, lequel n'osait plus depuis se présenter au
conseil\,; que son père, qui n'avait guère meilleure réputation, avait
pourri maître des requêtes, sans avoir jamais pu être intendant\,; que
le frère de celui-là, évêque de Bazas, était le mépris de la Gascogne\,;
qu'en un mot, s'ils voulaient déterminément la robe, ils nous donnassent
loisir de sortir honnêtement d'avec M\textsuperscript{me} d'Espinoy\,;
et que s'il voulait M\textsuperscript{lle} Pelletier, je pouvais faire
cette affaire-là par Coettenfao qui était leur ami intime et le mien\,;
qu'elle était fille d'un premier président, sœur d'un président à
mortier (depuis aussi premier président), petite-fille d'un ministre
d'État et contrôleur général, nièce de Pelletier de Sousy et de son fils
des Forts, tous deux conseillers d'État, et actuellement en place et en
grande considération\,; qu'au moins c'était une robe illustrée en son
état, et en situation de lui être utile. Ma mère et
M\textsuperscript{me} de Saint-Simon le pressèrent là-dessus comme je
venais de faire. Mais nous parlions un sourd et, qui pis étais, à un
amoureux, ce que nous ne sûmes qu'après.

C'était le matin. L'après-dînée M\textsuperscript{me} de Sandricourt
vint chez moi comme une furie. Je la laissai dire, comme on souffre les
fous. De chez moi elle monta chez ma mère, qui ne fut pas si endurante,
qui lui apprit sur sa future belle-fille ce qu'elle ne voulut pas
croire, quoique connu de tout le domestique de son père et de beaucoup
de gens, et lui prédit tout ce qui leur est arrivé depuis.
M\textsuperscript{me} de Sandricourt sortit plus en furie que jamais.
Son mari ne parut point chez nous. Cinq ou six jours après ils firent
leur mariage.

Le rare fut que ce bel époux alla de porte en porte, chez tout ce qu'il
put connaître de la robe, dire que je l'avais en telle horreur, que
j'avais rompu avec eux pour s'y être alliés. L'affaire du bonnet était
alors en grand mouvement\,; on peut juger de l'effet de ce discours qui
se répandit partout. Après un trait si noir d'ingratitude, de tromperie
et d'atroce calomnie, nous ne voulûmes plus ouïr parler d'eux, et
oncques depuis ne les avons vus.

Le père et la mère vécurent assez pour avoir vu et senti les vérités
dont ma mère avertit M\textsuperscript{me} de Sandricourt, la dernière
fois qu'elle l'ait jamais vue, e tous deux en sont morts dans la
douleur. Leur fils plus bénin, quelque temps amoureux, après mourant de
peur de sa femme, qui ne s'est guère embarrassée de mesures ni de
précautions, s'est mis à la mode en doux et soumis serviteur. Il n'a
point manqué d'enfants, mais souvent d'argent, sans pourtant en
dépenser, et a vécu obscur dans son quartier. Il n'a pas laissé de
servir et de devenir lieutenant général, jusqu'à la guerre de Bohême\,;
mais son peu d'esprit, son triste mariage, et l'obscurité qui en est
résultée, l'ont accablé, en sorte qu'on l'a laissé depuis en oubli, et
sans aucune sorte de récompense. M\textsuperscript{lle} Pelletier, que
je lui avais proposée, épousa depuis le marquis de Fénelon, longtemps
ambassadeur en Hollande, aujourd'hui lieutenant général, gouverneur du
Quesnoy, conseiller d'État d'épée, et chevalier de l'ordre.

Le vendredi 25 octobre, les obsèques solennelles du feu roi se firent à
Saint-Denis, où tout se passa dans une confusion si grande, et d'une
manière si éloignée de ce qui s'était pratiqué à celles de Louis XIII,
d'Henri IV et de tous les prédécesseurs, que je m'en épargnerai le
récit, qui ne pourrait se passer d'une longue dissertation.

Dreux était grand maître des cérémonies, comme on l'a vu en son temps,
par son mariage avec la fille de Chamillart. Son ignorance et sa
brutalité étaient égales, et au comble. Il a su montrer l'une et l'autre
à la guerre, où, malgré sa valeur et sa faveur, il s'est fait détester
et mépriser. Sa bêtise ne diminuait rien de son orgueil, qui, dans le
désespoir de la bassesse plus que très crasseuse de sa naissance, que sa
charge, son alliance, les richesses des usures de son père, ni le titre
de marquis, si plaisamment imposé par lui au nom de sa famille, ne
pouvaient recrépir, ne perdait pas une occasion de s'en venger contre la
vérité, contre le témoignage de ses registres, et contre son honneur,
dont en ce genre il ne faisait pas grand cas.

Je dis contre ses registres, parce que je les ai tous jusqu'à une époque
où pendant qu'il était à l'armée, sa femme, qu'il ne méritait pas, me
les prêta tous un à un, et je les fis copier et bien collationner\,; et
c'est sur cela que je dis qu'il allait contre ses registres, parce que
je l'y pris, et qu'il en demeura court lorsque M\textsuperscript{me} de
Saint-Simon conduisit un enfant de M\textsuperscript{me} la duchesse de
Berry à Saint-Denis. Il refusait un honneur qui était dû, je lui citai
son registre\,; il fut honteux et confus, et obligé de céder. Il avait
su apparemment, à son retour de l'armée, longtemps avant ce fait, que sa
femme m'avait prêté ses registres\,; il lui en fit un si étrange vacarme
que je n'ai pu y revenir depuis.

Je ne crois pas qu'il y ait de jugement téméraire à penser qu'il y aura
écrit tout ce qu'il lui aura plu. On a vu ( t. II, p.~80 et suiv.) le
silence de Sainctot, maître des cérémonies alors, dans les siens, et (t.
IV\,; p.~163) la fausseté de Châteauneuf dans ceux de l'ordre du
Saint-Esprit, dont je ne rappellerai point ici les sujets qui se
trouvent aux pages indiquées. Ces messieurs écrivent seuls dans les
ténèbres, sans contradicteur ni inspecteur, et prétendent faire ainsi
des lois. Les registres ne se faisaient pas autrefois de la sorte\,; et
la probité de ces nouveaux venus, si solennellement reconnue pour telle
qu'elle est par ces tristes découvertes, ne saurait plus faire
d'illusion à personne.

À l'égard de ces obsèques du roi, M. le duc d'Orléans ne se souciait
d'aucun ordre ni d'aucune règle. On ne fut pas longtemps à s'apercevoir
qu'il avait mis sa politique, tant en choses générales qu'en
particulières de toute espèce, à faire naître des disputes\,; et bientôt
ce mot favori lui échappa comme un axiome admirable dans la pratique\,:
\emph{Divide et regna}. Il laissa donc faire la pompe funèbre comme on
voulut\,: Dreux en fut le maître, et il y signala toutes ses bonnes
qualités.

Les ducs d'Uzès, de Luynes et de Brissac furent nommés pour porter la
couronne, le sceptre et la main de justice, comme les plus anciens à
pouvoir faire cette fonction. Ils étaient dans les hautes chaires, du
même côté que les trois princes du deuil, dont M. le duc d'Orléans était
le premier\,; et tout de suite après eux, une stalle vide entre le
dernier de ces trois princes et le duc d'Uzès, par conséquent au-dessus
de toutes les cours supérieures, et ils avaient aussi leurs carreaux.

La cérémonie commencée, Dreux s'étant approché au bas de la stalle de M.
le duc d'Orléans, pour en recevoir quelque ordre, M. d'Uzès s'avança par
devant les deux autres princes du deuil, et dit à Dreux qu'il le priait
de se souvenir que les trois ducs devaient être salués avant le
parlement. Dreux répondit net et court qu'il n'en ferait rien. Il était
fils de ce conseiller de la grand'chambre qu'on a vu qui avait fait la
lecture du testament du roi en la séance du parlement pour la régence.
Ainsi son fils n'avait garde de n'être pas pour le parlement, où la
charge de son père était, avant la sienne, le premier décrassement de sa
bassesse. M. d'Uzès se contenta de lui demander par quelle raison.
«\,Parce que cela ne se doit pas\,», répondit insolemment et faussement
ce menteur, car ses propres registres, que j'ai, portent que les ducs
furent sans difficulté salués avant le parlement aux obsèques de Louis
XIII, d'Henri IV, etc. Leur dignité le comporte, les symboles de la
royauté portés entre leurs mains l'exigent, leur séance actuelle
au-dessus du parlement le prouve avec évidence. M. d'Uzès insista, Dreux
brutalisa toujours, insista contre son su sur ses registres.

Ce n'était pas là le moment de les voir\,; il fut cru sur la plus que
périlleuse parole par M. le duc d'Orléans, qui était entre eux comme en
tiers, et qui n'entra que faiblement dans ce laconique pourparler. Il ne
se souciait pas des règles ni des dignités\,; il voulait ménager le
parlement, surtout dans ces commencements, il n'était pas fâché de
laisser naître une querelle de plus.

M. d'Uzès déclara très mal à propos à Dreux que les ducs ne lui
rendraient point le salut, s'ils ne le recevaient de lui, qu'après
l'avoir fait au parlement. Il fallait le lui refuser sans l'en avertir.
Dreux répondit avec impudence qu'il ne saluait point qui ne le saluait
pas\,; et, bien averti par la sottise de. M. d'Uzès, salua le parlement,
et ne salua point les ducs. Ils protestèrent au sortir de là sur tout ce
qui s'était passé, et il n'en fut autre chose.

On verra bientôt combien peu le régent eut lieu de s'applaudir de ses
égards, c'est trop peu dire, de son respect et de sa frayeur du
parlement, qui non seulement lui disputa toutes choses, mais jusqu'au
rang personnel, qu'il força le régent, de malepeur, à lui abandonner. Je
ne fais ici que cette remarque simple, le fait sera expliqué en son
temps.

Je n'avais senti que sa mollesse à la mort du roi, tant sur ce qui le
regardait si personnellement, et qui a été expliqué alors, que sur ce
qu'il me devait de justice sur l'inouïe scélératesse du duc de Noailles
à mon égard. Aussi voulus-je faire retraite, et je me tins chez moi sans
en sortir. M. le duc d'Orléans en fut en peine, et sans vouloir mieux
faire, ne voulut pas me laisser dépiter. Il m'envoya coup sur coup
l'abbé Dubois me conjurer de retourner chez lui, de ne l'abandonner
point dans cette première crise, de pardonner aux conjonctures, de
compter entièrement sur son amitié, sa confiance, sa reconnaissance, en
un mot les plus beaux discours du monde. J'eus grande peine à me
laisser, non pas persuader, mais aller à la bienséance\,; lui-même me
dit encore plus de merveilles, et quoique malgré moi, je me laissai
rengarier. C'était avant la formation arrêtée des conseils. Je ne fus
pas longtemps à m'apercevoir de pis que de mollesse.

Les conseils formés, et toutes les affaires en train, il fut question de
la nôtre avec le parlement. À tout ce qui s'était passé là-dessus, sous
le feu roi dans les derniers temps de sa vie, du su et sous les yeux de
M. le duc d'Orléans, et aussitôt après la mort de ce monarque, où la
parole du régent se trouvait engagée à nous d'une manière si formelle et
si redoublée, de plus encore si solennelle, en pleine séance du
parlement, il y avait lieu de compter que nous aurions enfin justice des
scélératesses du duc du Maine et de celles du premier président, gens
d'ailleurs si contraires à M. le duc d'Orléans. Je dois, quoi qu'il ait
fait, trop de respect à sa mémoire pour vouloir le montrer par un aussi
vilain côté que fut celui que nous en éprouvâmes\,; je dois aussi trop
de considération à mes confrères pour entrer dans un détail dont la
vérité serait si fâcheuse pour la plupart\,; je dois encore assez
d'égards au grand nom de l'ordre dont je suis moi-même, pour éclairer
toute la duperie, l'envie, la jalousie, le bas et aveugle intérêt de la
conduite de ceux qui nous, attaquèrent sous un nom si auguste, et si peu
celui de la plupart de ceux qui osèrent s'en couvrir, et qui se
dévouèrent à être le jouet du duc et de la duchesse du Maine, et la
honte de la véritable noblesse par la folie égale de leurs calomnies, de
leurs prétentions, et de leur abandon à celles des gens du parlement,
avec qui l'intérêt de leurs moteurs les avait amalgamés, à leur ruine,
et à la dérision et la compassion de tout ce qui n'avait pas pris les
folles impressions que soufflait tout l'art pernicieux du duc et de la
duchesse du Maine.

On vit la haute noblesse s'émouvoir et se rassembler en 1649, et
demander et obtenir l'adjonction des ducs contre les nouveaux rangs
accordés à MM. de Bouillon et de Rohan, comme injurieux à la noblesse et
nuisibles à l'État\footnote{{[}23{]}}. On lui vit obtenir ce qu'elle
demandait, qui fut rendu après l'orage à qui il avait été ôté. Enfin on
vit cette assemblée vouloir se mêler des affaires, et embarrasser la
cour, qui fut obligée de chercher les moyens de la séparer, et de
l'empêcher après de ses rassembler. Au moins avait-elle raison dans son
premier objet, puisque rien n'est en effet si injurieux à des maisons
illustres et anciennes que d'en voir d'autres qui ne sont pas
meilleures, ou qui sont même inférieures, distinguées d'elles par un
rang et une supériorité si marquée, accordés au seul titre de
naissance\,; et puisqu'il n'est rien de si pernicieux à un État, ni d'un
si corrupteur exemple, que d'accorder des grâces si nouvelles, si
inouïes, si étendues et si éclatantes pour prix d'une suite continuelle
de menées (comme aux Rohan), de complots, de révoltes ouvertes, de
pratiques dedans et dehors de royaume, de trahisons, de prises d'armes
contre le roi, d'un cercle sans fin d'abolition et de la haute noblesse
flétri par un tas de safraniers\footnote{}, mais reçus par les nobles
pour faire nombre, et prendre un objet tout opposé à celui de 1649.

Il ne s'agissait point alors des bâtards, ni d'y prendre parti, et nulle
apparence que la noblesse pût entrer à découvert dans celui du parlement
contre nous. Mais celui du duc du Maine voulait rassembler les borgnes
et les boiteux avec les forts et les sains, pour avoir force monde
ameuté tout prêt à ses ordres. Il fallait leur montrer un objet, leur
fasciner les yeux, profiter de leur ignorance, du peu de sens de la
multitude, la flatter, lui donner lieu et la satisfaction de faire du
bruit. Il fallait de plus un objet durable qui les tînt longtemps
attroupés, échauffés, qui aveuglât leur raison et leur intérêt
véritable, leur montrer une lune pour les faire aboyer, et les enivrer
tellement de la délicieuse nouveauté de se croire considérables et
importants qu'ils ne s'aperçussent point du piège qui leur était tendu,
et de la dérision secrète que faisaient d'eux ceux dont ils devenaient
les aveugles instruments, ni de la compassion que le gros sensé de la
véritable noblesse concevait de leur frénésie.

Elle fut telle que tout ce qui se présenta fut reçu, et que ces gens si
entêtés de leur noblesse consentirent à une parfaite égalité avec tous,
jusque-là que le marquis de Châtillon fit passer en faveur de son gendre
qu'ils signeraient tous en rond, pour bannir toute différence. Ce gendre
était colonel d'un régiment, et a été cassé depuis pour sa conduite. Il
était fils de Bonnetot, premier président de la chambre des comptes de
Rouen, et ce premier président était fils d'un laboureur de Normandie,
qui était devenu fermier, et par l'industrie de l'un et l'avarice de
l'autre un des plus riches bourgeois de Rouen. Je donne cet exemple
entre mille de ces reçus par ces messieurs soi-disant la haute noblesse.

L'objet pour les faire crier et les tenir ensemble fut bientôt trouvé.
Ce fut la calomnie du duc de Noailles, de la salutation du roi, et de là
des plaintes et des prétentions contre les ducs également folles et
absurdes, et qui n'avaient pas le plus léger fondement. À la place de
choses, c'étaient des inventions de minuties, qui auraient fait rire
dans un autre temps, et qui toutefois n'avaient ni réalité ni apparence.
On le leur démontrait, ils ne pouvaient combattre l'évidence, cela même
les irritait davantage.

Leur grande clameur était que les ducs ne voulaient pas être de l'ordre
de la noblesse. On leur demandait s'il y avait en France plus de trois
ordres, si les ducs se prétendaient de celui du clergé ou de celui du
tiers état, ou enfin s'ils ne voulaient être d'aucun des trois, et
s'exclure ainsi d'être François et du corps de l'État. Cette réponse, à
laquelle il n'y en avait point, les mettait en fougue, et la fin était
qu'eux ne voulaient pas que les ducs fussent de l'ordre de la noblesse.
On leur demandait duquel donc ils les voulaient mettre\,; on leur disait
encore que puisqu'ils ne voulaient point les ducs dans l'ordre de la
noblesse, ils ne devaient donc pas leur imputer de n'en vouloir pas
être, et en crier si haut. La fureur et le déraisonnement le plus inepte
était leur réplique, et cette ivresse était telle qu'à qui n'en a pas
été témoin, elle est entièrement incroyable.

Enfin après avoir bien battu l'air, il fallut les amuser, de peur de les
laisser se dissiper d'eux-mêmes. Les moteurs de ce fanatisme profitèrent
du premier objet par lequel ils avaient su les remuer et les
rassembler\,: et de cette calomnie du duc de Noailles sur la salutation
du roi, les conduisirent à attaquer les distinctions des ducs et des
duchesses, sans jamais parler de celles des princes étrangers, qui,
étant données par naissance, sont véritablement injurieuses à la
noblesse, au lieu que celles des ducs étant par dignité, tout noble peut
espérer d'y parvenir, comme ont fait ceux qui en sont revêtus. Cet
hameçon grossier fut saisi avec tout l'emportement que les promoteurs en
pussent désirer.

Le duc du Maine qui, par la perfidie si noirement pourpensée du bonnet,
s'était délivré de la crainte de l'union des ducs et du parlement contre
tout ce qu'il avait arraché du feu roi, n'avait pas moins de peur de la
réunion de tous les gens de qualité avec les ducs contre ces mêmes
choses. Par cette nouvelle adresse, il se délivrait de cette frayeur,
s'assurait au contraire de cet attroupement, et comptait de donner par
là une occupation de défense à ceux dont il redoutait les attaques.

Le parlement, d'autre part, qui ne voulait point répondre au régent sur
le bonnet, ni les autres choses qui regardaient les ducs, était ravi de
les voir attaqués de la sorte, et se réjouissait de la diversion. Peu
contents de leur nombre, ces messieurs écrivirent dans les provinces, y
procurèrent des assemblées et des adjonctions à eux par députés, et le
duc du Maine et le premier président firent par le bailli de Mesmes,
ambassadeur de Malte, que tous les chevaliers de Malte, comme noblesse,
s'y unirent aussi

Rien de plus scandaleux ni de plus vain\,: scandaleux, parce que nul
ordre ne doit et ne peut s'assembler que par ordre ou par permission du
roi, beaucoup moins pratiquer des adjonctions, et parce que la noblesse
ne peut être considérée comme telle, et comme faisant corps, que dans
les états généraux, ou dans une assemblée convoquée par le roi et formée
en conséquence dans les provinces, par bailliages, pour faire les
députations, comme il se pratique pour les états généraux. Ainsi cette
foule assemblée d'elle-même, cherchant à s'organiser de sa propre
autorité, ne pouvait être qu'un ramas informe, sans consistance, sans
nom, sans fonction, sans mouvement légitime, bien loin de pouvoir
prédire le nom de la noblesse et du second ordre de l'État. C'est à quoi
pas un d'eux ne pouvait répondre. Rien aussi de plus vain que leurs
clameurs et leurs démarches, et ils ne savaient que dire lorsqu'on leur
demandait ce qu'ils voulaient, et sur quel fondement\,; s'ils valaient
mieux que leurs pères et leurs ancêtres, qui n'avaient jamais imaginé de
se blesser de rien à l'égard des ducs\,; s'ils connaissaient un pays
policé dans le monde entier qui n'eût pas ses dignités, et ses grands
distingués de tous par leurs prérogatives, tant les monarchies que les
républiques, dans toutes les parties de l'univers et dans tous les
siècles\,; s'ils prétendaient que cela fût abrogé en France, où, comme
partout ailleurs, sous quelque nom que ç'ait été, il y en avait toujours
eu\,; s'ils voulaient dépouiller le roi du droit d'accorder ces grandes
récompenses, et eux-mêmes et les leurs de l'espérance d'y arriver\,;
enfin ôter toute émulation, toute ambition, toute envie de servir l'État
et ses rois, puisque, en détruisant les dignités, il ne pouvait plus y
avoir de distinction ni de préférence\,; que de l'un à l'autre personne
ne voudrait céder à un autre, et s'estimer inférieur à lui en noblesse,
dont chacun ne pouvait porter les titres sous son bras pour prouver
l'antiquité de la sienne par-dessus celle d'un autre. Toutes ces
raisons, et une foule d'autres que je tais, les accablaient et les
rendaient muets en raisons, et furieux en effet, jusque-là qu'il y en
eut, et de grand nom, que je veux bien taire, qui ne purent s'empêcher d
avouer que tout ce qu'on leur opposait était vrai\,; mais que,
n'espérant pas d'être ducs, ils en voulaient éteindre la dignité, et
rendre égaux tout le monde. Voilà jusqu'où le fanatisme fut poussé.

M. le duc d'Orléans, qui espérait de tout ce bruit que les ducs, trop
attaqués, lui donneraient plus de relâche sur leur affaire avec le
parlement, était si peu contraire à ces folies qu'il avait permis à ses
premiers officiers de s'y joindre, dont M. de Châtillon était le plus
ardent. Je représentai vainement à Son Altesse Royale le danger d'une
tolérance qui portait à une sorte de révolte de gens du plus grand nom
mêlés avec gens du plus bas, qui se devaient dire sans aveu que
d'eux-mêmes, s'attrouper, s'engager les uns aux autres en union par
leurs signatures, envoyer des lettres circulaires dans les provinces,
s'ériger en réformateurs, ou plutôt en refondeurs de l'État, sans avoir
pu articuler la preuve d'aucune de leurs plaintes contre les ducs, et
sans autre raison que leur bon plaisir et leur licence, contester aux
ducs ce qui a été de tout temps, et ce qui n'est pas en la puissance du
régent de leur ôter\,; que c'était être aveugle de ne voir pas la trame
de toute cette menée, tissue par le duc du Maine, son plus grand ennemi,
et par le premier président, qui ne l'était pas moins, et un avec le duc
du Maine, qui amusaient des gens sans connaissance, et qui profitaient
de leur vanité pour unir un nombreux groupe ensemble, le tenir en leurs
mains, disposer de leur aveuglement, et en temps et lieu s'opposer à lui
et à son gouvernement, à leur tête, et en unisson avec les provinces et
avec le parlement.

Je le priai de se souvenir de l'embarras que l'assemblée de 1649,
quoique avouée par Monsieur et par la reine régente, leur avait donné\,;
la juste crainte qu'ils en avaient enfin conçue, lorsqu'elle voulut
parler d'autre chose que du rang des Bouillon et des Rohan\,; enfin les
soins et les peines qu'il y eut à les séparer et à les empêcher de se
rassembler.

L'amour de la division et l'esprit de défiance qui, avec la plus étrange
faiblesse, dominaient le régent, le rendirent sourd à mes remontrances.
Il croyait que l'intérêt des ducs me faisait parler, et trouver le sien
dans ce vacarme\,; et dans la suite, la crainte de cette prétendue
noblesse le saisit et l'arrêta quand il eut commencé enfin à ouvrir les
yeux sur ses démarches. Dans tous ces divers temps, tantôt il convenait
avec moi, et promettait d'imposer, tantôt il esquivait. Je le
connaissais trop pour être la dupe de ses meilleurs propos. Un long
usage m'avait appris à lire dans ses yeux et dans sa contenance, quand
il me parlait vrai ou contre sa pensée. Mais je comptais faire mon
devoir de le poursuivre, et j'avouerai aussi que je me dépiquais en le
mettant au pied du mur. Il sentit trop tard la solidité de mes
représentations.

L'affaire du bonnet et des autres usurpations du parlement ne se suivait
pas avec moins de chaleur. Les ducs s'assemblaient fréquemment,
députaient au régent, et j'étais celui qui d'ailleurs lui parlais le
plus souvent et avec le plus de force. Il arrivait sans cesse que je le
mettais au désespoir par mes sommations de sa parole, et par celles que
je lui attirais des députations. Il sentait la force de la justice, et
celle de ses engagements publics avec nous\,; il craignait le parlement,
et le duc de Noailles, qui le redoutait encore plus sur son
administration des finances, le détournait de nous tenir ce qu'il nous
avait si solennellement promis, et l'avertissait et le fortifiait sur
les résolutions de nos assemblées.

J'en fus instruit avec preuves évidentes. Je les semai en une très
nombreuse assemblée chez M. de Laon, et aussitôt après je leur dis, en
regardant fixement le duc de Noailles\,: «\,Messieurs, nous avons ici
des traîtres qui mériteraient bien d'en être chassés avec toute
l'ignominie qui leur est due. Mais au moins vous les connaissez, vous ne
pouvez vous y méprendre. En attendant mieux à leur égard, méprisons-les,
suivons notre affaire avec courage, mettons toute notre force dans notre
union, et si nous savons tous marcher ensemble, nous aurons justice, et
nous pourrons après nous la faire de nos traîtres, et les livrer à toute
leur infamie.\,» J'avais souvent soupçonné le duc de Noailles, je lui
avais souvent donné des lardons en pleines assemblées. Pour cette fois,
assuré des faits, et en ayant montré l'évidence à la plupart avant de
nous asseoir, je donnai carrière à mon indignation.

Nous nous mettions toujours en rang d'ancienneté tout autour de la
chambre, pour opiner plus en ordre et moins en confusion. Il arriva que,
pendant ce court discours, chacun m'imita à regarder le duc de
Noailles\,; tous les yeux se fixèrent sur lui. Il ne put soutenir une si
forte épreuve\,; il rougit à l'excès, puis pâlit tout à coup, blanc
comme sa cravate\,; les lèvres lui tremblaient\,; il n'osa proférer un
seul mot de toute la séance, et se contenta d'approuver de la tête à
mesure qu'on convenait de quelque chose.

Je dis sur la fin, toujours regardant mon homme très fixement, qu'il ne
fallait pas douter que M. le duc d'Orléans, et peut-être le parlement
aussi, ne fussent promptement avertis, et de la première main, de tout
ce qui venait d'être débattu et résolu entre nous\,; mais qu'ayant pour
nous la vérité, l'équité, et l'engagement du régent le plus public et le
plus solennel, il n'y avait qu'à laisser rapporter nos traîtres, suivre
vivement ce qui était résolu, surtout maintenir l'union entre nous, et
la regarder comme notre salut unique, mais certain. Tous les regards
tombèrent encore, à cette reprise, sur le duc de Noailles, qui se leva
brusquement, dit un mot bas à Charost son voisin, et sortit tout de
suite comme un homme enragé. Cette manière de s'en aller n'échappa à
personne. Je la commentai, et j'expliquai plus au long les preuves de la
trahison du duc de Noailles, dont on ne douta plus. On convint de ne lui
plus rien communiquer, mais qu'il n'était pas possible de lui fermer la
porte de nos assemblées. Nous n'eûmes guère lieu d'en être embarrassés,
car il ne s'y présenta presque plus, c'est-à-dire de loin en loin, une
fois ou deux encore, et pour peu de moments, cachant sa turpitude sous
son importance, et le travail des finances qui ne lui donnait aucun
loisir.

Charost, au sortir de cette assemblée chez M. de Laon, dont je viens de
parler, me prit à part, et me voulut haranguer sur la façon dont j'avais
tancé le duc de Noailles. Je me moquai de lui, et lui demandai quel
ménagement méritait un traître, et d'ailleurs de Noailles à moi, le plus
noir et le plus perfide calomniateur, et à qui nous devions la frénésie
de toute cette prétendue noblesse. Charost répliqua que cela était bel
et bon, mais qu'il fallait donc que je susse que Noailles lui avait
parlé de moi avec menaces, comme mi homme qui voulait tirer raison de
moi si je recommençais à l'attaquer. Je me mis à rire, et lui dis qu'il
y avait longtemps que je lui en fournissais matière et occasion, s'il
était si mauvais garçon, et qu'il me semblait que la scène qu'il venait
d'essuyer était assez forte pour n'en attendre pas une nouvelle\,; que
ses complots, ses pratiques sous terre, ses noires impostures et ses
infernales machinations, étaient ses armes véritablement à redouter,
telles que je les avais éprouvées en très gratuite et très sublime
ingratitude, armes pour lui plus sûres et plus favorites que son épée,
qui tenait trop au fourreau pour craindre d'en être ébloui\,; qu'au
surplus c'était à lui à courir s'il en avait envie, et moi à l'attendre
comme je faisais depuis longtemps, sans la plus légère inquiétude, et
sans lui épargner nulle occasion ni aucun trait de l'y exciter, pour peu
qu'il fût homme à en avoir envie\,; que par conséquent cet avis qu'il
(Charost) me donnait ne me ralentirait pas le moins du monde.

En effet je ne manquai pas une occasion à tomber sur cet honnête
confrère, partout où je le pus, c'est-à-dire parmi nous, où, comme je
l'ai dit, il n'osa presque plus se montrer, au conseil et chez M. le duc
d'Orléans, qui étaient les seuls endroits où je pouvais le rencontrer,
où je recevais ses basses révérences, sans lui rendre la moindre
inclination, et où ma contenance et tant que j'y pouvais trouver jour,
mes propos et ma hauteur me vengeaient, et montraient avec évidence aux
assistants le coupable, qui n'osait jamais répondre un seul mot, ce qui
me paraîtrait à moi-même incroyable, si je ne l'avais sans cesse
expérimenté tous les jours huit ans durant, à la vue de toute la France,
tant le crime a de poids accablant jusque sur les plus méchants, les
plus impudents, les plus grandement établis, et qui ont le plus de
ressources d'ailleurs en eux-mêmes. Mais il faut me tenir ce que je me
suis proposé au commencement de cette triste matière, l'enrayer au plus
tôt, et devancer ici les temps pour n'avoir plus à y revenir.

Les mois s'écoulèrent en ces poursuites d'une part, en ces menées de
l'autre. Le parlement, pressé de la vérité, plus touché de son intérêt,
persuadé qu'il n'avait pas de quoi se défendre, prit un parti hardi que
lui inspira la faiblesse du régent\,; ce fut de laisser à côté la
défense des usurpations attaquées par les ducs, de montrer les dents à
M. le duc d'Orléans, et de refuser de lui répondre et de lui obéir
là-dessus. Conduit par d'Effiat et par Canillac, conseillé par le duc de
Noailles, appuyé du duc du Maine et de ce groupe si nombreux qu'il avait
su ameuter et s'unir sous le respectable nom de noblesse, le parlement
ne craignit point de se moquer d'un prince dont il voyait sans cesse les
ménagements pour lui, et en même temps la crainte qui les produisait.
Ces magistrats si bien guidés comprirent aisément qu'ils pouvaient tout
faire sans risquer rien, et que le régent, qui les ménagerait toujours
pour leur faire passer sans opposition les édits et les déclarations
qu'il voudrait faire sur les matières des finances et du gouvernement,
ne se compromettrait jamais avec eux pour chose qui au fond n'importait
en rien à sa personne, et dont il se souciait en effet fort peu. C'est
la conduite constante que le parlement tint dans toute la suite de cette
affaire, et qui lui réussit pleinement.

J'avais beau représenter à Son Altesse Royale la dérision publique que
le parlement faisait de son autorité, l'étrange exemple qu'il laissait
apercevoir, ou de sa faiblesse, ou de l'opinion qu'il n'avait pas le
pouvoir de faire répondre des magistrats sur des entreprises visibles
qui n'intéressaient qu'eux\,; qu'enfin il leur apprendrait, par une
conduite si peu digne du dépositaire de la plénitude de l'autorité
royale, qu'ils pouvaient lui résister en des choses qui
l'embarrasseraient fort dans l'exercice du gouvernement, et à lui
résister encore toutes fois et quantes il leur plairait de le faire. Ce
que je lui disais était évident, et il ne tarda pas longtemps à en faire
une honteuse expérience, comme je le raconterai en son temps. Mais je
parlais en vain, je le désespérais par la transcendance des raisons que
je lui apportais, auxquelles il ne pouvait répondre. Mais les mêmes
causes qui m'avaient fait échouer avec lui sur cette assemblée de
noblesse me procurèrent le même sort sur le parlement. Sa défiance lui
persuada que je ne lui parlais qu'en duc qui n'a que cet intérêt en
vue\,; son goût pour la division, qu'il la fallait entretenir entre les
ducs et le parlement, et entre les ducs mêmes\,; sa faiblesse, appuyée
des pernicieux conseils de Noailles, Besons, Effiat, Canillac et de bien
d'autres, qu'il fallait ménager le parlement en chose qui en intéressait
si vivement les principaux magistrats, et qui ne lui importait en rien à
lui-même, pour les trouver favorables et faciles à passer tout ce qu'il
leur voudrait envoyer à enregistrer. C'est-à-dire que ces bons et
fidèles conseillers comptaient pour rien la justice, la parole
solennelle et publique donnée aux ducs par le régent, et par lui
renouvelée en pleine séance au parlement, à l'ouverture de celle de la
régence, la dérision que le parlement et toute la France faisait de voir
un régent refusé par le parlement de lui répondre, et sur chose de cette
qualité qui n'intéressait que l'orgueil de quelques magistrats\,;
l'exemple et le courage que cette misère donnait à tout le monde, en
particulier au parlement pour en abuser dans les choses du
gouvernement\,; enfin de compter pour rien de manquer solennellement et
publiquement de foi, de parole, par conséquent d'honneur, à tout ce
qu'il y avait de grands en France.

Tout cela dura plusieurs années, et il faut que j'aie bien envie de
sortir d'une si dégoûtante matière pour en prévenir de si loin la fin,
qui arriva d'une part à force d'art, d'intrigues, de souplesses et
d'audace\,; de l'autre, de dépit, de dégoût et de guerre lasse.

Pendant cet intervalle, les protecteurs du parlement virent bien toute
la force que les ducs tireraient de leur union, qui faisait toute la
peine et l'embarras du régent sur cette affaire. Leur application se
tourna donc à les diviser\,; le duc de Noailles s'appliqua à regagner
les moins difficiles, et à effacer de leur esprit l'idée de ses
trahisons, tandis qu'il y était plus abandonné que jamais. J'avais eu,
dès avant la mort du roi, toutes les attentions imaginables à marquer à
chaque duc toute sorte de considération. On en a pu voir un échantillon
dans la façon dont je me raccommodai avec M. de Luxembourg, l'unique
avec lequel je fusse demeuré mal, car le roi vivait encore, et la
scélératesse du duc de Noailles à mon égard m'était alors inconnue.

Plus je parus depuis la mort du roi bien avec le régent, plus mes
attentions redoublèrent pour les ducs, et dans nos affaires connues
j'évitai avec le plus grand soin jusqu'au moindre air de faveur et
d'importance. Je parlais et j'opinais comme l'un d'eux\,; je soutenais
mes avis avec une modestie propre à les faire goûter, je puis dire que
je les traitai toujours avec un air de respect pour eux. Si je proposais
des partis fermes, j'en expliquais les raisons\,; si des partis hardis
et des propos de cette espèce à tenir au régent, je m'en chargeais ainsi
que de toutes les commissions difficiles. C'est une justice qui, quoi
qu'on ait fait, n'a pu m'être refusée, et que le duc de Tresmes entre
autres, sans être mon ami particulier, a bien su leur reprocher. Mais
cette conduite, toute mesurée qu'elle fût, ne put émousser l'envie.
Cette passion basse et obscure se blesse de tout\,; ma situation auprès
du régent l'excita, et le duc de Noailles en sut profiter.

La plaie de ma préséance n'était pas refermée dans le cœur de M. de La
Rochefoucauld, et le duc de Villeroy, toujours à sa suite, conservait le
même sentiment. Canillac cultivait l'hôtel de La Rochefoucauld, avec qui
il avait fait grande connaissance chez Maisons. La Feuillade était de
tout temps moins son ami que son esclave, et depuis sa disgrâce de Turin
il s'était accroché à M. de La Rochefoucauld et à M. de Liancourt, qui
dans les suites le reconnurent et lui fermèrent leur porte. La
Feuillade, je n'ai jamais su pourquoi, m'avait pris de tout temps en
aversion. Canillac, qui était l'envie même, et qui se persuadait qu'il
lui appartenait de gouverner le régent et l'État sans la plus légère
concurrence, n'était pas pour guérir La Feuillade ni La Rochefoucauld à
mon égard. Ils embabouinèrent le pauvre duc de Sully, connu auparavant
sous le nom de chevalier de Sully, qui sien repentit bien après qu'il
n'en fut plus temps, ainsi que le duc de Richelieu, qui ne faisait que
poindre, et que le bel air avait fait disciple très soumis de La
Feuillade. Noailles et Aumont s'amalgamèrent à eux dès qu'ils y purent
être reçus, et M. de Luxembourg se laissa entraîner à MM. de La
Rochefoucauld et de Villeroy, ses amis intimes de tous les temps, depuis
leur liaison commune avec feu M. le prince de Conti. Noailles, qui les
voulait gouverner, n'osa l'entreprendre à découvert\,: il crut le faire
plus aisément sous un autre nom, au poids duquel ces messieurs-là
fussent accoutumés. Il leur insinua de gagner le maréchal d'Harcourt,
qui n'avait plus ni tête ni presque de parole. La Rochefoucauld avait
toujours été lié avec lui et le duc de Villeroy, et Noailles l'avait été
à cause de M\textsuperscript{me} de Maintenon. Un tel mentor, qui n'en
avait plus que l'ombre, fut merveilleusement propre au duc de Noailles,
qui, dès qu'ils l'eurent gagné, devint le prêtre qui faisait parler
l'oracle.

Ce ne fut que pour contrecarrer tous les bons et sages partis que
voulaient prendre ceux qu'ils n'avaient pu débaucher, et qui étaient\,:
le cardinal de Mailly, archevêque de Reims\,; Clermont-Chatte, évêque de
Laon, qui avait pouvoir de faire pour son cousin de Tonnerre, évêque de
Langres\,; Rochebonne, évêque de Noyon, et de loin Noailles, évêque de
Châlons, qui suivait son frère le cardinal de Noailles, qui, malgré son
accablement des affaires de la constitution, et le besoin et les
liaisons qu'elles lui donnaient avec le parlement, fut un des plus
fidèles et des plus généreux de notre nombre. Les ducs de La Force, de
Tresmes, de Charost, le maréchal de Villars, et les ducs d'Antin et de
Chaulnes, aucun de ceux-là ne se démentit, aucun ne faiblit, tous
agirent et firent merveilles. C'était avec eux que j'étais uni.

Je laisse le reste des ducs qui ne parurent presque plus dans ce reste
de lutte avec le parlement et le régent, pour ne pas dire entre
nous-mêmes. Les uns absents, les autres enfants, ceux-ci lassés d'une
guerre plus qu'ingrate, ceux-là bas et timides sous un dehors politique
et prudent.

Le duc de Noailles ourdissait soigneusement sa trame pour nous désunir.
Tout l'invita à cet infâme travail. Se donner le mérite auprès du régent
de lui sacrifier l'intérêt de sa dignité\,; auprès du parlement, de le
délivrer en lui assurant le triomphe, avec ce ramas informe de noblesse
qu'il avait excitée et qu'il ne cessait de cultiver\,; de faire litière
de cette dignité qu'il lui avait plu de prendre en haine\,; enfin de
réparer en partie le peu de fruit qu'il avait recueilli de sa
scélératesse à mon égard.

Trop anciennement lié avec l'abbé Dubois, comme on l'a vu ailleurs, pour
avoir ignoré mon dégoût, mon commencement de retraite, et tout ce qui
s'était passé de la part du régent par Dubois pour me raccrocher, il
était au désespoir qu'une des choses dont il s'était le plus flatté eût
manqué. Il n'était pas moins confondu qu'après tant d'affreuses et de
noires pratiques pour me rendre l'objet de la fureur de toute cette
noblesse, pas un ne m'eût fait seulement la plus légère malhonnêteté. On
ne hait rien tant au monde qu'un homme à qui on doit, et que
gratuitement on a voulu perdre, qui le sait, qui le publie, qui en
connaît la cause, et qui la répand, qu'on n'a pu ni perdre ni même
affaiblir, et qui ne garde aucune sorte de mesure en quelque lieu ni en
quelque occasion que ce soit, avec lequel on ne peut éviter de se
rencontrer souvent, et que nulle patience, je n'oserait dire nuls
respects extérieurs, ne peuvent émousser. Outre le fruit que je viens
d'expliquer, qu'il se proposait pour soi-même du succès de ses travaux
pour nous désunir, il se flattait encore de me brouiller avec cette
partie des ducs qu'il aurait trompée, de me rendre à charge à ceux que
je voudrais maintenir en union, insupportable d'une part, et méprisable
de l'autre à M. d'Orléans par une opiniâtreté qui ne serait presque plus
soutenue de personne, par là de changer à son avantage ma situation
auprès de lui, et peut-être de dépit me faire quitter la partie, sans
craindre que le régent courût après moi comme la première fois.

Tant de puissants motifs pour une ambition démesurée qui, dans la
gangrène de son âme et la bassesse et la pourriture de son cœur, ne
trouvait ni remords ni obstacle, tirèrent de son art, de son esprit
aisé, liant, souple, fécond, séducteur, et de ces manèges obscurs où il
était si grand maître, tous les moyens de persuader des hommes qui ne se
défiaient plus de lui, et à qui il persuadait qu'il n'avait avec eux
qu'un seul et même intérêt.

À l'écorce plausible qu'il tâcha de donner à ses raisons, il n'oublia
pas de piquer la jalousie de ceux qui en purent être susceptibles, et de
me donner à eux comme un homme entêté de ses sentiments, gâté par la
faveur, désireux de dominer et d'emporter tout à ses avis, en un mot de
conduire et de gouverner ses égaux et ses confrères. On a dit par qui il
y fut aidé et pourquoi. Néanmoins la persuasion fut longue à prendre, et
nous fûmes bien avertis. Je ne crus pas devoir faire de démarche vers
aucun des ébranlés. Je me contentai de les laisser faire à ceux avec qui
j'étais uni qu'on n'avait pu rendre suspects aux autres, de me consoler
dans l'union et la fermeté des nôtres, surtout dans leurs sentiments, et
leur témoignage à tous de la droiture et de la simplicité de ma conduite
et de mon procédé dans tout le long cours de cette malheureuse affaire
si cruellement embarquée, malgré nous, sous la fin du feu roi, et j'ai
eu cette satisfaction encore que ces mêmes ducs sont tous demeurés mes
amis jusqu'à leur mort.

À force de temps, de ruses, d'artifices et de trames, Noailles vint à
bout de la division qu'il avait résolu de mettre entre nous. Il fit,
avec ceux qu'il séduisit, de petites assemblées secrètes\,; ensuite pour
leur donner du poids il y en eut de plus nombreuses chez le maréchal
d'Harcourt qui n était plus portatif, et qui n'étant plus en état de
rien comprendre, encore moins de disserter, les couvrit de son ombre, et
applaudissait de la tête avec de grands yeux ouverts et étonnés à ce que
Noailles expliquait, comme de sa part. Je voyais, il y avait du temps,
les progrès de cet Achitophel\,; je comprenais qu'il réussirait enfin\,;
je n'allais plus qu'à regret à nos assemblées chez l'ancien de nous qui
se trouvait à Paris, et souvent il fallait me presser pour m'obliger à
m'y rendre. Enfin un jour que nous fûmes tous avertis de nous trouver
chez le cardinal de Mailly, archevêque de Reims, nous le fûmes une heure
après pour nous rendre chez le maréchal d'Harcourt.

De ce moment je vis ce qui allait arriver, et je résolus de me tenir
chez moi. Je n'avais garde d'aller chez le maréchal d'Harcourt, où pas
un de notre union n'avait jamais été, et où pour la première fois nous
étions priés de nous trouver, parce que je ne voulus pas me livrer à des
disputes inutiles sur un parti bien pris entre eux, et qu'ils ne
voulaient que nous déclarer, pour rendre la division plus invariable par
tout ce qu'il était difficile qui n'accompagnât pas, dans les termes où
on était arrivé, l'action de cette assemblée, si nous nous y fussions
rendus\,; aussi pas un de nous n'en fut-il tenté.

Je ne voulais pas, non plus, aller chez le cardinal de Mailly, pour y
assister, pour ainsi dire, à nos funérailles, car ce les furent en
effet. Mais je fus si pressé de plusieurs, et le matin même par
M\textsuperscript{me} de Saint-Simon qui me représenta qu'il y aurait de
la honte d'abandonner ceux avec qui j'avais toujours été uni, que je m'y
en allai. Cela fit que j'y arrivai des derniers, qu'on y avait été dans
l'inquiétude de mon absence, et que je fus reçu avec de grands
témoignages de satisfaction. On attendit longtemps ceux qui étaient de
chez M. d'Harcourt. Tous les nôtres étaient chez le cardinal de Mailly,
et le duc de Rohan de plus qui déclama fort contre les autres, ainsi que
nous tous. Mais il ne s'y fit rien. Nous déplorâmes un schisme et une
scission fatale\,; et, après être demeurés ensemble fort tard, nous
résolûmes de ne plus battre l'air en vain, de céder à la trahison, d'une
part, et à l'entraînement de l'autre, et de laisser aux temps et aux
occasions à faire repentir le régent de son manquement de parole et de
son déni de justice, et à ces messieurs de chez M. d'Harcourt à se
mordre longuement les doigts de leur duperie et de leur conduite qui
perdait tout entre nos mains. Nous nous embrassâmes les uns les autres,
et nous nous promîmes une amitié et une union réciproques entre nous,
auxquelles pas un n'a manqué. À l'égard des autres, froideur et
civilité.

Ainsi par l'ambition et les artifices du duc de Noailles et de ses
consorts, et la simplicité de leurs dupes, se fit cette meurtrière
division qui mit fin à nos poursuites, donna lieu au parlement de
triompher moins de nous que du régent, et procura à ce prince un court
repos qu'il paya chèrement après. Prenons haleine après un si fâcheux
récit, et retournons sur nos pas, dont, pour l'achever de suite, il nous
a fort détournés.

\hypertarget{chapitre-xii.}{%
\chapter{CHAPITRE XII.}\label{chapitre-xii.}}

1715

~

{\textsc{M\textsuperscript{me} la duchesse de Berry obtient une
compagnie de gardes.}} {\textsc{- Le chevalier de Roye en est capitaine
et Rion lieutenant.}} {\textsc{- Ce que devient le chevalier de Roye.}}
{\textsc{- Harling est aussi capitaine des gardes de Madame, mais sans
compagnie.}} {\textsc{- M\textsuperscript{me} la duchesse d'Orléans
prend quatre dames auprès d'elle, tôt après imitée en cela par
M\textsuperscript{me} la Duchesse et par d'autres princesses du sang.}}
{\textsc{- Mort du comte de Poitiers, dernier mâle de cette grande et
illustre maison.}} {\textsc{- Mort d'Humbert.}} {\textsc{- Chirac en sa
place premier médecin de M. le duc d'Orléans.}} {\textsc{- Vergagne bien
singulièrement grand d'Espagne.}} {\textsc{- Mort de la princesse de
Cellamare.}} {\textsc{- Le fils de Matignon finit son mariage, et est
duc et pair de Valentinois.}} {\textsc{- Douze millions du clergé au
roi.}} {\textsc{- Vingt mille livres de rente sur les juifs de Metz au
duc de Brancas.}} {\textsc{- Pontchartrain reçoit ordre de donner la
démission de sa charge de secrétaire d'État, qui est en même temps
donnée à Maurepas, son fils.}} {\textsc{- Caractère du comte et de la
comtesse de Roucy.}} {\textsc{- Éclat entre le comte et la comtesse de
Roucy et moi, qui nous brouille pour toujours.}} {\textsc{- Le maréchal
d'Harcourt obtient pour son fils la survivance de sa charge de capitaine
des gardes du corps.}}

~

On vit à la cour des nouveautés singulières, qui en produisirent bientôt
après de plus étranges. Rien n'égalait l'orgueil de
M\textsuperscript{me} la duchesse de Berry, comme on l'a dit et montré
ailleurs, et son empire sur l'esprit de M. le duc d'Orléans était
toujours le même, quoique peu mérité. Elle se mit en tête de vouloir
avoir un capitaine des gardes. Jamais fille de France n'en avait eu.
C'était un honneur inconnu même aux reines mères et régentes, jusqu'à la
dernière, mère de Louis XIV, qui en eut un. Madame n'y avait jamais
songé, et M. le duc d'Orléans résista d'abord à cette fantaisie, mais il
y céda bientôt, et voulut en même temps que Madame en eût un,
puisqu'elle était de même rang que M\textsuperscript{me} la duchesse de
Berry, et il se chargea de le payer, parce que Madame, dont la maison
était grosse, et les revenus ne l'étaient pas, n'en voulut pas faire la
dépense. Elle choisit Harling, gentilhomme allemand, qui avait été
nourri son page, dont elle affectionnait la personne et la famille, qui
était lieutenant général, et qui s'était distingué à la guerre. Il était
fort honnête homme d'ailleurs, doux et simple, avec de l'esprit, et le
même qui fit avec Peri cette belle et singulière retraite d'Haguenau,
après l'avoir bien défendu, comme je l'ai raconté en son temps.

M\textsuperscript{me} la duchesse de Berry choisit le chevalier de Roye,
qui l'avait été de M. le duc de Berry. Il était le dernier des frères du
comte de Roucy, et n'avait rien\,; il épousa bientôt après la fille de
Prondre, un des plus riches financiers de Paris, dont il eut beaucoup.
Il prit le nom de marquis de La Rochefoucauld, mourut lieutenant général
à cinquante et un ans, en 1724, et ne laissa qu'une fille unique qui a
épousé M. de Middelbourg, frère du maréchal d'Isenghien.

Madame n'eut point de compagnie de gardes, et continua de se servir de
ceux de M. le duc d'Orléans. M\textsuperscript{me} la duchesse de Berry,
qui n'avait que peu de gardes et point de compagnie, en voulut une, dont
elle donna la lieutenance à Rion, et l'enseigne au chevalier de
Courtaumer. J'entre dans ce bas détail, parce qu'il sera fort mention de
Rion dans la suite, et que c'est ici la première fois qu'on ait ouï
parler de lui.

On a vu en son lieu que Madame aimait fort deux dames que Monsieur
haïssait fort, ce qui a été expliqué en son temps, et qu'à la mort de
Monsieur, le roi lui permit de les prendre auprès d'elle pour
l'accompagner, même à Marly. C'était la maréchale de Clérembault et la
comtesse de Beuvron, laquelle était morte il y avait longtemps, et qui
ne fut point remplacée. C'était le premier exemple de fille de France
qui eût eu des dames attachées à elle, autres que sa dame d'honneur et
sa dame d'atours. Les courses et les parties continuelles de
M\textsuperscript{me} la duchesse de Berry, ou seule, ou avec
M\textsuperscript{me} la duchesse de Bourgogne au commencement de son
mariage, obligèrent M\textsuperscript{me} de Saint-Simon à demander du
soulagement pour la suivre. Le roi lui permit de lui proposer quatre
dames, comme on a vu en son lieu\,; ce fut le second exemple. En France,
ils sont contagieux et s'étendent facilement par la vanité.
M\textsuperscript{me} la duchesse d'Orléans, petite-fille de France,
mais femme du régent, en profita pour s'assimiler, au moins en cette
partie, aux filles de France, et M. le duc d'Orléans n'était pas homme à
l'en refuser, sans pourtant se soucier de cette nouvelle distinction.

Elle prit donc quatre dames, qui furent la comtesse de Tonnerre,
petite-fille de la maréchale de Rochefort, sa dame d'honneur, et fille
de M\textsuperscript{me} de Blansac, qu'elle avait tant et si longtemps
aimée, et avec qui elle était brouillée depuis plusieurs années, et la
demeura toujours. Quoique M\textsuperscript{me} de Tonnerre fût mariée
dans une maison riche, elle avait besoin de se tirer d'avec un mari
imbécile, et qui pouvait pourtant avoir ses fantaisies et ses volontés.
M\textsuperscript{me} de Conflans fut la seconde\,: elle était veuve
d'un premier gentilhomme de la chambre de M. le duc d'Orléans, et fille
de M\textsuperscript{me} de Jussac, qui avait élevé
M\textsuperscript{me} la duchesse d'Orléans, et qu'elle avait toujours
fort aimée. Elle choisit encore M\textsuperscript{me} d'Épinay, fille de
M. et de M\textsuperscript{me} d'O, et c'était tout dire pour
M\textsuperscript{me} la duchesse d'Orléans. Ces deux-là trouvèrent une
subsistance et une occupation dans ces places.

On a vu en son lieu que nous avions marié, il y avait un an,
M\textsuperscript{lle} de Malause au comte de Poitiers, dernier mâle de
cette grande et illustre maison. Il venait de mourir en quatre jours de
la petite vérole, laissant sa femme grosse d'une fille, qui fut un grand
parti en tout sens, et qui a épousé le duc de Randan, fils aîné du duc
de Lorges. Ce fut un grand dommage de ce comte de Poitiers qui
promettait beaucoup et n'avait rien à reprendre. Sa veuve demeurait fort
jeune, sans belle-mère et fort menacée par une de ses belles-sœurs, qui
se proposait de lui redemander tout le bien du comte de Poitiers, si
elle accouchait d'une fille. Ces circonstances nous engagèrent à la
mettre chez M\textsuperscript{me} la duchesse d'Orléans, et je n'eus que
la peine de le lui demander\,; elle fut bien aise de me faire plaisir de
bonne grâce, et plus encore de meubler sa maison d'une femme de cette
qualité.

M. le duc d'Orléans perdit en ce même temps Humbert, un des plus grands
chimistes de l'Europe, et un des plus honnêtes hommes qu'il y eût, et
qui était le plus simple et le plus solidement pieux. C'était avec lui
que ce prince avait dressé sa fatale chimie, où il s'était amusé si
longtemps et si innocemment, et dont on essaya de faire contre lui un si
infernal usage. C'est ce même Humbert que M. le duc d'Orléans voulut
envoyer à la Bastille par le traîtreux conseil d'Effiat, à la mort de M.
{[}le Dauphin{]} et de M\textsuperscript{me} la Dauphine, comme on l'a
vu en son temps, et à qui il avait donné le titre de son premier
médecin. Il choisit pour lui succéder en cette qualité Chirac, qui
passait pour le plus grand médecin qu'il y eût, et qui l'avait suivi en
Italie et en Espagne. C'était d'ailleurs l'intérêt même en tout genre,
avec tout l'esprit et le savoir possibles. J'entre dans ce détail, parce
qu'il en sera mention ailleurs, et qu'il devint enfin premier médecin du
roi, après la mort de M. le duc d'Orléans.

M\textsuperscript{me} la Duchesse, qui n'avait jamais pu s'accoutumer à
voir sa sœur cadette si élevée au-dessus d'elle, ne put souffrir
longtemps de lui voir des dames sans en avoir aussi. Elle trouva de la
marchandise fort mêlée en tout génie, et des femmes qui, pour leur pain
et leur amusement, ne demandèrent pas mieux. La facilité de M. le duc
d'Orléans le souffrit, ainsi de toutes choses. D'autres princesses du
sang en eurent aussi après comme il leur plut.

Le régent favorisa aussi une autre nouveauté bien singulière. M. de
Nevers n'avait été duc qu'à brevet, c'est-à-dire point vérifié. On a vu
ailleurs que son fils unique était malvoulu du feu roi par sa conduite,
et par avoir également méprisé la guerre et la cour. On a vu aussi en
son temps que, hors de toute espérance d'obtenir la continuation,
c'est-à-dire un renouvellement du brevet de duc, il avait épousé la
fille aînée de Spinola, qui avait acheté la grandesse de Charles II,
qu'il servait de général en Flandre, et qui était veuf sans garçons et
hors d'état ou de volonté de se remarier.

Spinola ne mourait point, et son gendre, qui, par son mariage, avait
pris le nom de prince de Vergagne, s'ennuyait fort d'attendre la
grandesse si longtemps, et la duchesse Sforce pour le moins autant, qui
était sœur de sa mère, qui lui en avait toujours servi, et qui l'aimait
avec la même tendresse. Dès sa jeunesse, il était bien avec M. le duc
d'Orléans, et la débauche avait entretenu leur commerce et la
bienveillance du prince. On a vu à quel point d'amitié et de confiance
unique M\textsuperscript{me} Sforce était avec M\textsuperscript{me} la
duchesse d'Orléans, et qu'elle était aussi fort considérée de M. le duc
d'Orléans. Elle imagina d'avancer cette grandesse, de faire représenter
au roi d'Espagne que Spinola avait cédé sa grandesse à sa fille en la
mariant\,; qu'il désirait que le roi d'Espagne l'agréât, moyennant que
lui-même, qui était vieux et retiré, ne fût plus grand.
M\textsuperscript{me} Sforce fit parler et peut-être donner quelque
argent à Spinola. Il s'accorda tout, et M\textsuperscript{me} Sforce en
parla à M. {[}le duc{]} et M\textsuperscript{me} la duchesse d'Orléans.
Le régent ne voulut point en écrire au roi d'Espagne\,; mais il témoigna
à Cellamare qu'il prenait beaucoup de part en M. de Vergagne, et serait
fort touché des grâces que le roi d'Espagne lui voudrait faire. Ils
négocièrent en même temps en Espagne, et ils obtinrent la grandesse aux
conditions proposées.

Cellamare venait de perdre sa femme, qui était Borghèse et demeurait à
Rome. Elle avait épousé en premières noces le duc de La Mirandole, dont
elle avait eu le duc de La Mirandole qui avait pensé épouser la
princesse de Parme, depuis reine d'Espagne, et le cardinal Pico. Ce duc
de Mirandole, fils de M\textsuperscript{me} de Cellamare, s'établit
depuis en Espagne, où il fut grand, et il est aujourd'hui grand maître
de la maison du roi d'Espagne.

Matignon acheva dans ce même temps l'affaire du mariage et du duché de
son fils accordé par le feu roi, avec M. de Monaco. Le jeune homme alla
à Monaco, où le mariage fut célébré, et revint avec le nom et le rang de
duc de Valentinois, qui fut enregistré au parlement.

L'assemblée du clergé, depuis si longtemps occupée de l'affaire de la
constitution, harangua le roi à Vincennes par l'évêque d'Auxerre, pour
se séparer, et donna douze millions.

Le régent fit un don au duc de Brancas de vingt mille livres de rente
sur les juifs de Metz qui crièrent miséricorde, et qui ne purent
l'obtenir. Brancas, pauvre de lui-même et panier percé d'ailleurs, était
un famélique qu'on ne pouvait rassasier. J'en ai parlé ailleurs lorsque,
pour son pain, sa femme succéda à la duchesse de Ventadour chez Madame.
Il y aura lieu dans la suite de s'étendre plus commodément sur ce duc de
Brancas. Il serait bien étonné aujourd'hui, s'il vivait, des
établissements de sa famille.

Pontchartrain, à l'abri de la considération de son père et de la
protection d'Effiat et de Besons, vivait en assurance cramponné aux
stériles restes de sa place, alors totalement oisive, et il y survivait
infatigable aux affronts, soutenu par l'espérance d'en raccrocher un
jour les fonctions\,; tandis qu'il en conservait le titre. Il ne
manquait pas un conseil de régence, où il était réduit à demeurer muet,
où il n'était regardé ni accosté de personne, où il n'avait de fonction
que celle qu'il avait prise d'y moucher les bougies, ce qui s'était
également tourné en coutume de sa part, et en dérision sans contrainte
de celle de tous ceux qui y assistaient. Chacun y admirait un si bas et
triste personnage, et l'insensibilité qui le faisait ainsi se survivre à
soi-même dans un état si profondément humilié et si prodigieusement
distant de l'audace et de l'insolence de sa splendeur et de son autorité
passée. Chacun le souhaitait chassé, et ne se faisait faute de le
chasser à sa manière par l'extrême mépris qu'on lui marquait, comme pour
se dédommager de la considération et de la dépendance passée. M. le duc
d'Orléans admirait comme les autres sa patience\,; mais il ne songeait
point à le renvoyer. Nous nous en divertissions souvent à l'oreille, et
en nous poussant, le comte de Toulouse et moi, surtout lorsqu'il
s'agissait de marine, et que le comte ou le maréchal d'Estrées lui
lâchaient des lardons à bout portant, dont ils recherchaient même les
occasions, et le comte et moi nous plaignions souvent au conseil l'un à
l'autre de la plus que bonté du régent de laisser écouter ce qui s'y
passait à un néant inutile, assez méchant pour en abuser, et qui en cent
façons méritait d'être chassé. À la fin cette longue tolérance me devint
insupportable, et je me résolus à faire un effort pour la faire finir.

J'allai le dimanche 3 novembre chez M. le duc d'Orléans à Vincennes,
avant le conseil de régence qui se tenait le matin, et je lui demandai
s'il ne se laissait point d'y voir Pontchartrain ne pouvant dire mot,
écoutant tout, à qui personne ne parlait, et mouchant le soir les
bougies\,; s'il ne ferait point cesser ce ridicule pour le conseil
même\,; combien encore il avait résolu de nous laisser dégoûter et salir
par cette araignée venimeuse que chacun souhaitait dehors, et qu'il
était par trop indécent d'y laisser après les affronts fondés et
réitérés qu'il y avait reçus sur sa gestion de la marine, par les
mémoires détaillés et prouvés que le maréchal d'Estrées, et après lui le
comte de Toulouse, avaient lus et commentés en plein conseil devant nous
tous, en sa présence et en celle de Pontchartrain, qui depuis deux mois
n'avait pu trouver rien à y opposer. J'ajoutai l'indignation publique
contre cet ex-bacha, la surprise générale qu'il fût souffert si
longtemps, et l'applaudissement universel que recevrait sa chute. Le
régent convint de tout, mais il m'opposa le père, et me dit qu'il
n'avait pas le courage de lui donner un si grand déplaisir.

Je lui répondis que, s'il voulait, je lui fournirais un moyen de chasser
le fils, et que le père encore lui serait très sensiblement obligé. Le
régent fort surpris me demanda comment je ferais cela. Alors je lui
proposai d'ordonner à Pontchartrain de donner la démission pure et
simple, et à l'instant, de sa charge de secrétaire d'État, de la donner
sur-le-champ à Maurepas son fils aîné, qui, n'ayant guère que quinze
ans, ne se trouvait pas à portée d'exercer le peu qui en restait\,; d'en
charger La Vrillière à qui cela n'ajouterait pas une demi-heure de
travail par semaine, et de faire valoir au père la singularité de ce
présent, et l'attention de le mettre en dépôt, en attendant l'âge du
jeune homme, entre les mains d'un parent de même nom, très attaché au
père, et qui, étant lui-même secrétaire d'État, ne pouvait être tenté
d'embler cette charge. Le régent ouvrit les yeux et les oreilles bien
larges à cet expédient, et l'approuva. Je lui dis que, puisqu'il le
goûtait, rien n'empêchait de l'exécuter dès le lendemain. Il y consentit
encore, mais il voulut que je fisse sa lettre au père, et que je la lui
apportasse dans l'après-dînée même de ce dimanche au Palais-Royal. Je
n'eus garde de faire le difficile. Je voulais serrer la mesure et le
secret, je me souvenais de ce qui avait déjà sauvé Pontchartrain une
fois, au moment que je le comptais perdu\,; son père était à Paris, et
je craignais que quelqu'un n'eût le vent de ceci, et le temps de rompre
mes mesures.

Nous nous en allâmes tous dîner à Paris au sortir du conseil\,; je fis
la lettre de M. le duc d'Orléans au chancelier, tendre, honnête, pleine
d'estime et de considération. J'y en fis valoir la marque sans exemple
de laisser la charge dans sa famille, non en survivance, mais en titre,
à un homme de quinze ans, avec la précaution que je viens d'expliquer
sur La Vrillière, qui le formerait et lui apprendrait le métier, et je
finissais par lui dire bien ferme que devant être content pour sa
personne et pour sa famille, et le parti en étant fermement pris, Son
Altesse Royale voulait que, dans la matinée du lendemain lundi, son fils
donnât sa démission pure et simple, chez son père à l'Institution\,; que
l'abbé de Thesut s'y trouverait pour la lui apporter avant midi, et La
Vrillière pour que tout s'y fit en règle, et pour expédier les
provisions de la charge au jeune Maurepas dans l'après-dînée du même
jour, et le mener remercier le roi\,; surtout que ne voulant point être
fatigué de prières inutiles, il lui défendait de le venir trouver, de
lui écrire, et de lui faire parler par qui que ce fût, avant que tout
fût consommé\,: démission, provisions, etc. Je portai ce projet de
lettre tout fait au Palais-Royal tout de suite. M. le duc d'Orléans n'y
changea rien\,; je dictai la lettre, il l'écrivit de sa main, la signa,
la cacheta, y mit lui-même le dessus, et me la remit pour la rendre.

Il manda aussitôt La Vrillière et l'abbé de Thesut, à qui sous le secret
il donna ses ordres, en sorte que nous n'eûmes plus qu'à les exécuter.

Le lendemain matin sur les huit heures et demie j'envoyai la lettre de
M. le duc d'Orléans, enfermée dans une enveloppe cachetée où je mis le
dessus, au chancelier de Pontchartrain, et lui mandai que je serait
incontinent après chez lui. Je ne voulus pas être le porteur moi-même,
et je laissai une demi-heure d'intervalle exprès.

Comme j'allais chez lui, je rencontrai La Vrillière à la porte
Saint-Michel qui en revenait. Nous arrêtâmes, il monta dans mon carrosse
où je lui demandai ce qu'il pensait faire de s'en revenir ainsi. Il me
conta la surprise et la douleur du père qui convenait bien que son fils
méritait sa disgrâce, et que la grâce faite à son petit-fils était
infinie, mais qu'il était père, et qu'il voyait son fils perdu\,; qu'il
s'écriait que je lui avais bien dit que je perdrais son fils, et
néanmoins sans aigreur\,; et que lui La Vrillière, peiné de ces
lamentations, voyant que je n'arrivais point, avait pris le parti de
revenir. «\,Fort mal à propos, lui dis-je, et vous reviendrez tout à
cette heure avec moi. C'est un pauvre homme peiné sur son fils qui
bientôt sentira la joie de sa considération personnelle, et de la
conservation de sa charge dans sa famille, qui autrement tôt ou tard en
serait sortie, et qu'il ne faut point que vous perdiez de vue que la
démission ne soit signée et emportée par l'abbé de Thesut.\,»

Nous arrivâmes chez le chancelier, qui se promenait seul dans son
cabinet. Dès qu'il m'aperçut\,: «\,Ah\,! voilà de vos coups,
s'écria-t-il, je reconnais votre main\,; vous chassez mon fils, et vous
sauvez son fils pour l'amour de moi et de sa mère\,; vous m'aviez bien
promis que vous perdriez mon fils. --- Monsieur, lui dis-je, il est vrai
que je vous l'avais dit dès le temps du feu roi, et longtemps avant sa
mort\,; je ne vous ai point trompé, je vous tiens parole, mais je fais
plus que je ne vous avais promis, car votre famille est sauvée, votre
petit-fils en place, et sa place bien mise à couvert d'être emblée.
Quelle plus grande consolation pour vous\,! et quelle plus grande marque
possible de la plus grande considération pour vous et de la plus
distinguée\,! --- Eh\,! je le sens, me répondit-il, et que je le dois à
votre amitié\,;» et se jeta à mon cou, puis ajouta\,: «\,Mais je suis
père, et quoique je connaisse bien mon fils, il me perce le cœur d'être
perdu.\,» Il s'attendrissait, les larmes lui venaient aux yeux, puis se
remettait dans la vue de son petit-fils.

Quand il fut un peu calmé, je lui fis remarquer que c'était le salut de
sa famille, parce qu'il était impossible que son fils subsistât encore
longtemps, et qu'étant chassé, personne n'aurait imaginé de faire passer
sa charge à un homme de l'âge de son fils, et aussi peu au fils de celui
qu'on chassait. Il en convint, m'embrassa encore tendrement, puis nous
parlâmes tous trois assez confusément pour battre, pour ainsi dire, la
campagne.

De temps en temps le chancelier revenait à son fait, à son fils, et me
dit\,: «\, Vous avez fait la lettre, j'ai senti votre style et toutes
vos précautions. Vous n'avez pas voulu que je pusse approcher de M. le
duc d'Orléans, par la défense qui en est dans la lettre, ni que je lui
fisse parler, et vous étranglez mon fils par le peu de temps qu'elle
prescrit pour l'exécution de l'ordre. Oh\,! que je vous reconnais bien à
tout cela, et toutes les honnêtetés pour moi dont la lettre est
pleine\,! --- Eh bien\,! monsieur, lui répondis-je, quand cela serait,
ai-je eu tort\,? Vous m'y aviez attrapé l'autre fois, en allant trouver
M. le duc d'Orléans\,; je n'ai pas voulu manquer mon coup une seconde.
Croyez-moi, vous vous consolerez comme père\,; et comme grand-père, et
père de famille, vous vous réjouirez après, et vous me saurez gré. ---
Hé\,! si je vous en saurai, reprit-il vivement, je vous en sais déjà, et
j'en enrage, car il est vrai que c'est à vous que je dois la charge de
mon petit-fils et le salut de ma famille.\,» Il m'embrassa encore en
ajoutant qu'il ne laisserait pas ignorer à son petit-fils quelle
obligation il m'avait, et lui ordonnerait bien de ne la jamais oublier.
Il le fit en effet, et de manière que je m'en suis toujours fort aperçu
dans la conduite de M. de Maurepas avec moi, et dans tous les temps par
son amitié et sa confiance.

Sur ce propos l'abbé de Thesut arriva. Un moment après, le chancelier
regarda sa pendule, puis se tourna à moi et me dit\,: «\,J'ai envoyé
chercher mon pauvre fils\,; il va arriver\,; il ne saurait douter que le
coup qui l'écrase ne parte de votre main. Épargnez-lui la peine qu'il
aurait de vous trouver ici dans ce cruel moment.\,» Là-dessus il
m'embrassa encore en me disant\,: «\,Vous êtes un terrible homme, et
avec cela, il faut encore que je vous aime, et que je ne m'en puisse
empêcher. --- Monsieur, lui répondis-je, en vérité, vous me devez cette
amitié, et vous ne sauriez douter de la force de la mienne par cette
marque d'attachement que je vous donne jusqu'en cette occasion qui sauve
votre petit-fils et votre famille, dont vous sentirez la joie tout
entière après ce premier trouble passé.\,» Là-dessus je m'en allai, le
laissant avec La Vrillière et l'abbé de Thesut, en présence desquels se
devait faire et signer la démission.

Je rencontrai en m'en retournant Pontchartrain qui allait fort vite chez
son père. Il avait l'air fort effaré. La Vrillière me conta
l'après-dînée qu'il était demeuré fort abattu, et point du {[}tout{]}
consolé par la fortune de son fils. Il n'osa pas faire la moindre
difficulté en présence de son père et de l'homme de M. le duc d'Orléans,
qui reçut entre onze heures et midi cette démission, par l'abbé de
Thesut.

Cette nouvelle répandit la joie dans Paris, et après dans les provinces.
Chacun se disait qu'il y avait longtemps que cela aurait dû être fait\,;
quelques-uns demandaient s'il en serait quitte pour sa démission. On fut
surpris de la disposition de la charge, qui rehaussa autant la
considération du chancelier de Pontchartrain qu'elle accabla son fils
par son ignominie purement personnelle et si parfaitement et
universellement applaudie. Nous nous en félicitâmes les uns les autres
au conseil de régence. Le maréchal d'Estrées parut ravi, et M. le comte
de Toulouse, à qui je ne pus refuser de conter comment cela s'était
passé.

Depuis ce moment Pontchartrain demeura obscur au fond de sa maison,
abandonné de plus en plus. Il y vit encore dans la solitude et le plus
parfait néant, toujours enragé de jalousie et de dépit contre son fils
qui lui rend des devoirs et rien de plus. Cet ex-bacha si rude et si
superbe occupe son néant à compter son argent et en semblables misères,
et n'a presque plus paru nulle part depuis, qui est ce qu'il a fait de
mieux.

J'avais toujours eu dans le cœur et dans l'esprit de sauver la charge à
son fils en le perdant. J'aimais et je devais au père, j'avais aussi eu
lieu d'aimer fort la chancelière\,; M\textsuperscript{me} de Saint-Simon
avait passé sa vie comme moi avec eux dans la plus grande intimité et
réciproque confiance. La mémoire de M\textsuperscript{me} de
Pontchartrain m'était présente, et aussi vive et aussi tendre dans le
cœur de M\textsuperscript{me} de Saint-Simon qu'au jour qu'elle l'avait
perdue. Je n'avais donc cessé de ruminer en moi-même les moyens de
sauver Maurepas de la chute de son père, et je le voulais sauver par
adresse, ou par effort de crédit, à quelque prix que ce fût. J'allai
donc chez M. le duc d'Orléans dans cet esprit, dont la considération
pour le père me fournit heureusement l'expédient que je saisis. La
Vrillière, qui n'abhorrait guère moins son cousin que moi, fut ravi d'en
être défait, et eut encore la joie pour son nom et pour la personne du
chancelier, auquel il était fort attaché, de voir la charge sauvée, et
de l'avoir entre ses mains avec le jeune titulaire pour disciple avec ce
surcroît de chose et de considération qu'il sentit bien et me dit qu'il
me devait tout entière.

J'étais encore dans les premiers jours de la satisfaction d'avoir perdu
Pontchartrain et sauvé sa charge à son fils, qu'il m'arriva une de ces
aventures que nulle prudence ne peut prévoir ni parer, et qui ressemble
à la chute fortuite d'une cheminée sur un passant dans la rue. Je veux
parler de l'éclat subit qui changea la longue amitié du comte et de la
comtesse de Roucy avec moi en rupture ouverte, qui ne se réconcilia
plus. Je ne puis me refuser de la traiter à fond, et il est nécessaire
pour cela de remettre courtement sous les yeux plusieurs choses qui se
trouvent éparses dans ces Mémoires, et d'expliquer quels furent le comte
et la comtesse de Roucy, dont sans cette nécessité, je ne me serais pas
avisé de parler expressément, au peu de figure qu'ils ont fait à la cour
et dans le monde.

Il est donc à propos de répéter ici que la comtesse de Roye fut la sœur
favorite de M. le maréchal de Lorges qui, depuis sa sortie du royaume
avec son mari, un de ses fils et deux de ses filles, lors de la
révocation de l'édit de Nantes, prit soin de ceux de ses enfants qui
demeurèrent en France comme des siens propres, et sans nulle différence
d'intérêts, de soins et d'amitié, jusqu'à sa mort. Je trouvai cette
famille sur ce pied-là en me mariant. J'ai toujours fait grand cas de
l'union des familles. Je voulus plaire à mon beau-père, qui prit pour
moi une amitié de père qui a duré autant que sa vie, et pour qui j'eus
toujours le plus tendre attachement et le respect le plus fondé sur
l'estime que je conserve encore chèrement à sa mémoire. Je vécus donc
avec ses neveux et leurs femmes dans la plus grande amitié,
M\textsuperscript{me} de Saint-Simon de même, et dans un commerce le
plus continuel, dans la liberté et la familiarité qu'il donne entre si
proches quand ils sont en aussi grande liaison.

Cette famille était composée du comte de Roucy, de Blansac, des
chevaliers de Roucy et de Roye, qui prirent, en se mariant à la fille de
Ducasse e à la fille de Prondre, le nom de marquis de Roye et de marquis
de La Rochefoucauld. M\textsuperscript{me} de Pontchartrain était leur
soeur. On a vu quelle était l'union, l'intimité, la confiance entre elle
et M\textsuperscript{me} de Saint-Simon. On se souviendra aussi qui et
quelle était M\textsuperscript{me} de Blansac\,; et que la comtesse de
Roucy était dame du palais de M\textsuperscript{me} la duchesse de
Bourgogne, et fille de la duchesse d'Arpajon, dame d'honneur de
M\textsuperscript{me} la Dauphine de Bavière, soeur du marquis de
Beuvron, père du maréchal d'Harcourt. Les quatre frères étaient fort
unis, et les deux belles-sœurs, à l'heureuse mode ancienne qui
subsistait encore un peu quand les plus âgés d'entre eux arrivèrent dans
le monde. Ils en eurent un grand usage, mais d'esprit pas l'apparence,
et presque aussi peu de sens. Je me retrancherai au comte et à la
comtesse de Roucy, parce que ce n'est que d'eux qu'il est question ici.
Mais on se souviendra aussi des tristes aventures du comte de Roucy à la
bataille de la Marsaille, que j'eus tant de peine à replâtrer par
Chamillart, et du même et de Blansac à celle d'Hochstedt, où leurs
femmes eurent encore tous leurs recours à moi, où je fis tout ce qui me
fut possible auprès de Chamillart, qui les servit de son mieux, mais qui
ne put cependant faire revenir le roi des impressions qu'il avait
prises, en sorte que ni l'un ni l'autre ne purent jamais obtenir de
servir depuis. Roucy, à l'abri de Monseigneur, du jeu, de la chasse, du
duc de La Rochefoucauld et de la place de sa femme, ne laissa pas de ne
bouger de la cour comme auparavant. Mais n'ayant jamais été bien traité
du roi, il le fut encore moins qu'auparavant.

C'était un grand homme, fort bien fait, de bonne mine, mais qui ne
promettait rien, et qui par cela même n'était pas trompeuse\,; l'air
fort et robuste, qui sentait son homme de guerre, et qui par sa figure
et ses talents naturels était fort bien voulu des dames, qui avaient
après le plaisir de s'en moquer. De commerce, on n'en pouvait guère
avoir avec lui. Tout occupé de la cour de Monseigneur, avec qui il était
fort bien, et dont le choix n'était pas difficile, de le suivre à la
chasse, de jouer le plus gros jeu a la cour et à Paris, il était plus
sur les chemins qu'ailleurs. C'est lui le premier qui a mis les valets
sur le pied de la parure, de la familiarité, de l'insolence, des gros
gains, en gâtant les siens, contagion qui, à son exemple, a de l'un à
l'autre gâté une infinité de maisons. Lui et ses frères étaient les rois
de la canaille. Ils étaient familiers avec elle, ils connaissaient les
valets de tout le monde, ils savaient leurs gages, leurs profits, leurs
jalousies, leurs débats, pourquoi chassés, pourquoi pris et sur quel
pied\,; en plaçaient, les protégeaient, et par là sottement adorés du
vulgaire et des marchands et artisans qu'ils payaient en amitiés, en
services et en compliments, et qu'ils satisfaisaient tellement de la
sorte qu'ils avaient crédit et leur amitié, et encore celle de leurs
pareils.

Quoique le comte et la comtesse de Roucy n'eussent jamais un poulet chez
eux, et que l'un et l'autre mangeassent toujours où ils pouvaient, ils
n'en étaient pas mieux dans leurs affaires, avec un gros revenu et de
belles terres. Tous deux rogues et glorieux à l'excès, tous deux bas
jusqu'au servage devant les ministres et toute faveur, ils avaient vécu
de ce qu'on appelle faire des affaires tant que Barbezieux avait existé,
dont le comte de Roucy était le complaisant abject, et depuis, de celles
qu'à force de souplesses, de bassesses, de tourments, la femme, encore
plus âpre et assidue que le mari, pouvait tirer de Pontchartrain, qui se
plaisait à les faire acheter bien cher. Son père était désolé de tout ce
qui se passait là-dessus, s'en échappait quelquefois, et ne se
contraignait pas de montrer à la comtesse de Roucy et à
M\textsuperscript{me} de Blansac qu'elles lui étaient insupportables.
Elles remboursaient tout cela sans rien dire, et allaient toujours leur
train.

L'aigreur et l'orgueil de la comtesse de Roucy lui attiraient tous les
jours des querelles où les injures lui coûtaient peu, le plus souvent
avec d'autres dames du palais pour leur service, avec qui souvent
M\textsuperscript{me} de Saint-Simon était employée à la raccommoder, et
si entreprenante qu'on ne put jamais l'empêcher d'aller à Marly, un
voyage qu'elle prétendait être de son tour, qu'elle n'était point sur la
liste, et que M\textsuperscript{me} la duchesse de Bourgogne ne voulut
pas l'y mener. Dès le même soir qu'on y arriva, elle reçut ordre de s'en
retourner sur-le-champ. Le rare est que ces aventures ne la corrigeaient
de rien.

C'était une créature vive, haute, toujours haïssant assez de gens pour
des querelles, quelquefois pour de vieux procès ou pour d'autres
affaires, et ne contraignant ni ses discours ni ses manières à leur
égard\,; toutefois assidue aux dévotions, à la grand'messe de paroisse à
Versailles, les fêtes et dimanches, y communiant tous les huit jours\,;
avec cela l'envie et la jalousie même, et l'ambition, et se persuadant
que tout était dû à son mari et à elle, avec qui, à la vie qu'ils
menaient tous deux, et au peu au fond qu'ils se souciaient l'un de
l'autre, elle n'avait de commerce qu'en courant, en faisant toujours la
passionnée. Elle se faisait aussi des châteaux en Espagne, et les
débitait, soit qu'elle voulût persuader qu'ils étaient à portée de tout,
soit que, comme je l'ai toujours cru, elle s'en persuadât elle-même.

Étant un soir seule chez elle assez tard, quelque temps après la mort de
M. le maréchal de Lorges, elle me conta ce qui lui plut sur ce qu'elle
avait fait avec M\textsuperscript{me} de Maintenon, et m'assura que le
lendemain matin son mari serait fait duc ou capitaine des gardes, mais
qu'elle aimerait bien mieux qu'il eût cette charge de son oncle qui
sûrement le conduirait à être bientôt duc, que s'il était fait duc
alors, et n'aurait point de charge. Je me moquai d'elle sans pouvoir
jamais lui mettre là-dessus le moindre doute dans l'esprit.

C'était peu connaître la cour, pour une femme qui y était en quelque
place et depuis si longtemps. Le roi était buté à n'avoir pour capitaine
de ses gardes que des maréchaux de France, et même des ducs. Il avait
fait ducs tous les premiers gentilshommes de sa chambre, maréchaux de
France et souvent ducs tous les capitaines de ses gardes, et n'avait
jamais accordé pas une de ces charges, quand elles avaient vaqué, qu'à
des gens qui fussent ducs ou maréchaux de France, et souvent l'un et
l'autre. Il n'avait donc garde de changer de conduite à cet égard pour
un homme qu'il n'avait jamais bien traité, et pour qui son estime ne
paraissait pas, puisque depuis Hochstedt, il avait constamment refusé de
l'employer dans ses armées, quelques machines qui aient été remuées pour
l'obtenir. Il n'avait que les Marlys, où le roi ne lui parlait pas plus
qu'ailleurs, et où il ne le menait que comme joueur et chasseur. Il n'a
seulement jamais pu être menin de Monseigneur, quoiqu'il le suivît sans
cesse, et il est mort vieux sans charge, sans gouvernement, sans ordre
et sans dignité.

C'était en soi un homme fort rustre, brutal et désagréable, et dont les
bêtises se sont conservées à la cour, par exemple, le conseil qu'il
donna à la marquise de Richelieu, qui était incommodée et qui se
plaignait fort du bruit des cloches, de faire mettre du fumier dans sa
cour et devant sa maison, et bien d'autres de cette force. Envieux aussi
au dernier point\,: on en a vu un échantillon en son lieu à la mort du
duc de Coislin, frère de M. de Metz, à qui, par une autre raison, cela
coûta longtemps cher.

Telles étaient ces personnes avec qui M\textsuperscript{me} de
Saint-Simon et moi, depuis notre mariage, avions constamment vécu dans
la plus grande amitié et la plus grande union, jusqu'à l'aventure qu'il
s'agit maintenant de raconter.

Le maréchal d'Harcourt, comme on l'a vu en son lieu, ne voulait
qu'entrer dans le conseil, ne désirait que cela, ne travaillait qu'à
cela, et n'eut la charge de capitaine des gardes de mon beau-père que
malgré lui, parce qu'il n'avait osé ne la demander pas, et que le roi
fut bien aise de la lui donner pour, après une telle grâce, l'éconduire
plus nettement d'une place dans son conseil. Harcourt n'était pas riche,
il avait beaucoup d'enfants\,; sa santé était fort attaquée, il voyait
une longue minorité sans prévoir comment la cour se tournerait après\,;
il résolut de se défaire de sa charge. Le comte de Roucy en eut le vent,
et lui eu demanda la préférence.

Dans le moment qu'Harcourt la lui eut promise, qui était cousin germain
de sa femme, et en grande liaison avec eux, mais peu à portée de crédit
auprès de M. le duc d'Orléans, qui ne l'avait mis que par nécessité dans
le conseil de régence, le comte de Roucy vint tout courant à moi me
prier de lui obtenir l'agrément du régent. Je n'ignorais pas le vieux
levain de Meudon où, pour plaire, il n'avait gardé aucune mesure sur ce
prince, qui dans ces temps-là m'en avait souvent parlé avec dépit et
colère, contre un homme qu'il avait toujours bien traité partout où il
l'avait rencontré\,; mais je connaissais aussi sa débonnaireté parfaite
pour tous ceux qui lui avaient le plus étrangement manqué. Ainsi je ne
crus pas trouver de difficulté et je promis au comte de Roucy de parler
au régent et d'y faire de mon mieux. Je le fis dès le lendemain.

Ma surprise fut grande de trouver une barre de fer. J'insistai, et si
fort que la dispute se tourna en aigreur de sa part. Il me ramena tous
les propos de Meudon, leur amertume, leur énormité de la part du comte
de Roucy, les preuves qu'il en avait et qu'il m'avait dites dans le
temps, fort scandalisé que, informé de toutes ces choses, je lui
proposasse et j'insistasse pour faire un tel capitaine des gardes du
corps. Je cédai peu à peu, mis d'autres matières sur le tapis, et, quand
je crus voir ma belle, je demandai à M. le duc d'Orléans pourquoi cette
exception rigoureuse contre le comte de Roucy, quand il ne refusait rien
à tant d'autres qui lui avaient nui essentiellement, tandis que celui-ci
n'avait dit que des sottises pour plaire, et parler le langage du lieu
dont il espérait tout. Je fus bien plus étonné que la première fois\,;
le régent rougit, et avec une impétuosité qui lui était extrêmement
rare, insiste sur les choses de Meudon et leurs suites, sur la
différente conduite de Biron, Sainte-Maure et du Mont qui n'étaient pas
moins liés là et n'en attendaient pas moins que Roucy toute leur
fortune, et de là tomba avec furie sur la Marsaille et Hochstedt, et me
reprocha de lui vouloir faire faire un plaisant capitaine des gardes par
rapport au roi, à lui, et même au public en ce genre qui connaissait ce
qui s'était passé en ces deux combats. La conclusion fut de me défendre
de lui en plus parler, et un ordre de dire au comte de Roucy de sa part,
qu'il ne changerait rien là-dessus à la disposition constante du feu
roi, qui n'avait accordé ces charges-là qu'à des ducs ou à des maréchaux
de France, dont il suivrait exactement l'exemple, et se garderait bien
d'y manquer. Cela dit et répété fort sec, le régent entama d'autres
propos et différentes matières.

Pendant cette dernière partie de la conversation, convaincu qu'il n'y
avait plus à revenir au comte de Roucy, je pensai à mon beau-frère.
C'était la charge de son père. Je ne pus me résoudre à la demander pour
moi, pouvant l'espérer pour lui, quoique j'eusse tout lieu d'en être
très mal content, et que jamais il n'eût daigné se mettre à portée de
rien. La demander pour lui à la fin de la conversation, et l'obtenir, ce
fut la même chose. J'avais affaire à des gens peu faciles pour
l'arrangement du payement de quatre cent mille livres, quoique j'eusse
obtenu en même temps le même brevet de retenue. Je convins donc avec M.
le duc d'Orléans qu'il tiendrait l'agrément secret, jusqu'à ce que
toutes nos mesures fussent prises et arrêtées. Jamais il ne m'entra dans
l'esprit que le comte de Roucy pût avoir le plus léger soupçon de ma
conduite à son égard. La façon dont j'avais vécu avec lui toute ma vie,
et dont en toute occasion je l'avais servi, et la franchise et la
droiture dont j'étais connu, n'avaient pas permis de laisser entrer en
mon esprit aucune pensée de doute.

Je témoignai donc le lendemain au comte de Roucy, qui vint chez moi,
combien j'étais fâché de n'avoir pu réussir à lui faire obtenir ce qu'il
désirait, et d'avoir vu tous mes efforts inutiles. Roucy bien étonné, et
encore plus fâché, me demanda la cause de son malheur, et me pressa
tellement qu'il me força de lui rendre la réponse que j'avais ordre
positif de lui faire. Il n'en fallut pas davantage pour donner l'essor à
sa furie. Il cria contre cette prétendue nécessité d'être duc ou
maréchal de France pour être capitaine des gardes du corps, déclama
contre le régent, s'en alla chez lui, puis avec sa femme chez Harcourt,
où ils firent les hauts cris. Pour rendre la chose plus touchante d'une
part, plus injurieuse de l'autre, ils ajoutèrent à ma réponse que
j'avais eu tant de peine à lui rendre, et que j'avais adoucie le plus
que j'avais pu, ils ajoutèrent, dis-je, que je lui avais dit que Son
Altesse Royale ne voulait pas avilir ces charges en les donnant à des
gens non titrés, et on peut juger de l'effet de ce propos dans
l'effervescence qui s'entretenait encore avec tant d'art et de manège,
sur cette calomnie atroce inventée par le duc de Noailles, de cette
salutation du roi que j'ai expliquée en son lieu.

Le lendemain de ce vacarme, M. le duc d'Orléans tourmenté à souper par
les convives, et surtout par les dames curieuses d'apprendre qui aurait
la charge, tint bon longtemps, puis entre la poire et le fromage lâcha
le secret qu'il m'avait promis de garder. Ce fut la nouvelle du
lendemain matin.

Là-dessus le comte et la comtesse de Roucy prirent espérance de
m'embarrasser assez par un grand éclat contre moi, pour me forcer pour
l'amour de moi-même de mettre tout mon crédit à leur faire avoir la
charge. C'est au moins ce qui parut par tout l'artifice de leur
conduite, car dès ce même jour la comtesse de Roucy vint chez moi au
sortir de table comme pour m'apprendre, tout en douceur et en amitié, le
bruit que faisait cette affaire qui se répandait dans le monde\,;
qu'elle me connaissait trop et de trop longue main pour me soupçonner le
moins du monde d'avoir promis à son mari de parler pour lui, et de
n'avoir parlé que pour mon beau-frère\,; mais que le monde était si
méchant, et son mari si outré, qu'elle me conjurait, autant pour
moi-même que pour lui, de faire encore un effort.

Je lui répondis que je ne craignais point ces soupçons\,; que si j'avais
voulu la charge pour moi ou pour le duc de Lorges, rien ne m'empêchait
de le dire franchement au comte de Roucy, quand il vint me prier de
parler pour lui, et de m'en excuser, puis d'aller mon chemin à
découvert, à quoi personne ni lui-même n'aurait pu trouver quoi que ce
soit à reprendre\,; qu'aussi j'avais été pour lui rondement et
nettement\,; qu'à la vérité, me voyant éconduit pour lui à deux diverses
reprises, et telles qu'il n'y avait plus nul moyen d'y revenir une
troisième, la pensée m'était venue de proposer le duc de Lorges, sans
aucune qu'il en pût naître aucun soupçon\,; mais que, pour couper court,
je voulais bien faire encore un effort, et de toutes mes forces, puisque
je l'avais bien fait d'abord, mais à deux conditions, la première que ce
serait en présence du comte de Roucy qui serait témoin lui-même de tout
ce qui se dirait et se passerait, lui en tiers entre le régent et moi\,;
la seconde, que, puisque le monde s'avisait de soupçons, je monterais
actuellement dans son carrosse avec elle, et, sans la quitter, j'irais
prendre le comte de Roucy où qu'il fût, et, en sa présence à elle, le
mener sur-le-champ au Palais-Royal, où je lui répondais que, quoi que
pût faire M. le duc d'Orléans, nous le verrions sans remise\,; que je
n'entrerais qu'avec le comte de Roucy, et ne parlerais que devant lui.
J'ajoutai que cela était net et prompt, et court, exclusif de tout moyen
d'écrire, ou de faire parler à M. le duc d'Orléans, puisque je ne les
quitterais pas un instant l'un ou l'autre, ni ne parlerais bas à
personne dans l'entre-deux, ni à M. le duc d'Orléans en présence du
comte de Roucy que je ne quitterais pas un instant, et qu'en tiers avec
le régent et moi il serait témoin et juge si j'y allais bon jeu bon
argent, et verrait bien encore aux propos du régent, si mon langage
serait autre que n'avait été le premier.

La comtesse de Roucy, également aise et surprise, accepta la
proposition, et sur-le-champ nous montâmes tous deux dans son carrosse
que le mien suivit, et allâmes chez elle où son mari était, vis-à-vis
les Incurables. Elle fit apparemment ses réflexions en chemin, car elle
me dit que son mari était si outré, qu'elle me demandait en grâce de la
laisser entrer dans sa chambre pour lui parler avant que je le visse,
parce que mon procédé était si bon, et ma proposition si nette qu'elle
serait au désespoir qu'il fût mal reçu, comme cela pouvait arriver à un
homme fâché, dans la surprise. J'y consentis, mais à condition qu'elle
ne me laisserait attendre qu'en compagnie qui ne me quitterait pas
jusqu'à ce qu'elle revint. Il y en avait, en effet, dans la première
pièce avec qui je demeurai, à qui je ne cachai pas ce qui m'amenait, et
qui me parut dans l'étonnement et dans l'admiration de ce procédé.

Il y en avait d'autre dans la pièce d'après (je n'ai point su qui), où
était la comtesse de Roucy et où était son mari. Leur conseil fut long.
La conclusion fut que la comtesse de Roucy en sortit seule, me dit
qu'elle était outrée de douleur\,; que je connaissais son mari et
l'excès de son opiniâtreté\,; qu'il n'y avait jamais eu moyen de le
résoudre à me voir\,; que cela reviendrait, mais qu'elle me priait
d'aller encore au Palais-Royal, et de faire tout mon possible.

Alors je vis à découvert tout leur manège. Ils voulaient me forcer par
l'éclat à en faire ma chose propre, et à emporter la charge pour le
Roucy\,; si je réussissais, ils avaient leur compte, et le bâton haut\,;
si je n'obtenais rien, faire contre moi tout l'éclat imaginable\,; ce
qui ne se pouvait plus si le Roucy était témoin en tiers entre le régent
et moi, selon la condition que j avais mise. Aussi pris-je un autre ton
pour répondre à la comtesse de Roucy\,: je lui dis que je n'aurais pas
imaginé qu'une proposition aussi nette et aussi décisive du fait, aussi
facile, et que j'avais commencé à exécuter en venant chez elle avec
elle, pût être susceptible de refus\,; que j'estimais, au contraire,
qu'elle méritait toute autre chose\,; que je pensais que tout le monde
le trouverait ainsi, et verrait clair aux deux procédés\,; que, pour
cela même, je la faisais encore, et m'offrais de nouveau à l'exécuter à
l'instant, mais que si le refus persistait, j'entendrais ce que cela
voudrait dire, et que j'en serait fort étonné après une amitié de vingt
ans, telle qu'avait été la mienne. Tout cela se passa tout haut devant
ce que j'avais trouvé dans cette première pièce.

La comtesse de Roucy voulut répondre souplement, mais je la priai que
nous ne perdissions point le temps, et de retourner à son mari. Elle y
entra. Le parti était pris, elle y demeura peu, et revint me dire les
mêmes choses. Je lui répondis qu'après ce que j'avais fait, proposé,
commencé de ma part à exécuter en venant chez elle, avec elle, et encore
d'insister, je n'avais plus qu'à prendre congé d'elle, lui fis la
révérence, une autre à la compagnie, et m'en allai.

Dès ce même jour les cris redoublèrent, le comte et la comtesse de Roucy
coururent les maisons, et eurent beau jeu, parce que plus que content de
ce que j'avais fait, je ne pris pas la peine de m'en remuer. Trois ou
quatre jours se passèrent de la sorte. À la fin nous fûmes,
M\textsuperscript{me} de Saint-Simon et moi, avertis de tant d'endroits
des vacarmes et des propos du comte et de la comtesse de Roucy, qui
retentissaient partout, que j'allai au Palais-Royal où je trouvai M. le
duc d'Orléans avec M. le comte de Toulouse chez M\textsuperscript{me} la
duchesse d'Orléans, qui allait dîner seul à son ordinaire avec la
duchesse Sforce. Là je dis à M. le duc d'Orléans, devant cette courte
compagnie, tout ce qui s'était passé entre la comtesse de Roucy et moi,
que je viens de raconter, les clabauderies et les propos qui me
revenaient d'eux de toutes parts, enfin ce qu'il voyait bien que je ne
pourrais m'empêcher de faire, que j'avais voulu lui rendre ce compte
auparavant pour n'être pas au moins blâmé après par quelque autre tour
d'adresse. J'ajoutai que puisque M. le comte de Toulouse se trouvait là
heureusement présent, je le suppliais de vouloir bien lui dire de quelle
façon l'affaire de la charge s'était passée entre Son Altesse Royale et
moi, et d'avoir la bonté, puisque c'était chose passée, de lui confier
la raison personnelle et secrète de l'exclusion du comte de Roucy. M. le
duc d'Orléans fit l'un et l'autre, en sorte que le comte de Toulouse vit
à quel point toute raison, vérité, et net et bon procédé était de mon
côté.

Je voulus après m'en aller en ouvrant la porte aux plats et au service
qui avaient été arrêtés pendant toute cette conversation. M. le duc
d'Orléans me rappela et me retint malgré moi, jusqu'à faire tenir la
porte, et envoya sur-le-champ chercher le comte de Roucy, fort en colère
et bien plus que d'ordinaire à lui n'appartenait. Au bout de quelque
temps, je représentai si fortement le peu de convenance que je me
trouvasse présent à la vesperie\footnote{} qu'il voulait faire au comté
de Roucy, et le danger même de quelque manque de respect en sa présence,
que le comte de Toulouse m'aida à obtenir la permission de me retirer.
Je rencontrai le comte de Roucy sur le quai des galeries du Louvre, qui
allait à toutes jambes au Palais-Royal.

On l'y conduisit au lieu d'où je sortais, ou il trouva les mêmes
personnes et le dîner qui continuait, que M. le duc d'Orléans et le
comte de Toulouse, qui ne dînaient jamais, regardaient. M. le duc
d'Orléans, en leur présence, et sans renvoyer le service d'autour de la
table, parla au comte de Roucy un langage qu'il n'avait pas accoutumé,
dont le Roucy demeura étourdi et accablé. Cela mit fin à ses propos, à
ceux de sa femme, et même à ceux des gens qu'il avait mis à courir le
monde pour les répandre. Oncques depuis n'avons-nous ouï parler d'eux.

La comtesse de Roucy, qui ne communiait peut-être pas si souvent qu'elle
faisait à la cour du temps du roi et de M\textsuperscript{me} de
Maintenon, mourut à Paris un an après cet éclat, c'est-à-dire en
décembre 1716, sans avoir pensé à le réparer. Le comte de Roucy mourut
aussi à Paris, mais en novembre 1721, comme je venais de partir pour
l'Espagne. Quand il fut bien mal, il envoya prier M\textsuperscript{me}
de Saint-Simon de l'aller voir. Elle y fut, et cela se passa comme il
arrive en ces terribles moments, où la figure du monde s'éclipse, et où
la vérité seule paraît. Il la pria de me mander toutes sortes de choses
de sa part. Les autres Roucy, mâles et femelles, nous les avons revus,
quelques-uns même en amitié, qui n'avaient jamais approuvé ce qui
s'était passé à mon égard. Tout l'éloignement se concentra au fils du
comte de Roucy, qui mourut en 1725, mais surtout en sa femme, qui n'est
morte que depuis quelques années, aussi extraordinaire et aussi
follement glorieuse qu'elle était riche et de bas lieu. Elle n'a laissé
que deux filles, l'une duchesse d'Ancenis, l'autre de Biron, que
l'archevêque de Bourges a toutes deux mariées depuis sa mort. C'est le
seul fils qui reste du comte de Roucy, qui n'a pas pris les sentiments
de sa mère à notre égard, qui est commandeur du Saint-Esprit, nommé au
cardinalat et ambassadeur à Rome.

Pour la charge, M. de Lorges tira de sa mère tout ce qu'il put, aux
dépens de qui il appartiendrait, pour faire ses arrangements. Il ne tint
plus qu'à vendre sa petite guinguette de Livry pour achever la somme et
signer avec M. d'Harcourt. M. de Lorges ne se souciait point pour lui
d'être capitaine des gardes, encore moins pour son fils\,; il aimait
mieux ses plaisirs que tout. Quand il se fut bien assuré de ce que la
perspective si sûre et si prochaine de la charge de son père lui fit
obtenir de sa mère, il déclara qu'il ne vendrait point sa petite maison,
et au fond fut ravi de rompre le marché, et ne se soucia guère que je
l'eusse préféré à moi, étant à mon choix de prendre la charge, ni de
l'éclat qu'elle m'avait valu avec le comte de Roucy. Cet honnête
beau-frère se retrouvera ailleurs. Pendant tous ces négoces, la famille
du maréchal d'Harcourt se ravisa\,; il demanda sa charge pour son fils,
et il l'obtint. Ainsi il mangea l'huître dont le Roucy et M. de Lorges
n'eurent que les écailles, que je trouvai toutes deux fort dures. Il est
temps maintenant de parler des affaires étrangères.

\hypertarget{chapitre-xiii.}{%
\chapter{CHAPITRE XIII.}\label{chapitre-xiii.}}

1715

~

{\textsc{Mouvements d'Écosse.}} {\textsc{- Caractère de Stairs et ses
menées.}} {\textsc{- Rémond\,; quel.}} {\textsc{- Mouvements
d'Angleterre.}} {\textsc{- Conduite de l'Espagne.}} {\textsc{- Manèges
d'Albéroni pour gouverner seul.}} {\textsc{- Projets politiques
d'Albéroni.}} {\textsc{- Cause de la dépendance des Provinces-Unies de
l'Angleterre.}} {\textsc{- Albéroni éloigné de la France, encore plus du
régent, méprise les bassesses du duc de Noailles.}} {\textsc{- Il chasse
avec éclat le gouverneur du conseil de Castille.}} {\textsc{- Sa
correspondance avec Effiat.}} {\textsc{- Négociation de Stairs pour la
mutuelle garantie des successions de France et d'Angleterre.}}
{\textsc{- Le régent y veut engager la Hollande.}} {\textsc{- Stairs
presse le régent de faire arrêter le Prétendant, passant de Bar, caché,
en Bretagne pour s'embarquer.}} {\textsc{- Le Prétendant échappe aux
assassins de Stairs par le courage et l'adresse de la maîtresse de la
poste de Nonancourt, qui en est mal récompensée.}} {\textsc{- Il
s'embarque en Bretagne.}} {\textsc{- Impudence de Stairs et de ses
assassins.}}

~

Le feu roi était revenu à son goût naturel et à ses anciens principes
sur l'Angleterre, depuis la mort de la reine Anne, et l'éloignement de
tous emplois, et la disgrâce de toutes les personnes qui avaient sa
confiance et qui formaient son conseil. Le roi son successeur avait
remis en place tous ceux qu'elle en avait ôtés, les whigs en principal
crédit, et éloigné de tous les torys. On ne peut exécuter de si grands
changements, non seulement dans un gouvernement, mais dans tout un pays
naturellement porté aux factions, sans faire un grand nombre de
mécontents de toute espèce, d'autant plus que les nouveaux ministres et
favoris qui ne respiraient que vengeance contre ceux qui les avaient
chassés et pris leurs places sur les dernières années du dernier règne,
ne voulaient rien moins que les poursuivre et faire condamner
juridiquement ceux d'entre eux qui avaient eu le plus de part à la paix,
et à qui, par conséquent, la France avait le plus d'obligation. L'Écosse
ne se consolait point de se voir enfin tout à fait devenue province
d'Angleterre. Le duc d'Ormond se tenait caché dans Paris, en attendant
ce que le comte de Marr pourrait faire en Écosse, où il y avait un parti
en mouvement, et le Prétendant, pour parler un langage reçu, était à
Bar, qui n'attendait qu'une conjoncture un peu apparente pour passer la
mer, certain de la protection et des secours du roi et peut-être du roi
d'Espagne.

La mort du roi, qui entrait secrètement, mais de tout son cœur, dans ce
projet, qui pouvait même être bientôt favorisé par la Suède et la
Russie, qui avaient toutes deux grande envie de terminer leur guerre par
un traité de paix à ce dessein, le déconcerta. Une minorité, dans l'état
où le roi laissait l'intérieur de la France, n'était pas un temps propre
à risquer de rompre avec l'Angleterre, sans être bien assuré de ce dont
il est bien difficile de l'être, je veux dire d'une révolution subite et
entière, à peu près telle que fut celle qui plaça le roi Guillaume sur
le trône du roi son oncle et son beau-père, laquelle relierait en même
temps la France qui y aurait eu part avec l'Angleterre, et ne lui
laisserait d'ennemis qu'un électeur d'Hanovre, et ceux qui hors les îles
Britanniques se voudraient hasarder à prendre les armes pour lui. Le feu
roi, comme on l'a vu, avait laissé le trône de Philippe V bien raffermi,
l'union des deux couronnes parfaite, et toutes deux jouissant de la paix
avec toute l'Europe par les traités d'Utrecht et de Bade. M. le duc
d'Orléans voulait absolument conserver un bien si nécessaire

D'autres circonstances l'éloignaient encore de se prêter au projet du
feu roi en faveur du Prétendant. Le comte Stairs était en France de la
part du roi Georges plus d'un an avant la mort du roi, sans avoir encore
pris le caractère d'ambassadeur qu'il avait dans sa poche. C'était un
très simple gentilhomme écossais, grand, bien fait, maigre, encore assez
jeune, avec la tête haute et l'air fier. Il était vif, entreprenant,
hardi, audacieux par tempérament et par principe. Il avait de l'esprit,
de l'adresse, du tour\,; avec cela actif, instruit, secret, maître de
soi et de son visage, parlant aisément tous les langages, suivant qu'il
les croyait convenir. Sous prétexte d'aimer la société, la bonne chère,
la débauche qu'il ne poussait pourtant jamais, attentif à se faire des
connaissances et à se procurer des liaisons dont il pût faire usage à
bien servir son maître, et son parti à lui-même. C'était celui des whigs
et de tous ceux que le roi Georges avait remis en place, et la famille
et les amis du duc de Marlborough dont il était créature, à qui il avait
de tout temps été attaché, sous qui il avait servi, qui l'y avait avancé
et procuré un régiment et l'ordre d'Écosse. Il était pauvre, dépensier,
fort ardent et fort ambitieux, et il voulait servir de façon, dans son
ambassade, qu'avec les appuis qui le protégeaient, il pût faire une
grande fortune en Angleterre où son parti, auquel il était dévoué, et
ses patrons dominaient, et à qui il plaisait d'autant plus qu'il
haïssait la France autant qu'eux. On a vu que le feu roi fut promptement
et toujours après très mécontent de sa conduite\,; Torcy encore plus,
jusque-là qu'il refusa et cessa de le voir et de plus traiter avec lui.

Stairs vit de loin la décadence menaçante de la santé du roi. Il comprit
en même temps qu'il n'avait rien à espérer de l'autorité du duc du
Maine, qui, si elle prévalait, ne s'écarterait pas dans le gouvernement
du goût et des maximes du roi. Il sentit donc de bonne heure qu'il
n'avait de parti à prendre que celui de M. le duc d'Orléans qui avait
tout le droit de son côté, le flatter du secours de son maître, s'il en
avait besoin pour faire reconnaître sa régence et l'autorité qu'elle lui
donnait, l'enrôler, pour ainsi dire, de bonne heure avec le roi Georges,
par ces offres faites dans un temps douteux, le lier avec lui, en lui
persuadant que leurs intérêts étaient communs, et (pour en parler
franchement, car il ne craignit point d'en laisser échapper les propres
termes) que deux usurpateurs et aussi voisins se devaient soutenir
mutuellement, envers et contre tous, puisque tous deux étaient dans le
même cas, Georges à l'égard du Prétendant, M. le duc d'Orléans au faible
titre des renonciations à l'égard du roi d'Espagne, si un enfant tout
tendre, et aussi jeune qu'était le successeur de Louis XIV, venait à
manquer.

Sur ces principes Stairs songea de bonne heure à ce qui pouvait servir à
son dessein. Il ne dédaigna rien de ce qu'il crut l'y pouvoir conduire.
Il ramassa donc une de ces espèces qui ne peuvent guère être
caractérisées sous un autre nom. C'était un petit homme fort du commun,
et pis pour la figure, qui, à force de grec et de latin, de
belles-lettres et de bel esprit, s'était fourré où il avait pu, puis,
{[}à force{]} de débauche de toute espèce et de sentiments si
malheureusement à la mode, était parvenu à voir des femmes, et quelque
sorte de bonne compagnie. Il était galant, faisait des vers\,; il était
aussi philosophe, fort épicurien, grossier de fait, sublime et épuré de
discours, admirateur des savants anglais, et devenu un des commensaux à
Paris de la comtesse de Sandwich, qui s'y plaisait plus qu'à Londres. Il
y avait fait grande connaissance avec l'abbé Dubois qui n'en bougeait,
et par lui s'était produit à M\textsuperscript{me} d'Argenton et à M. le
duc d'Orléans, dont peu à peu il avait tiré un bouge au Palais-Royal, et
un autre à Saint-Cloud, où de fois à autre il allait faire le philosophe
solitaire, et n'y manquait pas M. le duc d'Orléans, quand rarement il
s'y allait promener. Il avait du manège, de l'entregent, de la
hardiesse, de l'audace même quand il s'y laissait aller, du débit
surtout, et devint peu à peu l'homme de l'abbé Dubois à tout faire. Il
s'appelait Rémond, et frappait à tout ce qu'il pouvait de portes. Stairs
l'écuma, et lui courtisa Stairs, de la connaissance, puis de la société
de qui il s'honora beaucoup avec raison, et peu à peu se livra
entièrement à lui.

Rien ne convenait davantage à l'abbé Dubois qui, déjà éloigné par M. le
duc d'Orléans pour avoir voulu trop se mêler, ne savait par où se
reprendre, et qui regarda sa liaison avec Stairs, et par lui avec
l'Angleterre, comme une ressource dont il se promit de grands avantages.
Rémond lia donc bien aisément ces deux hommes dont l'intérêt de chacun
le demandait également Dubois l'était, comme on l'a vu, déjà avec
Canillac et le duc de Noailles. Il l'était aussi avec Nocé. Il leur
persuada qu'il n'y avait de salut pour M. le duc d'Orléans que par
l'Angleterre contre tout ce qui s'opposerait à l'autorité que sa
naissance lui donnait de droit après le roi, et pour l'appuyer ensuite.

Il avait fait des promenades en Angleterre où il avait fait des
connaissances, et fort cultivé celle de Stanhope qu'il avait beaucoup vu
autrefois à Paris, et avec qui il avait ménagé quelque commerce
d'ancienne connaissance pendant qu'ils étaient en Espagne, l'un à la
tête des troupes anglaises, l'autre à la suite de M. le duc d'Orléans,
qui avait été souvent avec lui en débauche autrefois à Paris. Dubois
compta qu'en tournant ce prince du côté de l'Angleterre, il deviendrait
nécessairement l'entremetteur, et de là le négociateur, dont il se
promit toutes choses. Malheureusement il ne se trompa pas.

Rémond s'était fourré avec Canillac qu'il avait gagné par la conformité
de goût, et par des admirations de son esprit et de ses lumières, dont
il se moquait ailleurs, mais qui l'avaient mis dans sa confiance. Il lui
vanta Stairs, flatta sa vanité du désir de ce ministre de le connaître,
à qui il fit sa cour de le mettre en liaison avec un favori de M. le duc
d'Orléans. Il l'instruisit du faible du personnage\,; il les joignit, et
Canillac ne jura plus que par Stairs et par l'Angleterre. Tout cela se
fit de concert entre Dubois et Rémond, et comme Nocé leur était alors
fort uni, et qu'avec sa tête brûlée, mais son air de philosophe, il ne
laissait pas d'usurper d'habitude une sorte d'autorité sur M. le duc
d'Orléans, parce que sa philosophie n'excluait pas la débauche, ils
l'entraînèrent dans leurs idées anglaises, et dans la société de Stairs.

Tout cela se pratiquait à Paris, dans la dernière année du feu roi, vers
la fin duquel ils parlèrent à M. le duc d'Orléans des avantages uniques
qu'il ne pouvait tirer que de son union avec le roi Georges, et de là
des propos, puis des offres de Stairs. M. le duc d'Orléans, qui
craignait tout alors des dispositions du roi, et de sa dépendance de
M\textsuperscript{me} de Maintenon et du duc du Maine, écouta bien
volontiers ces propositions. Dubois et Canillac y firent entrer le duc
de Noailles, qui pour s'ancrer ne songeait qu'à les flatter et s'en
appuyer, et qui y donna tant qu'il voulurent. Cette pointe se poussa
jusque-là que M. le duc d'Orléans vit Stairs au Palais-Royal par les
derrières.

Il m'en parla tard et par hoquets. Il savait que je pensais sur
l'Angleterre comme le feu roi, et ne me fit cette confidence qu'après
coup pour ne me la pas cacher. À chose faite il n'y a plus rien à dire,
sinon que je le suppliai de ne s'engager pas trop avant, et de se bien
persuader que Stairs ne songeait qu'à soi et à son parti, et à profiter
des conjonctures présentes pour tirer de lui les partis les plus
avantageux, qu'il saurait après faire valoir d'une minière fort
embarrassante.

Voilà ce qui causa l'indécence de la présence de Stairs dans une
lanterne à la séance de la régence, où il voulut assister pour se faire
de fête auprès de M. le duc d'Orléans que les mêmes personnes
persuadèrent de le désirer même, pour montrer son union avec
l'Angleterre, et tenir le parlement et le duc du Maine en respect.

Canillac, que je ne voyais même guère, vint chez moi quelques jours
auparavant me vanter les intentions de Stairs, ses offres, leur utilité,
et me prier, s'il venait chez moi, de lui laisser la porte ouverte en
quelque temps que ce fût. Je pris pour bon tout ce qui était fait, et ne
voulus point de dispute avec un homme aussi infatué qu'il l'était de son
mérite et des Anglais. L'abbé Dubois, après ce qu'on a vu que Madame dit
et demanda à M. le duc d'Orléans de lui et pour son exclusion totale, se
sut bon gré de sa liaison anglaise, qui avait déjà servi à le faire
souffrir un peu mieux de M. le duc d'Orléans. Il la regarda de plus en
plus comme son unique ressource, et s'y livra à corps perdu.

Dès le milieu du mois d'octobre, Stairs eut une longue audience du
régent sur les alarmes de son maître, qui prétendit que le comte de
Peterborough avait découvert une conspiration prête à mettre le feu au
palais où demeure la maison royale, piller la banque, se saisir de la
Tour de Londres et proclamer le Prétendant. On avait surpris des lettres
de M. Hervey au Prétendant ou au duc d'Ormond, qui lui furent
représentées. Il voulut se tuer\,; mais ses blessures ne se trouvèrent
pas mortelles. Le grand nombre de mécontents, et qui parlaient haut dans
Londres et dans les provinces, donnèrent du corps à cette prétendue
conspiration dans l'esprit du roi Georges. Il demanda aux Hollandais le
corps de troupes qu'ils étaient obligés de lui fournir, qu'il voulait
envoyer au duc d'Argyle, pour s'opposer au comte de Marr, qui était fort
suivi, avait des succès et se conduisait sagement. Les États généraux
accordèrent trois mille Suisses, et autres trois mille suivant le traité
qui fixe ces secours à six mille.

L'Espagne se refroidit beaucoup à l'égard du Prétendant depuis la mort
de Louis XIV. Elle voulut au dehors satisfaire le roi Georges par toutes
sortes d'extérieur à cet égard, sans néanmoins rompre avec le malheureux
prince dans l'incertitude des événements, et l'Angleterre montra aussi
plus de ménagement pour l'Espagne.

L'abbé Albéroni commençait à gouverner cette monarchie. Il suivait, pour
y parvenir, en plein les traces de la princesse des Ursins. Comme elle,
il se servit de son crédit sur la reine, et de son ambition, pour lui
persuader de suivre les traces de M\textsuperscript{me} des Ursins, pour
posséder le roi, qui fut de l'enfermer, de l'obséder jour et nuit sans
aucun moment d'intervalle, d'empêcher personne d'en approcher, même son
service le plus indispensable, de l'accoutumer à ne travailler avec
aucun ministre qu'en sa présence, et de le dominer et le tenir de façon
que rien ne pût passer à lui, ni de lui à personne, qu'en sa présence et
de son aveu. Ce fut aussi ce qu'elle exécuta à la lettre\,; et par cette
adresse Albéroni les enferma tous deux, et les gouverna seul sans les
laisser approcher de personne\,; ce qui se verra ailleurs avec plus de
détail.

Albéroni se tint donc en grande mesure avec l'Angleterre, mais surtout
avec la Hollande dont l'union lui parut encore plus avantageuse. Il
senti bientôt le poids de l'influence de l'empereur sur un prince
d'Allemagne, qui, régnant en Angleterre, faisait intérieurement son
capital de ses premiers états, et qui avait besoin du chef de l'empire
pour se conserver l'usurpation qu'il avait faite sur la Suède, dans le
temps de ses derniers désastres, des duchés de Brême et de Verden.
Albéroni s'était encore mis dans la tête de chasser tous les étrangers
des Indes occidentales, surtout les Français, projet bien chimérique
auquel il se flatta de réussir par l'intérêt et le secours des
Hollandais, mais dont l'intérêt était plus que balancé par la crainte de
rupture des nations qu'on en voudrait chasser, et surtout avec
l'Angleterre, dont il ne leur était plus possible de se séparer.

Pour entendre ce point d'espèce de servitude de la Hollande à
l'Angleterre, il faut savoir qu'outre les liaisons intimes dont le roi
Guillaume avait uni ces deux puissances, par tous les liens qu'il avait
pu imaginer, tant qu'il fut à la tête de toutes les deux, la guerre sur
la succession d'Espagne y en avait ajouté un autre bien plus fort.
Heinsius, pensionnaire de Hollande, gouvernait cette république avec un
art qui l'en rendit tout à fait maître. Il était créature du roi
Guillaume, son confident, et l'âme de son parti dans tous les temps
avant et depuis son avènement à la couronne d'Angleterre. Il avait
pleinement hérité de s haine contre la France et contre la personne du
feu roi. Il était flatté des soumissions que lui prodiguèrent le duc de
Marlborough et le prince Eugène, qui lui déféraient tout, et qui avaient
un intérêt personnel et pressant de perpétuer la guerre qui était tout
leur appui à Vienne et à Londres, et qui leur valait infiniment en
particulier. Ils n'avaient pas honte d'attendre quelquefois des heures
entières dans l'antichambre d'Heinsius, par le moyen duquel ils firent
que les Hollandais suppléèrent à ce que l'empereur ne pouvait, et à ce
qu'on n'osait demander au parlement d'Angleterre, qui donnait souvent le
triple des engagements, et qu'on ne pouvait pousser au delà. De cette
façon la république se ruina si bien, que, si les sept provinces avaient
pu être vendues comme on vend une terre, le prix n'en aurait pas payé
les dettes.

Les plus riches du pays ne voyant donc plus de sûreté pour les fonds
qu'ils prêteraient à l'État, les mirent tant qu'ils purent sur la banque
d'Angleterre, en sorte que dans un État ruiné les particuliers
demeurèrent riches. Ces particuliers pour la plupart étaient toujours à
la tête des villes, des provinces, du conseil d'État, des était
généraux, et dans les premiers emplois et les principales commissions.
Ils étaient donc à peu près les maîtres des affaires, et le sont
toujours demeurés par leur nombre, leur succession des uns aux autres,
leur crédit. Mais en même temps leurs richesses, et même tout le bien de
la plupart étant entre les mains des Anglais, les met dans une telle
dépendance de l'Angleterre qu'ils se trouvent forcés d'en préférer les
intérêts à ceux de leur république, et de la faire consentir, contre son
propre avantage, à toutes les volontés des Anglais. C'est ce qui se voit
à l'œil, et se sent dans toutes les conjonctures, tellement que jusqu'à
ce jour que j'écris, la république ne s'est pas conduite autrement, et
avec peu ou point d'espérance d'aucun changement là-dessus. Albéroni
n'ignorait pas sans doute cette position, et il est surprenant qu'il ait
pu se flatter de se pouvoir servir des Hollandais pour chasser les
Anglais des Indes espagnoles.

On sentit bientôt, malgré toute son adresse, son peu d'inclination pour
la France, en particulier pour le régent, et pour son gouvernement. Je
ne sais si ce prince eut part ou non aux lettres misérables que le
maréchal d'Estrées et le duc de Noailles écrivirent à ce maître italien,
l'un pour lui donner part de ses nouveaux emplois, l'autre qui l'avait
méprisé en Espagne du temps de M. de Vendôme, pour lui demander
bassement son amitié. Ces recherches enflèrent Albéroni et ne le
changèrent sur rien\,; mais il continua la correspondance qu'Effiat
entretenait avec lui, qui pouvait lui être utile à plus d'une chose, à
ce qui a été expliqué de la perfide conduite d'Effiat. Albéroni, de plus
en plus avancé dans la faveur et le gouvernement, se voulut défaire des
principales têtes. Ne se sentant pas encore assez fort pour attaquer le
cardinal del Giudice, il le brouilla avec Tabarada, évêque d'Osma, qui
était gouverneur du conseil de Castille, et d'une insupportable fierté.
Il le rendit odieux à la reine, qui entreprit sa perte. Le roi voulait
se contenter d'une forte réprimande\,; mais la reine déclara que, s'il
ne se retirait, elle lui ferait donner des coups de bâton. Il s'enfuit
au plus vite en son évêché, et donna la démission de sa place.

Les troubles d'Angleterre augmentaient, et le comte de Marr avait des
succès en Écosse. Stairs était tout occupé d'empêcher la France de
donner aucun secours au Prétendant, et de lui couper le passage par le
royaume s'il voulait gagner les bords de la mer. Il avait de bons
espions\,; dès qu'il apprit que ce prince partait de Bar, il courut à M.
le duc d'Orléans pour lui demander de le faire arrêter. Stairs avait
proposé un traité de garantie des successions des royaumes de France et
d'Angleterre, et avait reçu pouvoir de le signer. Le régent y voulut
ajouter une alliance défensive entre ces deux couronnes et la Hollande,
qu'il jugeait nécessaire pour servir de base à la garantie réciproque.
Buys, ambassadeur de Hollande, y entra\,; mais Stairs, qui voulait
brusquer la garantie, s'éloignait de l'alliance défensive, dont il
craignait la longueur de la traiter. Il craignit aussi que le régent ne
cherchât à gagner du temps pour voir ce que deviendraient les affaires
d'Angleterre, et il s'échappa à dire à Son Altesse Royale que, s'il
regardait ces troubles avec indifférence, l'Angleterre aurait la même
pour ceux qu'elle pourrait voir naître en France. Ils en étaient en ces
termes, lorsque le Prétendant disparut de Bar, et que Stairs vint crier
à M. le duc d'Orléans sur son passage par la France, et lui demanda de
le faire arrêter.

Le régent, qui avec adresse nageait entre deux eaux, avait promis au
Prétendant de fermer les yeux et de favoriser son passage, pourvu que ce
fût sous le dernier secret\,; et en même temps accorda à Stairs sa
demande. Il fit partir sur-le-champ Contade qui lui était affidé, et
fort intelligent, et major du régiment des gardes, dont j'ai parlé plus
d'une fois, avec son frère lieutenant dans le même régiment, et deux
sergents à leur choix, pour aller a Château-Thierry attendre le
Prétendant, où Stairs avait des avis sûrs qu'il devait passer. Contade
partit la nuit du 9 novembre, bien résolu et instruit à manquer celui
qu'il cherchait. Stairs, qui ne s'y fiait que de bonne sorte, prit
d'autres mesures qui furent au moment de réussir\,; et voici ce qui
arriva\,:

Le Prétendant partit déguisé de Bar, accompagné de trois ou quatre
personnes seulement, vint à Chaillot où M. de Lauzun avait une ancienne
petite maison où il n'allait jamais, et qu'il avait gardée par
fantaisie, quoiqu'il eût celle de Passy dont il faisait beaucoup
d'usage. Ce fut où le Prétendant coucha, et où il vit la reine sa mère
qui était souvent et longtemps aux Filles de Sainte-Marie de Chaillot\,;
et de là partit pour aller s'embarquer en Bretagne par la route
d'Alençon, dans une chaise de poste de Torcy.

Stairs découvrit cette marche, et résolut de ne rien oublier pour
délivrer son parti de ce reste unique des Stuarts. Il dépêcha sourdement
des gens sur différentes routes, surtout sur celle de Paris à Alençon.
Il chargea particulièrement de celle-là le colonel Douglas, réformé dans
les Irlandais à la solde de France, qui, à l'abri de son nom, et par son
esprit, son entregent et son intrigue s'était insinué à Paris en
beaucoup d'endroits depuis la régence, s'était mis sur un pied de
considération et de familiarité auprès du régent, et venait assez
souvent chez moi. Il était de bonne compagnie, marié sur la frontière de
Metz, fort pauvre, avait de la politesse et beaucoup de monde, la
réputation de valeur distinguée, et quoi que ce soit qui pût le faire
soupçonner d'être capable d'un crime.

Douglas se mit dans une chaise de poste, s'accompagna de deux hommes à
cheval, tous trois fort armés, et courut la poste lentement sur cette
route. Nonancourt est une espèce de petite villette sur ce chemin, à
dix-neuf lieues de Paris, entre Dreux, trois lieues plus loin, et
Verneuil au Perche, quatre lieues au delà\,; ce fut à Nonancourt où il
mit pied à terre, y mangea un morceau à la poste, s'informa avec un
extrême soin d'une chaise de poste qu'il dépeignit et comme elle devait
être accompagnée, témoigna craindre qu'elle ne fût déjà passée et qu'on
ne lui dît pas vrai. Après des perquisitions infinies, il laissa un
troisième à cheval qui lui était arrivé depuis qu'il était là, avec
ordre de l'avertir lorsque la chaise dont il était en recherche
passerait, et ajouta des menaces et des promesses de récompenses aux
gens de la poste pour n'être pas trompé par leur négligence.

Le maître de la poste s'appelait Lospital. Il était absent, mais sa
femme était à la maison, qui se trouva heureusement une très honnête
femme, qui avait de l'esprit, du sens, de la tête et du courage.
Nonancourt n'est qu'à cinq lieues de la Ferté, et quand on n'y passe
point pour abréger, on avertit cette poste qui envoie un relais sur le
chemin. Je connaissais donc fort cette maîtresse de poste qui s'en
mêlait plus que son mari, et qui m'a elle-même conté toute cette
aventure plus d'une fois. Elle fit inutilement tout ce qu'elle put pour
tirer quelque éclaircissement sur ces inquiétudes. Tout ce qu'elle put
démêler fut qu'ils étaient Anglais, et dans un mouvement violent\,;
qu'il s'agissait de quelque chose de très important et qu'ils méditaient
un mauvais coup. Elle imagina là-dessus que cela regardait le
Prétendant, prit la résolution de le sauver, l'arrangea en même temps
dans sa tête, et sut heureusement l'exécuter.

Pour y réussir elle se fit toute à ces messieurs, ne refusa rien, se
contenta de tout, et leur promit qu'ils seraient infailliblement
avertis. Elle les en persuada si bien que Douglas s'en alla sans dire où
qu'à ce troisième, qui était venu le joindre, mais en lieu voisin pour
être averti à temps. Il emmena un des valets avec lui\,; l'autre demeura
avec ce troisième qui l'avait joint, pour attendre.

Un homme de plus embarrassa fort la maîtresse\,; toutefois elle prit son
parti. Elle proposa au monsieur, qui était ce troisième, de boire un
coup, parce qu'il avait trouvé Douglas hors de table. Elle le servit de
son mieux et de son meilleur vin, et le tint à table le plus longtemps
qu'elle put, et alla au-devant de tous ses ordres. Elle avait mis un
maître valet à elle, en qui elle se fiait, en sentinelle, avec ordre de
paraître seulement, sans dire mot, s'il voyait une chaise\,; et sa
résolution était prise d'enfermer son homme et son valet, et de relayer
la chaise avec ses chevaux qu'elle avait détournés par derrière. Mais la
chaise ne vint point, et l'homme s'ennuya de demeurer à table. Alors
elle fit si bien qu'elle lui persuada de s'aller reposer, et de compter
sur elle, sur ses gens, et sur ce valet que Douglas avait laissé.
L'Anglais recommanda bien à celui-là de ne pas désemparer le pas de la
porte, et de le venir avertir dès que là chaise paraîtrait.

La maîtresse mit ce monsieur reposer le plus qu'elle put sur le derrière
de sa maison, et toujours l'air dégagé, sort et s'en va chez une de ses
amies dans une rue détournée, lui conte son aventure et ses soupçons,
s'assure d'elle pour recevoir et cacher en son logis celui qu'elle
attendait, envoie quérir un ecclésiastique de leurs parents à toutes
deux, en qui elles pouvaient prendre confiance, qui vint, et qui prêta
un habit d'abbé et une perruque assortissante. Cela fait,
M\textsuperscript{me} Lospital retourne chez elle, trouve le valet
anglais à la porte, l'entretient, le plaint de son ennui, lui dit qu'il
est bien bon d'être si exact\,; que de la porte à la maison il n'y a
qu'un pas, lui promet qu'il y sera aussi bien averti que par ses yeux
sur la porte, lui persuade de boire un coup, donne le mot à un postillon
affidé, qui fait boire l'Anglais et le couche ivre-mort sous la table.
Pendant cette expédition, la maîtresse avisée va écouter à la porte du
monsieur anglais, tourne doucement la clef et l'enferme, et de là vient
s'établir sur le pas de sa porte.

Une demi-heure après vient le valet affidé qu'elle avait mis en
sentinelle\,: c'était la chaise attendue, à qui et à trois hommes qui
l'accompagnaient à cheval, on fit, sans qu'elle sût pourquoi, prendre le
petit pas. C'était le roi Jacques. M\textsuperscript{me} Lospital
l'aborde, lui dit qu'il est attendu et perdu s'il n'y prend garde, mais
qu'il ait à se fier à elle et à la suivre\,; et les voilà allés chez
l'amie. Là il apprend tout ce qui s'est passé, et on le cache le mieux
qu'il est possible, et les trois hommes de sa suite.
M\textsuperscript{me} Lospital retourne chez elle, envoie chercher la
justice\,; et, sur les soupçons qu'elle déclare, fait arrêter le valet
anglais ivre et le monsieur anglais, qui s'était endormi dans la chambre
où elle l'avait mené se reposer, et où elle l'avait en dernier lieu
enfermé, et aussitôt après, dépêche un de ses postillons à Torcy. La
justice cependant instrumente et envoie son procès-verbal la cour.

On ne peut exprimer quelle fut la rage de ce monsieur anglais de se voir
arrêté et hors d'état d'exécuter ce qui l'avait amené, ni quelle sa
furie contre le valet anglais qui s'était laissé enivrer. Pour
M\textsuperscript{me} Lospital il l'aurait étranglée s'il avait pu, et
elle eut très longtemps peur d'un mauvais parti.

Jamais l'Anglais ne voulut dire ce qui l'avait amené, ni où était
Douglas, qu'il nomma pour tâcher d'imposer par ce nom. Il se déclara
être envoyé par l'ambassadeur d'Angleterre, qui n'en avait pas encore
pris le caractère, et s'écria fort que ce ministre ne souffrirait pas
l'affront qu'il recevait. On lui répondit doucement qu'on ne voyait
point de preuves qu'il fût à l'ambassadeur d'Angleterre, ni que ce
ministre prit aucune part en lui\,; qu'on voyait seulement des desseins
très suspects pour la liberté publique et pour celle des grands
chemins\,; qu'on ne lui ferait ni tort ni déplaisir\,; mais qu'il
resterait en sûreté jusqu'à ce qu'on eût des ordres\,; et là-dessus il
fut civilement conduit en prison, ainsi que le valet anglais ivre.

Ce que devint Douglas n'a point été su, sinon qu'il fut reconnu en
divers endroits de la route, courant, s'informant, criant avec désespoir
qu'il était échappé, sans dire qui. Apparemment qu'il vint ou envoya aux
nouvelles, lassé de n'en point recevoir, et que le bruit d'un tel éclat
dans un petit lieu, comme est Nonancourt, vint aisément à lui dans le
voisinage où il s'était relaissé, et que cela le fit partir pour tâcher
encore de rattraper sa proie.

Mais il courait en vain. Le roi Jacques était demeuré caché à
Nonancourt, où, charmé des soins de cette généreuse maîtresse de poste
qui l'avait sauvé de ses assassins, il lui avoua qui il était, et lui
donna une lettre pour la reine sa mère. Il demeura là trois jours pour
laisser passer le bruit, et ôter toute espérance à ceux qui le
cherchaient\,; puis, travesti en abbé, il monta dans une autre chaise de
poste que M\textsuperscript{me} Lospital avait empruntée comme pour elle
dans le voisinage, pour ôter toute connaissance par les signalements, et
continua son voyage, pendant lequel il se vit toujours poursuivi mais
heureusement jamais reconnu, et s'embarqua en Bretagne pour l'Écosse.

Douglas, lassé de ses courses inutiles, revint à Paris où Stairs faisait
grand bruit de l'aventure de Nonancourt, qu'il ne traitait pas de moins
que d'attentat contre le droit des gens, avec une audace et une
impudence extrême\,; et Douglas, qui ne pouvait ignorer ce qui se disait
de lui, eut celle d'aller partout où il avait accoutumé, de se montrer
aux spectacles, et de se présenter devant M. le duc d'Orléans.

Ce prince ignora tant qu'il put un complot si lâche et si barbare, et à
son égard si insolent. Il en garda le silence, dit à Stairs ce qu'il
jugea à propos pour le faire taire, et lui rendit ses assassins anglais.
Douglas pourtant baissa fort auprès du régent. Beaucoup de gens
considérables lui fermèrent leur porte. Il tenta inutilement de forcer
la mienne\,; il osa me faire faire des plaintes là-dessus, qui ne lui
réussirent pas davantage\,; bientôt après il disparut de Paris. Je n'ai
point su ce qu'il était devenu depuis. Sa femme et ses enfants y
demeurèrent à l'aumône. Il y avait longtemps qu'il était mort delà la
mer, lorsque l'abbé de Saint-Simon passa de Noyon à Metz, où il trouva
sa veuve fort misérable.

La reine d'Angleterre fit venir M\textsuperscript{me} Lospital à
Saint-Germain, la remercia, la caressa comme elle le méritait, et lui
donna son portrait\,; ce fut tout\,; le régent, quoi que ce soit\,; et
longtemps après le roi Jacques lui écrivit et lui envoya aussi son
portrait. Conclusion\,: elle est demeurée maîtresse de la poste de
Nonancourt, et l'est demeurée, telle qu'elle l'était auparavant,
vingt-quatre ou vingt-cinq ans encore, jusqu'à sa mort\,; et c'est
encore son fils et sa belle fille qui tiennent cette même poste. C'était
une femme vraie, estimée dans son lieu\,; pas un seul mot de ce qu'elle
a raconté de cette histoire n'y a été contredit de qui que ce soit. On
n'oserait dire ce qui lui en a coûté de frais\,; jamais elle n'en a reçu
une obole. Jamais elle ne s'en est plainte\,; mais elle disait les
choses comme elles étaient, avec modestie et sans le chercher, à qui lui
en parlait. Telle est l'indigence des rois détrônés, et le parfait oubli
des plus grands périls et des plus signalés services.

Beaucoup d'honnêtes gens s'éloignèrent de Stairs, que l'insolence de ses
airs écartait encore. Il en combla la mesure par la manière
insupportable dont il s'expliqua toujours sur cette affaire, n'osant
toutefois l'avouer, sans s'en disculper non plus, ni en témoigner
d'autre peine que celle de son succès.

\hypertarget{chapitre-xiv.}{%
\chapter{CHAPITRE XIV.}\label{chapitre-xiv.}}

1715

~

{\textsc{Pensées de l'Espagne, où Albéroni gagne peu à peu la principale
autorité, et veut chasser le cardinal del Giudice.}} {\textsc{- Forte
brouillerie entre Rome et Madrid.}} {\textsc{- Adresse d'Albéroni pour
parvenir à la pourpre romaine.}} {\textsc{- Il veut faire des réformes
et établir une puissante marine.}} {\textsc{- Miraval, ambassadeur en
Hollande, choisi pour être gouverneur du conseil de Castille.}}
{\textsc{- La Mirandole éloigné.}} {\textsc{- Traité de la Barrière
signé entre l'empereur et les États généraux.}} {\textsc{- Soupçons
qu'il cause, favorables au Prétendant.}} {\textsc{- Inquiétude de la
France sur la conduite de l'Espagne, et la sienne en conséquence.}}
{\textsc{- Plaintes de l'Angleterre de la conduite de la France à
l'égard du Prétendant, et pareillement de celle d'Espagne.}} {\textsc{-
Le pape et le clergé d'Espagne assistent le Prétendant, dont les
affaires tournent mal.}} {\textsc{- L'Espagne se désiste, par un traité
fort avantageux aux Anglais, des articles ajoutés au traité d'Utrecht.}}
{\textsc{- Mesures de l'Espagne avec la Hollande sur le commerce.}}
{\textsc{- Vanteries d'Albéroni.}} {\textsc{- Naufrage de la flottille
d'Espagne richement chargée.}} {\textsc{- Plan d'Albéroni pour les
réformes.}} {\textsc{- Voir les pièces, et quelles elles sont tant sur
le détail des affaires étrangères que sur celles de la constitution.}}
{\textsc{- Duels réveillés.}} {\textsc{- Charost obtient pour son fils
la survivance de sa charge de capitaine des gardes du corps.}}
{\textsc{- Bals de l'Opéra.}} {\textsc{- Raisons de tenir la cour à
Versailles\,; celles de M. le duc d'Orléans pour Paris.}} {\textsc{- Les
médecins prolongent le séjour de Vincennes.}} {\textsc{- Les PP. Tellier
et Doucin chassés de Paris.}} {\textsc{- Les jésuites interdits par les
évêques de Metz et de Verdun.}} {\textsc{- Biron marie sa fille aînée à
Bonac, et son fils aîné à la fille aînée du duc de Guiche.}} {\textsc{-
Service du feu roi à Notre-Dame.}} {\textsc{- Mort d'une fille carmélite
du maréchal de Villeroy, et de M\textsuperscript{me} de Sourches.}}
{\textsc{- Mort de La Hoguette, archevêque de Sens\,; son éloge.}}
{\textsc{- Mort de M\textsuperscript{me} de Louvois.}} {\textsc{-
Curiosités sur elle.}} {\textsc{- Mort de la femme du czarowitz.}}
{\textsc{- Nouveau délai à Vincennes.}} {\textsc{- Les conseils de
régence sont partagés entre Vincennes et Paris.}} {\textsc{- Mort et
caractère du prince Camille.}} {\textsc{- Mort de l'électeur de Trèves
(Lorraine).}} {\textsc{- Mariage du marquis d'Harcourt avec
M\textsuperscript{lle} de Villeroy.}} {\textsc{- Caylus, réhabilité et
absous de son ancien duel, fait une grande fortune en Espagne.}}
{\textsc{- M. le duc d'Orléans a la faiblesse de pardonner à La
Feuillade, de le nommer ambassadeur à Rome, et de le combler de grâces
et de biens.}} {\textsc{- M. le duc dispute au duc du Maine et au comte
de Toulouse le traversement du parquet.}} {\textsc{- Réception du duc de
Valentinois au parlement différée.}} {\textsc{- Cruelle affaire suscitée
à Desmarets, dont il se tire bien.}} {\textsc{- Je lui pare l'exil et me
raccommode avec lui.}} {\textsc{- Peu après nous nous parlons très
franchement à la Ferté l'un à l'autre.}} {\textsc{- Valeur des espèces
augmentée.}} {\textsc{- D'Antin surintendant des bâtiments.}} {\textsc{-
Le roi à Paris.}}

~

L'Espagne jugeait que le régent voulait maintenir l'union avec elle et
la paix avec ses voisins, mais que son intelligence secrète avec
l'Angleterre était grande et allait à faire un traité de commerce. Elle
en concluait peu ou point d'espérance d'être secourue de la France, dont
les finances étaient en grand désordre, en cas d'attaque de l'empereur,
contre laquelle, si cette attaque arrivait, elle se préparait à se bien
défendre, en se maintenant en paix avec l'Angleterre et avec le
Portugal.

Albéroni gagnait toujours du terrain, et par degrés devenait en effet
premier ministre. Le cardinal del Giudice en était piqué au vif.
Cellamare, l'ami de l'un, neveu de l'autre, avait sagement entretenu
l'union entre eux. Il voulut donc s'en retourner en Espagne pour
empêcher leur rupture. Il demanda son congé\,; il se flatta de
l'obtenir\,; ce n'était pas l'intention d'Albéroni de bien vivre avec
Giudice. C'était pour lui un personnage d'un trop grand poids dont il
avait bien résolu de se défaire.

Il pensa y avoir une rupture entre les cours de Rome et de Madrid. On a
vu en son lieu quel était le cardinal Sala, et qu'il était mort. Il
avait eu du pape, à la recommandation de l'empereur, l'importante place
d'inquisiteur général d'Espagne. Le pape en disposa en faveur de
l'évêque d'Albaracin, aussi rebelle que l'avait été Sala. Le roi
d'Espagne voulait chasser Aldovrandi, nonce auprès de lui, et fermer la
nonciature. Le P. Daubenton, son confesseur, para ce coup avec bien de
la peine. Quelque jalousie qu'Albéroni eût de son crédit et de ses
fréquentes audiences secrètes du roi d'Espagne, il l'aida à calmer
l'esprit de ce prince. Albéroni qui voulait régner en Espagne sentait le
besoin qu'il avait de la pourpre pour s'y maintenir ou pour s'en
dédommager. Il ne sentait pas moins aussi l'excès de sa bassesse. Il
n'osait donc y prétendre ouvertement, mais il avait conçu le dessin que
la reine en fit toutes les démarches, comme à son insu, et pour lui
faire une surprise agréable. Pour parvenir à ce but, il fallait empêcher
que les deux cours ne se brouillassent, et ménager le jésuite Aubenton,
fabricateur de la bulle \emph{Unigenitus} avec le cardinal Fabroni,
comme on l'a vu en son lieu, lorsqu'il était assistant du général des
jésuites à Rome, après avoir été chassé d'Espagne et de la place de
confesseur du roi, où Rome et les jésuites avaient eu l'art de le faire
revenir, comme le plus habile instrument qu'ils pussent avoir en cette
cour, où il était le confident et le correspondant secret et immédiat du
pape.

Albéroni en même temps travaillait à réformer les dépenses des maisons
royales, des conseils, des tribunaux, et celle qui était destinée au
payement des pensions et des grâces. Il se plaignait que les gages des
officiers étaient montés au quadruple depuis que Philippe était en
Espagne. Cela le rendait fort odieux\,; mais il regardait une puissante
marine comme le fondement de la puissance solide de l'Espagne, et il
avait raison. Il cherchait donc à ramasser de tous côtés des fonds pour
parvenir à un but si nécessaire, et il flattait le roi d'Espagne de lui
armer quarante vaisseaux, pour l'année prochaine, en état d'assurer le
commerce des Indes espagnoles. Il avait l'adresse de vanter son
désintéressement, en ce que travaillant à toutes les affaires, et à
beaucoup encore de secrètes par la confiance du roi et de la reine, il
n'en avait pas encore reçu la moindre grâce, et ne vivait que des
cinquante pistoles que le duc de Parme, son maître, lui donnait tous les
mois, et en même temps laissait échapper doucement quelques plaintes de
l'ingratitude des princes.

Il continuait à donner tous les dégoûts possibles au cardinal del
Giudice, qui avait la direction des affaires étrangères qu'Albéroni lui
enlevait toutes, et le traversait sur ce qui regardait l'éducation du
prince des Asturies, dont ce cardinal était le gouverneur. Les choses
allèrent si loin que le cardinal et lui se querellèrent, et entrèrent
tous deux chez le roi pour lui porter leurs plaintes. Ni l'un ni l'autre
pour lors n'eurent l'avantage. Albéroni s'en prit au P. Daubenton, et il
en résulta que Miraval, ambassadeur en Hollande, eut ordre de revenir
pour remplir la place vacante de gouverneur du conseil de Castille, dans
lequel il avait passé sa vie. C'était un grand homme, froid, très
médiocre ambassadeur, et d'inclination autrichienne. J'aurai occasion
d'en parler ailleurs.

Albéroni, jaloux de tout ce qui pouvait aborder la reine, était fort
affligé de l'arrivée de sa nourrice, qu'elle avait fait venir d'Italie.
Il éloigna d'elle le duc de La Mirandole, qui avait l'honneur de lui
appartenir, et qui avait pensé l'épouser. Il était grand écuyer du roi,
et, comme on l'a vu ailleurs, fils du premier lit de la femme du prince
de Cellamare, et lié, par conséquent, avec le cardinal del Giudice. Non
content de porter des coups à ce dernier auprès du roi d'Espagne, qui
portèrent jusque contre le prince des Asturies, parce qu'il s'était
attaché à son gouverneur, Albéroni chercha à le rendre suspect au pape
sur les différends avec le roi d'Espagne, pour avoir seul le mérite d'y
servir Rome, dans sa vue du cardinalat et de brouiller Giudice
partout\,; il ne cherchait qu'à le réduire à force d'embarras et de
dégoûts à lui quitter la partie et a se retirer en Italie.

La signature du traité de la Barrière entre l'empereur et les États
généraux, après beaucoup de longueurs et de difficultés, fit naître
divers soupçons. Le roi Georges comptait entièrement sur la cour de
Vienne, et beaucoup moins sur M. le duc d'Orléans que lors de la mort du
roi. Il le crut dans les intérêts du Prétendant, et la cour d'Espagne,
qui s'était refroidie sur lui, lui fit compter cent mille écus, avec
espérance de plus grands secours, dans la crainte qu'elle conçut de la
liaison étroite entre l'empereur et le roi d'Angleterre. On conçut aussi
en France des soupçons de quelques projets de ligue entre l'Espagne et
les États généraux, dont le ministre à Madrid était traité avec une
grande distinction, et qui était tout à fait entré dans la confidence
d'Albéroni. C'était ce même Riperda qui succéda immédiatement à
Albéroni, lorsqu'il fut chassé d'Espagne. Le duc de Saint-Aignan eut
ordre d'en parler à cet abbé, de s'expliquer même sur les sujets
d'inquiétude, de lui offrir les mêmes secours et le même nombre de
vaisseaux qu'il prétendait tirer de Hollande, pour assurer la navigation
des Indes, et de lui demander une préférence là-dessus qu'il ne croyait
pas devoir être refusée aux Français. Il ajouta par le même ordre que,
si l'Espagne formait quelque entreprise contre l'Italie, contraire au
traité de neutralité, la France serait obligée de s'y opposer.

Albéroni, passionné du projet qu'il avait conçu de chasser les Français
et les Anglais des Indes espagnoles par le moyen des Hollandais, était
sourd à toute autre proposition. Riperda le rassurait sur l'Angleterre,
arrêtée à l'égard de l'Espagne par les vives représentations des États
généraux, et Albéroni attribuait la démarche du duc de Saint-Aignan à la
crainte que prenait la France de lui voir former une marine.

Les places frontières d'Espagne furent en ce même temps ravitaillées, et
leurs garnisons renforcées. Albéroni n'en fit aucune plainte, il
attribua cette précaution aux pensées de l'avenir. Capres, depuis duc de
Bournonville, qui briguait vainement l'ambassade de France, avait parlé
au roi d'Espagne de sa succession à cette couronne. Ce prince lui avait
répondu de manière à faire croire qu'il y pensait, en cas d'ouverture de
succession, sans néanmoins s'en expliquer. C'en était assez, si le
régent en avait su quelque chose, pour autoriser Albéroni dans sa pensée
sur ces précautions.

Il y avait alors de grands soupçons d'une alliance secrète signée entre
l'empereur et le roi d'Angleterre, par laquelle on croyait que
l'empereur promettait à Georges la garantie de la succession
d'Angleterre dans la ligne protestante, et celle de ce qu'il avait
usurpé sur la Suède, et qu'il y pourrait encore acquérir\,; et
réciproquement Georges, de donner des secours à l'empereur pour la
réunion de la Sicile cédée à Utrecht au duc de Savoie, avec le titre de
roi, au royaume de Naples possédé par l'empereur, comme aussi pour
s'emparer de la Toscane, lorsque la succession s'en ouvrirait. Ces
soupçons réchauffèrent les deux couronnes pour le Prétendant, qui ne
s'en cachèrent pas l'une à l'autre.

L'Angleterre, fort troublée au dedans et fort inquiète de l'Écosse, ne
se contentait pas que le régent eût refusé toutes sortes de secours au
Prétendant\,; elle en aurait voulu tirer contre lui de grands. Stanhope
reprocha à d'Iberville, chargé des affaires du roi à Londres, que le
régent se contentait de sauver les apparences, tandis qu'il assistait le
Prétendant en effet. Il allégua qu'on avait laissé passer et embarquer
le duc d'Ormond en Bretagne, tandis qu'on avait arrêté fort longtemps
des Anglais envoyés pour le suivre et reconnaître sa marche. C'est ainsi
qu'il déguisa l'affaire de Nonancourt. Il fit parade à d'Iberville des
forces et des alliances d'Angleterre, laissa échapper quelques menaces,
se plaignit du refus que M. le duc d'Orléans avait fait d'une nouvelle
alliance que Stairs lui avait proposée, dont j'ai parlé plus haut, et
qui n'était que suspendue pour y faire entrer les Hollandais\,; il finit
par déclarer qu'il ne parlait que comme particulier, se réservant de
faire des plaintes au nom du roi son maître, quand il serait temps de
les soutenir, et qu'il en serait chargé. Volkra, envoyé de l'empereur à
Londres attisait ce feu naissant, et on sut que Stairs ne travaillait
pas à l'éteindre par ses dépêches. Stanhope ne tint pas un langage plus
couvert ni plus modéré à Monteléon, ambassadeur d'Espagne, et même il
poussa les menaces plus loin.

Le pape, ayant appris que le clergé d'Espagne était disposé à faire sur
soi des impositions pour secourir le Prétendant, écrivit au roi
d'Espagne et au cardinal del Giudice, pour appuyer ces bonnes
dispositions, et fit toucher à ce malheureux prince cinquante mille écus
de son propre argent.

Stanhope parla enfin si haut, et les affaires d'Écosse prirent un si
mauvais tour, l'incertitude du débarquement du Prétendant fut si grande
jusqu'à la fin de cette année, qu'Albéroni prit enfin le parti de
terminer tous les différends de l'Espagne avec les Anglais, et de les
satisfaire. Elle se désista donc des articles ajoutés au traité
d'Utrecht, dont ils avaient fait tant de plaintes, et fit signer à
Madrid, par le marquis de Bedmar, avec un secrétaire que l'Angleterre
tenait en cette ville, un traité dont les conditions furent si
avantageuses aux Anglais que Riperda, ambassadeur de Hollande à Madrid,
s'en réjouit comme de la ruine du commerce de France. Cet abbé se vanta
que le pensionnaire de cette république, charmé des vertus politiques de
la reine d'Espagne, avec force autres louanges, lui offrait dix
vaisseaux armés pour assurer la navigation des Indes, sans prétendre
faire le commerce, mais pour aider seulement les Espagnols à le faire à
l'exclusion de toute autre nation, et qu'il s'en rapportait à l'abbé
pour régler le payement suivant les temps du retour des flottes.

Sur ces offres, le roi d'Espagne ne prit que six vaisseaux pour faire
seulement le commerce du Mexique, auxquels il ajouta quelques-uns des
siens, et résolut d'envoyer le plus tôt et le plus secrètement qu'il
serait possible cinq navires dans la mer du Sud, pour surprendre tout ce
qu'ils y trouveraient de vaisseaux étrangers, particulièrement de
français dont le nombre était grand, nonobstant les plaintes
continuelles de l'Espagne, et les défenses du feu roi fort mal observées
pour empêcher ce commerce, qui donnait de la jalousie à toutes les
nations de l'Europe, lesquelles s'en plaignaient hautement.

L'Espagne alors venait de recevoir la nouvelle que la flottille,
revenant en ce royaume, avait échoué dans le canal de Bahama\,; que
douze vaisseaux du roi d'Espagne y avaient péri avec quatre cents hommes
et Ubilla qui la commandait. Elle était chargée de dix-huit millions
d'écus, et il y en avait pour presque autant en marchandises, dont les
principales étaient de l'indigo et de la cochenille. Ces nouvelles
ajoutaient en même temps qu'on avait déjà repêché plus des deux tiers de
l'argent.

Parmi toutes ces occupations, Albéroni travaillait toujours à la réforme
dont on a parlé, et à celle des troupes, indépendamment d'aucun autre
ministre, et tous les soirs en rendait compte au roi et à la reine. Son
plan était de réduire toutes les troupes à cinquante mille hommes, y
compris les officiers. Il prétendait trouver dans la seule réduction des
gardes du corps de huit cents à quatre cents un profit de cinquante
mille pistoles par an, et pour laisser repeupler l'Espagne, il voulait
{[}qu'elle{]} prît un corps de Suisses. Plein de ces projets, il se
vantait que si l'empereur lui laissait seulement deux ou trois ans, il
aurait à son tour de quoi lui donner à penser.

En même temps, il se laissait de ne faire que comme en secret les
fonctions de premier ministre. Il en voulait avoir publiquement la
qualité, renvoyer incessamment en Italie le cardinal del Giudice, qui
n'avait plus que l'ombre du soin des affaires étrangères, et en sa
place, mais sous soi, y commettre Grimaldo, duquel j'aurai ailleurs
beaucoup d'occasions de parler.

On trouvera, parmi les Pièces, beaucoup de détails curieux, tant sur les
affaires étrangères que sur celles de la constitution, recueillis sur
les lettres de la poste par M. de Torcy en plusieurs volumes\,; pendant
qu'il en a été le surintendant, et qu'il a bien voulu me communiquer
depuis. Ils méritent tous d'être lus d'un bout à l'autre\,; on y
trouvera une instruction infinie et beaucoup de plaisir dans une grande
simplicité. Je les ai fait copier tout entiers, comme les meilleures
pièces originales qu'il soit possible de ramasser\footnote{}. Revenons
maintenant en France.

Ferrant, capitaine au régiment du roi, et Girardin, capitaine au
régiment des gardes, se battirent familièrement sous la terrasse des
Tuileries, le mardi 12 novembre. L'un était de ces Ferrant, du
parlement\,; l'autre fils de Vauvray, qui était du conseil de marine,
comme en ayant été longtemps intendant à Toulon. Ce dernier fut fort
blessé. C'étaient deux hommes faits tout exprès, par leur conduite et
leur petit état, pour servir d'exemple de toute la sévérité des duels.
Le régent parut d'abord le vouloir\,; sa facilité se laissa bientôt
vaincre. Ils perdirent leurs emplois, et leurs emplois n'y perdirent
rien. Ce mauvais exemple réveilla les duels, qui étaient comme éteints.
L'étrange est que M. le duc d'Orléans n'en fut pas trop fâché.

Néanmoins, M. de Richelieu et le comte de Bavière ayant peu de jours
après pris querelle ensemble, à Chantilly, et leurs mesures pour se
battre au bois de Boulogne le jour d'une grande chasse que M. le Duc
devait y donner aux dames, le régent les envoya chercher tous deux, leur
lava la tête, prit leurs paroles, et leur déclara que, s'ils y
manquaient, il ne les manquerait pas. La chose finit ainsi.

Charost me pria de demander au régent pour M. d'Ancenis, son fils, la
survivance de son gouvernement de Calais et de sa lieutenance générale
unique de Picardie. Je lui dis qu'il l'aurait toujours aisément, après
celle de sa charge de capitaine des gardes, et pourquoi il ne l'aurait
pas aussi bien que le maréchal d'Harcourt. Je l'obtins le lendemain.

Le chevalier de Bouillon, qui depuis la mort du fils du comte d'Auvergne
avait pris le nom de prince d'Auvergne, proposa au régent qu'il y eût
trois fois la semaine un bal public dans la salle de l'Opéra, pour y
entrer en payant, masqué et non masqué, et où les loges donneraient la
commodité de voir le bal à qui ne voudrait pas entrer dans la salle. On
crut qu'un bal public, gardé comme l'est l'Opéra aux jours qu'on le
représente, serait sûr contre les aventures, et tarirait ces petits bals
borgnes épars dans Paris où il en arrivait si souvent. Ceux de l'Opéra
furent donc établis avec un grand concours et tout l'effet qu'on s'en
était proposé. Le donneur d'avis eut dessus six mille livres de pension,
et on fit une machine d'une admirable invention, et d'une exécution
facile et momentanée pour couvrir l'orchestre et mettre le théâtre et le
parterre au même plain-pied et en parfait niveau. Le malheur fut que
c'était au Palais-Royal, et que M. le duc d'Orléans n'avait qu'un pas à
faire pour y aller au sortir de ses soupers, et pour s'y montrer souvent
en un état bien peu convenable. Le duc de Noailles, qui cherchait à lui
faire sa cour, y alla, dès la première, si ivre qu'il n'y eut point
d'indécence qu'il n'y commît.

M. le duc d'Orléans était fort importuné de Vincennes\,: il voulait
avoir le roi à Paris. J'avais fait ce que j'avais pu pour qu'on
retournât à Versailles. On n'était là qu'avec la cour, loin de toute
cette sorte de monde qui ne découche point de Paris que pour aller à la
campagne. Tout ce qui avait des affaires y trouvaient en une heure de
temps tous les gens qu'ils avaient à voir, au lieu qu'à Paris il fallait
aller dix fois chez les mêmes et courir tous les quartiers. Ceux qui
étaient chargés des affaires n'auraient point eu à Versailles les
dissipations et les pertes de temps qui se trouvaient à Paris\,; et ce
que je considérais davantage, c'est que loin du tumulte du parlement,
des halles, du vulgaire, on n'y était point exposé, comme à Paris, à des
aventures de minorité, telles que {[}celles que{]} Louis XIV y avait
essuyées, et qui l'en firent sortir furtivement une nuit de la veille
des Rois. J'étais touché aussi d'éloigner M. le duc d'Orléans des
pernicieuses compagnies avec qui il soupait tous les soirs, de l'état
auquel il se montrait souvent aux bals de l'Opéra, et du temps qu'il
perdait à presque toutes les représentations de ces spectacles. Mais
c'était précisément ce qui l'attachait au séjour de Paris, duquel il n'y
eut pas moyen de le tirer. Il fit même faire une grande consultation de
médecins pour ramener le roi à Paris\,; mais ceux de la cour et de la
ville se trouvèrent du même avis, qu'on n'y devait mener le roi qu'après
que les premières gelées auraient purifié l'air, et éteint le grand
nombre de petites véroles, même dangereuses, qui régnaient alors à
Paris.

Son Altesse Royale régla la réforme des troupes, qui fut exécutée
presque aussitôt après.

Ce prince ne s'était pas bien trouvé de ne m'avoir pas cru sur les PP.
Tellier et Doucin. Ils firent tant de pratiques si dangereuses, et si
hautement, que Son Altesse Royale fut obligée de les chasser. Il eut
encore la facilité de permettre au premier de se retirer à Amiens, dont
l'évêque, aussi fanatique que lui, mais fort sot, était sa créature. On
verra qu'il fallut encore le sortir de cet asile, où il faisait encore
pis qu'à Paris. Les jésuites firent tant d'impertinences à Metz et à
Verdun que M. de Metz se trouva obligé de les interdire, et y fut tôt
après imité par l'évêque de Verdun, au grand scandale de son cousin
Charost, plus fanatique qu'eux, si cela pouvait être possible.

Biron, qui n'avait point de bien et beaucoup d'enfants, trouva à se
défaire de l'aînée avec soixante mille livres pour tout, à Bonac, neveu
de Bonrepos. Bonac avait de la capacité pour les affaires étrangères, où
il avait presque toujours été employé dans le nord et en Espagne. Lassay
fils, nommé par le feu roi pour aller en Prusse, aima mieux, après sa
mort, demeurer auprès de M\textsuperscript{me} la Duchesse, qui ne le
désirait pas moins. Bonac fut destiné à le remplacer, quoique destiné à
l'ambassade de Constantinople, où il alla pourtant à la fin. M. de
Lauzun, frère de la mère de M\textsuperscript{me} de Biron, fit la noce.

Biron fit un autre mariage en même temps bien différent de celui-ci, ce
fut de Gontaut son fils avec la fille aînée du duc de Guiche, grande et
singulièrement belle et bien faite, et spirituelle, à qui son père donna
vingt mille livres. Gontaut en avait conté à des personnes en qui M. le
duc d'Orléans prenait part, il n'avait été ni discret ni modeste, il
avait été chassé. Lassé de tuer des lièvres à Biron, au fond de la
Gascogne, il était venu vivre à l'abbaye de Saintes qu'avait une sœur de
sa grand'mère et de M. de Lauzun. Ce fut là où on lui envoya permission
de revenir pour faire le mariage, qui avait toutes les apparences d'être
le plus heureux, et qui néanmoins tourna le plus malheureusement du
monde.

On fit le jeudi 28 novembre les obsèques solennelles du feu roi à
Notre-Dame avec les cérémonies. Maboul, évêque d'Aleth, y prononça
l'oraison funèbre. Le cardinal de Noailles y officia et donna à
l'archevêché un grand repas aux trois princes du deuil qui furent les
mêmes qu'à Saint-Denis, et à beaucoup de gens de la cour.

Le maréchal de Villeroy perdit une fille, qu'il avait carmélite à Lyon,
dont il parut fort affligé. Le grand prévôt, qui avait donné sa charge à
son fils, perdit sa femme de la même maladie dont le roi était mort, et
du même âge. Ces circonstances la consolèrent de mourir. Elle était de
cette ancienne et illustre maison de Monsoreau qui est éteinte.

L'archevêque de Sens, Fortin de La Hoguette, conseiller d'État d'Église,
mourut aussi dans un grand âge. On a vu ailleurs quel il était, et son
illustre et modeste refus de l'ordre du Saint-Esprit. Toute sa vie ne
l'avait pas été moins par la pureté de ses mœurs, la probité de sa
conduite, l'assiduité dans ses diocèses, car il avait été évêque de
Poitiers, {[}et par{]} tous les devoirs d'un excellent pasteur. Il était
extrêmement considéré, et avait beaucoup d'amis. Il l'était fort de mon
père, et j'avais entretenu cette amitié avec le soin qu'elle méritait,
et que j'ai toujours cultivée dans tous les amis de mon père.

M\textsuperscript{me} de Louvois mourut en même temps. Ce fut une perte
fort grande pour sa famille, pour ses amis et pour les pauvres, et un
exemple singulier de ce que peut une conduite sage, digne, suivie,
dirigée par l'honnêteté, la piété et le seul bon sens\,: C'était une
grande héritière d'une race dont l'illustration ne passait pas le
maréchal de Souvré, père de son grand-père\,; mais ce maréchal fut
illustre, et eut des enfants qui le furent aussi, et qui tous ensemble
mirent le nom de Souvré sur un pied dans le monde, qui n'aurait pas
gagné en approfondissant, et qui eut sa source dans l'esprit, le mérite,
la faveur et les grands emplois de ce maréchal, qu'il couronna par celui
de gouverneur de la personne de Louis XIII et de premier gentilhomme de
sa chambre, laquelle passa à son fils avec le gouvernement de Touraine
et de Fontainebleau. Tous deux aussi furent chevaliers de l'ordre. Un
autre de ses fils fut grand prieur de France, figura beaucoup et eut des
emplois distingués au dedans et au dehors.

Le maréchal de Souvré eut deux filles qui y contribuèrent pour le moins
autant\,: M\textsuperscript{me} de Lansac, gouvernante du feu roi, qui
de mère en fille en a transmis la charge jusqu'à la duchesse de Tallard,
et M\textsuperscript{me} de Sablé, si connue par son esprit et par la
singulière considération qu'elle sut s'acquérir et se conserver toute sa
vie.

Leur frère avait épousé la sœur du premier maréchal de Villeroy, dont,
de cinq enfants qu'il en eut, il ne lui resta qu'un fils, qui mourut
même avant lui, et qui d'une Barentin n'eut qu'une fille unique, qui
naquit même posthume, et qui, excepté sa mère qui n'avait ni nom ni
famille et qui se remaria à M. de Boisdauphin, perdit tous ses proches
avant l'âge nubile. Il ne lui resta que le premier maréchal de Villeroy,
frère de sa grand'mère, qui fut son tuteur. C'était un homme avisé, qui
ne fit pas pour rien une si grande fortune, et qui ne se donna pas moins
de peine pour la conserver. De tant de gens distingués qui le
courtisaient pour le mariage de cette nièce, belle, grande, bien faite
et si riche, dont il disposait seul, il préféra M. de Louvois, au
scandale de toute la France\,; mais M. Le Tellier son père était lors au
plus haut point de sa faveur et au plus florissant état de son
ministère. Villeroy voulut se concilier de tels amis par un service si
fort, surtout alors, au delà de leur portée, et compta pour rien tout ce
qui se dirait du sacrifice de sa petite-nièce qu'il se faisait à
lui-même.

Elle avait la plus grande mine du monde, la plus belle et la plus grande
taille\,; une brune avec de la beauté\,; peu d'esprit, mais un sens qui
demeura étouffé pendant son mariage, quoiqu'il ne se puisse rien ajouter
à la considération que Louvois eut toujours pour elle et pour tout ce
qui lui appartenait.

Au lieu de tomber à la mort de ce ministre, elle se releva, et sut
s'attirer une véritable considération personnelle, qui de sa famille, où
elle régna, passa à la cour et à la ville, où elle se renferma, et où
elle sut tenir une grande maison, sans sortir des bornes de son état et
de son veuvage. Elle y rassembla sa famille et ses amis, et passa sa vie
dans les bonnes œuvres, sans enseigne et sans embarras. Il est immense
ce qu'elle faisait d'aumônes, et combien noblement et ordonnément elle
les distribuait. Elle allait à la cour y coucher une nuit, une ou deux
fois l'année, toujours accompagnée de toute sa famille. C'était une
nouvelle que son arrivée. Elle allait au souper du roi, qui lui faisait
toujours beaucoup d'accueil, et toute la cour à son exemple. Du reste,
presque point de visites, pas même à Paris. Tout l'été à sa belle maison
de Choisy avec bonne compagnie, mais décente et trayée, convenable à son
âge. En un mot elle mena une vie si honorable, si convenable, si décente
et si digne, dont elle ne s'est jamais démentie en rien, que sa mort,
qui fut semblable à sa vie, fut le désespoir des pauvres, la douleur de
sa famille et de ses amis, et le regret véritable du public. En elle
finit la maison de Souvré.

La princesse de Wolfenbüttel, sœur de l'impératrice régnante, et femme
du czarowitz qui a fait depuis une fin si tragique, mourut d'un coup de
pied que son mari lui donna dans le ventre, étant grosse. La vanité d'un
petit prince son grand-père, la sacrifia à des barbares que l'empereur
se voulait acquérir. Sa figure, son esprit, sa vertu méritait un
meilleur sort. Elle fut toujours malheureuse avec le plus Russe des
Russes, et ne reçut de protection et de douceur que du fameux czar son
beau-père.

On assembla encore les médecins sur le retour du roi à Paris, qui
demandèrent encore quelques semaines, sur quoi M. le duc d'Orléans prit
le parti de ne donner plus que deux conseils de régence à Vincennes par
semaine, et de tenir les deux autres à Paris dans l'appartement du roi
aux Tuileries. Ce fut un grand soulagement pour tous ceux qui en
étaient, à qui ces courses continuelles à Vincennes, en plein hiver,
étaient fort pénibles, et faisaient perdre beaucoup de temps.

Le prince Camille, un des fils de M. le Grand, mourut à Nancy. C'était
un homme très bien fait, très adroit dans tous les exercices, qui avait
de l'esprit, du sens, des vues, même du Guise, mais triste, sombre,
particulier, silencieux, dédaigneux, extrêmement glorieux. Las de sa
pauvreté, encore plus du joug domestique, à son âge, d'un service
militaire qui ne le menait à rien, solitaire par son goût au milieu du
monde, il trouva moyen, comme on a vu, de s'accrocher en Lorraine, d'y
avoir la première charge de cette petite cour, avec une subsistance de
commodités très abondante, outre vingt-quatre mille livres de pension ou
d'appointements, et seize mille livres qu'il tirait de France, moitié
d'une pension sur l'archevêché d'Auch, moitié d'un don du roi sur les
litières. L'ennui le poursuivit en Lorraine comme ailleurs. Il aimait
fort le vin et la table\,; mais il y était sans agrément aucun, comme
partout. On a vu que M. de Vaudemont lui tomba dessus comme une bombe,
avec cette préséance que M. de Lorraine lui donna immédiatement après
ses enfants et ses frères. Camille s'absenta toujours pendant les
séjours de Vaudemont. Ce dégoût lui rendit son état fort triste. Il ne
fut point marié, et ne fut regretté de personne, pas même de qui que ce
fût de sa famille.

L'électeur de Trèves, frère du duc de Lorraine, mourut à Vienne, en même
temps, de la petite vérole. Celui-là fut fort regretté pour sa personne
et pour ses établissements. Son élection avait coûté fort cher au duc de
Lorraine. Il était aussi évêque d'Osnabrück, et avait d'autres
bénéfices. Un autre frère, abbé de Stavelo, et grand prieur de Castille,
était mort de la même maladie l'année précédente.

Le fils aîné du maréchal d'Harcourt, nouveau survivancier de sa charge,
épousa la fille aînée du duc de Villeroy. Le maréchal de Villeroy fit
une noce fort magnifique

M. le duc d'Orléans, facile, comme je l'ai déjà remarqué, sur les duels,
permit à Caylus de venir purger le sien, dont j'ai parlé en son lieu,
avec le fils aîné du comte d'Auvergne, mort il y avait longtemps. Il
vint d'Espagne exprès, où il avait toujours depuis servi avec
distinction, et il y était lieutenant général. Trois ou quatre jours de
conciergerie terminèrent son affaire, et trois ou quatre autres ses
visites à ce qui lui restait de connaissances, après quoi il s'en
retourna prendre le commandement de l'Estrémadure, vacant par la mort du
marquis de Bay, que le roi d'Espagne lui avait donné. Il y a fait depuis
la plus complète fortune. J'aurai lieu de parler de lui ailleurs. Il
était frère de l'évêque d'Auxerre, et beau-frère de
M\textsuperscript{me} de Caylus, nièce favorite de M\textsuperscript{me}
de Maintenon, de laquelle il a été ici fait mention plus d'une fois.

La faiblesse de M. le duc d'Orléans, qui gâta tout en lui toute sa vie,
se montra en ce temps-ci par un trait le plus marqué, et qui lui fit un
tort extrême par l'opinion qu'on en conçut, et qui, à son égard, régla,
ou pour mieux dire, dérégla la conduite de beaucoup de gens. On a vu, à
mesure que les occasions s'en sont présentées, que personne n'avait
offensé ce prince si souvent, ni si gratuitement, que La Feuillade, ni
si cruellement. On a vu quelle fut sa conduite à Turin, ses propos
publics à la mort de M. {[}le Dauphin{]} et de M\textsuperscript{me} la
Dauphine\,; que c'est le seul homme contre lequel, cette dernière
occasion, il s'emporta jusqu'à lui vouloir faire donner des coups de
bâton, que j'eus toutes les peines du monde à empêcher. La Feuillade
avec sa fausseté, son masque de philosophie, son épicurienne morale, sa
bassesse jusqu'à l'indignité pour la faveur, son ambition démesurée, qui
se permettait tout, et sa hauteur insupportable dans la fortune, n'avait
pas deviné que M. le duc d'Orléans deviendrait le maître. Il se désolait
donc de n'être délivré par la mort du roi d'une disgrâce profonde, que
rien n'avait pu diminuer depuis Turin, que pour retomber dans une autre,
d'autant plus fâcheuse qu'il se l'était creusée lui-même par ses
gratuits forfaits. Il se désespérait de n'y voir point d'issue, quand un
coup de baguette changea son sort en un instant.

On a vu que l'infâme débauche et d'autres circonstances l'avaient
intimement lié avec Canillac, qui l'aimait d'autant plus chèrement que
son orgueil était flatté de la supériorité que La Feuillade lui avait
laissé prendre sur lui, jusqu'à en être regardé et traité comme son
oracle. Ce même orgueil de Canillac, joint à l'amitié, lui fit
entreprendre d'abuser de celle de M. le duc d'Orléans jusqu'à le trahir,
et de rendre la vie à l'ambition de La Feuillade. Canillac ne
connaissait que trop à fond le prince à qui il avait affaire. Il fit
l'effort de se taire sur ce projet qui ne pouvait réussir que par le
secret. Il piqua le régent de peur, d'intérêt et d honneur, l'un aussi
mal à propos que l'autre, étala son bien-dire d'un ton d'autorité, et
fit si heureusement son personnage que le régent, qui ne s'était montré
inexorable sur le comte de Roucy que parce que ce n'était pas un homme,
reçut presque comme un service l'occasion qui lui fut présentée par
Canillac de regagner La Feuillade, duquel, par l'étoffe qu'il y
connaissait, on lui fit aisément accroire qu'il y avait à craindre et à
espérer.

L'occasion du marché du gouvernement de Dauphiné, que Canillac persuada
à M. le duc d'Orléans, qui ne songeait à rien moins, d'acheter de La
Feuillade, qui avait grand besoin d'argent, pour M. le duc de Chartres,
fut habilement saisie, pour devenir une source de pluies de grâces et de
bienfaits sur La Feuillade, {[}comme{]} on le verra bientôt. Elles
indisposèrent étrangement le monde, parfaitement instruit de ce que La
Feuillade méritait du régent. Elles retirèrent aussi du nouveau favorisé
tous ses amis, ennemis du gouvernement, avec qui il frondait et
moralisait sans cesse, dont plusieurs étaient considérables à divers
égards, et qui ne se crurent plus en sûreté sur rien avec un homme à
transitions si entières et si subites. On verra dans la suite quelle fut
la conduite et la parfaite ingratitude de La Feuillade, et la
catastrophe des deux amis. Dès que la réconciliation fut faite, La
Feuillade fut nommé ambassadeur à Rome.

Avec tout son esprit, son brillant, ses discours étalés, il ne savait
quoi que ce soit au monde, n'eut jamais ni gravité ni maintien, se vêtit
et vécut toujours comme à dix-huit ans, et les propos souvent de même\,;
il n'avait d'homogène avec les Italiens chez qui on l'envoyait, au
milieu du feu de la constitution, que la foi et les mœurs. Aussi ne
songeât-il jamais sérieusement à y aller, mais à toucher gros pour ses
équipages, dont il ne fit que lentement un seul carrosse, et à se faire
payer ses appointements, comme s'il eût été à Rome. Ce manège dura
plusieurs années, au bout desquelles il ne fut plus question
d'ambassade, dont il se serait sûrement aussi bien acquitté qu'il avait
fait du siège de Turin.

Le nouveau duc de Valentinois pressait pour se faire recevoir au
parlement, et les pairs, à cette occasion, pressaient aussi pour faire
finir les usurpations dont ils se plaignaient. M. le Duc prétendit que
le duc du Maine et le comte de Toulouse ne devaient plus traverser le
parquet. Tout cela fit surseoir la réception du duc de Valentinois, et
une nouvelle aigreur entre M. le Duc et le duc du Maine.

La Garde, commis confident de Desmarets, avait été attaqué pour de
grosses sommes où son maître, du temps de son ministère, se trouvait
fort mêlé. Une créature du peuple, qu'on appelait M\textsuperscript{me}
La Fontaine, donna des avis contre lui, qui parurent si importants,
qu'après l'examen du conseil des finances, on jugea à propos de renvoyer
l'affaire au parlement. Le duc de Noailles, après ce qu'on a vu de
Desmarets, qui, à son retour, disgracié d'Espagne, l'avait réchauffé
dans son sein, le seul homme en place qui l'eût reçu, et qui de plus lui
avait appris tout ce qu'il avait voulu sur les finances, n'eut pas honte
de se montrer publiquement le protecteur de M\textsuperscript{me} La
Fontaine\,; ce qui fit beaucoup soupçonner qu'il l'avait instruite et
suscitée. Les amis de Desmarets en crièrent beaucoup. Le maréchal de
Villeroy et d'Effiat ne s'y épargnèrent pas, et protégèrent leur ami de
toutes leurs forces. Ils ne purent toutefois empêcher qu'il n'essuyât
des décrets et d'autres procédures fort désagréables. On en parla
quelque temps diversement. Le souvenir de l'affaire des pièces de quatre
sous rendit les accusations plausibles, et Desmarets y paya l'intérêt de
ses insolences et de ses brutalités passées. Il s'en tira pourtant fort
bien, et le duc de Noailles en eut toute la honte. Rien n'en passa au
conseil de régence\,; ainsi je profitai de pouvoir rester là-dessus dans
un entier silence. Mais Desmarets n'était pas au bout.

À peine jouissait-il de la satisfaction de s'être tiré nettement
d'affaires, que le duc de Noailles, enragé d'y avoir succombe, persuada
au régent que Desmarets, qui avait été en place l'ami et le protecteur
des principaux financiers, les tenait tous encore dans sa main, et par
ses manèges avec eux faisait avorter tout le fruit de son travail dans
les finances. Ainsi Desmarets, poursuivi sans relâche par ce
reconnaissant ami, fut averti que son exil était résolu et lui allait
être annoncé.

Louville avait épousé sa nièce, et m'avait, comme on l'a vu, voulu
raccommoder avec lui tout à la fin de la vie du roi, dont je n'avais pas
voulu entendre parler. Il vint me conter la triste situation de cette
mouche pourchassée par l'araignée, et prête à tomber dans ses toiles. Il
me demanda si je serait inexorable. Il n'oublia rien pour me piquer de
générosité, et mon courage aussi sur le plaisir de lui faire manquer son
coup. Je n'oserait dire que ce dernier tour fut inutile. Je m'étonnai
qu'avec d'Effiat et le maréchal de Villeroy en croupe, Desmarets, au
point où nous en étions, me fit rechercher dans son pressant besoin.
Louville me laissa entendre qu'ils étaient émoussés de l'affaire de
cette La Fontaine, et que j'étais la seule ressource à qui on pût avoir
recours. Je me complus un peu à me faire prier\,; et à voir l'ex-bacha
que j'avais perdu pour avoir méprisé mon ancienne amitié, ce vizir si
rogue, si brutal, si insolent, se jeter pour ainsi dire à mes pieds par
Louville, et me demander protection contre les traits de notre ingrat
commun. Je la lui accordai à la fin\,; et Louville, ravi, courut lui en
porter la nouvelle.

Dès le lendemain, je parlai au régent des bruits qui couraient de l'exil
de Desmarets. Il me répondit que la lettre de cachet en allait être
expédiée, et m'en expliqua plus au long les raisons que je viens de
rapporter, sans faire façon avec moi de nommer Noailles, et les plaintes
qu'il lui avait portées. Je souris, et lui dis qu'il savait de reste que
je n'aimais pas ces deux hommes, mais que j'aimais sa réputation à
lui\,; qu'il venait de voir par l'affaire de La Garde, et par celle de
cette M\textsuperscript{me} La Fontaine, qui avec tant d'éclat l'avait
suivie de si près, qu'on cherchait tout ce qu'on pouvait déterrer pour
perdre Desmarets\,; que, malgré l'art, le crédit et la volonté la plus
déployée, il était sorti net de toutes les deux\,; que je trouvais donc
fort peu décent de punir en coupable un homme qui venait de prouver la
fausseté de pareilles imputations, et que lui régent, qui passait
souvent pour trop bon, se mettait, par la complaisance de cet exil, de
moitié avec ceux qui par cette troisième poursuite acquéraient dans le
public avec raison l'odieux nom de persécuteurs\,; qu'au fond, les
plaintes qu'on lui avait portées n'avaient qu'une accusation vague, et
qui pouvait tomber sur tout homme instruit des finances et qui s'en
serait mêlé avec quelque autorité\,; que tout au plus elles pouvaient
mériter d'en faire avertir Desmarets, pour rendre sa conduite plus sage
et plus circonspecte, mais non pas un châtiment pour chose où il y avait
toute apparence qu'il n'était pas tombé, après l'exemple de son gendre
chassé en Bourgogne sur pareille accusation, et nouvellement instruit
par les deux affaires dont il venait de sortir, où {[}on{]} n'avait
cherché qu'à le perdre. Bref, je parlai si bien que non seulement le
régent me promit de ne plus songer à exiler Desmarets, mais me permit de
lui faire dire de sa part de n'en avoir plus d'inquiétude\,; et le
régent me tint parole.

l'avertis promptement Louville de ce que j'avais obtenu qui, après
louange et remerciement, me demanda si je refuserait de les recevoir de
Desmarets. Il alla lui porter la bonne nouvelle, et revint aussitôt me
conjurer de lui permettre de venir chez moi. J'eus la malice de me
laisser encore presser, puis je consentis à le voir cinq ou six jours
après chez Louville, comme par hasard, pour ne pas joindre de si près un
raccommodement public à ce qui venait de se passer. On peut juger de ce
que Desmarets me dit chez Louville\,; il vint après chez moi, et nous
nous revîmes.

Le printemps d'après j'allai passer quelques jours à la Ferté dans un
intervalle de conseils. Desmarets se trouva chez lui à Maillebois, qui
en est à quatre lieues. Il vint dîner à la Ferté, et fut curieux de voir
beaucoup de choses que j'avais faites dans le parc depuis bien des
années qu'il n'y était venu. Il était goutteux\,; le parc est grand\,;
nous montâmes tous deux dans une calèche. La conversation se porta
bientôt sur le gouvernement passé et présent. Nous nous parlâmes de
bonne foi l'un à l'autre. Je lui rappelai ce qui, par son humeur et sa
plus que négligence à mon égard, m'avait fâché, et lui racontai
franchement comment je l'avais fait chasser de sa place. Lui, avec la
même sincérité, m'avoua que la tête lui avait tourné\,; que ses
précédents malheurs, qui devaient l'avoir instruit sur les places, la
cour et le monde, et l'attacher à ses anciens amis, n'avaient pu le
rendre sage dans la pratique dans son retour, ni le préserver de
l'entraînement\,; qu'il était vrai qu'il avait compté pour tout le roi
et M\textsuperscript{me} de Maintenon, et tout le reste pour rien\,;
vrai encore qu'accoutumé depuis si longtemps à leur règne, et par son
retour à les approcher tous les jours, il les avait crus immortels, et
n'avait jamais imaginé qu'ils pussent mourir\,; qu'il se comptait très
bien avec eux\,; qu'il ne songeait qu'à s'y maintenir, et qu'il n'avait
d'attention que pour ceux qui étaient assez bien avec eux pour y pouvoir
contribuer. J'ai cent fois repassé en moi-même une conversation si
singulière. Elle dura toute la promenade, et effaça toute la beauté de
mon parc sans que j'y prisse garde. Elle ne finit pas sans dire deux
mots chacun de notre bon et estimable ami le duc de Noailles. Après une
ouverture si égale des deux parts et si extraordinaire, l'heure de s'en
aller nous sépara à regret, et jusqu'à sa mort nous nous sommes vus sur
le pied d'amitié et de franchise. Je devais le surlendemain aller dîner
à Maillebois\,; mais le lendemain il m'envoya dire qu'il était pris
d'une forte néphrétique, et qu'il me priait de n'y pas aller. Je sus
après qu'elle avait {[}été{]} violente, et lui avait duré plusieurs
jours. Je ne sais si ma franchise lui avait causé cette révolution. Je
fus obligé de retourner à Paris\,; il y revint bientôt après. J'ai cru
que cette aventure méritait d'avoir place ici pour sa curieuse rareté.

Un matin que le conseil de régence se tenait aux Tuileries sur les
affaires étrangères, nous fûmes surpris que le duc de Noailles demanda à
entrer pour une affaire pressée. Il parla un moment, à un coin, à M. le
duc d'Orléans, puis proposa le rehaussement des espèces. La surprise fut
grande. Le régent parla après lui sur le malheur de cette nécessité,
mais comme ayant pris son parti. On opina assez confusément, entre la
répugnance et la crainte de déplaire. Quand ce vint à moi, j'exposai
tous les inconvénients de toucher à la monnaie, par les histoires et par
les exemples de nos jours, et l'illusion d'un soulagement présent qui
entraînait de si longues et de si funestes suites pour le change et pour
la place, et pour toute sorte de commerce, et je conclus à la laisser
sur le pied qu'elle était, puisqu'on n'était pas en état de la
rapprocher en la baissant de sa valeur intrinsèque. Je fus applaudi,
mais tondu. Cela ne laissa pas d'exciter quelque murmure, et beaucoup
dans le public.

M. le duc d'Orléans déclara d'Antin surintendant des bâtiments, comme
Torcy des postes. Il y eut de la difficulté au parlement et à la chambre
des comptes.

Ce prince assista, comme faisait feu Monsieur, aux dévotions de Noël à
Saint-Eustache et aux pères de l'Oratoire de Saint-Honoré. Moins de
dévotions de calendrier, et moins de licence {]}es soirs, aurait formé
une vie plus unie et plus décente. Il n'est pas encore temps d'en
parler, non plus que du détail de ses journées. Il faut un peu plus
avancer pour s'y étendre plus à propos.

Enfin le lundi 30 décembre le roi partit de Vincennes après son dîner
pour venir à Paris, placé dans son carrosse aussi peu décemment qu'il
l'avait été en venant de Versailles à Vincennes. Il était au fond entre
M. le duc d'Orléans et la duchesse de Ventadour\,; le maréchal de
Villeroy au devant, entre M. du Maine et le prince Charles, grand
écuyer\,; le maréchal d'Harcourt, capitaine des gardes en quartier, à la
portière du roi, c'est-à-dire à droite M. le Premier souffla l'autre de
vitesse au duc d'Albret, grand chambellan, que M. le duc d'Orléans avait
appelé.

J'ai déjà expliqué le droit des places du carrosse du roi, lors du
voyage de Versailles à Vincennes. J'ajouterai seulement que M. du Maine,
ni le maréchal de Villeroy, n'avaient aucun fondement de s'y mettre tant
que le roi était entre les mains des femmes, et leurs places auraient
été remplies avec raison par le duc de Tresmes, premier gentilhomme de
la chambre en année, et par le duc d'Albret. L'anticipation des hommes
de l'éducation avait commencé à Vincennes, où ils eurent des logements.
Aux Tuileries le maréchal de Villeroy eut un beau logement, et ensuite
il prit celui de la reine, contigu à celui du roi, et M. du Maine eut en
bas le bel appartement des Dauphins. M. de Fréjus en eut un en haut. Les
sous-gouverneurs, etc., y en eurent aussi. La ville harangua le roi à
son arrivée, qui trouva grande foule jusque dans son appartement. Ainsi
finit l'année 1715.

\hypertarget{chapitre-xv.}{%
\chapter{CHAPITRE XV.}\label{chapitre-xv.}}

1716

~

{\textsc{1716.}} {\textsc{- M. du Maine me fait une visite sans cause.}}
{\textsc{- Je visite M. {[}le duc{]} et M\textsuperscript{me} la
duchesse du Maine, qui me tiennent des propos fort singuliers, mais fort
polis.}} {\textsc{- Abbé Dubois conseiller d'État d'Église.}} {\textsc{-
Force évêchés et abbayes donnés.}} {\textsc{- Prédiction sur Cambrai
singulière.}} {\textsc{- Conseil de commerce.}} {\textsc{- M. le Duc et
le duc du Maine entrent au conseil de guerre.}} {\textsc{- Mort des
reines douairières de Suède et de Pologne.}} {\textsc{- Mort, caractère
et succession de la duchesse de Lesdiguières-Gondi.}} {\textsc{- Mort de
M\textsuperscript{me} de Grancey.}} {\textsc{- Mort et caractère de
Coulanges, et celui de sa femme.}} {\textsc{- Mort de Cavoye.}}
{\textsc{- Veuvage de sa femme respectable et prodigieux.}} {\textsc{-
Mort de M\textsuperscript{lle} d'Acigné.}} {\textsc{- Mort de
Parabère.}} {\textsc{- Mariage du fils unique de M. de Castries.}}
{\textsc{- Singularité étrange de M\textsuperscript{me} la duchesse
d'Orléans.}} {\textsc{- Mariage de Broglio, mort maréchal de France et
duc, avec une Malouine.}} {\textsc{- Mariage de Bellegarde avec la fille
unique de Vertamont, à qui on donne un râpé de l'ordre.}} {\textsc{-
Foule étrange de ces râpés et vétérans.}} {\textsc{- Mariage de Maubourg
avec une fille du maréchal de Besons.}} {\textsc{- Mariage du duc de
Melun avec une fille du duc d'Albert.}} {\textsc{- Mariage conclu, puis
rompu avec éclat, du marquis de Villeroy avec la fille aînée du prince
de Rohan, qui ne le pardonne pas.}} {\textsc{- Il marie sa fille au duc
de La Meilleraye, et le marquis de Villeroy épouse la fille aînée du duc
de Luxembourg.}} {\textsc{- Courtenvaux marie son fils à la dernière
fille de la maréchale de Noailles, et lui donne sa charge des
Cent-Suisses.}}

~

Avant de commencer {[}à{]} rapporter les événements de cette année 1716,
il faut, pour un moment, remonter dans la précédente, sur la préparation
de ce qui en fut les premiers. M. du Maine et moi étions toujours sur le
même pied ensemble, depuis l'étrange visite que je lui avais rendue,
lorsqu'il nous fit casser sur le corps la corde du bonnet qu'il nous
avait si malicieusement tendue. Nous nous voyions sans cesse au conseil
de régence\,; il y cherchait à s'attirer quelque civilité de moi par
toutes celles dont il me prévenait, sans toutefois oser me parler\,; il
me trouvait également sec et roide, lent et bref à lui rendre les
révérences longues et marquées dont il m'accablait. Le roi n'était
plus\,; M\textsuperscript{me} de Maintenon n'était plus à craindre. De
leur temps je ne l'avais pas ménagé, ni ne m'étais montré plus poli à
son égard depuis ce sourd éclat. Il comprenait que je m'en contraindrais
bien moins encore\,; il me voyait dans la plus grande liberté avec le
régent, et dans une confiance qui me rendait un personnage\,; sa
timidité s'en alarmait\,; il ne savait comment me rapprocher.

Dans cette situation réciproque, je fus très surpris, sur la fin du
séjour de Vincennes, qu'un matin que j'y avais couché, je vis entrer le
duc du Maine dans ma chambre. Il couvrit son embarras d'un air aisé, et,
avec mille prévenances, m'entretint comme si nous n'eussions jamais rien
eu ensemble, et sans me parler de quoi que ce soit du passé. C'était
l'homme du monde qui menait mieux la parole et toutes sortes de
conversations. Il usa de ce talent avec toutes ses grâces, et n'oublia
rien pour me plaire, sans toucher le moins du monde à rien
d'intéressant. Il fallut bien, chez moi, tâcher de payer de même
monnaie. Quoique la partie ne fût pas égale, je m'en tirai
raisonnablement bien, avec assez de langage et de politesse pour ne rien
mettre contre moi, avec assez de retenue, sur les compliments
principalement, pour ne rien donner du mien. Cela dura plus d'une
demi-heure tête-à-tête\,; c'était avant le conseil de régence du matin,
et point du tout l'heure des visites. Ce temps qu'il avait pris m'avait
encore été par là suspect\,; quand il fut sorti, je me trouvai
doublement à mon aise d'en être délivré, et que ce fût simplement une
visite. Ce fut la première chose que je dis au régent, un moment avant
de nous mettre au conseil. Nous rîmes ensemble de la frayeur de cet
homme, qui le comptait naguère pour si peu, et moi, comme de raison,
pour infiniment moins. Il m'exhorta cependant à lui rendre sa visite, et
puisqu'il avait fait cette première démarche, à lui montrer moins
d'éloignement et de sécheresse dans les lieux où nous nous trouvions
nécessairement tous deux. Quelque raisonnable que fût ce conseil, il me
coûta à suivre après ce qui s'était passé, et que j'ai raconté en son
lieu. Je n'ai jamais été faux\,: il me semblait de la fausseté à vivre
avec le duc du Maine comme avec un autre homme indifférent. Néanmoins je
m'y pliai comme je pus par la nécessité de la bienséance, d'assez
mauvaise grâce, je crois, et toujours évitant le plus que je le pouvais
de me trouver à portée de sa conversation, et toujours peiné de la
prostitution de ses révérences, et de toutes les agaceries dont il
tâchait sans cesse de me rapprocher et de me prévenir.

L'arsenal était renversé pour y bâtir un beau logement pour lui. La
maison qu'il se faisait au bout de la rue de Bourbon, sur la rivière,
était à peine commencée\,; il logeait à l'emprunt dans la maison du
premier président, rue Sainte-Avoye, au Marais, lequel par sa place
habitait au Palais. Ce fut là que je l'allai voir dans les premiers
jours que le roi fut revenu de Vincennes à Paris, et je pris une fin de
matinée pour avoir un prétexte sûr de ne point voir
M\textsuperscript{me} la duchesse du Maine. Je n'y gagnai rien\,; je fus
reçu avec des empressements, même des remerciements. Bientôt après,
voulant m'en aller, il me dit que M\textsuperscript{me} la duchesse du
Maine ne lui pardonnerait jamais de me laisser sortir sans la voir.
J'eus beau faire et beau dire, il m'y mena malgré moi, et me mit dans un
fauteuil au chevet de son lit, et lui vis-à-vis de moi. L'accueil fut le
même\,; car la femme ne faisait pas moins d'elle et de sa langue tout ce
qu'elle voulait, ni avec moins de grâce et de politesse, quand il lui
plaisait, que le mari. Je crus au moins en être quitte pour ces sortes
de langages\,; point du tout\,; les cajoleries cédèrent à du sérieux,
qui me surprit fort et ne m'embarrassa point. Il y avait là sept ou huit
hommes ou femmes de leur maison avec nous. M\textsuperscript{me} du
Maine, à propos de la maison où je la voyais, me mit sur le premier
président, car ce fut elle qui tint toujours le dé, et M. du Maine ne
fit que se mêler dans la conversation. Je répondis que l'amitié que je
lui savais pour ce magistrat me fermait la bouche en sa présence. Elle
me pressa, et tant, qu'elle eut contentement, et moi aussi. Elle n'en
fit que rire, et M. du Maine, qui excellait en ces sortes de propos, les
allongea encore. Je voulus prendre congé\,; ils s'écrièrent tous deux
que c'était pour eux tant de plaisir de me voir qu'ils le voulaient
faire durer davantage. Cela voulait dire si nouveau et si rare, car
depuis la visite que j'avais reçue de M. du Maine, je n'avais point
encore été chez lui, et lorsque, avant l'affaire du bonnet, je le
voyais, c'était extrêmement rarement, et toujours sans aller chez
M\textsuperscript{me} la duchesse du Maine, qui d'ailleurs n'était comme
jamais à la cour. Tout de suite, et comme de peur de manquer à tenir ce
chapitre avec moi, elle me parla de M. le Duc et d'eux, dont les démêlés
fermentaient sans beaucoup paraître encore. Je voulus éviter d'entrer en
cette matière, mais elle m'y força par des interrogations sans fin,
doucement aiguisées par le duc du Maine, en sorte que je me trouvai là
comme sur la sellette, écouté et regardé attentivement de ce petit
groupe de gens qui nous environnaient. À la fin j'en sortis par leur
dire que M. du Maine, et elle par conséquent, devaient savoir, il y
avait longtemps, ce que je pensais là-dessus, puisque je le lui avais
dit plus d'une fois à lui-même.

J'avais espéré couper court par cette réponse, qui disait tout et
n'expliquait rien en détail. M\textsuperscript{me} du Maine ne s'en
contenta point, et avec une plaisanterie à M. du Maine de ce qu'il ne
lui disait pas tout, elle me pressa de parler plus clairement. Ce
procédé me mit intérieurement en colère. Je lui dis donc que puisqu'elle
voulait absolument entendre de nouveau ce qu'elle ne me persuaderait pas
que M. du Maine ne lui eût pas appris dans les temps, je lui obéirais,
pourvu qu'elle voulût bien se souvenir qu'elle me le commandait\,; et
là-dessus je lui répétai que j'étais fort content qu'ils fussent princes
du sang, succédant à la couronne, parce qu'avec ceux-là nous n'avions
rien à démêler\,; que tant qu'ils seraient dans cet état, nous n'avions
rien à dire\,; mais qu'ils prissent bien garde à se le conserver, parce
que, s'ils venaient à en déchoir, nous ne supporterions pas leur rang
intermédiaire, et que nous ferions tout ce qui serait en nous pour ne
les pas voir entre les princes du sang et nous. Tous deux, au plus loin
de leur pensée, trouvèrent que j'avais raison, et qu'ils n'avaient point
à se plaindre dès que nous trouvions bon l'état dont ils jouissaient.
«\,Mais, ajouta-t-elle, n'exciterez-vous point les princes du sang
contre nous\,? --- Madame, lui répondis-je, ce ne sont pas là nos
affaires, mais celles des princes du sang qui n'ont pas besoin de notre
conseil, et qui aussi ne nous le demandent point.\,» Je dansai ainsi sur
la corde sur une si délicate question. Ils demeurèrent satisfaits de
tout ce que je leur dis, parce qu'ils le voulurent être, et moi encore
plus de m'en être tiré sans broncher d'un côté ni d'autre. Les
gentillesses recommencèrent à l'envi de leur part, et je les quittai
enfin après une grosse heure au moins, qui m'en parut le double.
Conduite de M. du Maine et compliments à l'infini. Oncques depuis je
n'ai vu M\textsuperscript{me} du Maine chez elle, et M. du Maine
extrêmement rarement aux Tuileries. Mais au conseil, et quelquefois chez
M\textsuperscript{me} la duchesse d'Orléans où je le rencontrais, il se
surpassait à mon égard, et je faisais aussi la meilleure mine que je
pouvais, qui, pour en dire la vérité, n'était pas trop bonne, et
toujours avec grande réserve, et jamais n'attaquant, ni presque jamais
m'en approchant, et tant que je pouvais honnêtement, évitant de m'en
laisser joindre.

Je n'étais pas sur ce ton avec le comte de Toulouse. Celui-là, comme je
l'ai dit ailleurs, était fort vrai et fort honnête homme. Il n'avait eu
nulle part aux grandeurs que son frère avait accumulées en Titan pour
escalader les cieux, beaucoup moins encore à l'affaire du bonnet. Sa
façon d'opiner, d'aller au bien pour le bien, à la justice pour la
justice, m'avait gagné. Je le voyais souvent chez M\textsuperscript{me}
la duchesse d'Orléans, et je vivais avec lui en ouverture, et lui avec
moi, ce qui s'était peu à peu amené réciproquement des deux côtés, sans
néanmoins de ces confiances d'amis intimes, et sans nous voir l'un chez
l'autre, mais ailleurs presque tous les jours, très souvent en tiers
avec M\textsuperscript{me} la duchesse d'Orléans, quelquefois la
duchesse Sforce en quatrième, où nous parlions fort librement\,;
toujours auprès de lui au conseil, où nous nous parlions de même, et
quelquefois tète à tête avant et après.

L'autre affaire qui oblige à rétrograder est la vacance d'une place de
conseiller d'État d'Église par la mort de La Hoguette, archevêque de
Sens. L'abbé Dubois m'avait toujours fort courtisé, comme on l'a souvent
vu dans ces Mémoires. Depuis la décadence de la santé et la mort du roi,
il avait redoublé. Lors de cette grande époque, il était tombé auprès de
son maître, et Madame, comme je l'ai raconté en son lieu, avait achevé
de le tuer auprès de lui. Dans cet état d'éloignement, il avait eu
recours a moi, et jusqu'à ce qu'il ait été secrétaire d'État, je l'ai
souvent, et pendant des années, trouvé dans son carrosse, rangé dans la
rue près de chez moi, attendant que je rentrasse, sans vouloir entrer
lui-même avant moi, et en plein hiver souvent, ni jamais souffrir que
son carrosse fût ailleurs que dans la rue. J'avais effectivement trouvé
qu'il était traité trop durement, après avoir eu tant de privance. Je
l'avais représenté à M. le duc d'Orléans, l'exhortant néanmoins à le
tenir éloigné de toute affaire, mais à le traiter d'ailleurs avec plus
de bonté. J'avais réussi sur ce dernier article depuis quelque temps\,;
plût à Dieu que sur l'autre j'eusse été cru de même\,!

L'abbé Dubois voulut être conseiller d'État, et me vint prier d'en
rompre la glace auprès du régent. Il s'appuyait sur ce que les évêques
ne voudraient plus d'une place dans laquelle l'abbé Bignon les
précéderait\,; et, en effet, c'est ce qui les en a exclus, au déshonneur
du conseil. Ma franchise ne put se taire. Je répondis à l'abbé Dubois
que je lui souhaitais toute sorte de bien, mais que pour cette place je
le priais de regarder un peu derrière lui, et de voir si elle lui
convenait, le dépit qu'en auraient les conseillers d'État, et si son
attachement pour M. le duc d'Orléans lui pouvait permettre de lui
attirer par là la haine de tout le conseil et de tous les prétendants,
et tous les discours du monde, tous ceux qui se tiendraient sur
lui-même, et les mauvais offices qui sûrement naîtraient de ce choix. Il
fut un peu étonné, mais il n'eut point de bonne réplique\,; nous ne
laissâmes pas de nous séparer fort bien. Quatre jours après, l'abbé
Dubois revint chez moi, qui d'abordée\,: «\, Je viens, me dit-il, vous
rendre compte que je suis conseiller d'État,\,» transporté de joie.
«\,Mon cher abbé, lui répondis-je, j'en suis ravi, et d'autant que je
n'y ai point de part\,; vous êtes content, et moi aussi. Prenez
seulement garde aux suites, et puisque l'affaire est faite, tenez-vous
gaillard, et veillez-y seulement sans les craindre.\,» Je l'embrassai,
et il s'en alla fort satisfait de moi. Je n'en dis pas un mot au régent
ni lui à moi. Ma coutume était de ne lui jamais parler des choses faites
que je désapprouvais\,; la sienne, de ne me rien dire de celles qu'il
avait faites, et qu'il sentait faites mal à propos. Sur les grâces, je
ne voulais desservir personne\,; ainsi je n'allais point à la parade,
mais je me réservais tout entier pour tout ce qui était affaires, et
empêcher celles que je croyais mauvaises. Les suites furent telles que
je les avais prévues. Il n'y eut personne, depuis le chancelier jusqu'au
dernier des maîtres des requêtes, qui ne se crut personnellement
offensé, et qui ne le montrât. Ni eux ni les prétendants ne
contraignirent leurs plaintes ni leurs discours. L'abbé Dubois, qui ne
pensait qu'à soi, avait ce qu'il avait voulu, et ne se soucia point du
bruit ni de son maître.

Quatre jours après, M. le duc d'Orléans donna ce grand nombre de
bénéfices, dont le P. Tellier n'avait jamais pu venir à bout de
persuader au roi de disposer pour en disposer lui-même. Pour cette fois,
ils furent assez bien donnés. L'abbé d'Estrées eut Cambrai. Je me
souviens très bien qu'à la mort du célèbre Fénelon, son prédécesseur, il
courut une prophétie de je ne sais qui de ce diocèse\,; que ses trois
premiers successeurs n'y entreraient jamais. On rit avec raison de ce
conte, qui pourtant s'est trouvé exactement accompli. L'ancien évêque de
Troyes obtint Sens pour son neveu, qui était évêque de Troyes, homme de
vertu, de savoir, de mœurs et de mérite, et qui valait bien mieux que
lui. L'abbé de Castries, à qui Troyes fut donné, le refusa\,; il crut
que c'était trop peu de chose pour un homme de son âge, qui avait été
aumônier ordinaire de M\textsuperscript{me} la Dauphine, et qui avait
acheté la charge de premier aumônier de M\textsuperscript{me} la
duchesse de Berry. Il était frère du chevalier d'honneur de
M\textsuperscript{me} la duchesse d'Orléans, tellement que pour cette
fois la mère et la fille se trouvèrent d'accord à soutenir l'abbé de
Castries. Je proposai au régent de mettre les prétendants à Bayeux
d'accord, sans jalousie, au profit du roi, en le donnant au cardinal de
La Trémoille qui était un panier percé, et qu'il fallait bien soutenir à
Rome par des pensions ou par des bénéfices. Celui-là valait quatre-vingt
mille livres de rente\,; on en prit dix en pensions. Je proposai aussi
l'abbé de Beaumont pour Saintes. Je ne le connaissais point du tout\,;
mais il était fils d'une sœur de M. de Fénelon, archevêque de Cambrai,
et homme de bonnes mœurs, qui avait été lecteur des princes, et chassé
d'auprès d'eux avec son oncle. La mémoire, toujours vivante en moi, du
duc de Beauvilliers, agit seule en moi en cette occasion. Un abbé
d'Entragues, aumônier du feu roi et de celui-ci, eut Clermont. Je le
nomme parce que Bentivoglio, qui le crut mal affectionné à la
constitution, lui rendit tant de si mauvais offices à Rome que ses
bulles retardèrent toutes les autres. La vérité est qu'il estimait la
constitution sa juste valeur, et qu'il connaissait les jésuites. Il ne
s'en contraignit pas pendant son épiscopat, qui ne fut pas bien long.
C'était un très homme de bien, mais de peu de savoir. Il y eut quatorze
ou quinze abbayes données\,: le cardinal Gualterio eut Saint-Victor, à
Paris\,; et le cardinal Ottoboni, Saint-Paul de Verdun. Le régent donna
Saint-Ouen de Rouen à l'abbé de Saint-Albin\,; c'était un nom de guerre,
et un bâtard qu'il avait eu de la comédienne Florence, qu'il n'a point
reconnu. L'abbé de Thésut, secrétaire de ses commandements, eut celle de
Saint-Martin de Pontoise\,; et celle de Sainte-Madeleine fut donnée à un
chanoine de Notre-Dame de Paris, frère de La Roche, qui avait
l'estampille et la confiance du roi d'Espagne, qui l'avait fort
recommandé. Enfin Moissac fut donné à Biron pour un fils qu'il voulait
pousser dans l'Église, et qui n'a jamais voulu étudier, ni être prêtre.

Le régent établit un nouveau conseil de commerce, sur le modèle de celui
qui se tenait sous le feu roi, où entraient et entrèrent les douze
députés des douze principales places de commerce du royaume, élus chacun
par sa ville. Au lieu de M. d'Aguesseau qui présidait seul, on y mit le
maréchal de Villeroy, comme chef du conseil des finances, qui ne fut
proprement que \emph{ad honores}, comme il était au conseil des
finances. Le duc de Noailles, qui y faisait tout, fut le second, mais le
véritable président de ce conseil de commerce, où le maréchal d'Estrées
eut liberté d'entrer quand il le voudrait comme président du conseil de
marine. Quatre conseillers d'État y furent mis\,: MM. d'Aguesseau\,;
Amelot, qui, pour avoir longtemps gouverné la marine, les finances et le
commerce d'Espagne, en savait plus que tous\,; Nointel et Rouillé du
Coudray, qui avec M. de Noailles était le maître des finances et de tout
ce qui y avait rapport. On y fit entrer aussi un cinquième conseiller
d'État qui fut M. d'Argenson, mais comme lieutenant de police, et trois
maîtres des requêtes. La nomination des inspecteurs du commerce dans les
places de commerce fut attribuée à ce conseil, dont les patentes furent
données au nom du maréchal de Villeroy, excepté celui de Marseille, dont
la dépendance fut réservée au conseil de marine. Valossière, produit par
le duc de Noailles, fut secrétaire du conseil de commerce. Cet
établissement était fort bon, et aurait été fort utile, si les intérêts
particuliers, qui gâtent toujours tout en France, n'en eussent point
traversé l'administration.

M. le Duc pressa tant le régent de lui permettre d'entrer au conseil de
guerre qu'il l'obtint, à condition de n'y présider point, quoique à la
première place, et de ne s'y mêler de rien. La même faiblesse qui lui
fit accorder cette entrée ne la put refuser au duc du Maine, qui faisait
en tout le singe des princes du sang, et aux mêmes conditions. Mais
comme il avait les Suisses et l'artillerie, elles ne purent si bien être
exécutées à son égard qu'à celui de M. le Duc, qui n'avait point de
charges militaires. Il voulut donc dans la suite se mêler peu à peu,
comme avait fait le duc du Maine, et cela causa des embarras qui
retardèrent les affaires, et qui fatiguèrent souvent M. le duc d'Orléans
et ce conseil, et l'obligèrent d'y entrer plus souvent qu'il n'eût
voulu. Ces tracasseries mirent plus que du froid entre M. le Duc et le
maréchal de Villars, lequel à la fin demeura le maître, et les dégoûta
de ce conseil, où ils n'allèrent presque plus\,; mais ce ne fut qu'après
assez longtemps.

Deux reines moururent tout au commencement de cette année, dont la perte
ne fit pas grand bruit dans le monde\,: la reine mère de Suède, à près
de quatre-vingts ans, qui était Holstein-Gottorp\,; et la reine de
Pologne à Blois, La Grande-Arquien, veuve du fameux roi Jean Sobieski.
On a vu en son temps que son orgueil l'avait rendue la plus vive ennemie
de la France, et comment aussi elle y fut reçue quand, lasse de Rome,
elle voulut s'y retirer. Elle y fut laissée avec toute l'inconsidération
qu'elle méritait, et y vécut et mourut comme une particulière. Elle fut
traitée de même après sa mort, et sa petite-fille aussi qui était auprès
d'elle. Elle s'en alla, sans aucun honneur de la part de la cour,
joindre en Silésie son père Jacques Sobieski, qui y vivait retiré sur
ses grands biens. Il la maria depuis au roi Jacques d'Angleterre à Rome.
Elle n'eut pas même permission de passer par Paris. On ne sait ce qui la
retint à Blois quatre ou cinq mois encore après avoir perdu sa
grand'mère.

La duchesse de Lesdiguières mourut à Paris dans son bel hôtel. Elle
n'était point vieille, mais veuve depuis très longtemps, et avait perdu
son fils unique, gendre de M. de Duras. C'était le reste de ces Gondi
amenés en France par Catherine de Médicis, qui y avaient fait une si
prodigieuse fortune et tant figuré. Aussi laissa-t-elle des biens
immenses. C'était de tous points une fée, qui avec de l'esprit ne
voulait voir presque personne, moins encore donner à manger à aucun de
ce peu qu'elle voyait\,; jamais à la cour, et presque jamais hors de
chez elle. Sa maison, dont la porte était toujours ouverte, était aussi
toujours fermée d'une grille qui laissait voir un vrai palais de fée,
tel que les dépeignent les romans. Le dedans presque désert, mais de la
dernière magnificence, y répondait par là et par sa singularité, que ne
démentait pas son train, sa livrée, la housse jaune de son carrosse, et
ses deux grands Maures avec tout leur appareil. Elle laissa gros à ses
domestiques et en legs pieux\,; rien à sa belle-fille, quoique pauvre,
et qu'elle lui rendît beaucoup de devoirs\,; six mille livres viagers à
la sœur de Vertamont, veuve sans enfants du duc de Brissac, qui avait
été mon beau-frère en premières noces, et qui était son cousin germain,
laquelle duchesse de Brissac n'avait pas de pain, beaucoup d'esprit et
de mérite, et la voyait fort\,; huit mille livres viagers et la
jouissance d'une terre de dix mille livres de rente à la duchesse de
Lesdiguières-Canaples, qui était Mortemart, qu'elle aimait fort. Le
maréchal de Villeroy et ses enfants héritèrent de plus de trois cent
mille livres, outre sa belle maison, et une grande quantité de meubles
magnifiques.

La mère du maréchal de Villeroy était soeur du duc de Lesdiguières,
beau-père de cette fée\,; et la mère de cette même fée et celle de la
femme du maréchal de Villeroy étaient soeurs. La branche de Lesdiguières
et la maison de Gondi étaient éteintes\,; et le duc de Brissac, frère de
la maréchale de Villeroy, n'avait point eu d'enfants. Ainsi les Villeroy
héritèrent des deux côtés de tout à la fois, parce que le duc de
Lesdiguières, fils de la fée, lui avait laissé tous ses biens par son
testament. Qui eût prédit cette succession aux ducs, maréchal, cardinaux
de Gondi et de Retz, au connétable de Lesdiguières et au maréchal de
Créqui son gendre, qui avaient tous vu M. de Villeroy secrétaire d'État,
et d'où il était sorti, ils se seraient étrangement indignés, le
maréchal de Créqui surtout, qui eut tant de peine à consentir au mariage
de sa fille, que le connétable son beau-père le força de faire avec M.
de Villeroy, petit-fils du secrétaire d'État, parce qu'il avait la
survivance du gouvernement de Lyon, Lyonnais, etc., de M. d'Alincourt
son père, et que le connétable, gouverneur de Dauphiné, commandant de
Provence, et comme roi dans ces deux provinces, le voulut être encore
dans le gouvernement de Lyon, Lyonnais, etc.

Médavy perdit en même temps sa fille unique, qu'il avait mariée à
Grancey son frère, qui n'en eut point d'enfants.

Le monde perdit aussi Coulange. C'était un très petit homme, gros, à
face réjouie, de ces esprits faciles, gais, agréables, qui ne produisent
que de jolies bagatelles, mais qui en produisent toujours et de
nouvelles et sur-le-champ, léger, frivole, à qui rien ne coûtait que la
contrainte et l'étude, et dont tout était naturel. Aussi se fit-il
justice de fort bonne heure. Il se défit d'une charge de maître des
requêtes, renonça aux avantages que lui promettaient sa proche parenté
avec M. de Louvois, et ses alliances avec la meilleure magistrature,
uniquement pour mener une vie oisive, libre, volontaire, avec la
meilleure compagnie de la ville, même de la cour, où il avait le bon
esprit de ne se montrer que rarement, et jamais ailleurs que chez ses
amis particuliers. La gentillesse, la bonne mais naturelle plaisanterie,
le ton de la bonne compagnie, le savoir-vivre et se tenir à sa place
sans se laisser gâter, le tour aisé, les chansons à tous moments qui
jamais n'intéressèrent personne, et que chacun croyait avoir faites, les
charmes de la table sans la moindre ivrognerie ni aucune autre débauche,
l'enjouement des parties dont il faisait tout le plaisir, l'agrément des
voyages, surtout la sûreté du commerce, et la bonté d'une âme incapable
de mal, mais qui n'aimait guère aussi que pour son plaisir, le firent
rechercher toute sa vie, et lui donnèrent plus de considération qu'il
n'en devait attendre de sa futilité. Il alla plus d'une fois en
Bretagne, même à Rome, avec le duc de Chaulnes, et fit d'autres voyages
avec ses amis\,; jamais ne dit mal ni ne fit mal à personne\,; et fut
avec estime et amitié l'amusement et les délices de l'élite de son
temps, jusqu'à quatre-vingt-deux ans, dans une santé parfaite de tête et
de corps, qu'il mourut assez promptement. Sa femme, qui avait plus
d'esprit que lui, et qui l'avait plus solide, eut aussi quantité d'amis
à la ville et à la cour, où elle ne mettait jamais le pied. Ils vivaient
ensemble dans une grande union, mais avec des dissonances qui en
faisaient le sel et qui réjouissaient toutes leurs sociétés. Ils
n'eurent point d'enfants. Elle l'a survécu bien des années. Elle avait
été fort jolie, mais toujours sage et considérée. Coulange était un
petit homme fort gras, de physionomie joviale et spirituelle, fort égal
et fort doux, dont le total était du premier coup passablement
ridicule\,; et lui-même se chantait et en plaisantait le premier.

Cavoye mourut en même temps. Je me suis assez étendu sur lui et sur sa
femme pour n'avoir rien à y ajouter. Cavoye, sans cour, était un poisson
hors de l'eau\,; aussi n'y put-il longtemps résister. Si les romans ont
rarement produit ce qu'on a vu de sa femme à son égard, ils auraient
peine à rendre le courage avec lequel cet amour pour son mari si durable
la soutint pour l'assister dans sa longue maladie et à sa mort, voulant,
disait-elle, qu'il fût heureux en l'autre vie, ni la sépulture à
laquelle elle se condamna à sa mort, et qu'elle garda fidèlement jusqu'à
la sienne. Elle conserva son premier deuil toute sa vie, jamais ne
découcha de la maison où elle l'avait perdu, ni n'en sortit que pour
aller deux fois le jour à Saint-Sulpice prier dans la chapelle où il est
enterré. Elle ne voulut jamais voir d'autres personnes que celles
qu'elle avait vues dans les derniers temps de la maladie de son mari, ou
le jour de sa mort, ne s'occupa que de bonnes œuvres de toutes les
sortes, presque toutes relatives au salut de son mari, et se consuma
ainsi en peu d'années, sans avoir jamais faibli ni reculé d'une ligne.
Une véhémence si égale et si soutenue, sans relâche ni amusement de quoi
que ce soit, et toujours surnagée de religion, est peut-être un exemple
unique et bien respectable.

La mort de M\textsuperscript{lle} d'Acigné délivra le duc de Richelieu,
fils de sa sœur, d'un retour de partage de cent mille écus qu'elle lui
demandait.

Parabère mourut aussi. Pour le personnage qu'il faisait en ce monde, il
eût mieux valu pour lui de le quitter plus tôt. Il était gendre de
M\textsuperscript{me} de La Vieuville, dame d'atours de
M\textsuperscript{me} la duchesse de Berry. J'aurai lieu ailleurs de
parler de M\textsuperscript{me} de Parabère.

Ce commencement d'année produisit aussi plusieurs mariages. Celui du
jeune Castries avec la fille de Nolent, conseiller au parlement, dont le
frère avait été major du régiment des gardes, donna une ridicule scène.
Pour la faire entendre, il faut dire que le père de M. de Castries était
lieutenant général de Languedoc, gouverneur de Montpellier, chevalier de
l'ordre en 1661, et que sa mère était sœur du cardinal Bonzi, archevêque
de Narbonne et grand aumônier de la reine. Il aimait fort sa sœur, et
avait obtenu le gouvernement de Montpellier pour son neveu, à la mort de
son beau-frère. M. du Maine le maria à une fille de M. de Vivonne qui
n'avait rien. Outre l'honneur de l'alliance, il espérait en étayer son
oncle par M. du Maine, gouverneur de Languedoc, fils de la sœur de M. de
Vivonne, contre la persécution de Bâville, intendant, ou plutôt roi de
Languedoc. Cette proximité fit dans la suite, et à distance, le mari
chevalier d'honneur de M\textsuperscript{me} la duchesse d'Orléans, et
la femme sa dame d'atours, qui les aimait fort l'un et l'autre, et
M\textsuperscript{me} de Montespan beaucoup, qui depuis longtemps
n'était plus à la cour. M\textsuperscript{me} de Castries était une
figure de tout point manquée pour la forme et pour la matière, mais tout
âme, tout esprit et charmant, toujours nouveau, et de ce rare chrême des
Mortemart, avec beaucoup de lecture et de savoir sans le montrer jamais.
Le mari s'était fort distingué à la guerre, et y aurait été loin sans un
asthme et une santé fort triste, qui le força à quitter.

Avec une si médiocre place, et un esprit qui ne l'était guère moins, sa
vertu et son mérite lui avaient acquis des amis distingués, et en nombre
et une considération personnelle où peu d'autres sont parvenus. Ils
avaient un seul fils, fort bien fait, et qui promettait beaucoup, dont
ils étaient idolâtres. Ils avaient fort peu de bien\,; ils voulurent le
richement marier. Ils trouvèrent une beauté parfaite avec toutes les
grâces possibles, plus admirable, à ce qu'on disait, d'âme et d'esprit
que de corps\,; car elle parut et passa comme une fleur. L'affaire
conclue, il en fallut parler à M\textsuperscript{me} la duchesse
d'Orléans par respect, étant a elle, mais sans avoir de grâce à lui
demander. Cette princesse qui, comme Minerve, n'avait point de mère, et
ne reconnaissait de parents que ceux de Jupiter, n'avait jamais laissé
apercevoir aux Castries la moindre idée de parenté, quelque amitié,
quelque familiarité, quelque confiance qu'elle eût en eux, et eux de
leur côté auraient commis un crime irrémissible à son égard, s'il leur
en était échappé la moindre apparence. À la mention de ce mariage, elle
se douta pour la première fois qu'il pouvait être que
M\textsuperscript{me} de Castries fût sa cousine germaine, et tout aussi
{[}tôt{]} chausse le cothurne sur l'indigne alliance des Nolent. Ce
n'était pas qu'elle eût un autre parti à leur proposer, moins encore à
leur fournir de quoi prétendre à mieux\,; mais de ce mariage, elle n'en
voulut pas entendre parler, le traita d'offense pour elle, et fit tant
de bruit qu'il en demeura tout court\,; il fallut attendre, et cela dura
six mois. Cependant ce mariage n'en fut point rompu, parce qu'il était
réciproquement désiré. À la fin le duc du Maine et le comte de Toulouse
obtinrent la levée de l'interdit, et le mariage s'acheva. Mais depuis ce
moment, tout fut si dédaigneux de la part de M\textsuperscript{me} la
duchesse d'Orléans que la jeune femme n'osait presque s'y présenter, et
que M. et M\textsuperscript{me} de Castries étaient eux-mêmes fort
empêchés de leurs personnes. Les pauvres jeunes gens ne durèrent guère.
Ce ne fut que par leur mort, qui arriva à quatre jours l'un de l'autre,
que M\textsuperscript{me} la duchesse d'Orléans se rapprocha de M. et de
M\textsuperscript{me} de Castries, qui en pensèrent mourir de douleur,
et ne s'en consolèrent jamais.

Broglio cadet, et qui a fait depuis une si étrange fortune, épousa une
très riche Malouine, qui s'est vue assise veuve, sans l'avoir pu être,
mariée. Car son mari a vu la cour bien peu, maréchal de France, fait
bien bizarrement duc en Bohême, d'où presque aussitôt, il revint perdu,
exilé, et mourut peu après dans cette disgrâce, sans avoir eu permission
d'approcher la cour depuis son retour.

D'Antin maria son second fils à la fille unique de Vertamont, premier
président du grand conseil, riche à millions, et plus avare, s'il se
peut, que riche. Elle manquait de bas et de souliers chez son père, dans
un grenier où elle ne voyait jamais de feu. Ses naïvetés aussi,
quoiqu'elle ne manquât pas d'esprit, et ses surprises de l'abondance et
de la magnificence qu'elle trouva chez d'Antin, furent longuement
divertissantes. Son mari prit le nom de marquis de Bellegarde. En même
temps d'Antin procura à Vertamont le râpé de la charge de greffier de
l'ordre que Lamoignon, président à mortier, vendit à Le Bas de
Montargis, garde du trésor-royal. On cria fort de voir l'ordre sur
Montargis, et cela renouvela contre Crosat. On trouva étrange aussi que
six hommes vivants demeurassent parés du cordon successif de la même
charge, qui étaient\,: La Vrillière, les chanceliers de Pontchartrain et
Voysin, Lamoignon, Vertamont et Montargis. Les trois autres charges
avaient aussi leurs vétérans et leurs râpés, mais non chacune en si
grand nombre.

Le maréchal de Besons maria aussi une de ses filles, belle et bien
faite, à Maubourg, brigadier de cavalerie, et très bon officier, veuf
depuis un an d'une fille de La Vieuville, mari de la dame d'atours de
M\textsuperscript{me} la duchesse de Berry.

Le duc de Melun épousa une fille du duc d'Albret. M\textsuperscript{me}
d'Espinoy, sa mère, mit sa fille dans les Rohan\,; elle était Lorraine,
comme on a vu souvent\,; elle voulait peu à peu poulier son fils à la
principauté que son mari avait toujours eue dans la tête.

Le mariage du fils aîné du duc de Villeroy fut arrêté avec la fille
aînée du prince de Rohan. On a vu plus d'une fois ici ce que toute leur
vie furent l'un à l'autre le maréchal de Villeroy et la duchesse de
Ventadour, grand-père et grand'mère de ce mariage. L'affaire publique et
les compliments reçus, les Rohan crurent que rien ne la pourrait rompre.
Alors ils proposèrent qu'en cas que les mâles, issus du prince de Rohan
ou de son fils, vinssent à manquer, cette fille aînée reçût quelque
légère augmentation de dot, mais que tous les biens de cette branche
passassent à celle de Guéméné, et déclarèrent qu'ils les avaient
substitués de la sorte. Ce n'était pas que le maréchal de Villeroy se
souciât de biens, ni qu'il espérât que cette fille vît mourir tous les
mâles de sa branche, mais il ne voulut pas être la dupe des Rohan, moins
encore leur valet, et faire un mariage avec une condition qui lui sembla
honteuse, et qui ne lui fut déclarée qu'après que tout eut été convenu.
Il rompit donc avec le plus grand éclat. Mais le vieil amour du maréchal
de Villeroy et de la duchesse de Ventadour ne put souffrir un long
divorce. Il remit même peu à peu quelque sorte de bienséance entre les
Rohan et les Villeroy, qui en firent même les avances pour plaire à
M\textsuperscript{me} de Ventadour. Mais ils ne le pardonnèrent jamais
au maréchal de Villeroy, et furent les sourds mais principaux
instigateurs de sa catastrophe. Mais ils s'en cachèrent tant qu'ils
purent, à cause de M\textsuperscript{me} de Ventadour qu'ils avaient un
si grand intérêt de ménager et de gouverner, comme ils ont fait toute sa
vie, et dont le coeur était depuis tant d'années si inséparablement
attaché au maréchal de Villeroy. Il eut bientôt lieu d'être dépiqué par
la figure, le bien et la naissance, en quoi il ne perdit rien aux Rohan.
Six semaines après, il maria son petit-fils à la fille aînée du duc de
Luxembourg.

Les Rohan, de leur côté, ne voulurent pas demeurer en reste. Ils
tonnelèrent aisément le duc de Mazarin, qui consentit à leur
substitution, et le mariage se fit du duc de La Meilleraye, son fils
unique, qui n'avait que quinze ans, un mois après celui du marquis de
Villeroy avec M\textsuperscript{lle} de Luxembourg.

La maréchale de Noailles maria sa huitième et dernière fille au fils de
Courtenvaux, qui devait être très riche. Le duc de Noailles obtint pour
cela du régent que le père cédât à son fils sa charge de capitaine des
Cent-Suisses, et d'en conserver les appointements et la survivance.
Ainsi le maréchal d'Estrées fut beau-frère de tous deux\,: du père, mari
de sa soeur\,; du fils, son neveu, qui épousa la sœur de la maréchale
d'Estrées.

\hypertarget{chapire-xvi.}{%
\chapter{CHAPIRE XVI.}\label{chapire-xvi.}}

1716

~

{\textsc{Je fais donner à La Vrillière voix au conseil de régence.}}
{\textsc{- M. de Châtillon mestre de camp général, et M. de
Clermont-Tonnerre commissaire général de la cavalerie.}} {\textsc{- La
charge de secrétaire d'État de la guerre supprimée\,; celle des affaires
étrangères rétablie sans fonction, donnée à Armenonville, qui en paye
quatre cent mille livres au chancelier Voysin.}} {\textsc{- Les
conseillers d'État prétendent que la place de conseiller d'État est
incompatible avec la charge de secrétaire d'État, et perdent leur procès
contre Armenonville.}} {\textsc{- Avaraye ambassadeur en Suisse, et
Bonac à Constantinople.}} {\textsc{- Maupertuis et Vins, capitaines des
deux compagnies des mousquetaires, se retirent\,; Artagnan et Canillac
leur succèdent.}} {\textsc{- Réforme des troupes.}} {\textsc{- Querelle,
combat, procédure et jugement entre le duc de Richelieu et le comte de
Gacé.}} {\textsc{- Princes du sang, bâtards, pairs.}} {\textsc{- Épées
aux prisons.}} {\textsc{- Querelle et combat entre MM. de Jonzac et de
Villette.}} {\textsc{- Mort de Sourches, ci-devant grand prévôt, et de
Lyonne, premier écuyer de la grande écurie, à qui succède le neveu de
Sainte-Maure.}} {\textsc{- Chambre de justice contre les financiers.}}
{\textsc{- Accident à un oeil de M. le duc d'Orléans.}} {\textsc{-
Payements se commencent.}} {\textsc{- Misère étrange des ministres
employés par la France au dehors.}} {\textsc{- Mortification, puis don,
aussi mal à propos l'un que l'autre, à Desmarets.}} {\textsc{- Cheverny
gouverneur de M. le duc de Chartres ad honores.}} {\textsc{-
M\textsuperscript{me} la duchesse de Berry usurpe des honneurs qu'elle
ne conserve pas.}} {\textsc{- Son démêlé avec M. le prince de Conti.}}
{\textsc{- S'abandonne à Rion.}} {\textsc{- Quel est Rion.}} {\textsc{-
Il la maîtrise fort durement.}} {\textsc{- Contrastes de
M\textsuperscript{me} la duchesse de Berry avec elle-même, et dans le
monde, et aux Carmélites.}} {\textsc{- M\textsuperscript{me} d'Aydie
dame de M\textsuperscript{me} la duchesse de Berry, au lieu de la mère
du marquis de Brancas, qui rend sa place.}}

~

La Vrillière aurait dû être content de son sort, dont il ne s'était pas
tant promis lui-même. Je l'avais sauvé seul du naufrage des secrétaires
d'État, à force de temps et de bras, et je lui avais fait attribuer à
lui seul toutes les fonctions pour lesquelles on ne se pouvait
commodément passer d'un secrétaire d'État, et qui s'étendaient par tout
le royaume pour tous les ordres en commandement, outre le secret et la
direction de la police de Paris. De cinquième roue d'un chariot qu'il
était sous le feu roi, avec une place caponne, car sa charge de
secrétaire d'État n'avait que ses provinces et point de département
particulier, il était devenu un personnage à qui tout le monde avait
affaire. Malgré tant de différence dans la situation nouvelle où il se
trouvait, il avait un ver qui le rongeait, et qui depuis l'expulsion de
Pontchartrain ne lui laissait point de repos, quoique depuis la mort du
roi, jusqu'à sa dernière chute, Pontchartrain fût devenu un simulacre
qu'on ne cessait de bafouer sans cesse et sans mesure. Mais tandis qu'à
ce prix il entrait encore au conseil de régence, comme secrétaire
d'État, où toutefois il n'eut jamais d'autre fonction que de moucher les
bougies, La Vrillière, avec ce pendant d'oreille, n'osa parler de ce qui
le tourmentait. Quand Pontchartrain fut chassé, La Vrillière prit plus
de hardiesse, parce qu'il se trouva seul dans le cas, et bientôt après
vint à moi comme à son protecteur, sur sa privation de voix au conseil
de régence. J'essayai de lui faire entendre raison\,; mais lui et sa
femme revinrent si souvent à la charge, il faut tout dire, pleurèrent
tant chez moi l'un et l'autre, que l'amitié l'emporta en moi sur la
raison. Je parlai au régent qui avait une facilité et un mépris de
toutes choses qui lui en faisait faire litière, quand il n'était pas
retenu par quelqu'un, et j'obtins facilement ce que La Vrillière
regardait lors comme le comble de ses vœux.

La Vrillière vendit alors sa charge de mestre de camp général de la
cavalerie à M. de Châtillon, qui en était commissaire général, et gendre
de Voysin, qui a fait depuis une fortune si grande et si peu espérée,
dont l'extrême brillant s'est enfin changé en de tristes ténèbres. Il
vendit la sienne au marquis de Clermont-Tonnerre.

Je m'impatientais de ce que le chancelier ne se défaisait point de sa
charge de secrétaire d'État de la guerre, dont il ne faisait plus aucune
fonction depuis l'établissement des conseils. C'était la condition sous
laquelle le maréchal de Villeroy avait dans les derniers jours de la vie
du roi arraché pour lui la conservation des sceaux, comme je l'ai
raconté en son lieu, de la misère de M. le duc d'Orléans\,; car c'est le
terme qui convient à une telle faiblesse. Je pressais le régent de finir
cela, et à la fin, j'en vins à bout. Armenonville dont j'ai parlé plus
d'une fois, et duquel j'avais eu lieu d'être content toute ma vie, me
vint demander instamment de le servir pour obtenir ce qui n'était plus
qu'une carcasse inanimée de charge, mais qui pouvait se relever, et
passer à son fils. Voysin, qui, jusqu'au dernier moment du roi, ne
s'était pas oublié, en avait obtenu tout à la fin de sa vie un brevet de
retenue de quatre cent mille livres sur cette charge, et par la
condition obtenue par le maréchal de Villeroy, en lui faisant conserver
les sceaux, il fallait que la charge fût vendue. J'obtins donc
l'agrément pour Armenonville, qui fut pourvu de celle dont Torcy avait
été récompensé en s'en démettant, et donna quatre cent mille livres au
chancelier Voysin, qui fut enragé encore, parce qu'il avait trouvé à la
vendre le double. La sienne demeura supprimée en entier, et celle des
affaires étrangères n'eut aucune sorte de fonction.

Cette affaire fit naître une ridicule prétention. Armenonville était si
avancé dans le conseil qu'il touchait presque au décanat\,; ce décanat
emporte honneur et profit. Armenonville était d'âge et de santé à en
jouir longtemps, et ce n'était pas l'intérêt de ceux qui avaient envie
d'y parvenir. Les anciens conseillers d'État imaginèrent une
incompatibilité dans les deux places dont il était revêtu, et peu à peu
la persuadèrent aux autres conseillers d'État. Ils citaient des exemples
vrais et faux là-dessus dont pas un ne faisait au fond de la chose. Il
est vrai que les secrétaires d'État et le contrôleur général des
finances étaient si supérieurs en considération, en fonctions, en
autorité aux conseillers d'État, qui ne jugent que des procès, que ceux
d'entre eux qui sous le feu roi avaient été pris d'entre les conseillers
d'État pour remplir ces grandes places, s'étaient démis de celle de
conseillers d'État. Cela même était d'autant plus raisonnable que le
service du conseil le demandait, parce qu'il n'y a que vingt-quatre
conseillers d'État de robe, dont il y en a toujours intendants dans les
grandes provinces, intendants des finances souvent, prévôts des
marchands, dont l'absence des bureaux et du conseil retarde
l'expédition, et nuit souvent aux affaires. Un conseiller d'État, devenu
secrétaire d'État ou contrôleur général, était encore de moins au
conseil où il n'avait plus le temps de vaquer, et de plus cette place
n'était pour lui d'aucune ressource, parce que, venant à déplaire assez
pour perdre la principale, il ne se serait pas réduit à retourner faire
le simple conseiller d'État au conseil, et à devenir, comme on dit,
d'évêque meunier. Il était faux que M. de Croissy, président à mortier
au parlement de Paris, quand il fut secrétaire d'État à la place de M.
de Pomponne, se fût défait de sa charge de président à mortier. M. de
Croissy eut la charge de président à mortier en\ldots{}\footnote{{[}27{]}}
de M\ldots.\footnote{{[}28{]}}, fut en 1679 secrétaire d'État, eut
en\ldots.\footnote{{[}29{]}} la survivance de sa charge de président à
mortier pour M. de Torcy son fils.

En 1689, le roi ordonna au premier président de Novion de donner la
démission de sa charge, moyennant une charge de président à mortier pour
son petit-fils, M. de Novion, qui, après la régence, a été premier
président. M. de Croissy lui vendit sa charge de président à mortier, et
M. de Torcy, qui en avait la survivance, eut en la place celle de
secrétaire d'État de M. de Croissy. Or un secrétaire d'État des affaires
étrangères, par ses occupations, et par être nécessairement toujours à
la cour et jamais à Paris, est bien moins compatible avec les fonctions
journalières de président à mortier que ne le sont les places de
secrétaire d'État et de conseiller d'État. Si de là on passe à l'être de
ces places, il se trouve que l'être de secrétaire d'État est {[}celui{]}
de conseiller d'État. La charge de secrétaire d'État lui en donne le
titre, l'entrée et la voix au conseil, le rang d'ancienneté partout
parmi les conseillers d'État du jour qu'il a été secrétaire d'État, et
comme secrétaire d'État a rang de conseiller d'État, et n'en a point
d'autre. Si par la puissance de leurs charges ils ont regardé les places
de conseiller d'État au-dessous d'eux, c'est une idée qui a pu entrer
dans leur tête, mais qui n'a pas changé l'essence de leurs charges et de
leur condition, qui, par ce qui vient d'être expliqué, est homogène aux
places de conseillers d'État, et ne peut être incompatible avec elles.
Aussi les conseillers d'État eurent-ils beau s'assembler, députer au
régent, présenter des mémoires imprimés, solliciter les membres du
conseil de régence, et l'ancien évêque de Troyes chargé par le régent
d'y rapporter l'affaire, bien défendue par Armenonville, ce dernier y
gagna son procès tout d'une voix. Comme sa nouvelle charge ne lui
donnait aucune occupation, il continua ses fonctions de conseiller
d'État comme auparavant, et devint doyen du conseil. Nous lui verrons
donner les sceaux dans la suite, avec lesquels il ne mourut pas.

Avaray, bon militaire et rien de plus, fut choisi pour l'ambassade de
Suisse, et Bonac pour celle de Constantinople. C'était un neveu paternel
de Bonrepos, qui avait eu l'honneur d'épouser la fille aînée de Biron, à
la vérité fort chargé d'enfants, et pour rien. Il avait de l'esprit, de
l'expérience, et de la capacité dans les négociations, où il avait passé
sa vie, alors assez peu avancée. On l'avait employé de bonne heure en
Allemagne, puis dans le Nord, et en Pologne longtemps, enfin en Espagne,
et on avait eu lieu partout d'en être content. L'emploi délicat, mais
fort lucratif de Constantinople, parut tout à la fois une dot et une
récompense pour lui.

Artagnan, qui depuis longtemps commandait les mousquetaires gris sous
Maupertuis qui avait plus de quatre-vingts ans, et qui ne s'en mêlait
presque plus, lui donna cent cinquante mille livres et en fut capitaine
à sa place. Trois mois après, Canillac, cousin de celui qui était dans
le conseil des affaires étrangères, et qui commandait les mousquetaires
noirs sous M. de Vins, qui n'était guère moins vieux que Maupertuis, et
qui désirait fort de se retirer, lui donna aussi cent cinquante mille
livres, et fut capitaine à sa place. Ce fut la première fois qu'on est
monté à ces compagnies pour de l'argent. Il est vrai que si on n'eût eu
égard qu'au mérite, Maupertuis et Vins n'auraient pas eu de tels
successeurs.

Après bien des projets différents, on fit enfin la réforme des troupes.
On ne conserva que cent cinquante escadrons de cavalerie à cent maîtres
chacun, sans majors ni aumôniers, et les dix-sept escadrons de la maison
du roi et de la gendarmerie, de laquelle les compagnies furent réduites
de soixante à trente-cinq maîtres. On conserva aussi les quatorze
régiments de dragons à un escadron chacun, dont la moitié à pied. Le
tiers des Suisses fut réformé, en sorte que des dix-huit mille hommes on
n'en conserva que douze mille en ôtant une compagnie par régiment\,; et
les régiments sur le pied étranger, excepté les Suisses à qui leurs
capitulations furent conservées, et les Irlandais, on les mit sur le
pied français infiniment moins cher, en donnant à leurs colonels huit
mille livres de pension, en dédommagement de ce qu'ils y perdirent.

Il y eut force bals dans Paris, outre ceux de l'Opéra. Il arriva en l'un
de ces derniers une querelle entre le duc de Richelieu et le comte de
Gacé, fils aîné du maréchal de Matignon. Ils sortirent, se battirent
dans la rue de Richelieu et se blessèrent légèrement tous deux. Le
parlement, certain de la faiblesse du régent, et de la misère des ducs à
qui il ne pardonnait point de ne pas essuyer toutes ses usurpations avec
le dernier respect, se promit bien de profiter du temps et de
l'aventure, et sans lettres patentes, comme il est de l'ordre, du droit
et de l'usage, se mit à informer, sous prétexte que M. de Richelieu
n'était pas reçu au parlement, comme s'il était moins pair de France
faute de cette réception, après celle de son père. Il y eut en bref un
ajournement personnel, et se rendre dans quinzaine à la conciergerie du
Palais, avant l'expiration duquel M. le duc d'Orléans les envoya à la
Bastille. Ce nonobstant, le parlement leur fit signifier en leurs
domiciles l'ajournement personnel, et de se rendre à la Conciergerie.
Ces messieurs furent fort visités à la Bastille. Cette prétendue
noblesse excitée par M. et M\textsuperscript{me} du Maine, dont on a
parlé en son temps, fermentait toujours, et trouva fort mauvais que les
ducs qui allaient voir les deux prisonniers à la Bastille gardassent
leurs épées, et qu'ils fussent obligés de laisser les leurs à la porte.
Grand bruit, à leur ordinaire\,; mais de ce bruit il n'en fut autre
chose sinon que le régent qui savait bien ce qui en était et devait
être, eut la complaisance de faire perquisition de l'usage, qui se
trouva tel qu'il se pratiquait et que cette prétendue noblesse s'en
plaignait. Ainsi elle continua à laisser les épées à la porte de la
Bastille, et les ducs à la conserver en entrant dans cette prison et
dans toutes les autres où ils vont voir quelqu'un, comme du temps du feu
roi il m'est arrivé au For-l'Évêque, sans qu'on y ait songé à me parler
de quitter mon épée, ce que je n'aurais pas souffert aussi.

Le régent, qui se plaisait aux \emph{mezzo-termine} favorables à sa
faiblesse et à son goût politique d'abaissement et de confusion, et de
tenir tout brouillé, laissa faire le parlement, et fit seulement écrire
une lettre du roi à chaque prince du sang, bâtard, et autre pair pour se
trouver au jugement du duc de Richelieu. Les princes du sang furent
piqués de ce que cette qualité se trouva également mise à la suscription
de leurs lettres et de celles des bâtards. M. le Duc, M. le prince de
Conti et le duc du Maine déclarèrent qu'ils n'iraient point au jugement
du duc de Richelieu comme étant ses parents trop proches. Ce fut une
défaite que le régent leur suggéra pour éviter noise. Les princes du
sang s'étaient vantés qu'ils empêcheraient les bâtards de traverser le
parquet, et quand ce fut à l'exécution, ils se trouvèrent encore plus
contents de cette raison d'en éviter l'occasion, que ne le fut le régent
même qui la leur fournit. Le prince de Dombes et le comte de Toulouse
s'y trouvèrent avec les autres pairs. Le parlement, ne pouvant pis après
tout ce qu'il avait entrepris et usurpé dans cette affaire, ordonna un
plus amplement informer, et garder prison deux mois.

Quand le jour du jugement définitif s'approcha, il fut dit que le roi
n'écrirait qu'aux pairs, et point aux princes du sang, ni à MM. du Maine
et de Dombes comme exclus par leur parenté. M. de Dombes y avait
pourtant assisté une fois, mais on prit ce milieu pour faire en sorte
que le comte de Toulouse se laissât persuader de n'y point aller, et
d'avoir cette déférence pour les plaintes amères que M. le Duc avait
faites, et continuait de porter au régent de ce que le prince de Dombes
et lui s'étaient trouvés à la dernière séance. Le prince de Dombes se
voulait bien exclure de celle-ci comme parent, ainsi que son père, par
M\textsuperscript{me} la duchesse du Maine. Mais le comte de Toulouse,
qui n'avait point cette raison, persista à s'y vouloir trouver. Ainsi
fit-il, et traversa le parquet. Les pairs tous convoqués par le roi y
assistèrent. Il y eut arrêt de plus amplement informer pendant trois
mois, et, cependant mis en liberté. Ils sortirent le même jour de la
Bastille\,; il y avait six mois que cela durait. J'ai cru devoir
rapporter cette affaire tout de suite.

Dans ce même temps de la querelle du duc de Richelieu et du comte de
Gacé, il y eut un badinage de rien entre deux jeunes gens ivres à souper
chez M. le prince de Conti à Paris, à quoi eux-mêmes ni personne n'eut
pris garde sans la malice des convives, excités par l'exemple du maître
de la maison, qui leur apprit le lendemain qu'ils avaient eu une affaire
la veille, et qui voulut faire semblant de les accommoder. L'un était
Jonzac, fils d'Aubeterre, l'autre Villette, frère de père de
M\textsuperscript{me} de Caylus. M. le Duc, qui ne voulut pas que les
maréchaux de France se mêlassent d'une affaire arrivée chez M. le prince
de Conti, les envoya chercher deux jours après et les accommoda. Mais
ceux qui de rien avaient fait une affaire se mirent si fort après eux,
que les familles s'en mêlèrent et les crurent déshonorés s'ils ne se
battaient pas. Tous deux y résistèrent\,; mais enfin poussés à bout, ils
se battirent en fort braves gens, et montrèrent ainsi que leur
résistance ne venait que de ne savoir pourquoi se battre. Tous deux
furent blessés, Villette plus considérablement, et disparurent. Ce fut
le premier fruit de l'impunité effective du premier duel de la régence,
sur le quai des Tuileries, en plein jour, de la plus grande notoriété,
entre deux hommes qui ne valaient pas, en quoi que ce fût, la peine
d'être ménagés, et qui en produisit bien d'autres. L'affaire dont je
viens de parler avait trop éclaté et trop longtemps pour pouvoir être
étouffée. Le parlement procéda, Villette sortit du royaume et mourut
bientôt après\,; Jonzac se cacha longtemps, et ne se présenta que bien
sûr de ce qui arriverait de son affaire. Il en fut quitte pour une assez
longue prison, absous après, et ne perdit point son emploi. Cette
affaire pourtant réveilla celle de Girardin et de Ferrant, qui furent
obligés de s'absenter, et qui à la fin furent condamnés, effigiés, et
perdirent leurs emplois. Ce fut un remède qui vint beaucoup trop tard.

Deux hommes, qui étaient devenus fort inutiles au monde, moururent en ce
même temps\,: Sourches, fort vieux, qui avait cédé à son fils sa charge
de grand prévôt, et Lyonne, premier écuyer de la grande écurie, qui
n'avait jamais exercé cette charge, et qui passait sa très obscure vie
avec les nouvellistes de Paris. Sainte-Maure crut faire merveilles de
faire prendre cette charge à son neveu. Ce n'en était pas une pour un
homme de sa qualité, mais il y brilla aussi peu que son prédécesseur.

Le duc de Noailles et Rouillé voulurent absolument une chambre de
justice contre les financiers. On a vu ce que j'avais pensé là-dessus\,;
mais ces deux hommes étaient maîtres absolus de ce qui était finance\,;
cela passa donc au conseil de régence. Lamoignon et Portail, présidents
à mortier, furent mis à la tête de six maîtres des requêtes, dix
conseillers du parlement, huit maîtres des comptes, et quatre
conseillers de la cour des aides. Fourqueux, neveu de Rouillé, et
procureur général de la chambre des comptes, fut procureur général de ce
nouveau tribunal. Portail et lui y acquirent beaucoup de réputation par
leur intégrité\,; Lamoignon y gagna de l'argent et se déshonora. L'édit
de cette création fut enregistré tel qu'il fut présenté au parlement le
12 mars, et le chancelier alla le 14 mars faire l'ouverture de ce
nouveau tribunal aux Grands-Augustins, où il tint ses séances. La
frayeur se mit parmi les financiers. On prétendait que les traitants
avaient profité de dix-huit cent millions. Parmi les assignations qui
furent données à ceux qu'on voulut ressasser, le duc de Noailles
n'oublia pas M. d'Auneuil, maître des requêtes, frère de
M\textsuperscript{me} la maréchale de Lorges, dont le père était entré
en plusieurs affaires du temps de M. Colbert, avait été depuis garde du
trésor royal avec autant de bonne réputation que ces gens-là en peuvent
avoir, et avait longtemps avant sa mort quitté sa charge et toute
affaire, et entièrement apuré ses comptes à la chambre des comptes. Dès
que j'appris cette malice, j'allai trouver M. le duc d'Orléans, qui
sur-le-champ et devant moi envoya ordre au duc de Noailles de retirer
cette assignation et de la lui apporter. Il eut un peu la tête lavée,
tout favori qu'il était, avec défense de toucher à d'Auneuil en quoi que
ce pût être, et l'assignation bien déchirée. Ils avaient tous bien envie
d'attaquer Pontchartrain, et M. le duc d'Orléans aussi\,; mais la
considération de son père borna ce dessein aux désirs et aux regrets\,;
M. le duc d'Orléans s'y porta de lui-même. Je n'eus ni la peine ni le
mérite de parer ce coup.

Ce prince, qui avait la vue fort basse et un œil bien moins mauvais que
l'autre, jouant à la paume qu'il aimait fort en ce temps-ci, se donna
sur ce bon œil un coup de raquette qui le mit en danger de le perdre.
Mais s'il le conserva il n'en fut guère mieux\,; il n'en vit presque
plus le reste de sa vie\,; et le mauvais œil, dont il se servait le
moins, devint le bon, sans en être meilleur qu'il n'était.

Il commença à faire faire des payements. Ce qu'il y avait de plus pressé
étaient les ministres de France dans les pays étrangers. Ils étaient
tellement en arrière qu'il y en avait plusieurs qui, depuis plusieurs
mois, n'avaient pas de quoi retirer leurs lettres de la poste et les y
laissaient. On comprend l'inconvénient de cette misère pour les
affaires, et par le mépris où ils ne pouvaient éviter de tomber dans les
divers pays où ils étaient employés, et où ils mouraient de faim, après
s'être endettés partout. Ce fut aussi par où on commença. On donna aussi
quelque chose à la marine, qui était depuis longtemps encore pis qu'à
sec, moins pour la relever, comme je l'expliquerai bientôt, que pour
apaiser un peu le comte de Toulouse et le conseil de marine.

Les délations portées à la chambre de justice attirèrent une
mortification à Desmarets, et un ridicule à qui la lui donna. On se
persuada sur ses rapports qu'il avait caché beaucoup d'argent dans
l'abbaye d'Hières près Paris, dont sa soeur était abbesse. On y envoya
fouiller partout, et on y remua bien la terre\,; on n'y trouva rien du
tout. Le rare est qu'aussitôt après le maréchal de Villeroy, ami de
Desmarets de tout temps, fit valoir au régent une prétendue promesse du
feu roi à Desmarets de lui donner cent mille écus au prochain
renouvellement des fermes générales. Le roi était mort auparavant, et
Desmarets avait été chassé. Dans l'extrême disette où on était d'argent,
dont on avait besoin pour tant de choses également importantes et
pressées, et le régent par aucun coin tenu d'acquitter de pareilles
grâces du feu roi, il eut la faiblesse de se laisser entraîner aux
propos du maréchal de Villeroy, et de faire payer Desmarets de ce don à
mille pistoles par mois.

Ce prince choisit Cheverny pour gouverneur de M. son fils. Il était
homme de qualité et fort capable de faire quelque chose de bon d'un
pupille qui lui aurait été sérieusement remis. Mais il avait depuis
longtemps de Court dont le nom n'était point faux, et qui de plus était
un pédant achevé. Son frère avait toujours été au duc du Maine, et y
était mort. C'en était assez pour avoir toute la confiance de
M\textsuperscript{me} la duchesse d'Orléans, qui n'avait d'yeux que pour
ses frères, et qui de préférence à tout voulait inculquer à son fils sa
manie là-dessus. Ainsi Cheverny ne fut mis que \emph{ad honores}, ravi
de n'en avoir ni les soins ni la peine, et qui laissa faire de Court
sans se mêler de rien. M. le duc d'Orléans, partie connaissance de ce
qu'il avait à espérer de M. son fils, partie négligence, laissa faire.
M\textsuperscript{me} la duchesse d'Orléans réussit à la vérité
parfaitement à coiffer son fils de la bâtardise. Du reste on voit
comment cette éducation a réussi.

Le roi sortit pour la première fois des Tuileries pour aller au
Palais-Royal voir Madame, M. {[}le duc{]} et M\textsuperscript{me} la
duchesse d'Orléans. Quelque temps après il sortit pour la seconde fois,
et alla voir M\textsuperscript{me} la duchesse de Berry au Luxembourg.
Les prétentions et l'indécision firent ôter le strapontin de son
carrosse pour n'y laisser que les deux fonds. Le roi était étouffé au
derrière par M\textsuperscript{me} de Ventadour et le duc du Maine. Au
devant ses deux fils et M\textsuperscript{me} de Villefort,
sous-gouvernante\,; c'est-à-dire toutes personnes sans droit aucun d'y
être, excepté la duchesse de Ventadour. J'ai expliqué ailleurs les deux
règles des places du carrosse, celle de droiture et celle de nécessité,
mais la confusion sur tout était uniquement en règne, et s'y établit de
plus en plus.

M\textsuperscript{me} la duchesse de Berry en profitait de son côté pour
usurper tous les honneurs de reine, malgré les représentations de
M\textsuperscript{me} de Saint-Simon, et les dégoûts dont elle l'assura
que de telles entreprises seraient suivies. Elle marcha dans Paris avec
des timbales sonnantes, et tout du long du quai des Tuileries où le roi
était. Le maréchal de Villeroy en porta le lendemain ses plaintes à M.
le duc d'Orléans, qui lui promit que tant que le roi serait dans Paris,
on n'y entendrait d'autres timbales que les siennes, et oncques depuis
M\textsuperscript{me} la duchesse de Berry n'y en a eu. Elle alla aussi
à la comédie, y eut un dais dans sa loge, quatre de ses gardes sur le
théâtre, d'autre, dans le parterre, la salle bien plus éclairée qu'à
l'ordinaire, et fut avant la comédie haranguée par les comédiens. Cela
fit un étrange bruit dans Paris comme avait fait son haut dais au
parterre de l'Opéra. Néanmoins elle n'osa retourner aux spectacles de la
sorte\,; mais pour ne pas reculer aussi, elle renonça à voir la comédie
dans son lieu ordinaire, et elle prit à l'Opéra une petite loge où elle
n'était qu'à peine aperçue, et comme incognito. Elle ne le vit plus
ailleurs, et comme la comédie venait jouer sur le théâtre de l'Opéra
pour Madame, cette petite loge servit pour les deux spectacles.

Allant un jour à l'Opéra, ses gardes firent arrêter le carrosse de M. le
prince de Conti qui y arrivait, et maltraitèrent son cocher, ce prince
étant dans son carrosse. La vérité est que ce n'était qu'entreprises de
toutes parts. Les princes du sang n'osaient pas nier tout à fait leur
devoir d'arrêter devant les filles de France, car il n'y avait point de
fils de France alors, mais ils les évitaient et de fait ne voulaient
point arrêter devant elles\,; d'autre part, c'était bien assez de le
faire arrêter de haute lutte, sans maltraiter son cocher, lui dans son
carrosse\,; Il s'en plaignit au marquis de La Rochefoucauld, capitaine
des garde de M\textsuperscript{me} la duchesse de Berry, qui n'eut pas
l'esprit de lui répondre de manière à le contenter, et à faire tomber la
chose\,; M. le prince de Conti, piqué, s'adressa à M. le duc d'Orléans,
qui obligea M\textsuperscript{me} la duchesse de Berry de le prier de
venir chez elle. Il y vint\,; la conversation se passa en public fort
mal à propos, et pour en dire le vrai, avec tout son esprit, elle s'en
tira fort mal\,; elle fit des reproches à ce prince de ne s'être pas
adressé à elle\,; elle voulut accuser le cocher et excuser son garde,
puis voyant qu'elle ne réussissait pas, et que M. le duc d'Orléans
voulait être obéi, elle dit à M. le prince de Conti que, puisqu'il
voulait que ce garde allât en prison, il y irait, mais qu'elle le priait
qu'il n'y fût guère. Cela fut pitoyable. En effet, à peine le garde se
fut-il remis qu'il sortit à la prière de M. le prince de Conti. Le point
était qu'on l'avait fait arrêter, qu'il n'osait le contester ni s'en
plaindre. Voilà pour le rang à couvert et bien décidé\,; le reste était
une sottise dont il fallait savoir sortir galamment.

Après maintes passades, elle s'était tout de bon éprise de Rion, jeune
cadet de la maison d'Aydie, fils d'une sœur de M\textsuperscript{me} de
Biron, qui n'avait ni figure ni esprit. C'était un gros garçon court,
joufflu, pâle, qui avec force bourgeons ne ressemblait pas mal à un
abcès. Il avait de belles dents, et n'avait pas imaginé causer une
passion qui en moins de rien devint effrénée, et qui dura toujours, sans
néanmoins empêcher les passades et les goûts de traverse. Il n'avait
rien vaillant, mais force frères et soeurs qui n'en avaient guère
davantage. M. et M\textsuperscript{me} de Pons, dame d'atours de
M\textsuperscript{me} la duchesse de Berry, étaient de leurs parents, et
de même province. Ils firent venir ce jeune homme, qui était lieutenant
de dragons, pour tâcher d'en faire quelque chose. À peine fut-il arrivé
que le goût se déclara, et qu'il devint le maître à
Luxembourg\footnote{}. M. de Lauzun, dont il était petit-neveu, en riait
sous cape. Il était ravi\,; il se croyait renaître en lui à Luxembourg,
du temps de Mademoiselle\,; il lui donnait des instructions.

Rion était doux et naturellement poli et respectueux, bon et honnête
garçon. Il sentit bientôt le pouvoir de ses charmes qui ne pouvaient
captiver que l'incompréhensible fantaisie dépravée d'une princesse. Il
n'en abusa avec personne, et se fit aimer de tout le monde par ses
manières, mais il traita M\textsuperscript{me} la duchesse de Berry
comme M. de Lauzun avait traité Mademoiselle. Il fut bientôt paré des
plus belles dentelles et des plus riches habits, plein d'argent, de
boîtes, de joyaux et de pierreries. Il se faisait désirer\,; il se
plaisait à donner de la jalousie à sa princesse, à en paraître lui-même
encore plus jaloux. Il la faisait pleurer souvent. Peu à peu il la mit
sur le pied de n'oser rien faire sans sa permission, non pas même les
choses les plus indifférentes. Tantôt prête de sortir pour l'Opéra, il
la faisait demeurer\,; d'autres fois il l'y faisait aller malgré elle.
Il l'obligeait à faire bien à des dames qu'elle n'aimait point, ou dont
elle était jalouse, mal à des gens qui lui plaisaient, et dont il
faisait le jaloux. Jusqu'à sa parure, elle n'avait pas la moindre
liberté. Il se divertissait à la faire décoiffer ou lui faire changer
d'habits quand elle était toute prête, e cela si souvent, et quelquefois
si publiquement qu'il l'avait accoutumée à prendre le soir ses ordres
pour la parure et l'occupation du lendemain, et le lendemain il
changeait tout, et la princesse pleurait tant et plus. Enfin elle en
était venue à lui envoyer des messages par des valets affidés\,; car il
logea presque en arrivant au Luxembourg\,; et ses messages se
réitéraient plusieurs fois pendant sa toilette, pour savoir quels rubans
elle mettrait\,; ainsi de l'habit et des autres parures, et presque
toujours il lui faisait porter ce qu'elle ne voulait point. Si
quelquefois elle osait se licencier à la moindre chose sans son congé,
il la traitait comme une servant, et les pleurs duraient quelquefois
plusieurs jours. Cette princesse si superbe, et qui se plaisait tant à
montrer et à exercer le plus démesuré orgueil, s'avilit à faire des
repas avec lui et des gens obscurs, elle avec qui nul homme ne pouvait
manger s'il n'était prince du sang.

Un jésuite, qui s'appelait le P. Riglet, qu'elle avait connu enfant, et
qui l'avait toujours cultivée depuis, était admis dans ces repas
particuliers sans qu'il en eût honte, ni que M\textsuperscript{me} la
duchesse de Berry en fût embarrassée. M\textsuperscript{me} de Mouchy,
dont j'ai parlé ailleurs, était la confidente de tous ces étranges
particuliers\,; elle et Rion mandaient les convives, et choisissaient
les jours. La Mouchy raccommodait souvent sa princesse avec son amant,
qui en était mieux traitée qu'elle, sans qu'elle osât s'en apercevoir,
de crainte d'un éclat qui lui aurait fait perdre un amant si cher, et
une confidente si nécessaire. Cette vie était publique\,: tout à
Luxembourg s'adressait à M. de Rion, qui de sa part avait grand soin d'y
bien vivre avec tout le monde, même avec un air de respect qu'il
refusait, même en public, à sa seule princesse. Il lui faisait devant le
monde des réponses brusques qui faisaient baisser les yeux aux
spectateurs, et rougir ceux de M\textsuperscript{me} la duchesse de
Berry, qui ne contraignait point ses manières soumises et passionnées
pour lui devant les compagnies. Le rare est que, parmi cette vie, elle
prit un appartement aux Carmélites du faubourg Saint-Germain, où elle
allait quelquefois les après-dînées, et toujours coucher aux bonnes
fêtes, et souvent y demeurer plusieurs jours de suite. Elle n'y menait
que deux dames, rarement trois, presque point de domestiques\,; elle
mangeait avec ses dames de ce que le couvent lui apprêtait, allait au
chœur ou dans une tribune à tous les offices du jour, et fort souvent de
la nuit\,; et outre les offices, elle y demeurait quelquefois longtemps
en prières, et y jeûnait très exactement les jours d'obligation.

Deux carmélites de beaucoup d'esprit, et qui connaissaient le monde,
étaient chargées de la recevoir et d'être souvent auprès d'elle. Il y en
avait une fort belle\,; l'autre l'avait été aussi. Elles étaient assez
jeunes, surtout la plus belle, mais d'excellentes religieuses, et des
saintes, qui faisaient cette fonction fort malgré elles. Quand elles
furent devenues plus familières, elles parlèrent franchement à la
princesse, et lui dirent que, si elles ne savaient rien que ce qu'elles
en voyaient, elles l'admireraient comme une sainte\,; mais que
d'ailleurs elles apprenaient qu'elle menait une étrange vie, et si
publique, qu'elles ne comprenaient pas ce qu'elle venait faire dans leur
couvent. M\textsuperscript{me} la duchesse de Berry riait et ne s'en
fâchait point\,; Quelquefois elles la chapitraient, lui nommaient les
gens et les choses par leurs noms, l'exhortaient à changer une vie si
scandaleuse, et, avec esprit et tour, poussaient ou enrayaient à propos,
mais jamais sans lui avoir parlé ferme. Elles le contaient après à
celles de ses dames qui étaient les plus propres à goûter leurs peines
sur l'état de M\textsuperscript{me} la duchesse de Berry, qui ne cessa
de vivre comme elle faisait à Luxembourg et aux Carmélites, et de
laisser admirer un contraste aussi surprenant, et qui du côté de la
débauche augmenta toujours. Rion lui fit venir de sa province une de ses
soeurs, mariée à M. d'Aydie, pour remplir la place de
M\textsuperscript{me} de Brancas la mère, de laquelle j'ai quelquefois
fait mention, à qui le feu roi avait donné une place de dame auprès
d'elle, et qui était toujours demeurée en Provence, où elle était
retournée quand elle y fut nommée, et finalement n'en voulut point
revenir.

\hypertarget{chapitre-xvii.}{%
\chapter{CHAPITRE XVII.}\label{chapitre-xvii.}}

1716

~

{\textsc{Vie, journées et conduite personnelle de M. le duc d'Orléans.}}
{\textsc{- Le régent impénétrable sur les affaires dans la débauche,
même dans l'ivresse.}} {\textsc{- Ses maîtresses.}} {\textsc{- Roués de
M. le duc d'Orléans.}} {\textsc{- Énormités ecclésiastiques.}}
{\textsc{- Démêlé des cours de Rome et de Turin sur le tribunal de la
monarchie de Sicile.}} {\textsc{- Naissance de don Carlos, roi des
Deux-Siciles.}} {\textsc{- Prince palatin électeur de Trèves.}}
{\textsc{- Cabale qui, par intérêts particuliers, attache pour toujours
le régent à l'Angleterre.}} {\textsc{- M. le duc d'Orléans n'a jamais
désiré la couronne, mais le règne du roi et par lui-même.}} {\textsc{-
Je propose au régent l'indissoluble et perpétuelle union avec l'Espagne,
comme le véritable intérêt de l'État, dont la maison d'Autriche et les
Anglais sont les ennemis essentiellement naturels.}} {\textsc{-
Stralsund pris.}} {\textsc{- Le roi de Suède échappé et passé en
Suède.}}

~

M\textsuperscript{me} la duchesse de Berry rendait avec usure à M. son
père les rudesses et l'autorité qu'elle éprouvait de Rion, sans que la
faiblesse de ce prince en eut moins d'assiduité, de complaisance, il
faut le dire, de soumission et de crainte pour elle. Il était désolé du
règne public de Rion et du scandale de sa fille, mais il n'osait en
souffler, et si quelquefois quelque scène également forte et ridicule
entre l'amant et la princesse avait percé en public, M. le duc d'Orléans
osait en faire quelque représentation, il était traité comme un nègre,
boudé plusieurs jours, et bien empêché comment faire sa paix. Il n'y
avait jour qu'ils ne se vissent, le plus souvent au Luxembourg. Il est
temps de parler un peu des occupations publiques et particulières du
régent, de sa conduite, de ses parties, de ses journées.

Toutes les matinées étaient livrées aux affaires, et les différentes
sortes d'affaires avaient leurs jours et leurs heures. Il les commençait
seul avant de s'habiller, voyait du monde à son lever, qui était court
et toujours précédé et suivi d'audiences auxquelles il perdait beaucoup
de temps\,; puis ceux qui étaient chargés plus directement d'affaires le
tenaient successivement jusqu'à deux heures après midi. Ceux-là étaient
les chefs des conseils, La Vrillière, bientôt après Le Blanc dont il se
servait pour beaucoup d'espionnages, ceux avec qui il travaillait sur
les affaires de la constitution, celles du parlement, d'autres qui
survenaient\,; souvent Torcy pour les lettres de la poste\,; quelquefois
le maréchal de Villeroy pour piaffer\,; une fois la semaine, les
ministres étrangers\,; quelquefois les conseils\,; la messe dans sa
chapelle en particulier, quand il était fête ou dimanche. Les premiers
temps il se levait matin\,; ce qui se ralentit peu à peu, et devint
après incertain et tardif, suivant qu'il s'était couché. Sur les deux
heures ou deux heures et demie, tout le monde lui voyait prendre du
chocolat\,; il causait avec la compagnie. Cela durait selon qu'elle lui
plaisait\,; le plus ordinaire en tout n'allait pas à demi-heure. Il
rentrait et donnait audience à des dames et à des hommes, allait chez
M\textsuperscript{me} la duchesse d'Orléans, puis travaillait avec
quelqu'un ou allait au conseil de régence\,; quelquefois il allait voir
le roi, le matin rarement, mais toujours matin ou soir, avant ou après
le conseil de régence, et l'abordait, lui parlait, le quittait avec des
révérences et un air de respect qui faisait plaisir à voir, au roi
lui-même, et qui apprenait à vivre à tout le monde.

Après le conseil, ou sur les cinq heures du soir, s'il n'y en avait
point, il n'était plus question d'affaires\,; c'était l'Opéra ou
Luxembourg, s'il n'y avait été avant son chocolat, ou aller chez
M\textsuperscript{me} la duchesse d'Orléans où quelquefois il soupait,
ou sortir par ses derrières, ou faire entrer compagnie par les mêmes
derrières, ou si c'était en belle saison, aller à Saint-Cloud ou en
d'autres campagnes, tantôt y souper, tantôt à Luxembourg ou chez lui.
Quand Madame était à Paris, il la voyait un moment avant sa messe\,; et
quand elle était à Saint-Cloud, il allait l'y voir, et lui a toujours
rendu beaucoup de soins et de respect.

Ses soupers étaient toujours en compagnie fort étrange. Ses maîtresses,
quelquefois une fille de l'Opéra, souvent M\textsuperscript{me} la
duchesse de Berry, et une douzaine d'hommes, tantôt les uns, tantôt les
autres, que sans façon il ne nommait jamais autrement que ses
\emph{roués}. C'était Broglio, l'aîné de celui qui est mort maréchal de
France et duc\,; Nocé\,; quatre ou cinq de ses officiers, non des
premiers\,; le duc de Brancas, Biron, Canillac, quelques jeunes gens de
traverse, et quelques dames de moyenne vertu, mais du monde\,; quelques
gens obscurs encore sans nom, brillant par leur esprit ou leur débauche.
La chère exquise s'apprêtait dans des endroits faits exprès, de
plain-pied, dont tous les ustensiles étaient d'argent\,; eux-mêmes
mettaient souvent la main à l'œuvre avec les cuisiniers. C'était en ces
séances où chacun était repassé, les ministres et les familiers tout au
moins comme les autres, avec une liberté qui était licence effrénée. Les
galanteries passées et présentes de la cour et de la ville sans
ménagement\,; les vieux contes, les disputes, les plaisanteries, les
ridicules, rien ni personne n'était épargné. M. le duc d'Orléans y
tenait son coin comme les autres, mais il est vrai que très rarement
tous ces propos lui faisaient-ils la moindre impression. On buvait
d'autant, on s'échauffait, on disait des ordures à gorge déployée, et
des impiétés à qui mieux mieux, et quand on avait bien fait du bruit, et
qu'on était bien ivre, on s'allait coucher, et on recommençait le
lendemain. Du moment que l'heure venait de l'arrivée des soupeurs, tout
était tellement barricadé au dehors que quelque affaire qu'il eût pu
survenir, il était inutile de tâcher de percer jusqu'au régent. Je ne
dis pas seulement des affaires inopinées des particuliers, mais de
celles qui auraient le plus dangereusement intéressé l'État ou sa
personne, et cette clôture durait jusqu'au lendemain matin.

Le régent perdait ainsi un temps infini en famille et en amusements, ou
en débauches. Il en perdait encore beaucoup en audiences trop faciles,
trop longues, trop étendues, et se noyait dans ces mêmes détails que, du
vivant du feu roi, lui et moi lui reprochions si souvent ensemble. Je
l'en faisait quelquefois souvenir\,; il en convenait, mais il s'en
lassait toujours entraîner. D'ailleurs mille affaires particulières, et
quantité d'autres de manutention de gouvernement qu'il aurai pu finir en
une demi-heure d'examen le plus souvent, et décider net et ferme après,
il les prolongeait, les unes par faiblesse, les autres par ce misérable
désir de brouiller, et cette maxime empoisonnée qui lui échappait
quelquefois comme favorite\,: \emph{Divide et impera ;} la plupart par
cette défiance général de toutes choses et de toutes personnes, et de
cette façon des riens devenaient des hydres dont lui-même après se
trouvait souvent fort embarrassé. Sa familiarité et la facilité de son
accès plaisait extrêmement\,; mais l'abus qu'on en faisait était
excessif. Il allait quelquefois au manque de respect\,; ce qui, à la
fin, eut des inconvénients d'autant plus dangereux qu'il ne put, quand
il le voulut, réprimer des personnages qui l'embarrassèrent plus
qu'eux-mêmes ne s'en trouvaient e ne s'en trouvèrent embarrassés. Tels
furent Stairs, tels les chefs de la constitution, tels le maréchal de
Villeroy, tels le parlement en particulier, et en gros la magistrature.
Je lui représentais quelquefois tant de choses importantes à mesure que
les occasions s'en offraient\,; quelquefois j'y gagnais quelque chose,
et je parais des inconvénients\,; plus souvent il me glissait de la main
après être demeuré persuadé de ce que je lui disais, et sa faiblesse
l'entraînait.

Ce qui est fort extraordinaire, c'est que ni ses maîtresses, ni
M\textsuperscript{me} la duchesse de Berry, ni ses roués, au milieu même
de l'ivresse, n'ont jamais pu rien savoir de lui de tant soit peu
important, sur quoi que ce soit du gouvernement et des affaires. Il
vivait publiquement avec M\textsuperscript{me} de Parabère\,; il y
vivait en même temps avec d'autres\,; il se divertissait de la jalousie
et du dépit de ces femmes\,; il n'en était pas moins bien avec toutes,
et le scandale de ce sérail public, et celui des ordures et des impiétés
journalières de ses soupers était extrême, et répandu partout.

Le carême était commencé, et je voyais un affreux scandale ou un
horrible sacrilège pour Pâques, qui ne ferait même qu'augmenter ce
terrible scandale. C'est ce qui me résolut d'en parler à M. le duc
d'Orléans, quoique depuis longtemps je gardasse le silence sur ses
débauches par avoir perdu toute espérance là-dessus. Je lui représentai
donc que le détroit où il allait tomber à Pâques me paraissait si
terrible du côté de Dieu, si fâcheux de celui du monde qui veut bien mal
faire, mais qui le trouve mauvais d'autrui et surtout de ses maîtres,
que, contre ma coutume et ma résolution, je ne pouvais m'abstenir de lui
en représenter toutes les conséquences, sur lesquelles je m'étendis à
l'égard du monde\,; car de celui de la religion, malheureusement il n'en
était pas là. Il m'écouta fort patiemment\,; puis me demanda avec
inquiétude ce que je lui voulais proposer. Alors je lui dis que c'était
un expédient, non pour ôter tout scandale, mais pour le diminuer et
empêcher les excès des propos, et même des sentiments auxquels il devait
s'attendre, s'il ne le prenait pas, et qui était très aisé. C'était
d'aller passer chez lui à Villers-Cotterets les cinq derniers jours de
la semaine sainte, e le dimanche et le lundi de Pâques, c'est-à-dire
partir le mardi saint, et revenir la troisième fête de Pâques\,; n'y
mener ni dames ni roués, mais six ou sept personnes à son gré, de
réputation honnête, avec qui causer, jouer, se promener, s'amuser,
manger maigre où il pouvait faire aussi bonne chère qu'en gras, ne point
tenir de mauvais propos à table, et ne la pas allonger par trop ; aller
le vendredi saint à l'office, et le dimanche de Pâques à la
grand'messe\,; que je ne lui en demandais pas davantage, et qu'avec
cela, je lui répondais de tous les discours. J'ajoutai que personne
n'ignorait ce que faisaient ou ne faisaient pas des princes de son
élévation, par conséquent qu'il n'aurait point fait ses Pâques\,; mais
qu'il y avait toute différence entre ne les faire point tête levée avec
un air, qui qu'on pût être, d'insolence et de mépris au milieu de la
capitale, sous les yeux de tout le monde, et changer de lieu avec un air
de honte, de respect et d'embarras\,; que le premier fait abhorrer un
pécheur audacieux, et révolte contre lui jusqu'aux libertins\,; le
second donne une charitable compassion aux honnêtes gens, et arrête
toutes les langues. Je m'offris de l'accompagner en ce voyage, s'il
m'avait agréable, et de lui sacrifier celui que j'avais coutume de faire
en ce temps-là tous les ans chez moi, et je lui fis faire réflexion que
cette conduite était celle des personnes un peu marquées, qui se
trouvaient à Pâques embarrassées de leurs personnes. Je lui fis encore
remarquer que les affaires ne souffriraient point de son absence en des
jours qui les suspendent toutes, la proximité de Villers-Cotterets, la
beauté du lieu, le nombre d'années qu'il ne l'avait vu, et la convenance
qu'il y allât faire un tour.

Il prit la proposition à merveille\,; il s'en trouva soulagé\,; il ne
savait ce que je lui voulais proposer\,; il n'y trouva rien que d'aisé,
même d'agréable, me remercia fort d'avoir pensé à cet expédient, et de
vouloir aller avec lui. Nous raisonnâmes sur ceux qu'il pourrait
mener\,; ce qui ne fut pas difficile à trouver, et la chose demeura
arrêtée Nous crûmes également lui et moi qu'il ne fallait rien afficher
d'avance, et qu'il suffirait qu'il donnât ses ordres dans la semaine de
la Passion. Nous en reparlâmes encore une fois ou deux, et il était
véritablement persuadé que ce voyage était sage, et qu'il devait le
faire. Le malheur était que ce qu'il avait résolu de bon s'exécutait
rarement, par le nombre de fripons dont il était environné, et dont
c'était rarement l'intérêt ou pour lui plaire, ou pour le tenir de près,
ou par des raisons encore plus perverses. C'est ce qui arriva de ce
voyage.

Quand je lui en parlai à un jour ou deux du dimanche de la Passion, je
trouvai un homme embarrassé, contraint, qui ne savait que me répondre.
Je sentis aisément ce qui en était, je redoublai mes efforts, je le pris
par l'approbation qu'il y avait donnée\,; je le défiai de me montrer le
plus léger inconvénient de ce voyage\,; je frappai fortement sur les
discours qu'il ferait tenir par l'audace de sauter par-dessus les
Pâques, au milieu de Paris\,; sur l'ennui dans lequel il ne pouvait
éviter de tomber pendant les jours saints, s'il y voulait garder quelque
mesure, et tout ce qu'il ferait dire contre lui, s'il les passait, comme
il faisait les autres jours\,; enfin je ramassai toutes mes forces pour
lui représenter l'exécration d'un sacrilège, toute l'horreur que le
monde aurait de lui, tout ce qu'il le mettrait en droit de dire, et la
licence avec laquelle toutes les bouches s'en expliqueraient, même les
plus libertines, et jusqu'à quel point cette horrible action éloignerait
de lui tous les gens de bien, ceux qui se piquaient ou qui sont d'état à
l'être, enfin tous les honnêtes gens. J'eus beau dire\,; je ne trouvai
que du silence, du triste, du morne, de misérables raisons que je
détruisis toutes, et de la ténuité desquelles je ne remplirai pas ce
papier\,; en un mot, un parti pris au premier mot qu'il s'en était
laissé entendre qui avait donné l'alarme aux maîtresses et aux roués.
Qu'on ne soit pas surpris si ce mot m'échappe souvent. M. le duc
d'Orléans ne leur donnait point d'autre nom, ni lui, ni
M\textsuperscript{me} la duchesse de Berry\,; M\textsuperscript{me} la
duchesse d'Orléans même en parlant à lui, et tous trois, parlant d'eux à
quiconque, ne les appelaient jamais autrement. Cela avait donné le ton,
et tout le monde sans exception ne parlait plus d'eux\,: que par ce
terme. Ils craignirent que ce prince ne s'accoutumât à vivre avec
d'honnêtes gens, et qu'à son retour ils ne fussent plus admis et seuls à
l'ordinaire. Les maîtresses n'eurent pas moins de frayeur, et ce bon
groupe fit tant sur un prince facile, que le voyage, dès la première
mention, fut absolument rompu. Prenant congé de lui pour m'en aller chez
moi, je le conjurai de se contenir au moins pendant les quatre jours
saints, c'est-à-dire le jeudi, vendredi, samedi et dimanche, et sur
toutes choses de ne pas commettre un sacrilège gratuit où il perdrait du
côté du monde qu'il croirait captiver par là, infiniment plus qu'en s'en
abstenant, parce que sa vie, la même devant et après, le décèlerait tout
aussitôt, et très publiquement.

Je m'en allai là-dessus à la Ferté, espérant du moins avoir paré ce
comble. J'eus la douleur d'y apprendre qu'après avoir passé les derniers
jours de la semaine sainte moins même qu'équivoquement, quoique avec
plus de cacherie, il avait été à la plupart des fonctions de ces jours
saints, suivant l'étiquette de feu Monsieur, qui les passait presque
toujours à Paris\,; qu'il était allé le jour de Pâques à la grand'messe
à Saint-Eustache, sa paroisse, et qu'en grande pompe il y avait fait ses
pâques. Hélas\,! ce fut la dernière communion de ce malheureux prince,
et qui, du côté du monde, lui réussit comme je l'avais prévu. Sortons
d'une si triste matière pour entrer en celle de ce qui se passait au
dehors.

Avant d'entrer dans la narration de ce qui regarde les affaires
étrangères des premiers mois de cette année, il faut, pour éviter une
digression, expliquer une affaire que la cour de Turin eut avec celle de
Rome, qui, pour le dire en passant, fait voir jusqu'à quel excès de
tyrannie et d'oppression les ecclésiastiques tiennent les laïques qui
sont assez simples pour souffrir leurs prétentions se tourner en droit
sous le spécieux prétexte de religion, dont les rois ont été souvent les
victimes, et qui le seraient encore si on les laissait faire, quoique
ces maîtres en Israël trouvent bien écrit dans l'Évangile que la
domination leur est très précisément défendue par Jésus-Christ, et qu'il
leur dise que son royaume n'est pas de ce monde.

Ces Roger, Normands qui conquirent la Sicile et une partie du royaume de
Naples sur les Sarrasins, y régnèrent quelque temps sous le nom de ducs.
Leur piété donna la troisième partie des revenus de la Sicile en
fondations d'évêchés, d'hôpitaux, de monastères, et ils voulurent bien,
par dévotion de ce temps-là, faire relever leur conquête du saint-siège.
Mais en princes avisés, ils y mirent des conditions que les papes se
trouvèrent heureux d'accepter et de confirmer de la manière la plus
solide\,: la première, qu'il fut consenti de part et d'autre que le pape
l'érigerait en royaume, et les en reconnaîtrait rois héréditaires pour
leur postérité\,; l'autre fut pour parer à ce que ces princes voyaient
pratiquer partout où les papes et les ecclésiastiques le pouvaient, qui
dans ces temps d'ignorance usurpaient tout par la terreur de
l'excommunication. Ces princes, qui ne songèrent qu'au solide et à
demeurer vraiment maîtres chez eux, passèrent l'honneur au pape,
moyennant quoi il fut convenu qu'il y aurait en Sicile un tribunal
perpétuellement subsistant, dont les membres, tous laïques, seraient
toujours à la nomination, disposition et en la main des rois de Sicile,
uniquement, sans autre attache ni dépendance, lequel, en vertu du
privilège bien nettement expliqué qu'il recevrait du pape une fois pour
toutes, et irrévocablement en toutes ses parties, et sans jamais être
sujet en aucun cas possible à renouvellement ni à confirmation, jugerait
en dernier ressort souverainement et sans appel de toutes les causes
ecclésiastiques quelles qu'elles pussent être, soit entre laïques e
ecclésiastiques, soit entre ecclésiastiques en tous cas civils et
criminels, excommunications et autres censures, même de la personne des
archevêques, évêques, prêtres, moines, chapitres, tant civilement que
criminellement, tant en première instance que par appel, sans pouvoir
jamais être soumis en aucun cas à rendre raison de sa conduite, sinon
aux rois de Sicile seuls, ni être encore moins sujets pour quelque cause
que ce pût être, à citations, censures ni excommunications, ni troublés
en sorte quelconque en leurs fonctions par Rome, ni par qui que ce pût
être. Avec ce sage et puissant correctif, les immunités et privilèges du
clergé furent admis en Sicile\,; et depuis ces temps reculés ce
tribunal, qu'on appelle \emph{de la monarchie}, a continuellement et
entièrement subsisté, joui et usé de toute l'étendue de sa juridiction.

Il arriva, dans l'été précèdent qu'un fermier de l'évêque d'Agrigente
porta des pois chiches au marché pour les vendre. Des commis aux droits
de M. de Savoie, roi de Sicile, pour lors reconnu et en possession par
le dernier traité de paix de Ryswick\footnote{{[}31{]}}, voulurent faire
payer à l'ordinaire pour l'étalage. Le fermier, sans dire qui il était
les envoya promener, et par cette conduite se fit saisir ses pois
chiches. Fier de l'immunité ecclésiastique qui affranchit de tous
droits, il alla trouver son maître qui, sans autre information ni délai
aucun, fulmina une excommunication. Les commis n'apprirent que par là, à
qui ces pois chiches appartenaient, les rapportèrent tout aussitôt, se
plaignirent de ce que le fermier n'avait daigné finir la querelle d'un
seul mot en disant qui il était, et à qui ces pois chiches
appartenaient. Une réponse et une défense si raisonnable ne put
satisfaire l'évêque. Il demeura ferme, et menaça de pis si ces commis
n'en passaient par tout ce qu'il lui plairait, et comme il voulut
beaucoup exiger d'eux, ils n'osèrent rien promettre sans l'ordre de
leurs supérieurs. Ceux-ci tentèrent vainement d'apaiser l'évêque\,; ils
n'en reçurent qu'une nouvelle excommunication. Le tribunal de la
monarchie trouva que c'était bien du bruit pour des pois chiches rendus
dès qu'on avait su à qui ils appartenaient, et il essaya de terminer
doucement cette affaire.

Ce tribunal incommodait extrêmement la cour de Rome, qui n'avait jamais
pu y donner atteinte par la jalouse attention des souverains de la
Sicile à le maintenir dans tout son entier. Un duc de Savoie devenu roi
seulement de Sicile, parut à Rome plus aisé à entamer que ses puissants
prédécesseurs jusqu'alors. Ainsi la cour de Rome s'aigrit à dessein, et
tant fut procédé que l'évêque d'Agrigente excommunia le tribunal de la
monarchie, quoique juge de sa personne et de ses excommunications, et
soumis à aucune. Le coup parti, le modeste prélat se jeta dans une
barque qu'il avait toute prête, et passa la mer de peur de la prison. Le
tribunal de la monarchie ne souffrit pas patiemment une entreprise si
folle\,; mais les autres évêques, animés par la cour de Rome, ou
l'évêque d'Agrigente avait été reçu à bras ouverts, la soutinrent, en
sorte que, quelque temps après, tous les diocèses de Sicile furent mis
en interdit et les fulminations redoublées. Tous les évêques s'enfuirent
en même temps delà la mer, et y furent bientôt suivis par une
innombrable multitude de prêtres et de moines pour se mettre à couvert
de la prison et des autres peines infligées aux prêtres et aux moines
qui voulaient observer l'interdit.

Rome ne fut pas peu embarrassée de l'inondation de tant de peuple sacré,
réduit à la mendicité par la saisie exacte du temporel de ses biens tant
patrimoniaux qu'ecclésiastiques, qui ne pouvaient subsister que des
libéralités de celui qui causait leur proscription, et qui avait mis le
comble à leur misère par ses censures confirmatives. La vigueur avec
laquelle toute la Sicile se soutenait et se tenait unie contre une
tyrannie si violente et si hors d'exemple depuis plusieurs siècles lit
d'autant plus regretter l'embarquement qu'il était demeuré en Sicile
assez de prêtres, même de religieux sages et fidèles, pour que le
service divin s'y continuât partout, et que les puissances de la
communion romaine commencèrent à lui montrer, surtout la France, par les
procédures et l'arrêt du parlement de Paris rendu à ce sujet, qu'elles
regardaient l'affaire de Sicile comme commune avec elles.

Les jésuites qui ont de grands biens et de superbes maisons en Sicile,
comme par toute l'Italie, et il faut dire partout, excepté en France, se
roidirent tous à demeurer en Sicile, à y observer rigoureusement
l'interdit, et à en animer l'observation exacte de toutes leurs forces.
Le roi de Sicile, qui sentit la conséquence dangereuse de cette
audacieuse conduite, envoya secrètement ses ordres au comte Maffei qu'il
y avait laissé vice roi, duquel il est parlé t. XI, p.~239, qui les sut
exécuter avec un ordre, un secret et une industrie tout à fait
admirable. Il profita de la situation d'une île environnée de la mer de
toutes parts, dont les meilleures villes et autres habitations se
trouvent ou sur les côtes, ou peu avant dans le pays. En un même matin
tous les jésuites, pères et frères, jeunes et vieux, sains ou malades
sans exception d'aucun, furent enlevés dans toutes leurs maisons,
sur-le-champ jetés dans des voitures, conduits à la mer et embarqués
tout de suite, sans leur laisser emporter quoi que ce fût. Les bâtiments
qui étaient tout prêts à les recevoir les passèrent sur les côtes de
l'État ecclésiastique, où ils les laissèrent devenir ce qu'ils
pourraient, sans leur fournir la moindre chose du monde.

On peut juger de l'effet que c e coup fit en Sicile, de l'étonnement de
ces religieux, et de l'embarras du pape et de leur général. Où en placer
un si grand nombre tout à la fois, et faire vivre ces milliers
d'athlètes de leur cause\,? Pour tout cela, il ne s'en rabattit rien des
deux côtés. Mais la chambre apostolique à bout de fournir du pain à ce
nombre immense qui fourmillait à Rome et aux environs, et qui n'en avait
point d'autre, même les évêques siciliens, que celui que cette chambre
leur donnait, on vit un beau jour un édit affiché à Rome qui ordonnai à
tous ces proscris de vider la ville sous des peines, et un trois jours
sans exception, et sans leur fournir ni leur indiquer de quoi vivre,
juste salaire de la sédition, mais qui ne donna pas de réputation à qui
tant d'insensés s'étaient abandonnés, et en devenaient les martyre.
Maffei cependant faisait garder toutes les côtes avec grand exactitude
contre les émissaires et les commerces de Rome, tellement que, lorsque
la plupart de ces proscrits abandonnés voulurent enter de retourner en
Sicile, l'entrée leur en fut fermée\,; {[}ce{]} qui acheva de les mettre
au désespoir.

La fermeté égale des deux côtés laissa les choses en cet État, sans
toutefois que Rome osât attaquer directement le roi de Sicile ni aucun
de ses ministres de terre ferme, jusqu'à ce que, par les événements qui
se trouveront en leur lieu et que j'ai cru devoir prévenir ici pour
achever cette affaire de suite, la Sicile changea de maître et demeura
l'empereur, en donnant la Sardaigne au duc de Savoie, pour lui conserver
la dignité royale. Alors toute l'affaire ecclésiastique tomba, et Rome
se trouva heureuse d'en être quitte pour laisser le tribunal de la
monarchie dans la totalité de l'exercice ordinaire de sa juridiction,
qu'il ne fut plus parlé de rien de tout ce qui s'était passé à
l'importante occasion des pois chiches de l'insolent fermier d'un évêque
impudemment et follement séditieux, et que l'empereur, devenu roi de
Sicile, ayant de Naples et Milan, voulut bien ignorer une entreprise
poussée si loin et aussi destitué de raison, de justice, de la plus
légère apparence, mais qui doit être un puissant rafraîchissement de
leçon à toutes les puissances temporelles des monstrueux excès de
l'ambition ecclésiastique qui, dans tous les temps, ne peut être
contenue que par ne lui passer rien du tout, même de plus léger sous
aucun prétexte, et une vigilance bien exacte à la tenir dans la plus
entière impuissance d'oser seulement songer à s'y livrer.

Pour n'avoir point à retourner sur nos pas, il faut dire que la reine
d'Espagne était accouchée, le 20 janvier de cette année, à Madrid de son
premier enfant. Ce fut un prince qui reçut le nom de Charles ou don
Carlos, qui est( depuis devenu roi de Naples et de Sicile, et que le 20
février, le grand maître de l'ordre Teutonique, coadjuteur de Mayence et
frère de l'électeur palatin, fut élu archevêque et électeur de Trèves.

J'ai répandu en divers endroits, suivant que les occasions s'en sont
offertes, les caractères des personnages de tous États qui ont eu à
entrer dans les matières que j'expose, pour la nécessité ou la curiosité
de les bien connaître. C'est donc de ces caractères dont il faut bien se
souvenir pour ceux qu'on voit entrer et figurer sur la scène, et avoir
présentement recours à ceux du duc de Noailles, de Canillac, de l'abbé
Dubois, de Nocé, d'Effiat, de Stairs, même de Rémond, enfin du maréchal
d'Huxelles.

On a vu en son lieu le commencement du projet d'Écosse, le voyage secret
du Prétendant pour aller s'embarquer en Bretagne, et comment il échappa
aux assassins de Stairs, par l'esprit et le courage de la maîtresse de
la poste de Nonancourt, enfin l'audace avec laquelle cet ambassadeur se
fit rendre les scélérats qui avaient manqué leur coup, et qui avaient
été arrêtés à Nonancourt. Ce projet d'Écosse avait été résolu avec le
feu roi, et avec le roi d'Espagne qui en voulurent bien faire les grands
malheurs du roi Jacques III. La mémoire de ce monarque était trop
récente, lors du voyage secret du Prétendant pour s'aller embarquer en
Bretagne, pour que la France parût changer de sentiment. On le laissa
donc faire, mais sans dessein d'aucun secours, à moins d'y être de
Nonancourt ayant rendu l'embarquement suspect en Bretagne, Bolingbroke,
qui avait lors la conduite et le secret des affaires du Prétendant, qui
était son secrétaire d'État caché à Paris, lui fréta un vaisseau en
Normandie où le Prétendant vint s'embarquer, non en Normandie, mais à
Dunkerque, où on avait fait passer le vaisseau.

On a vu encore, en parlant de Stairs sur la fin de 1715, que ce ministre
anglais ne perdait pas son temps à Paris, et les liaisons utiles à ses
vues pour l'avenir qu'il y avait faites. Les moindres, qu'il ne
négligeait pas, le conduisirent à de plus importantes. Rémond, bas
intrigant, petit savant, exquis débauché, et valet à tout faire, pourvu
qu'il fût dans l'intrigue et qu'il pût en espérer quelque chose, avait
beaucoup d'esprit, et à force de s'être fourré dans le monde par bel
esprit et débauche raffinée, il le connaissait fort bien, et s'attacha
de bonne heure à l'abbé Dubois, qui savait faire usage de tout, et à
Canillac. Il les captiva tous deux par ses respects et ses adulations,
l'abbé par l'intrigue, le marquis par le même goût d'obscure débauche
grecque, et par l'admiration de son esprit et de sa capacité. Ravi de se
faire de fête, il leur vanta le génie supérieur de Stairs\,; à Stairs
tout l'usage qu'il pouvait tirer d'eux auprès de M. le duc d'Orléans\,;
il fit a chacun, comme en étant chargé, des avances mutuelles, et il fit
si bien qu'il les mit en commerce, d'abord de civilité par estime
réciproque, qui se tourna bientôt en commerce d'affaires.

Canillac, comme on l'a vu, avec tout son esprit, avait fort peu de sens.
Un lumineux, qui éblouissait à force de frapper singulièrement bien sur
les ridicules, tenait chez lui la place du jugement\,; et un flux
continuel de paroles, qu'une passion conduisait toujours, et l'envie
plus qu'aucune autre, noyait son raisonnement et le rendait presque
toujours faux. Stairs, bien instruit par Rémond, n'oublia ni respects ni
prostitutions\,; c'était le faible de Canillac. Les cajoleries
continuelles de Stairs le gagnèrent\,; il ne put résister au plaisir de
sentir le caractère d'ambassadeur ployer devant son mérite, et l'audace
du personnage s'humilier devant lui. À son tour il admira son esprit, sa
capacité, ses vues\,; la brouillerie ouverte de Stairs avec tout le
gouvernement du feu roi fut un autre attrait très puissant pour
Canillac, qui haïssait les gens en crédit et en place, le feu roi et
tous ceux qu'il y avait mis. Stairs prit grand soin de le cultiver et de
le séduire, et bientôt Canillac ne vit plus rien que par ses yeux. Son
union avec le duc de Noailles lui fit souhaiter celle de Stairs avec
lui. Noailles, qui l'avait conquis par la même voie, qui avait si bien
réussi à Stairs, avait pour maxime de ne le contredire jamais et de
l'admirer toujours\,; ainsi la connaissance fut bientôt faite, et de là
les raisonnements politiques entre eux.

Pour l'abbé Dubois, la liaison fut bientôt faite\,; il ne la souhaitait
pas moins que Stairs. Stanhope était secrétaire d'État et ministre
confident du roi Georges. Il avait autrefois passé quelque temps à
Paris\,; il y avait vu Dubois chez M\textsuperscript{me} de Sandwich,
qui fut beaucoup d'années de suite en France, et qui était en galanterie
avec l'abbé. Lui et Stanhope firent grande amitié de voyageur et de
débauche\,; l'abbé le fit connaître à M. le duc d'Orléans, qui le vit
familièrement depuis, et l'admit en quelques-unes de ses parties.
Stanhope et Dubois se firent faire souvent des compliments par
M\textsuperscript{me} de Sandwich, depuis le retour de Stanhope en
Angleterre. Il se trouva à la tête des troupes anglaises en Espagne,
lorsque M. le duc d'Orléans et l'abbé Dubois y étaient, où d'armée à
armée ils eurent tout le commerce que put permettre l'état d'ennemis. On
a vu en son lieu combien le prince et son abbé comptaient sur ce général
anglais, dans ce que j'ai rapporté de l'affaire d'Espagne de M. le duc
d'Orléans. Un autre Stanhope avait succédé à celui-ci au commandement
des troupes en Espagne, dont la catastrophe a été marquée en son temps,
et le lord Stanhope, connu de l'abbé Dubois et de M. le duc d'Orléans,
était devenu secrétaire d'État. Dubois, à qui l'ambition et le goût de
l'intrigue ne laissait point de repos, bâtissait en esprit sur ses
anciennes liaisons avec Stanhope. Il voulait pour cela même tourner M.
le duc d'Orléans vers le roi Georges\,; il n'était pas alors en
situation auprès de lui d'y réussir\,; il désirait d'apprivoiser Stairs
pour se procurer des occasions de parler d'affaires au régent, et de lui
faire valoir leur ancienne connaissance avec Stanhope, et Stairs
souhaitait pour le moins autant que Dubois de se familiariser avec lui
pour se procurer accès personnel auprès de M. le duc d'Orléans, et lui
faire passer par l'abbé Dubois, qu'il s'imaginait en être à portée,
quoiqu'il n'y fût point du tout encore, des choses qui feraient plus
d'impression d'une autre bouche que de la sienne. Rien n'allait mieux à
leurs vues communes, mais réciproquement ignorées, que l'union que
Rémond avait procurée, de concert avec Dubois, de Stairs et de Canillac,
et de celle que celui-ci avait faite du ministre anglais avec Noailles.

Le triumvirat était déjà formé entre Noailles, Canillac et Dubois, comme
je l'ai expliqué sur la fin du règne du feu roi. Dubois, pour ses vues
cachées, n'oublia rien pour confirmer Canillac dans son infatuation pour
Stairs, et pour y jeter le duc de Noailles. Celui-ci, toujours pris par
les nouveautés, et qui était homogène à M. le duc d'Orléans par
l'enchantement des voies détournées, eut une forte raison, et peut-être
deux, pour se livrer à cette complaisance. Il sentait la sécheresse des
finances, et tous les embarras de joindre les deux bouts, et il voyait
une grande épargne à refuser tout secours au Prétendant, et à faire
échouer une entreprise qu'il aurait fallu soutenir devenant heureuse, et
peut-être soudoyer longtemps, et fortement. L'autre raison, que
j'imagine peut-être, me regardait. Nous avions vécu trop longtemps
confidemment ensemble pour qu'il pût ignorer que j'étais parfaitement
jacobite, et très persuadé de l'intérêt de la France à donner à
l'Angleterre une longue occupation domestique, qui la mit hors d'état de
songer au dehors, et d'empiéter encore le commerce d'Espagne et le
nôtre, et que nous n'en avions pas un moindre à n'avoir plus affaire à
un roi d'Angleterre, s'il était possible, qui par ses États et ses
intérêts en Allemagne était plus Allemand qu'Anglais, et toujours en
crainte, en brassière, et tant qu'il pouvait en union avec l'empereur.
Peut-être lui était-il revenu que Stairs m'avait tourné inutilement par
M. de Lauzun, qui aimait à voir les étrangers, et qui, malgré tout ce
qu'il devait, et tout ce qu'il était à la cour de Saint-Germain, aimait
tous les Anglais, voyait fort Stairs, mangeaient l'un chez l'autre, et
n'avait pu me résoudre à répondre aux avances qu'il me faisait pour
Stairs, et à son empressement de nous joindre à dîner ensemble, que par
de simples compliments, tels qu'ils ne se peuvent refuser.

Pensant comme je faisais sur l'Angleterre, je ne pouvais goûter une
liaison avec son ambassadeur, dont l'audace et la conduite me
repoussaient d'ailleurs, bien plus encore depuis l'affaire de
Nonancourt. Noailles put donc comprendre qu'avec le secours de Canillac
et les manèges de Dubois, il ne serait pas difficile de tourner le
régent vers le roi Georges, et qu'en venant à bout, il ne serait pas
difficile de me rendre suspect à cet égard, et d'entamer la confiance
générale dont Son Altesse Royale m'honorait, en lui persuadant de me
faire un mystère de son union avec l'Angleterre. Quoi qu'il en soit de
ces raisons, Noailles s'embarqua avec Stairs, tout aussi avant que ses
deux amis Canillac et Dubois, et ils persuadèrent M. le duc d'Orléans de
se conduire à cet égard par une maxime purement personnelle,
conséquemment détestable\,; Cette maxime était que le roi Georges était
un usurpateur de la couronne de la Grande-Bretagne, et, si malheur
arrivait au roi, M. le duc d'Orléans serait aussi usurpateur de la
couronne de France\,; conséquemment même intérêt en tous les deux, et
raison de se cultiver l'un l'autre, de se conduire au point de se
garantir ces deux couronnes mutuellement, et de ne jamais faire aucun
pas qui put le moins du monde écarter de ce grand objet, en quoi,
ajoutaient-ils, le prince français gagnait tout pour assurer son
espérance, tandis que l'Anglais en possession, par cela même n'y gagnait
presque rien, d'autant plus qu'il n'avait affaire qu'à un Prétendant
sans biens, sans état, sans secours, au lieu que le cas advenant, M. le
duc d'Orléans aurait pour compétiteur un roi d'Espagne établi et
puissant, et par mer et par terre limitrophe de tous les côtés de la
France.

M. le duc d'Orléans avala ce poison présenté avec tant d'adresse par des
personnes sur l'esprit, la capacité et l'attachement personnel
desquelles il croyait devoir compter, qui toutefois lui prouvèrent bien
dans la suite que leur e prit était faux, leur capacité nulle, leur
attachement vain et uniquement relatif à eux-mêmes. Ce prince n'avait
que trop de pénétration pour apercevoir le piège, et le prodige est ce
qui le séduisit\,: ce fut le contour tortueux de cette politique, et
point du tout le désir de régner. Je m'attends bien que si jamais ces
Mémoires voient le jour, cet endroit fera rire, en décréditera les
autres récits, et me fera passer pour un grand sot, si j'ai cru
persuader mes lecteurs, ou pour un imbécile, si je l'ai cru moi-même.
Telle est pourtant la vérité toute pure, à laquelle je sacrifie tout ce
qu'on pensera de moi. Quelque incroyable qu'elle paroisse, elle ne
laisse pas d'être vérité. J'ose avancer qu'il y en a beaucoup de telles
ignorées dans les histoires, qui surprendraient bien si on les savait,
et qui ne sont ignorées que parce qu'il n'y en a presque aucune qui soit
écrite de la première main.

Cette vérité-ci, et plusieurs autres que j'ai vues, m'en persuadent, qui
sont trop peu importantes à l'histoire de ce temps pour que je les aie
écrites, et d'autres encore dont j'ai inséré ici les principales que
j'ai sues de mon père, et qui sont demeurées dans l'oubli, ou qui de
Louis XIII, a qui elles appartiennent, ont été transportées au cardinal
de Richelieu. Je le répète, et je le dois à la vérité qui règne
uniquement dans ces Mémoires, comme on le voit sur M. le duc d'Orléans
lui-même par le portrait que j'en ai donné, jamais ce prince n'a désiré
la couronne\,; il a très sincèrement souhaité la vie du roi\,; il a plus
fait, il a désiré qu'il régnât par lui-même, comme on le verra dans la
suite. Jamais de lui-même il n'a pensé que le roi pût manquer, ni aux
choses qui pouvaient suivre ce malheur, qu'il regardait sincèrement
comme tel, et pour lui-même, si jamais il arrivait. Il ne faisait que se
prêter aux réflexions qui là-dessus lui étaient présentées, incapable
entièrement d'y penser de lui-même, ni aux mesures à prendre sur la
considération que cela était possible. Je ne dirai pas que, le cas
arrivant, il eût abandonné le droit que lui donnait la renonciation
réciproque, garantie de toute l'Europe\,; mais j'ajoute en même temps
que la possession de la couronne y eût eu la moindre part, et que
l'honneur, le courage, sa propre sûreté l'aurait eue tout entière\,:
encore une fois, ce sont des vérités que ma très parfaite connaissance,
ma conscience et mon honneur m'obligent à rapporter.

Pour achever de suite la matière de cet engagement qui éclaircira tout
ce que j'aurai à rapporter de ses suites, ces messieurs ne réussirent
pas entièrement dans leur projet à mon égard, si mon soupçon sur le duc
de Noailles a été véritable. Le régent ne put me cacher longtemps
l'inclination supérieure qu'il avait prise pour l'Angleterre. Je
l'approuvai jusqu'à un certain point, pour entretenir la paix dont
l'épuisement de la France et un temps de minorité avaient tant de
besoin, et pour retenir le trop dangereux penchant du roi Georges vers
l'empereur. Mais je ne pus approuver des dispositions à aller plus loin.

Je répétai au régent ce que je lui avais souvent dit, et ce que j'avais
plus d'une fois opiné au conseil de régence, que l'intérêt essentiel de
l'État était la plus solide et la plus inaltérable union avec
l'Espagne\,; que la même maison et encore presque au premier degré
unissait, et qu'aucune prétention ni intérêt véritable ne divisait, dont
trois choses confirmaient l'évidence\,: l'exemple de la maison
d'Autriche qui n'avait bâti cette formidable grandeur, si longtemps près
de la monarchie universelle, que par l'union de ses deux branches que
nul effort n'avait jamais pu séparer\,; l'extrême frayeur conçue par
toute l'Europe d'un fils de France devenu roi d'Espagne, cause unique de
la dernière guerre qui a tant coûté à toutes ses puissances\,; enfin
l'avantage infini à tirer pour cette union et pour la mutuelle grandeur
de la contiguïté des terres et des mers des deux monarchies qui leur
procure réciproquement des facilités que la nature avait refusées aux
deux branches d'Autriche, dont elles auraient bien su grandement
profiter\,; que la politique de cette habile maison devait être en ce
point le modèle de la notre, et le pôle dont rien, pour spécieux qu'il
fût, ne nous devait faire perdre la vue la plus fixe\,; que cette maxime
posée, il fallait compter sur deux choses, et se raidir contre toutes
les deux fort diversement, l'une les brouillards d'intérêts particuliers
des personnages de cette cour et de celle de Madrid, les fantaisies du
roi et de la reine d'Espagne, les travers de leur ministère qu'il
fallait esquiver, flatter, cajoler\,; surtout ne se jamais fâcher\,;
faire revenir à raison avec patience, douceur, amitié\,; captiver ces
têtes qui influaient\,; se persuader que les cours de Vienne et de
Madrid s'étaient souvent donné réciproquement les mêmes embarras
domestiques sans qu'ils aient jamais éclaté ni qu'ils les aient
refroidies l'une pour l'autre en ce qui était affaires\,; que nous ne
devons pas moins faire qu'elles à cet égard, ni en espérer un moindre
succès\,; enfin, imiter la sagesse des familles particulières, qui ont
leurs humeurs, leurs dépits, leurs défauts, mais qui n'en laissent rien
apercevoir au dehors, et qui présentent toujours à l'opinion publique
une union qui fait leur force, leur crédit, leur considération\,;
l'autre qu'il fallait se bien attendre à tous les ressorts que la
politique des autres puissances ne se lasserait point de faire
successivement jouer pour parvenir à jeter du froid, puis de la division
entre les deux couronnes\,; que la paix, qui enfin avait terminé la
longue, ruineuse et sanglante guerre causée par la succession d'Espagne,
n'en avait pas éteint l'extrême jalousie, ni par conséquent amorti le
moins du monde la passion de les brouiller et de les désunir\,; que
toutes regardaient ce point comme le but de leur plus grand intérêt et
comme un ouvrage auquel leur concert et leur politique ne devait jamais
se lasser de travailler\,; que pour cela tous les partis spécieux,
toutes les propositions éblouissantes, toutes les perspectives de
crainte et de danger seraient sans cesse employées dans l'une et l'autre
cour, même des réalités qui, jusqu'à un certain point, seront offertes
et réputées à gain d'être acceptées, sachant bien quel grand intérêt à
en retirer\,; que le moyen de déconcerter tant de suite est d'en avoir
soi-même à tenir les yeux bien ouverts, et de refuser toute espèce
d'avantage, quelque considérable qu'il pût être offert, qui pourrait
entraîner de la division avec l'Espagne\,; se rendre inaltérable sur ce
point capital\,; se mettre avec l'Espagne sur un pied d'assez de
confiance pour s'entre-communiquer toutes ces diverses tentatives et en
profiter pour resserrer de plus en plus l'étroite et indissoluble
union\,; que cette conduite avait été celle des deux branches d'Autriche
depuis Charles-Quint jusqu'au prédécesseur de Philippe V\,; que c'est ce
qui avait porté leur puissance à un si haut point, et une leçon a
prendre dans nos deux branches sans s'en écarter jamais\,; enfin que la
facilité en était d'autant plus grande, qu'il n'y avait rien à craindre
pour la sûreté des courriers, et parce que le roi d'Espagne avait le
coeur entièrement français.

J'ajoutai, parce que le régent et moi étions tête-à-tête, comme il
arrivait presque toujours, qu'après le paquet de son affaire d'Espagne,
et sa réconciliation, de plus dans sa position personnelle par rapport
aux renonciations, rien ne lui tournerait personnellement plus à bien ou
à mal en France et dans le reste de l'Europe, ni avec plus de suites et
de conséquences, que de tenir avec l'Espagne la conduite que je
proposais, ou une différente. J'appuyai sur ce qu'à Rome, qui dans ces
temps-là était encore le centre des affaires, et dans toutes les autres
cours, les intérêts des deux branches d'Autriche avaient sans cesse été
les mêmes, et jusque dans l'intérieur domestique des affaires de
l'empire\,; que nulle puissance ne pouvait toucher à l'une, que l'autre
n'intervint incontinent comme commune en tout et partout, ainsi qu'il
avait paru en toutes les guerres et en tous les traités particuliers et
généraux, jusque-là que le reste de l'Europe s'était depuis longtemps
dépris de songer à les désunir, et n'avait plus pensé qu'à se soutenir
contre elles. Que c'était la le modèle que nous avions à suivre si nous
voulions prospérer dedans et dehors, et nous élever jusqu'au point de
devenir les dictateurs de l'Europe, comme il était arrivé à la maison
d'Autriche, même après avoir tacitement renoncé à la monarchie
universelle, où elle avait enfin senti qu'elle ne pouvait atteindre.

Je suppliai ensuite le régent de se souvenir que les véritables ennemis
de la France étaient la maison d'Autriche et les Anglais. Que la
connaissance qu'il avait de l'histoire ne lui présentait autre chose,
dans toute sa suite, que cette haine et cette jalousie d'une couronne
qui seule pouvait arrêter leur ambition\,; que cette passion avait pris
un nouvel accroissement par la compétence\footnote{} de Charles-Quint et
de François Ier, et par les vains efforts de Philippe II, du temps de la
Ligue\,; et depuis, à l'égard de l'Angleterre, par la haine
irréconciliable du feu roi pour le prince d'Orange et par le dépit de ce
dernier de n'avoir pu l'amortir par vingt ans de soumissions, lequel
s'était tourné en rage, de laquelle on avait senti les effets par toute
l'Europe, dont il avait excité toutes les puissances\,; enfin par son
invasion d'Angleterre, par la protection que le feu roi avait prise de
Jacques II et de sa famille\,; en dernier lieu par sa reconnaissance de
Jacques III, nonobstant le traité solennel de Ryswick, et les
conjonctures où il l'avait faite, dont le roi Guillaume avait bien su se
servir dans toute l'Europe, et tout mourant qu'il était, l'unir contre
la France, et porter à cette occasion la haine des Anglais jusqu'à la
rage. Que si une intrigue de femme et de la cour de la reine Anne avait
sauvé la France des derniers malheurs par sa séparation d'avec ses
alliés, et les traités de paix qui en furent la suite, et elle
l'instrument, il fallait bien distinguer une cabale de cour qui y trouva
son intérêt pour s'élever sur la ruine de ses ennemis qui auparavant
avaient tout pouvoir en Angleterre, d'avec la nation, et même la
totalité de la cour.

D'ailleurs la médaille avait tourné par la mort d'Anne et l'arrivée de
son successeur en Angleterre, qui avait chassé tous ceux à qui nous
devions la paix, remis en place ceux qu'Anne en avait ôtés, et abandonné
nos amis à la fureur des whigs, et aux procédures d'un parlement furieux
de cette paix, que la cour excitait encore contre eux. De cet exposé je
conclus qu'il était insensé de se proposer de lier avec l'Angleterre une
amitié véritable qui ne serait jamais que frauduleuse et traîtresse,
jamais offerte ou acceptée que dans l'unique vue de diviser la France
d'avec l'Espagne, et d'en profiter\,; que de se rabattre à l'espérance
de nouer au moins cette amitié de roi à roi, c'était encore un leurre
fort grossier, qui ne pouvait tirer nulle force de celle qui avait été
entre le feu roi et Charles II\,; qu'outre que Charles II était son
cousin germain, qu'il avait la reine sa mère établie en France depuis
les premiers {[}malheurs{]} de Charles Ier, et Madame, sa sœur, épouse
de Monsieur, qui avait la confiance et l'amitié personnelle des deux
rois, dont elle avait été le lien tant qu'elle avait vécu, et dont la
mémoire leur était toujours demeurée chère, on n'avait pas laissé
d'avoir grand besoin de soutenir cette amitié par beaucoup d'argent, et
par tout le crédit de la duchesse de Portsmouth, dont Charles II était
possédé, et qui était française au point de tout confier aux
ambassadeurs de France, et de se gouverner uniquement par eux. Et si,
malgré une amitié si bien cimentée, vit-on les Anglais forcer la main à
leur roi, et le réduire malgré lui à se déclarer contre la France, et
s'unir à ses ennemis, dans une conjoncture qui fit abandonner au roi ses
vastes conquêtes des Pays-Bas\,; qu'il y avait donc bien loin d'un roi
d'Angleterre tel que Charles II, d'avec le roi Georges, qui ne devait
tout ce qu'il possédait de grand qu'à l'empereur, qui l'avait fait
électeur, et qui favorisait son occupation des duchés de Brême et de
Verden, en pleine paix, sur la Suède, mais sans lui en donner
l'investiture pour le contenir par là\,; et aux Anglais, au feu roi
Guillaume, au protestantisme et aux whigs, qui de tous les Anglais
haïssent le plus la France, qui n'ont jamais voulu de paix, qui font le
procès aux ministres de la reine Anne pour l'avoir procurée, et qui ont
été remis par Georges dans toutes les grandes, médiocres et petites
charges, et emplois dans toute la Grande-Bretagne, par Georges, dis-je,
qui sent que les whigs sont son appui en Angleterre, et l'empereur pour
ses États et ses prétentions d'Allemagne, et qui, par de si puissants
intérêts, est radicalement incapable d'aucune véritable ni durable
liaison avec la France\,; enfin, que de telles barrières étaient
insurmontables par leur nature, bien différente des petits intérêts
particuliers des deux cours de France et d'Espagne, des travers de leurs
ministres, des fantaisies de Sa Majesté Catholique, d'un roi d'Espagne,
oncle paternel du roi, dont le coeur est tout français, et dont
l'autorité et le pouvoir est despotique dans sa monarchie, et ne connaît
ni formes, ni torys, ni whigs, ni parlements, et dont la religion est la
même que la nôtre, et les intérêts homogènes aux nôtres contre toutes
les puissances qui n'ont rien oublié pour le détrôner, en particulier
les maritimes, rivales jusqu'au transport du commerce de toutes les
autres et singulièrement de celui d'Espagne, et du nôtre par notre union
avec elle. Enfin que, quelque intimité que, par impossible, on pût
supposer entre la France et l'Angleterre, on ne pouvait jamais espérer,
pour l'utilité et la grandeur de la première, rien d'approchant de celle
qu'il était visible qui résulterait de celle de deux rois si proches, et
de même maison, et de deux si puissantes monarchies si parfaitement
limitrophes, qui n'ont aucuns intérêts opposés, et de même religion.

Le régent, qui m'avait écouté avec grande attention, n'eut rien à
opposer à la force naturelle de ces raisons. Il convint des principes et
des faits. Il m'assura aussi que son dessein était de se lier tant qu'il
pourrait avec l'Espagne, mais que ce n'était pas une résolution à
laisser pénétrer trop avant à l'Espagne même, gouvernée par une reine
ambitieuse, et par un ministre très dangereux, qui tournaient le roi
d'Espagne tout comme ils voulaient, et très capables d'abuser de cette
connaissance\,; encore moins la trop montrer à l'Angleterre et aux
autres puissances, qui s'en refroidiraient pour nous, redoublerait leur
jalousie et leurs efforts pour nous diviser d'avec l'Espagne, et leur
persuaderait de ne nous jamais considérer que comme ennemis\,; que ce
ménagement était d'autant plus nécessaire que je n'ignorais pas que la
grande maxime de la cour de Vienne, surtout depuis la paix de Ryswick,
était une liaison indissoluble avec les puissances maritimes, laquelle
avait été pareillement fondée entre l'Angleterre et la Hollande par le
roi Guillaume, que la jalousie du commerce n'avait pu altérer depuis, et
qui trouvaient leur compte dans l'alliance de l'empereur pour nous
l'opposer, lequel était le maître de l'empire, et de le faire armer sans
autre cause que sa volonté et son intérêt particulier.

Je convins avec le régent de la solidité de la précaution qu'il se
proposait, pourvu que ce ne fût que précaution, et qu'il convînt aussi
de la nécessité de suivre les maximes que je venais de lui proposer. Il
m'assura beaucoup que c'était sa ferme intention\,; et la conversation
finit de la sorte, en me remontrant avec combien de mystère et de mesure
il devait aider le Prétendant débarqué en Écosse, et cacher les secours
qu'il lui donnerait sous les plus épaisses ténèbres, à moins d'un succès
rapide et inespéré.

Il m'apprit en même temps que les Danois et les Prussiens avaient enfin
pris Stralsund qu'ils assiégeaient depuis longtemps, mais que le roi de
Suède, qui depuis son retour de Bender s'était jeté dedans, avait
échappé il leur vigilance, et était passé en Suède.

\hypertarget{chapitre-xviii.}{%
\chapter{CHAPITRE XVIII.}\label{chapitre-xviii.}}

1716

~

{\textsc{Traité de commerce avantageux à l'Angleterre signé à Madrid.}}
{\textsc{- Albéroni a seul la confiance du roi et de la reine
d'Espagne\,; fait la réforme des troupes.}} {\textsc{- Revenus de la
couronne d'Espagne.}} {\textsc{- Lenteurs de l'échange des ratifications
du traité de la Barrière et du rétablissement des électeurs de Cologne
et de Bavière.}} {\textsc{- Semences de mécontentement entre l'Espagne
et l'Angleterre.}} {\textsc{- Albéroni tient le roi et la reine
d'Espagne sous sa clef.}} {\textsc{- Sa jalousie du cardinal del
Giudice, qu'il veut perdre, et du P. Daubenton, qu'il veut subjuguer.}}
{\textsc{- Quel est ce jésuite.}} {\textsc{- Albéroni pointe au
cardinalat, et se mêle des différends avec Rome.}} {\textsc{-
Aubrusselle, jésuite français, précepteur du prince des Asturies.}}
{\textsc{- Dégoût del Giudice.}} {\textsc{- Fâcheux propos publics sur
la reine et Albéroni qui prend un appartement dans le palais et se fait
rendre compte en premier ministre.}} {\textsc{- Anglais et Hollandais
veulent chasser les François des Indes.}} {\textsc{- Brocards sur
Albéroni.}} {\textsc{- Friponneries de Stairs.}} {\textsc{- Haine des
Anglais pour la France.}} {\textsc{- L'empereur tenté d'attaquer
l'Italie.}} {\textsc{- Crainte de l'Italie de l'empereur et des Turcs.}}
{\textsc{- Traité de la Barrière conclu.}} {\textsc{- Le régent propose
la neutralité des Pays-Bas\,; les Anglais, un renouvellement d'alliance
aux Hollandais, dangereuse à la France, et y veulent attirer le roi de
Sicile.}} {\textsc{- Le pape implore partout du secours.}} {\textsc{-
Situation et ruses d'Albéroni.}} {\textsc{- Plaintes et disgrâces que
cause sa réforme des troupes.}} {\textsc{- Le duc de Saint-Aignan s'en
mêle mal à propos.}} {\textsc{- Hersent père\,; son caractère\,; son
état.}} {\textsc{- Le Prétendant échoue en Écosse et revient.}}
{\textsc{- L'Espagne lui refuse tout secours, caressée par l'Angleterre
aigrie contre la France.}} {\textsc{- Impostures de Stairs pour l'aigrir
encore plus.}} {\textsc{- Soupçons réciproques des puissances
principales.}} {\textsc{- Adresse de Stanhope pour brouiller la France
et l'Espagne, et pour gagner le roi de Sicile à son point.}} {\textsc{-
Triste opinion générale de l'Espagne.}} {\textsc{- Ombrages d'Albéroni
qui promet un grand secours au pape.}} {\textsc{- Triste et secrète
entrevue du Prétendant et de Cellamare.}} {\textsc{- Berwick et
Bolingbroke mal avec le Prétendant, qui prend Magny.}} {\textsc{- Quel
est Magny.}} {\textsc{- Violents offices de l'Angleterre partout contre
tout secours et retraite à ce prince.}} {\textsc{- Fausses souplesses à
l'Espagne, jusqu'à se liguer avec elle pour empêcher l'empereur de
s'étendre en Italie, et secourir le roi d'Espagne en France si le cas
d'y exercer ses droits arrivait.}} {\textsc{- But du secours d'Espagne
au pape.}} {\textsc{- Le roi et la reine d'Espagne ne perdent point
l'esprit de retour, si malheur arrivait, en France.}} {\textsc{-
Albéroni les y confirme.}} {\textsc{- Ses ombrages\,; ses manèges\,; son
horrible duplicité.}} {\textsc{- Inquiétude de Riperda.}} {\textsc{-
Crainte du roi de Sicile.}} {\textsc{- Liberté de discours du cardinal
del Giudice.}} {\textsc{- Étrange scélératesse de Stairs confondue par
elle-même.}} {\textsc{- Faux et malin bruit répandu sur les
renonciations.}} {\textsc{- Propositions très captieuses contre le repos
de l'Europe faites par l'Angleterre à la Hollande, qui élude sagement.}}
{\textsc{- Frayeur égale du pape, de l'empereur et du Turc.}} {\textsc{-
Stanhope propose nettement à Trivié de céder à l'empereur la Sicile pour
la Sardaigne.}} {\textsc{- Stanhope emploie jusqu'aux menaces pour
engager la Savoie contre la France.}} {\textsc{- But et vues de
Stanhope.}} {\textsc{- Préférence du roi Georges de ses États
d'Allemagne à l'Angleterre, cause de ses ménagements pour l'empereur.}}
{\textsc{- Conseil de Vienne et celui de Constantinople divisés sur la
guerre.}} {\textsc{- Escadres anglaise et hollandaise vont presser le
siège de Wismar.}} {\textsc{- Nouvelles scélératesses de Stairs.}}
{\textsc{- Intérêt du ministère anglais de toujours craindre la France
pour tirer des subsides du parlement.}} {\textsc{- Continuation
d'avances infinies de l'Angleterre à l'Espagne.}} {\textsc{- Monteléon
en profite pour s'éclaircir sur la triple alliance proposée par
l'Angleterre avec l'empereur et la Hollande.}} {\textsc{- Souplesse de
Stanhope.}} {\textsc{- Crainte domestique du ministère anglais qui veut
rendre les parlements septénaires.}}

~

Le traité qui se négociait l'année dernière entre le roi d'Angleterre et
le roi d'Espagne venait d'être signé à Madrid, et par la satisfaction
extrême qu'on en témoignait à Londres, semblait promettre la plus grande
liaison entre les deux monarques. Monteléon, ambassadeur d'Espagne à
Londres, comptait d'en augmenter sa considération personnelle et sa
fortune, et y fondait de grandes espérances pour le service du roi
d'Espagne, non seulement présentement, mais au cas qu'il arrivât en
France des choses sur lesquelles Leurs Majestés Catholiques et leurs
ministres, qui n'étaient pas Espagnols, tenaient toujours leurs yeux
ouverts. C'était de quoi Stanhope l'entretenait souvent pour engager
l'Espagne à prendre avec l'Angleterre des engagements plus étroits, dans
le mécontentement où Stairs entretenait sa cour sur les secours et la
protection qu'il mandait que le régent accordait au Prétendant, ignorant
ou voulant bien ignorer que l'Espagne n'en faisait pas moins là-dessus
que la France\,; ce qui était caché même à Monteléon par sa propre cour.
Elle n'avait point de vaisseaux en mer, ni de préparatifs pour en armer.
La Hollande lui en avait offert pour assurer le commerce des Indes,
mais, contente de voir son offre acceptée, la république ne se pressait
pas, dans la vue d'obtenir à cette occasion quelques avantages pour son
commerce. Dans cet intervalle, l'Angleterre offrit aussi des vaisseaux à
Monteléon, comme par reconnaissance de la manière dont le dernier traité
venait d'être signé. Monteléon se prévalut de ces démonstrations
d'amitié pour s'éclaircir sur les liaisons secrètes qui l'inquiétaient
entre le roi d'Angleterre et l'empereur. Stanhope lui répondit, avec un
air d'ouverture, que l'opposition qu'ils remarquaient de la France à
leurs intérêts les avait engagés pour faire des alliances, parce qu'ils
n'avaient pas douté que l'Espagne ne suivit la France\,; qu'il n'y avait
rien de conclu avec l'empereur au préjudice de l'Espagne\,; et que, le
traité de commerce venant d'être signé si à propos à Madrid avec
l'Angleterre, elle n'écoulerait aucune proposition directe ni indirecte
qui pût intéresser l'Espagne.

Cette couronne, qui regardait la Sicile comme pouvant un jour lui
revenir selon les traités, prit vivement ses intérêts à Rome sur
l'interdit fulminé contre ce royaume à l'occasion des pois chiches de
l'évêque d'Agrigente. Albéroni avait seul la confiance du roi et de la
reine d'Espagne. Il était seul chargé des réforme, des troupes, des
dépenses de la marine, de celles de la maison royale, et des principales
affaires d'État. Il s'ouvrit à quelqu'un que le produit des revenus de
1716, qui devaient se toucher dans son courant, ne se montaient qu'a
seize millions, et les dépenses nécessaires de la même année à vingt et
un millions, sans les extraordinaires qui pouvaient survenir. Il
travaillait tous les soirs avec Leurs Majestés Catholiques sur la
réforme des troupes. Il y fut résolu qu'il ne serait conservé que deux
compagnies des quatre des gardes du corps, et d'autres détails de
réforme dans les deux conservées, en quoi Albéroni comptait épargner
soixante mille pistoles par an\,; de dix bataillons des gardes n'en
garder que deux, dont un espagnol, l'autre wallon. Il comptait que la
réforme du seul état major de ces régiments réduits à deux bataillons
irait à une épargne de quatre cent mille réaux par an. Il résolut aussi,
après la réforme exécutée, de lever six mille dragons, dont la moitié à
pied, et de les laisser toujours dans la Catalogne. Les autres réformes,
ainsi que les règlements nouveaux pour les conseils et pour le palais,
ne devaient venir qu'ensuite.

Cellamare, ambassadeur d'Espagne à Paris, n'était pas moins attentif que
les ministres des autres puissances aux semences de division qui y
éclataient, et dont celles qui avaient signé la paix d'Utrecht avec tant
de dépit espéraient des troubles et un renouvellement de guerre,
l'accomplissement du traité de la Barrière mettait du malaise entre
elles. La Hollande différait d'en donner sa ratification avant que
l'Angleterre eût fourni la sienne. Les Impériaux menaçaient d'en venir
enfin aux voies de fait. Ceux qui étaient aux Pays-Bas trouvaient que
ces délais de les mettre en possession donnaient de la hardiesse aux
peuples qui leur devaient devenir soumis de se mêler de trop
d'informations. Ils avaient même secrètement consulté Bergheyck, dont
j'ai si souvent parlé, sur les droits qu'on voulait tirer d'eux, et
avaient fait partir leurs députés pour aller porter leurs remontrances à
Vienne. Surtout les Impériaux et les Anglais ne goûtaient point la
proposition de la neutralité des Pays-Bas, faite par le régent, à
laquelle la Hollande paraissait assez favorable. Une autre affaire
occupait l'empereur. C'était l'entier rétablissement des électeurs de
Cologne et de Bavière. L'électeur de Mayence, directeur de l'empire, le
sollicitait ardemment pour contrebalancer l'autorité des protestants
dans le collège électoral. L'empereur sentait la nécessité d'y faire
rentrer ces deux électeurs par leur accorder leur investiture, mais il
leur excusait ses délais sur ceux de la France à restituer quelques
bailliages à l'électeur palatin, et à satisfaire d'autres particuliers
qui se plaignaient à cet égard de l'inexécution des traités de Rastadt
et de Bade. Cet aveu fut appuyé de l'espérance que l'empereur leur donna
de finir leur rétablissement, si la France demeurait opiniâtre, pour les
en détacher et faire retomber sur elle les délais de leurs désirs,
ajoutant qu'il verrait après à trouver les moyens d'obliger la France à
exécuter les traités. Le régent, instruit de cette malice, et qui avait
chargé le comte du Luc, ambassadeur de France à Vienne, de convenir des
limites de l'Alsace, jugea sagement qu'il devait ôter à l'électeur
palatin l'occasion du recours à l'empereur, et tout prétexte à Sa
Majesté Impériale à l'égard des électeurs de Cologne et de Bavière en
faisant de lui-même justice au palatin. Les autres particuliers ne
l'avaient pas de leur côté, ni la considération d'influer rien dans les
affaires.

Il se trouva bientôt que la reconnaissance de l'Angleterre pour
l'Espagne du dernier traité de commerce entre elles, où Philippe V
s'était si légèrement désisté des articles qu'il avait fait ajouter au
traité de paix d'Utrecht, qui grevaient tant le commerce anglais,
n'était qu'en paroles et en compliments. Ils ne cessèrent point
d'insister injustement sur les prétentions qu'il leur plaisait de
former, comme en conséquence de leur traité de
l'\emph{Asiento}\footnote{{[}33{]}} des nègres, en sorte que le roi
d'Espagne se persuadait que le roi Georges avait pris des liaisons
fortes avec ses ennemis, et qu'Albéroni cherchait à découvrir. Cela
n'empêcha pas ce ministre de résoudre la réforme qu'il avait fait agréer
au roi d'Espagne. Ce prince, par ce plan, conservait environ
quarante-trois mille hommes et huit mille chevaux.

Albéroni avait persuadé à la reine d'Espagne de tenir le roi, son mari,
enfermé comme avait fait la princesse des Ursins. C'était le moyen
certain de gouverner un prince que le tempérament et la conscience
attachait également à son épouse, qui par là, comme sa première, le
conduisait toujours où elle voulait, et le meilleur expédient, dès qu'il
s'y abandonnait lui-même, pour n'être pas contredite, et que le roi ne
sût rien de quoi que ce fût que par elle et par Albéroni, qui était la
même chose. Tous les officiers du roi, grands, médiocres et petits,
furent donc écartés, les entrées et les fonctions auprès du roi ôtées.
Il ne vit plus dans l'intérieur que trois gentilshommes de sa chambre,
toujours les mêmes, et encore des moments de services, à son lever, et
peu à son coucher, et quatre ou cinq valets, dont deux étaient Français.
Ces trois gentilshommes de la chambre étaient\,: le marquis de
Santa-Cruz, majordome-major de la reine, très bien avec elle\,; le duc
del Arco, grand écuyer, grand veneur et gouverneur de presque toutes les
maisons royales, que le roi aimait fort, qui ne ploya jamais sous
Albéroni qui ne put jamais l'écarter, qui n'était même point mal avec la
reine, et dont l'esprit doux, sage et médiocre était d'autant moins à
craindre qu'il se bornait à ses emplois, et ne se voulait mêler de rien.
Il était ami intime du marquis de Santa-Cruz, qui avait beaucoup
d'esprit et de politique, et qui haïssait les Français. Le troisième
était Valouse, écuyer particulier de M. le duc d'Anjou, en sortant de
page, qui l'avait suivi en Espagne, et qui était premier écuyer. C'était
un honnête homme, mais fort borné, qui mourait de peur de tout, qui
était toujours bien avec qui gouvernait, aimé du roi, bien avec tout le
monde, attaché au grand écuyer et incapable de se vouloir mêler de la
moindre chose. Je m'étendrai dans un plus grand détail sur cette clôture
intérieure lorsque mon ambassade me donnera lieu de traiter
particulièrement d'Espagne\,; ce détail, fait ici, détournerait trop. Il
suffit de dire que le roi d'Espagne se laissa enfermer dans une prison
effective et fort étroite, gardé sans cesse à vue par la reine, en trous
les instants du jour et de la nuit. Par là elle-même était geôlière et
prisonnière\,; étant sans cesse avec le roi, personne ne pouvait
approcher d'elle, parce qu'on ne le pouvait sans approcher du roi en
même temps. Ainsi Albéroni les tint tous les deux enfermés, avec la clef
de leur prison dans sa poche.

Néanmoins il ne put d'abord exclure absolument le cardinal del Giudice,
qui était grand inquisiteur, gouverneur du prince des Asturies, et qui
végétait encore dans les affaires, où il avait eu autrefois une
direction principale. Le jésuite Daubenton avait aussi nécessairement,
comme confesseur du roi, de fréquentes audiences. On aura tout dit de
lui pour le faire bien connaître en faisant souvenir qu'il avait été
chassé de cette place, qu'il s'était retiré à Rome, qu'il y avait été
fait assistant du général de la compagnie, et que c'était lui seul, et
dans le dernier secret, qui sous les yeux du cardinal Fabroni avait fait
la constitution \emph{Unigenitus}. Quand M\textsuperscript{me} des
Ursins fit renvoyer le P. Robinet, trop homme de bien et d'honneur pour
se maintenir dans la place de confesseur, Rome et les jésuites
n'oublièrent rien pour y faire rappeler le P. Daubenton, qui la reprit,
et qui y porta toute la confiance personnelle du pape, avec lequel il
eut un commerce secret et immédiat de lettres, et qui n'était pas sans
vues, sans projets et sans la plus sourde et forte ambition. Ces deux
hommes incommodaient infiniment Albéroni qui se résolut à perdre le
cardinal, et à subjuguer le jésuite qu'il sentit trop de difficulté à
faire chasser. Ainsi l'abbé Albéroni, simple ministre du duc de Parme, à
Madrid, s'y trouvait en effet premier ministre tout-puissant.

Ce grand crédit et son incertitude sur lequel était fondée sa puissance,
lui fit lever les yeux jusques au cardinalat pour fixer sa fortune. Il
songea donc à se procurer la nomination d'Espagne. Ceux qui
l'approchaient de plus près lui faisaient leur cour de cette idée, et de
le presser d'y travailler. Il en mourait d'envie, mais il ne le pouvait
que par la reine qui, dans ce commencement de ce grand essor, n'ajustait
pas dans sa tête la bassesse de ce favori étranger avec la nomination du
roi d'Espagne, au mépris de tous prétendants. Cette froideur déconcerta
Albéroni, et il ne l'était pas moins du silence à cet égard
qu'Aldovrandi, nonce à Madrid, observait avec lui. On a vu que ce
ministre du pape y était plutôt souffert que reçu\,; la nonciature était
toujours fermée depuis les démêlés des deux cours, et\,; la
reconnaissance forcée de l'empereur comme roi d'Espagne par le pape. Sa
Sainteté prétendait différentes choses de la cour de Madrid, entre
autres la dépouille des évêques d'Espagne\,; et Aldovrandi profitait
doucement et finement de l'ambition du ministre et du confesseur, pour
avancer peu à peu les affaires de son maître.

Les dégoûts accueillirent de plus en plus le cardinal del Giudice.
Aubenton en profita pour donner au prince des Asturies un précepteur de
sa compagnie, qu'il fit venir de Paris. Giudice n'en fut instruit que
deux jours avant son arrivée. On resserra beaucoup le prince des
Asturies en même temps sur les chasses et sur les promenades, dont il
n'eut plus la liberté. Ce dépit, qu'on voulut faire à ses dépens à
Giudice qu'il aimait fort, tourna en fort mauvais discours, et fort
publics, sur les desseins qu'on prêtait à la reine et à son confident.
Ce hardi Italien, ébloui d'une situation si flatteuse, voulut la faire
éclater de plus en plus à Rome pour s'y faire compter et favoriser ses
vues\,; à Madrid pour s'y faire redouter par la montre extérieure de son
pouvoir. Il se fit donc donner la commission secrète de conférer et de
travailler avec le confesseur sur les différends avec Rome, qui
jusqu'alors en était chargé seul, et en même temps ce qui était sans
exemple, un appartement au palais, près de celui de la reine, où les
secrétaires des finances, de la guerre et de la marine eurent ordre
d'aller travailler avec lui, sans la participation du conseil, sur
toutes les affaires de leurs départements, et de ne faire aucune
expédition sans les lui communiquer. Un reste de considération mourante
du cardinal del Giudice en excepta le seul Grimaldo. En cet état,
Albéroni ne doutait de rien. Il comptait d'autant plus sur le
rétablissement des finances que le roi d'Espagne était le seul monarque
qui n'eût point de dettes, parce qu'il n'avait pas eu le crédit d'en
contracter. Il s'assurait sur les compliments des ministres
d'Angleterre, qui ne tenaient à Madrid qu'un secrétaire fort malhabile
et sans expérience, et sur ceux de Riperda qui lui succéda depuis, lors
ambassadeur de Hollande à Madrid, qui n'avait ni estime ni considération
dans sa république, qui, se croisant d'ailleurs, s'unissaient pour
chasser les Français des Indes, et s'en flattaient par la persuasion où
ils étaient que le roi d'Espagne s'éloignait de plus en plus de la
France, et par la facilité d'Albéroni à passer aux Anglais des articles
si favorables au dernier traité de commerce qu'il se disait hautement
qu'il en avait reçu force guinées, que les moins mal intentionnés
l'accusaient de grossière ignorance, et on l'appelait publiquement par
dérision le comte-abbé, par allusion au comte-duc d'Olivarez, qui avait
eu sous Philippe III\footnote{{[}34{]}} la même autorité que celui-ci
exerçait sous Philippe V.

La cour de Londres, inquiète des mouvements domestiques, croyait avoir
intérêt à former des liaisons avec l'Espagne, et caressait Monteléon son
ambassadeur. Wolckra, envoyé de l'empereur, s'en aperçut, et les fit
craindre à Vienne comme peu compatibles avec celles de ces deux cours,
tandis que Stairs ne s'occupait qu'à aigrir les ministres d'Angleterre
contre le régent, dont il interprétait sinistrement toutes les actions,
et lui en supposait même pour assister puissamment le Prétendant, sur
lequel Stanhope se laissa emporter à plus que des plaintes amères. Les
deux partis qui divisaient l'Angleterre s'animaient également contre la
France\,: les torys l'accusaient d'ingratitude par son indifférence pour
le Prétendant\,; les whigs au contraire, de manquer aux paroles données
à l'entrée de la régence en soutenant ce prince de tout son pouvoir, sur
quoi ils s'emportèrent violemment\,; et tinrent dans la chambres des
communes les discours les plus vifs là-dessus. L'Espagne à cette
occasion était aussi louée que la France blâmée, et on redoublait les
protestations d'amitié à Monteléon. On savait que l'empereur était
pressé par plusieurs de ceux qui l'approchaient de plus près, même par
quelques-uns de ses ministres, de porter la guerre en Italie. Ils lui
représentaient qu'il n'en retrouverait jamais une occasion si favorable,
par l'extrême faiblesse de tous les princes d'Italie, qui n'avaient même
aucune préparation de défense\,; et c'était ce nouvel incendie que
Monteléon se crut en situation de prévenir par l'Angleterre. L'empereur
goûtait plus ce projet d'Italie qu'il ne s'en laissait entendre. Il
était armé\,; mais les Turcs, enflés de la conquête de la Morée et de
leurs victoires sur les Vénitiens, le tenaient en respect, tandis que
l'Italie craignait également une invasion de l'empereur, ou une du Turc
approché d'elle par la Morée.

Le traité de la Barrière venait enfin d'être conclu sous la médiation et
la garantie de l'Angleterre, où on ne se contraignait pas de laisser
entendre que, dès que les mouvements d'Écosse seraient finis, la France
verrait éclore des desseins que les divisions domestiques avaient
suspendus. La proposition de la neutralité des Pays-Bas que le régent
avait faite, et qui avait été assez goûtée en Hollande, était également
suspecte à l'empereur et à l'Angleterre. Aussitôt donc qu'elle vit
l'affaire de la Barrière finie, elle proposa aux Hollandais un projet de
renouvellement de leurs anciennes alliances, avec une garantie
réciproque en cas d'agression. En même temps Stairs eut ordre de
travailler auprès du ministre de Sicile à Paris pour engager son maître
dans une ligue contre la France, à quoi il n'épargna pas ses soins. On
découvrait sans cesse les mauvaises intentions de l'Angleterre, et de
nouveaux motif de l'occuper et de souhaiter le succès de l'entreprise du
Prétendant.

Pendant ces diverses intrigues que le régent conduisait de l'oeil pour
en éviter les dangers, et en tire s'il se pouvait quelque avantage, le
pape mourait de peur du Turc. Il s'adressa à l'Espagne et au Portugal
pour obtenir du secours\,; et au milieu de ses rigueurs pour la France,
il n'eut pas honte de lui en faire demander aussi par Bentivoglio, qui
n'oubliait rien pour la brouiller et y mettre le schisme. La vérité
était que jamais les princes d'Italie ne furent plus faibles ni plus
divisés\,; et la république de Venise était brouillée avec la France sur
l'affaire des Ottobon, et avec l'Espagne pour avoir reconnu l'empereur
en qualité de roi de cette monarchie.

Les plaintes contre l'administration d'Albéroni étaient infinies\,: il
était chargé de tout\,; il ne pensait qu'à sa fortune et ne remédiait à
rien. Il est vrai qu'il ne pouvait suffire au poids qui l'accablait, et
que sa jalousie ne lui en permettait pas le partage ni même le
soulagement. Il fallait exécuter la réforme projetée\,; il en craignait
le moment et les cris qu'elle exciterait contre lui. Il éloigna les
officiers de Madrid, et engagea le roi à écrire de sa main tout le plan
de la réforme, pour lui donner, disait-il, plus de poids, en effet, s'il
l'eut pu, pour se cacher et la faire passer pour son ouvrage. Elle parut
à la fin de janvier, et souleva non seulement les intéressés, mais leurs
parents et leurs amis.

Le duc de Popoli, capitaine de la compagnie des gardes du corps
italienne, parle fortement en faveur des deux compagnies des gardes du
corps réformées, et des officiers qu'on réformait dans les deux que l'on
conservait. Le duc d'Havré, colonel du régiment des gardes wallonnes, en
avait {[}fait{]} autant sur les bataillons qu'on en réformait\,; et ces
deux seigneurs avaient déclaré au roi d'Espagne que, en conservant une
aussi faible garde, il les mettait hors d'état de pouvoir répondre de sa
personne, et le marquis de Bedmar, chargé des affaires de la guerre, les
avait fort soutenus, et le prince Pio cria tant qu'il put de Barcelone,
où il commandait en Catalogne. Il est pourtant vrai que les Espagnols,
qui n'avaient jamais vu de compagnies ni de régiments des gardes à leurs
rois avant celui-ci, et qui étaient fâchés de le voir armé et par là
plus autorisé, avaient habilement flatté l'épargne d'Albéroni pour le
confirmer à faire cette réforme. Le duc d'Arcos et le marquis de
Mejorada en furent les principaux instigateurs. On remarqua plusieurs
grands qui ne venaient presque jamais au palais s'y rendre assez
fréquemment, n'y parler à pas un étranger\,: et on s'aperçut que cette
faction espagnole mourait d'envie du rappel des exilés, et de se
délivrer de tous ces étrangers, Italiens, Wallons, Irlandais, etc. Ils
s'assemblaient là-dessus entre eux, et ils entretenaient des
correspondances secrètes avec les Espagnols retirés à Vienne, même avec
quelques-uns qui entraient dans les conseils de l'empereur.

Le duc de Saint-Aignan, touché du préjudice que le service du roi
d'Espagne souffrait, lui représenta fortement qu'une résolution de cette
conséquence, et dans la conjoncture des grands armements de l'empereur
et des dispositions visibles de l'Angleterre n'aurait pas dû être prise
sans la participation de la France. Il proposa une suspension de trois
mois\,; et quoiqu'en effet il n'eût reçu aucun ordre là-dessus, il fit
entendre qu'il ne parlait pas de son chef. Cette représentation réussit
fort mal et demeura sans réponse\,; mais le prince de Cellamare eut
ordre d'exposer au régent le plan de la réforme, de lui faire entendre
qu'elle ne tombait que sur les états-majors\,; que le nombre de troupes
demeurait le même, parce qu'elles n'étaient pas complètes\,; et de
demander un ordre du roi au duc de Saint-Aignan de s'abstenir de se
mêler du détail et de l'intérieur du gouvernement d'Espagne, comme
lui-même, de sa part, ne s'était point mêlé du changement fait dans le
gouvernement à la mort du roi, ni de la réforme des troupes que le
régent avait réglée. On attribuait moins les démarches de Saint-Aignan à
des ordres reçus de les faire qu'à des liaisons particulières avec des
seigneurs et des dames du palais intéressés pour leurs parents, et
{[}à{]} son intimité avec Hersent, \emph{guardaropa} du roi d'Espagne,
homme d'esprit, de conduite, de mérite, que le roi avait donné à son
petit-fils en partant de France. C'était un homme d'honneur, haut sans
se méconnaître, fort au-dessus de son état par ce qu'il valait, très
bien et librement avec le roi d'Espagne, qui se faisait compter, qui
avait des amis considérables, et qui prenait grande part à cette réforme
parce qu'il avait ses deux fils capitaines dans le régiment des gardes
wallonnes, qui avaient de l'honneur et de la valeur et qui y étaient
considérés.

Albéroni s'aigrit d'autant plus fortement contre le duc de Saint-Aignan
qu'il mourait de peur des menaces publiques des réformés, qui ne se
prenaient qu'à lui de leur malheur, et qui ne le menaçaient pas moins
que de le pendre à la porte du palais, et les moins emportés de le rouer
de coups de bâton. Il se résolut donc à un coup d'éclat. Il fit exiler
le duc d'Havré, donner le régiment des gardes wallonnes au prince de
Robecque, et ôter la place de dame de palais de la reine à sa femme,
fille de la duchesse Lanti, sœur de la princesse des Ursins qui l'y
avait mise. Ils se retirèrent en France et dans leurs terres. Le marquis
de La Vère, lieutenant-colonel et officier général, frère du prince de
Chimay, et grand nombre d'officiers distingués de ce régiment, du nombre
de ceux qui n'avaient pas été réformés, quittèrent\,; et le cadet des
fils d'Hersent, qui avait été un des députés de ce corps à Albéroni, fut
arrêté, et conduit à Ségovie, et très resserré en prison, puis exilé,
après envoyé dans un cachot à Mérida, sous de fausses accusations
qu'Albéroni ne voulut jamais être jugées, et sans que jamais son père
pût l'en faire sortir. Il trouva enfin, au bout de plusieurs mois, la
liberté, par la disgrâce d'Albéroni, de gagner le Portugal et de
repasser en France, où il a servi depuis. Son père ne le pardonna pas à
Albéroni.

Ce ministre, voyant les affaires du Prétendant tourner mal en Écosse,
arrêta les secours d'argent qu'il avait commencé à lui faire payer.
Monteléon, apprenant les plaintes générales et les soupçons des secours
fournis au Prétendant, contenus dans la harangue du roi d'Angleterre au
parlement eut hardiment là-dessus une explication avec Stanhope, qui
l'assura de la satisfaction du roi Georges de la conduite du roi
d'Espagne à cet égard et de son désir de la reconnaître, jusqu'à
promettre de ne prendre jamais d'engagements contraires à ses intérêts,
à quoi il ajouta de grandes plaintes contre la France sur le Prétendant.
L'Espagne était toutefois inquiète de l'opinion générale qu'il y avait
une ligue secrète formée entre l'empereur et l'Angleterre, tandis que
les ministres impériaux n'étaient pas moins agités d'une nouvelle union
entre l'Espagne et l'Angleterre, depuis le traité de commerce signé avec
l'Angleterre à Madrid, et n'étaient pas en moindre soupçon des
dispositions intérieures de la Hollande, qui n'était pas sans en avoir
aussi de l'empereur, sur l'exécution du traité de {[}la{]} Barrière, et
si alarmée des bruits répandus d'une prochaine rupture de l'Angleterre
avec la France, qu'elle s'excusait déjà d'y entrer sur l'épuisement où
la dernière guerre l'avait mise. Le Prétendant avait repassé la mer avec
le duc de Marr\,; le roi Georges paraissait plus affermi que jamais, et
Stairs n'oubliait rien pour l'animer contre la France, jusqu'aux plus
grossiers mensonges, tels que celui-ci\,:

Le secrétaire d'Angleterre à Madrid eut ordre de confier au roi
d'Espagne que le régent avait voulu faire entendre à Stairs que
l'Espagne avait fait plus que la France en faveur du Prétendant, mais
que le roi d'Angleterre avait tant de confiance en l'amitié et en la
bonne foi du roi d'Espagne, qu'il l'avertissait des soupçons que le
régent tâchait de lui inspirer. En même temps les Anglais cherchaient à
concilier et à attacher le roi de Sicile à l'empereur. Les ministres
anglais, qui désiraient le renouvellement de la guerre avec la France,
ne laissaient pas d'y être embarrassés dans la crainte domestique du
mécontentement général des peuples d'Angleterre, et de ce qui fumait
encore en Écosse. Ils craignaient encore l'effet que produiraient enfin
en France les plaintes sans fin de leur ambassadeur, et ses mémoires
menaçants présentés coup sur coup au régent, ils n'en étaient que plus
déterminés à rechercher l'amitié de l'Espagne, et tous les moyens de
semer la division entre elle et la France. Stanhope, pour confirmer la
confidence qu'il avait fait faire au roi d'Espagne, montra à Monteléon
une lettre de Stairs, qui rapportait les termes suivants, qu'il
prétendait avoir entendus du régent, et qu'il lui dit\,: \emph{Enfin,
monsieur, vous voilà amis de l'Espagne\,; cependant je vous assure que
le roi d'Espagne a fait pour le Prétendant ce que moi je n'ai pas voulu
faire}. Monteléon répondit que ce propos lui paraissait incroyable,
qu'il y soupçonnait plus de malice que de vérité, néanmoins qu'il en
rendrait compte au roi son maître, et qu'il priait Stanhope d'en écrire
à l'agent d'Angleterre à Madrid. Toutefois il ne laissa pas de recevoir
assez d'impression de cette confidence pour se resserrer beaucoup avec
d'Iberville, que le régent tenait à Londres, avec ordre de lui
communiquer tous ses ordres, et de le consulter sur tout, quoique
d'ailleurs ils fussent amis, et de se prendre de plus en plus aux
cajoleries de Stanhope, qui l'assurait ainsi que les ministres allemands
du roi d'Angleterre, que quoi qu'en publiassent les bruits publics, ils
ne voulaient point de guerre avec la France, mais conserver un bon pied
de troupes et de vaisseaux\,; en même temps ils ne laissaient point de
travailler à unir le roi de Sicile à l'empereur par un traité.

Après avoir été longtemps, eux et Trivié, ambassadeur de Sicile à
Londres, à qui parlerait le premier, Stanhope s'étendit sur le préjudice
que la Sicile causait à la maison de Savoie, et montra ainsi à dessein
que le premier article qui serait demandé par l'empereur serait la
cession de cette île. Trivié, qui n'avait point douté de ce projet, cria
bien haut, mais en ministre d'un prince faible, qui pourtant ne veut pas
se laisser dépouiller\,; il en prit occasion de s'éclaircir de la
situation de l'Angleterre avec l'empereur, sur quoi Stanhope répondit
qu'elle en était fort recherchée, mais qu'il n'y avait rien de conclu
entre eux. Les menaces anglaises de rompre avec la France, en traitant
avec l'empereur, aboutirent pourtant à suspendre une levée ordonnée de
seize régiments, et l'armement de douze vaisseaux de guerre, et à écrire
dans toutes les cours pour leur demander de refuser tout asile et
retraite au Prétendant dans leurs États. Le roi d'Espagne refusa
retraite et secours à ce malheureux prince, à qui il en avait assez
libéralement fourni dans l'espérance de succès. Cellamare en parla au
régent qui approuva cette dernière résolution de l'Espagne à cet égard,
qui n'était pas en état de se brouiller, ni de soutenir une guerre
contre l'Angleterre qui cultivait toujours Sa Majesté Catholique, et
avait toujours fait semblant d'ignorer qu'elle eût secouru le
Prétendant.

Les étrangers s'apercevaient et déploraient même le mauvais état de
l'Espagne et de son gouvernement\,; ils regardaient le roi d'Espagne
comme le plus faible de ceux qui avaient porté cette couronne, Albéroni
comme maître à baguette, uniquement attentif à s'enrichir et à s'élever,
très indifférent aux intérêts de l'État qu'il gouvernait. Ils avaient
beaucoup rabattu de l'opinion qu'ils avaient prise de l'esprit et des
talents de la reine\,; sa nourrice, qu'elle avait fait venir de Parme
depuis quelques mois, alarmait infiniment Albéroni, qui ne voulait
partager la confiance avec personne. Il n'était guère moins inquiet sur
le P. Daubenton, aussi ambitieux et plus pénétrant que lui, et tous deux
cherchaient à se concilier la faveur de Rome. Vers le milieu de février,
Albéroni déclara au nonce que le roi d'Espagne secourrait le pape,
contre l'invasion qu'il craignait des Turcs, de sic vaisseaux de guerre,
quatre galères, douze bataillons faisant huit mille hommes, les
officiers compris, et de quinze cents chevaux\,; que ces troupes
seraient sous les étendards du pape, commandées par deux lieutenants
généraux, qui obéiraient au général de Sa Sainteté, lesquelles seraient
aux frais du pape, dès qu'elles lui seraient livrées armées, et les
cavaliers montés. Le roi d'Espagne se chargeait des frais de la marine,
et quant au transport des troupes de Barcelone à Civita-Vecchia, il
comptait que ce serait par les vaisseaux d'Espagne et de Portugal. Le
rare est qu'Albéroni parlait en même temps aux ministres d'Angleterre et
de Hollande, pour avoir des vaisseaux, et qu'ils en promettaient en
doutant fort que l'intérêt du commerce de Levant permit à leur maître
d'en fournir.

Le roi Jacques, caché près de Paris, hors d'espérance de tout secours de
la part du régent, essaya encore de toucher l'Espagne\,; il obtint avec
peine de Cellamare une entrevue secrète avec lui dans un coin du bois de
Boulogne. Là il lui fit une peinture vive et touchante de sa situation,
de son embarras sur le lieu de sa retraite et sur les moyens de
subsister, rejeta le mauvais succès de son entreprise sur la conduite
suspecte de Bolingbroke, qu'il venait de destituer de sa place de
secrétaire d'État, et se plaignit amèrement du duc de Berwick, qui
n'avait jamais voulu passer en Écosse. Il pria Cellamare de ne leur rien
confier de ses affaires, mais d'en conférer seulement avec Magny qu'il
avait choisi. C'était un choix bien étrange, comme on le verra dans la
suite. Ce Magny était fils de Foucault\footnote{}, conseiller d'État
distingué et riche, qui avait eu le crédit de le faire succéder en sa
place. Intendant de Caen, il y avait fait tant de sottises qu'il n'y put
être soutenu, et de dépit et de libertinage avait vendu sa charge de
maître des requêtes, et s'était fait introducteur des ambassadeurs, où
il ne put durer longtemps. Jacques témoigna à Cellamare que sa retraite
à Rome serait fort préjudiciable à ses affaires en Angleterre\,; qu'il
n'espérait plus que le duc de Lorraine voulût le recevoir, laissa
entrevoir, mais sans insister, son désir de l'être en Espagne, dit qu'il
ne voyait qu'Avignon, mais qu'en quelque lieu que ce fût il avait grand
besoin de secours tant pour lui que pour ceux qui avaient tout perdu
pour le suivre. Il finit par demander cent mille écus au roi
d'Espagne\,; Cellamare s'en tira le plus honnêtement qu'il put, mais
sans engagement dont il comprenait les conséquences. Georges demandait
formellement à toutes les puissances de l'Europe de refuser tout secours
et toute retraite à son ennemi et à ses adhérents. Stairs venait de
faire cette demande au régent par un mémoire très fort, et l'agent
d'Angleterre était chargé du même office auprès du roi d'Espagne. La
cour d'Angleterre était d'autant plus vive là-dessus qu'elle connaissait
la mauvaise disposition des peuples et la haine du sang qu'elle avait
répandu\,; ce qui l'engagea à entretenir dans les trois royaumes jusqu'à
trente-cinq mille hommes et quarante vaisseaux de guerre. Dans cette
situation douteuse, le ministre anglais chercha de plus en plus à
s'assurer l'Espagne. Les flatteries et les confidences ne furent pas
épargnées, jusqu'à montrer de la jalousie de la puissance de l'empereur
en Italie\,; et enclins à se liguer avec l'Espagne pour l'empêcher de
s'y étendre, à lui confier que l'Angleterre avait refusé un traité
proposé par l'empereur, parce qu'il y voulait stipuler qu'elle lui
garantirait la Toscane, à la flatter de l'attention à ne rien faire à
son préjudice, enfin à leurrer le roi d'Espagne de ses secours dans les
cas qui pourraient arriver en France, qui donneraient lieu à ses grands
droits.

Rien ne pouvait être plus agréable à la cour d'Espagne que l'alliance
que le roi d'Angleterre lui proposait. Le but véritable du secours
offert au pape était d'avoir un corps de troupes en Italie pour tâcher,
suivant les événements, d'y regagner quelque chose de ce qu'elle y avait
perdu\,; et si le pape, dans la crainte de se rendre suspect, refusait
un si grand secours, il devait être donné aux Vénitiens qui en
demandaient aussi à l'Espagne\,; mais ce qui toucha le plus la reine et
Albéroni, pour ne pas dire le roi d'Espagne, ce fut la corde de ses
grands droits en France adroitement pincée par Stanhope, qui produisit
le plus doux son à leurs oreilles. Quelque intérêt qu'Albéroni parût
avoir de préférer l'Espagne qu'il gouvernait sans obstacle, à la France
où il ne pouvait espérer la même autorité qu'après bien des concurrences
et de dangereux travaux, il ne laissait pas d'être véritable qu'il
exhortait sans cesse le roi d'Espagne à n'abandonner pas le trône de ses
pères, si le roi son neveu venait à manquer, et qu'il n'appuyât ses
raisons de tous les artifices et de toutes les lettres vraies ou fausses
qu'il disait qu'il recevait de France. Il n'inspirait pas ce désir à la
reine avec moins d'application\,; et on peut avancer avec confiance
qu'il y réussit fort bien auprès de l'un et de l'autre. Quelque bien
établie qu'il fût en toute confiance et en toute autorité, il était
alarmé des Italiens, des Parmesans surtout et de la nourrice. Il
n'oubliait rien pour les faire renvoyer sous prétexte de la dépense
qu'ils causaient\,; et la reine s'étant souvenue de quelques-uns qu'elle
eut envie de faire venir, et à plus d'une reprise, il l'empêcha toujours
à son insu, par le moyen du duc de Parme qui le craignait et le
ménageait beaucoup. Il ne perdait point d'occasion de vanter au roi et à
la reine la nécessité et l'utilité de ses conseils\,; et sur l'avis
donné par l'Angleterre du prétendu discours du régent à Stairs sur le
Prétendant, rapporté ci-dessus, Albéroni fit souvenir le roi d'Espagne
du conseil qu'il lui avait donné à la mort du roi son grand-père de ne
se pas fier au régent, mais de se conduire avec lui comme s'il devait
être son plus grand ennemi. En même temps il faisait écrire à Son
Altesse Royale que Leurs Majestés Catholiques étaient parfaitement
contentes de ses sentiments, et que lui, Albéroni, n'oubliait rien pour
maintenir une parfaite intelligence entre les deux couronnes. L'union de
l'Espagne et de l'Angleterre, qui se resserrait toujours, inquiéta enfin
l'ambassadeur de Hollande à Madrid, qui comprit que les Anglais y
trouvaient leur compte, et que ce ne pouvait être qu'au préjudice du
commerce des Provinces-Unies. Par cette considération il pressa ses
maîtres de gagner les Anglais de la main, en se hâtant d'achever la
négociation commencée avec l'Espagne pour lui fournir des vaisseaux.

Le roi d'Espagne avait protesté contre la bulle qui révoquait le
tribunal de la monarchie en Sicile. Le roi de Sicile, qui craignait
quelque secrète intelligence entre le pape et l'empereur pour le
dépouiller de cette île, pressait le roi d'Espagne de s'employer plus
fortement à Rome pour ses intérêts. Son ministre s'adressait toujours au
cardinal del Giudice, qui n'avait plus que le nom de premier ministre,
qui ne se contraignit pas de lui répondre qu'il n'avait rien à espérer
de la faiblesse d'un aussi mauvais gouvernement qui, aussi bien que
celui de France, ne se souciait que de demeurer en paix.

Stairs commit en ce même temps une scélératesse complète\,: il manda
faussement au roi son maître que la France armait puissamment pour le
rétablissement du Prétendant, avec tous les détails des ports, des
vaisseaux et des troupes. Ce bel avis mit l'alarme en Angleterre\,; les
fonds publics y baissèrent aussitôt. Le roi d'Angleterre était prêt
d'aller au parlement demander des subsides pour la guerre inévitable
avec la France et la sûreté de l'Angleterre. Monteléon, qui sentit
l'intérêt que l'Espagne avait d'empêcher la rupture de l'Angleterre avec
la France, parla si ferme et si bien à Stanhope, qu'il l'arrêta tout
court\,; que ce ministre, voyant ensuite clairement que cet avis n'avait
point d'autre fondement que la malignité de celui qui l'avait donné,
changea tout a coup de système. Il avait commencé à proposer à Monteléon
une union entre l'Angleterre et l'Espagne pour la neutralité de
l'Italie, et même pour la garantie au roi de Sicile de ce qu'il
possédait en vertu du traité d'Utrecht\,; il sentait le mécontentement
universel qui fermentait dans toute la Grande-Bretagne du gouvernement,
et l'importance de l'affranchir de l'inquiétude des secours que la
France et l'Espagne pourraient donner au Prétendant\,; il revint donc à
souhaiter que la France entrât dans l'union dont on vient de parler, et
{[}voulût{]} se porter en même temps pour garante de la succession à la
couronne de la Grande-Bretagne dans la ligne protestante, conformément
aux actes du parlement. Ainsi la scélératesse de Stairs et cet
infatigable venin qui lui faisait empoisonner les choses les plus
innocentes, et controuver les plus fausses pour brouiller la France avec
l'Angleterre, fit un effet tout opposé à ses intentions\,; et cette
époque fut le commencement du chemin de l'union tant souhaitée par
l'abbé Dubois entre la France et l'Angleterre, et la base première de la
grandeur de cet homme de rien, qui en sut très indignement profiter pour
l'État, et très prodigieusement pour sa fortune. Stairs présenta un
mémoire de différents griefs, qui, excepté les secours à refuser au
Prétendant, n'étaient pas grand'chose. Le mémoire fut répondu de manière
qu'on en fut content en Angleterre\,; ce qui fit tomber la pensée qu'on
y avait eue de prendre le roi d'Espagne pour médiateur de ces petits
différends.

Un autre bruit aussi malicieux fut répandu en même temps à Paris, dans
le dessein sans doute d'examiner l'impression qu'il ferait. On parlait
d'un traité fort secret, signé par le prince Eugène et le maréchal de
Villars, qui seuls en avaient eu la conduite, qui annulait les
renonciations du roi d'Espagne à la couronne de France, et qui en ce cas
assurait celle de l'Espagne au roi de Sicile. Ce bruit était fomenté
avec soin\,; le régent n'en prit pas la plus légère inquiétude\,; mais
on remarqua {[}que{]} Leurs Majestés Catholiques parurent depuis bien
plus attentives à tout ce qui pouvait regarder cette succession.

Le roi d'Angleterre, toujours inquiet de sa situation domestique, fit
deux propositions aux Hollandais, l'une de fortifier et de rendre plus
nombreuse la garantie de la succession au trône de la Grande-Bretagne
dans la ligne protestante, l'autre de s'expliquer sur l'alliance
défensive à faire entre l'empereur, l'Angleterre et les États généraux.
Ils répondirent sur le premier qu'ils verraient avec plaisir la garantie
fortifiée par d'autres princes, et qu'ils étaient disposés à entrer avec
Georges dans le concert de la manière dont ce projet pourrait
s'exécuter. La seconde leur parut très délicate pour le repos de
l'Europe, et en particulier sur les intérêts du roi d'Espagne. Ils se
tinrent d'autant plus réservés que Walpole montrait plus de chaleur sur
cette affaire à la Haye, et que le résident de l'empereur cabalait
ouvertement dans le même esprit à Amsterdam. Ils ne songèrent donc qu'à
éluder et à gagner du temps, et répondirent qu'ils en délibéreraient, et
en diraient après plus particulièrement leur pensée.

Le grand armement des Turcs obligeait cependant l'empereur à se préparer
tout de bon à n'être pas prévenu, et jetait l'Italie dans l'effroi. Le
pape sans défense et sans moyens sollicitait des secours de France et
d'Espagne\,; en même temps il craignait encore plus l'empereur. Il
savait que ce prince ne consentirait jamais, sous quelque prétexte que
ce pût être, de laisser entrer des troupes françaises ou espagnoles en
Italie\,; ainsi le pape refusa celles qui lui furent offertes, et
demanda des vaisseaux et des galères dont l'empereur ne pouvait prendre
d'ombrage.

Quelque satisfaction que la cour d'Angleterre eût témoignée de la
réponse du régent au mémoire de Stairs, dont on vient de parler,
l'animosité nourrie par cet ambassadeur se manifestait encore. Le roi de
Sicile, qui n'avait pu tirer aucune protection du roi d'Espagne à Rome,
qui lui-même avait plusieurs grands démêlés avec cette cour, en chercha
en Angleterre pour son accommodement avec l'empereur qui était toujours
suspendu. Trivié, son ambassadeur à Londres, y employa Monteléon auprès
de Stanhope, parce qu'il l'en voyait toujours fort caressé, et le
ministre anglais entra en matière avec le Piémontais. Ce dernier fut
étrangement surpris quand après les compliments et les préfaces
ordinaires il entendit Stanhope lui déclarer que la Sicile arrêterait
toujours tout accommodement\,; lui vouloir persuader après que cette île
était à charge à la maison de Savoie, enfin revêtir le personnage du
ministre de l'empereur et lui proposer en échange la Sardaigne pour
conserver à son maître la dignité royale. Trivié répondit qu'il ne
pouvait négocier sur une condition qu'il était sûr que son maître
n'accepterait jamais. Stanhope entreprit de lui démontrer la facilité
que l'empereur avait de se tendre maître de la Sicile, lui dit que
l'affaire serait déjà faite si le roi d'Angleterre eût seulement
consenti à le laisser agir\,; qu'il s'y était opposé jusqu'alors, et
tout nouvellement encore. Trivié pria Stanhope de se souvenir qu'il n'y
avait que cinq ou six mois qu'il lui avait dit qu'il ne tenait qu'à la
France et à l'Espagne que l'Angleterre n'eût moins de déférence pour
l'empereur, d'où il lui demanda pourquoi donc ils déféraient tant à la
cour de Vienne.

Stanhope répliqua que les choses étaient changées\,; qu'alors ils
avaient lieu de croire que le régent voulait vivre en parfaite
intelligence avec le roi d'Angleterre, mais que depuis ils ne le
pouvaient regarder que comme un ennemi caché, incapable de repos,
toujours prêt à exciter des troubles dans la Grande-Bretagne, à y faire
tout le mal qu'il pourrait à la maison régnante, dont le remède était à
former une ligue contre elle où le roi de Sicile entrât pour terminer
par là ses différends avec l'empereur. Il ajouta qu'il n'y aurait point
de guerre en Hongrie cette année, mais ailleurs\,; n'oublia rien pour
persuader Trivié des grands avantages que le roi de Sicile retirerait
d'une guerre contre la France, étant soutenu d'aussi puissants alliés,
lui fit valoir le service que l'Angleterre lui avait rendu en arrêtant
l'empereur jusqu'alors sur la Sicile, lui déclara que si le roi de
Sicile hésitait encore, le roi d'Angleterre ne pourrait plus empêcher
l'empereur d'exécuter ses projets. Trivié tacha inutilement de lui
rendre suspecte pour l'Angleterre même la puissance de la maison
d'Autriche. Stanhope voulait susciter de puissants ennemis à la France,
et n'en trouvait point de plus dangereux à porter la guerre dans
l'intérieur du royaume que le duc de Savoie par sa situation. Il
craignit en même temps que les ministres de France et d'Espagne, que
Trivié voyait souvent, ne traversassent son projet, et mit tout en œuvre
pour les lui rendre suspects. Monteléon bien qu'amusé par l'apparente
confiance et les caresses de Stanhope et par l'espérance d'une ligue
défensive de l'Espagne avec l'Angleterre et la Hollande, avait pénétré
qu'il se traitait une alliance défensive entre ces deux dernières
puissances et l'empereur, et que la conclusion n'en était arrêtée que
par l'espérance de l'Angleterre de rendre cette ligue offensive.
Néanmoins les affaires domestiques de l'Angleterre ne lui permettaient
pas de songer tout de bon à l'offensive. Le ministre impérial à Londres
s'en plaignit, et embarrassa. Le roi d'Angleterre ne regardait point sa
couronne comme un bien solide\,; ses États d'Allemagne l'occupaient bien
autrement\,; par cette raison il voulait plaire à l'empereur, et le
mettre en état d'agir lorsque l'intérêt commun des puissances, engagées
dans la dernière ligue contre Louis XIV et Philippe V, demanderait
qu'elles se réunissent et reprissent les armes. Il prenait tous les
soins à lui possibles pour détourner le Grand Seigneur de faire la
guerre à l'empereur, que le grand vizir et le prince Eugène voulaient,
que presque tous les ministres impériaux, surtout les Espagnols,
craignaient, et que le mufti détournait. Le prince Eugène prétendait que
si l'empereur différait à attaquer les Turcs lorsqu'il le pouvait avec
avantage, il le serait lui-même par eux l'année suivante avec un grand
désavantage.

Cette attention prépondérante du roi d'Angleterre pour ses États
d'Allemagne l'occupait fort de la guerre du nord et de chasser les
Suédois de ce qui leur restait dans l'empire. De toutes leurs anciennes
conquêtes ils n'avaient conservé que Wismar. Il fut donc résolu en
Angleterre d'envoyer vingt vaisseaux presser la reddition de cette
place, auxquels les Hollandais en joignirent douze des leurs. C'était
bien plus qu'il n'en fallait pour accabler les Suédois dans la réduction
déplorable où ils étaient\,; mais le gouvernement d'Angleterre faisait
toujours semblant de craindre un secours que le régent n'était ni en
volonté ni en pouvoir de donner. Ce n'était pas que les ministres
anglais et allemands pussent douter de ses intentions, mais il était de
l'intérêt de ce ministère de maintenir les alarmes d'une guerre
prochaine avec la France, pour continuer d'obtenir des subsides du
parlement, qu'il aurait refusés dans une paix bien assurée. Ainsi bien
servis par Stairs pour continuer les défiances et les jalousies, il leur
mandait faussement que le régent lui avait promis de chasser tous les
Anglais rebelles et qu'il manquait à sa parole, et leur suggérait de
solliciter Son Altesse Royale de poursuivre le Prétendant jusque dans
Avignon, et d'obliger le pape à l'en faire sortir s'il s'y voulait
retirer. En même temps ils ne pouvaient ignorer les secours que
l'Espagne avait donnés à cet infortuné prince\,; mais résolus de
l'ignorer, ils n'épargnaient aucunes assurances de l'amitié et de
l'union la plus intime avec elle. Le roi d'Angleterre déclara qu'il se
croyait comme engagé par le traité d'Utrecht à garantir la neutralité de
l'Italie, et qu'il était disposé à former de nouvelles liaisons avec le
roi d'Espagne pour la maintenir, et de plus pour confirmer et renouveler
toutes les alliances précédentes. Monteléon profita de tant
d'empressement extérieur pour parler à Stanhope de la triple alliance
proposée par l'Angleterre entre elle, l'empereur et la Hollande, dont
Walpole avait depuis peu présenté le projet aux États généraux.

Stanhope ne put désavouer un fait public, mais il assura Monteléon que
ce projet n'avait rien de contraire aux traités de paix, aux intérêts du
roi d'Espagne, ni au renouvellement proposé entre l'Angleterre et
l'Espagne des anciennes alliances, ni à prendre avec elle un nouvel
engagement pour la neutralité de l'Italie. Il lui fit valoir le refus de
l'Angleterre à d'autres propositions que l'empereur lui avait faites, et
finit par beaucoup d'aigreurs et de plaintes contre la France, qu'il dit
chercher à négocier avec l'Angleterre, laquelle ne l'écouterait point
qu'elle n'eût des preuves de sa sincérité, et qu'elle ne sût ce que le
Prétendant deviendrait et ceux qui suivaient sa fortune. Stanhope tirait
ainsi avantage de la disposition de la France à conserver la paix, et de
ce qu'elle avait agréé les offres que lui avait faites Duywenworde de
travailler au rétablissement d'une parfaite intelligence entre elle et
l'Angleterre, laquelle en même temps recherchait le roi d'Espagne, au
point que Monteléon lui manda qu'il dépendait de Sa Majesté Catholique
de faire seule une alliance avec l'Angleterre ou d'y faire comprendre la
France.

Parmi tant de mouvements contraires et de propositions trompeuses, les
ministres d'Angleterre étaient fort occupés au dedans. Leur parti whig,
qui avait triomphé des torys par la mort de la reine Anne et la faveur
de Georges son successeur, craignait la vengeance de la tyrannie qu'il
avait si cruellement exercée, si le parti opprimé, soutenu du
mécontentement général du gouvernement, reprenait le dessus. Le
parlement rendu triennal n'avait plus qu'une année à durer\,; il était
de l'intérêt des ministres de le prolonger encore de quelques années, en
quoi s'accordait celui de la chambre basse, dont les membres continués
épargnaient les brigues et l'argent d'une autre élection. Celle des
seigneurs y était opposée, parce que, ne craignant point de changement
pour elle, la plupart en désiraient dans celle des communes contre le
gouvernement présent\,; mais en Angleterre comme dans les autres pays,
ce n'était plus le temps des seigneurs. Les ministres et les principaux
de leurs amis des communes travaillaient donc de concert à cette grande
affaire, qui absorbait presque toute l'application des ministres, parce
que les autres affaires n'étaient que celles de l'État et que celle-ci
était la leur même, et la plus importante à la conservation de leurs
places et de leur autorité. C'était aussi la principale du roi
d'Angleterre. Leur projet était de faire passer un acte de prolongation
du parlement pour quatre années\,; mais ils voulaient être certains d'y
réussir avant de le présenter.

\hypertarget{chapitre-xix.}{%
\chapter{CHAPITRE XIX.}\label{chapitre-xix.}}

1716

~

{\textsc{Le régent ne peut être dépris de l'Angleterre.}} {\textsc{-
Scélératesse de Stairs et de Bentivoglio.}} {\textsc{- Sa faiblesse à
leur égard\,; comment conduite.}} {\textsc{- Le parti de la constitution
n'oublie rien pour me gagner, jusqu'à une tentation horrible.}}
{\textsc{- Conduite du duc de Noailles avec moi, et de moi avec lui.}}
{\textsc{- Le cardinal de Noailles bénit la chapelle des Tuileries.}}
{\textsc{- Mort du duc d'Ossone.}} {\textsc{- Entreprises du grand
prieur à la fin arrêtées\,; se plaint de moi inutilement.}} {\textsc{-
Je l'empêche d'entrer dans le conseil de régence.}} {\textsc{- Mort de
la duchesse de Béthune\,; son état.}} {\textsc{- Mort de l'abbé de Vassé
et du chevalier du Rosel, et de Fiennes, lieutenants généraux.}}
{\textsc{- Mort de Valbelle et de Rottembourg, et du duc de Perth.}}
{\textsc{- La Vieuville se remarie.}} {\textsc{- Forte scène entre le
prince et la princesse de Conti.}} {\textsc{- M\textsuperscript{me} la
duchesse de Berry mure les portes du jardin de Luxembourg, et fait
abréger les deuils.}} {\textsc{- Elle est la première fille de France
qui souffre dans sa loge les dames d'honneur des princesses du sang, et
fait La Haye gentilhomme de la manche du roi.}} {\textsc{- Vittement
sous-précepteur du roi.}} {\textsc{- Elle achète la Muette
d'Armenonville, qui en est bien récompensé.}} {\textsc{-
M\textsuperscript{me} la princesse de Conti, première douairière, achète
Choisy.}} {\textsc{- M. le duc d'Orléans achète pour le chevalier
d'Orléans la charge de général des galères\,; donne au comte de
Charolais soixante mille livres de pension\,; fait revenir les comédiens
italiens.}}

~

Quelque soin que prit Stairs de cacher ses scélératesses en France, de
voiler et d'affaiblir celles dont il ne pouvait dérober la connaissance,
il n'évita pas d'y passer pour un brouillon qui abusait de son
caractère, et d'y être fort haï, à quoi son air audacieux ajoutait
encore\,; mais il fut heureux au Palais-Royal\,; ce triumvirat, qu'il
avait captivé, aurait cru se faire tort de revenir à son égard sur
soi-même. Dubois à toute reste\footnote{{[}36{]}} voulait percer par
l'Angleterre, parce qu'il ne s'en voyait pas d'autre moyen\,; Noailles,
qui avait compris de bonne heure que cet homme-là, tôt ou tard,
reprendrait auprès de M. le duc d'Orléans, s'était fait un principe de
se le dévouer tandis qu'il avait besoin de lui, de ne le jamais
contredire, d'être toujours prêt à l'aider en tout pour le retrouver
après à son tour\,; et Canillac, incapable de la même souplesse, mais
sans aucun jugement, demeurait dans son premier engouement, nourri par
les déférences et les admirations de Stairs pour lui. Longepierre, fade
savantasse, mais dont les louanges avaient épris le duc de Noailles,
insinué chez Stairs par Rémond, et Rémond lui-même, trouvaient leur
compte à se mêler des messages des uns aux autres et s'en croyaient
importants, tellement que le régent eut beau voir clair dans la conduite
de Stairs et de ses maîtres, il n'eut pas la force de secouer cette
pernicieuse maxime des deux usurpateurs qu'on lui avait inculquée, ni de
résister aux discours continuels de ces trois hommes, qui de concert,
tantôt ensemble, tantôt séparément, le tenaient toujours en haleine et
mettaient un obstacle continuel à tout ce qui n'était pas dans leurs
vues par rapport à Stairs et à l'Angleterre. J'eus souvent des prises
là-dessus avec le régent si j'avais moins connu sa faiblesse, j'aurais
souvent espéré le faire changer de boussole\,; mais je n'étais qu'un
contre trois, dont l'assiduité successive renversait aisément tout ce
que j'avais dit, démontré, même persuadé, et le régent contre son gré
flottant était toujours raccroché par eux. Il s'en dédommageait par des
brocards sur eux, auxquels Dubois était accoutumé, et dont Noailles ne
faisait que secouer les oreilles, mais dont l'orgueil de Canillac était
souvent blessé. Le régent le laissait bouder, riait et quelquefois après
le caressait, tant son jargon important l'avait accoutumé à le
considérer.

Stairs et Bentivoglio étaient deux têtes brûlées qui, pour leur fortune,
n'avaient rien de sacré, et ne travaillaient qu'à culbuter la France\,;
et si l'un des deux était plus corrompu, plus noir, plus scélérat que
l'autre, c'était assurément Bentivoglio\,; tous deux imposteurs publics
assez pris sur le fait, assez connus, assez déshonorés jusque dans leurs
propres cours, où ils avait perdu croyance pour qu'elles ne puissent
refuser leur rappel s'il était demandé avec quelque force. Mais si
Stairs était à l'abri par ses trois protecteurs déclarés, Bentivoglio
n'en avait pas de moins bons\,: Effiat, sans croire en Dieu, lui était
vendu, et il imposait à son maître. La faiblesse de ce prince craignait
le maréchal de Villeroy et les cardinaux de Rohan et Bissy, ses ardents
et très intéressés protecteurs. Je parle des cardinaux, car le maréchal,
ce n'était que par sottise d'habitude du feu roi. Ainsi le régent, sous
le nom et le caractère de nonce du pape et d'ambassadeur d'Angleterre,
conserva près de lui les deux plus grands et plus dangereux boute-feu,
et les deux plus grands ennemis que la France et sa personne pussent
avoir. On en verra quelques traits de cet infâme nonce, qui n'était
point honteux d'entretenir une fille de l'Opéra, dont il eut deux filles
qui y entrèrent depuis, si publiquement connues pour telles, qu'on ne
les nomma jamais que la Constitution et la Légende.

Si j'avais grossi ces Mémoires de ce qui s'est passé en détail sur la
constitution pendant la régence et la nonciature de Bentivoglio, ce
n'est point employer un terme trop fort que dire, et dans toute son
étendue, que les cheveux se dresseraient dans la tête à la lecture de la
conduite véritable et journalière de Bentivoglio. Il était encore
soutenu par l'ancien évêque de Troyes, qui avait pensé tout différemment
autrefois, mais que son ami le maréchal de Villeroy, les Rohan et la
cabale avait su retourner, et qui s'en croyait plus à la mode d'une
part, plus compté de l'autre.

Ce parti, dès aussitôt après la mort du roi, avait travaillé à me
gagner, du moins à ne m'avoir pas contraire. Il n'ignorait pas mes
sentiments par le P. Tellier, à qui je ne les avais pas\,; cachés\,; on
a vu en leur temps ce qui s'est passé là-dessus entre lui et moi. Le
cardinal de Bissy, et quelque temps après le prince et le cardinal de
Rohan, tous deux ensemble, m'en parlèrent. Je répondis civilement et
modestement. Je dis que je n'étais point évêque, et aussi peu docte ou
docteur\,; je me battis en retraite de la sorte. Cela ne les contenta
pas. Le duc de La Force, de tout temps livré aux jésuites à l'occasion
de sa conversion, en effet pour plaire au feu roi, et s'en approcher
s'il eût pu, était par même raison initié avec les cardinaux de Rohan et
de Bissy, et les chefs accrédités de leur parti. Ils me le détachèrent
pour faire un dernier effort. Ce n'était pas que j'eusse levé aucun
étendard sur cette affaire\,; je me contenais même tout à fait dans les
bornes où doit s'arrêter un homme en situation de parler et de dire son
avis au conseil de régence, ou en particulier au régent\,; mais ils
savaient, dès le temps du feu roi, sur quoi compter là-dessus par la
raison que je viens de dire, et ils étaient alarmés de ma liaison avec
le cardinal de Noailles. La force argumenta avec moi sur le fond de la
matière. Il savait et débitait bien ce qu'il savait\,; mais comme la
politique était sa religion, et que, pour persuader, il faut être
persuadé soi-même, ce n'est pas merveille s'il n'y put réussir avec moi.

À bout enfin de raisons et de raisonnements, il se jeta sur l'intérêt
présent et futur du régent de ménager Rome, les jésuites, le grand
nombre des évêques, et s'étendit beaucoup là-dessus. Mais comme la
politique et l'intérêt ne peuvent jamais être mis en la place de la
religion et de la vérité, sa politique fut aussi vaine avec moi que sa
doctrine. Ne sachant plus que faire, il en vint à un argument \emph{ad
hominem}, dont j'ai su depuis que ceux qu'il servait, et lui-même,
avaient tout espéré. Il me dit qu'il avouait qu'il ne me comprenait
point, et qu'il ne pouvait allier mon esprit avec ma conduite\,; que
j'étais ennemi du duc de Noailles sans mesure, sans ménagement, sans
pouvoir être adouci par tout ce qu'il ne se laissait point d'employer
pour cela\,; que je m'en piquais même\,; que je lui rompais en visière à
tous moments en plein conseil de régence, et partout où je le pouvais
rencontrer\,; et que tandis que je ne me cachais pas du désir que
j'avais de le perdre, j'en négligeais le moyen sûr que j'en avais en
main\,; et que j'étais l'ami et le soutien du cardinal de Noailles. Je
demandai à La Force quel était donc ce moyen sûr de perdre le duc de
Noailles, et je l'assurai qu'il me ferait grand plaisir de me
l'apprendre. «\,Perdre, me répondit-il, son oncle\,; et il ne tient qu'à
vous en vous tournant au parti contraire. L'oncle perdu, le neveu tombe
nécessairement avec lui, et vous êtes vengé.\,» L'horreur me fit monter
la rougeur au visage. «\,Monsieur, lui répondis-je vivement, est-ce
ainsi que se traitent des affaires de religion\,? Persuadez-vous bien
une fois pour toutes, et le dites nettement à vos amis, que, quelque
certain que je pusse être de la chute totale et sans retour du duc de
Noailles en arrachant seulement un cheveu de la tête de son oncle, il
serait de ma part en pleine sûreté. Non, monsieur, encore une fois,
ajoutai-je avec indignation, j'avoue qu'il n'est rien d'honnête à quoi
je ne me portasse pour écraser le duc de Noailles\,; mais de le tuer à
travers le corps du cardinal de Noailles, il vivra et régnera plutôt
deux mille ans.\,» Le duc de La Force me parut confondu, et depuis cette
réponse, ils n'ont plus songé à me gagner. Je n'en voulus rien dire au
cardinal de Noailles, ni à personne qui pût le lui rapporter.

Il est vrai que ma conduite avec le duc de Noailles allait peut-être
jusqu'à abuser des involontaires remords d'un aussi grand coupable à mon
égard. Nous ne nous rencontrions qu'en nos assemblées sur nos affaires
du parlement, que ses trahisons, et la jalousie ou la sottise de
quelques autres, finirent bientôt, et dont, avant leur fin, mes propos
directs et publics le bannirent, sans qu'il osât jamais me répondre un
mot\,; mais à la dernière, il dit au duc de Charost, près duquel il
était assis, que je le poussais de façon que je l'obligerais d'en avoir
raison l'épée à la main\,: raison, il ne l'a ni eue ni même demandée, et
l'épée est demeurée doucement dans son fourreau. Partout il me saluait
d'une façon très marquée\,; je le regardais un peu hagardement, et
passais sans m'incliner le moins du monde\,; et de part et d'autre cela
se répétait sans jamais y manquer, partout où nous nous rencontrions\,;
quelque accoutumé qu'on y fût, c'était un spectacle. Si je passais près
de lui, il se rangeait aussitôt sans que je daignasse y prendre garde\,;
et jamais nous ne nous parlions qu'en conseil sur les affaires, et tout
haut, devant tout le monde, sèchement et laconiquement de ma part, de la
sienne avec toute la politesse, je n'oserait dire l'air de respect,
l'onction et la circonspection qu'il y pouvait mettre.

Il vint une fois au conseil de régence un jour de conseil d'État, sous
prétexte d'une affaire de finance pressée. Le conseil était un peu
commencé\,; il fit dire au régent qu'il était à la porte\,; il le fit
entrer. Je me levai parce que tout le conseil se leva\,; il s'assit
au-dessous de moi, tout près de moi, et se mit à débiter ce qui
l'amenait, qui n'était pas grand'chose. Comme il achevait, je dis à
l'oreille au comte de Toulouse que je joignais de l'autre côté, que le
duc de Noailles avait pris ce prétexte pour tenter de demeurer au
conseil. «\,Je le croirais bien comme vous, me répondit-il eu souriant.
--- Oh\,! bien, répliquai-je, nous allons voir, laissez-moi faire.\,»
Tout ce qui regardait la finance achevé, le duc de Noailles demeura, et
après quelques moments d'intervalle M. le duc d'Orléans regarda le
maréchal d'Huxelles et lui dit\,: «\,Allons, monsieur, continuons.\,» M.
de Troyes lisait les dépêches pour soulager le maréchal, parce qu'il
avait la voix et la prononciation bonnes, et qu'il lisait fort bien. Il
commença\,; au second mot, je l'interrompis et je lui dis\,: «\,Attendez
donc, monsieur\,; voilà M. de Noailles qui n'est pas sorti.\,» Et je me
tourné tout de suite à regarder le duc de Noailles. M. de Troyes se tut
tout court, et tous les yeux regardaient. Je tournai un peu mon siège
ployant, pour donner plus d'aisance à M. de Noailles pour sortir, qui,
au bout de quelques moments de silence, voyant celui de M. de Troyes et
celui du régent, me tourna le dos avec impétuosité, et, sans saluer
personne, s'en alla. Je regardai M. le comte de Toulouse qui riait, M.
le duc d'Orléans qui ne sourcilla pas, et toute la compagnie qui me
regardait aussi, et qui riait ou souriait. Ce fut après la nouvelle
qu'il avait fait la tentative, et que je l'avais chassé du conseil. Le
comte de Toulouse, M. du Maine, M. le Duc, le maréchal de Villeroy et
quelques autres, m'en parlèrent au sortir de la séance, et approuvèrent
ce que j'avais fait, et moi je les blâmai de ne l'avoir pas fait
eux-mêmes. J'en parlai après au régent, qui n'osa me désapprouver, à qui
je reprochai sa faiblesse, et lui demandai si, pour être du conseil, il
ne tenait qu'à y entrer pour un moment sous quelque prétexte, et avoir
après l'impudence d'y rester.

Une autre fois que c'était {[}conseil{]} de finance, et que le duc de
Noailles y était, toujours auprès et au-dessous de moi, il se mit à
pérorer sur la licence de vendre et de porter des étoffes défendues, sur
le tort que cela faisait aux manufactures du royaume, et s'étendit
surtout avec une emphase merveilleuse sur l'abus de porter des toiles
peintes, dont la mode l'emportait sur toute règle et raison, et que les
plus grandes dames, et toutes les autres à leur imitation et à l'abri de
leur exemple, portaient publiquement et impunément partout, avec le plus
scandaleux mépris public des défenses et des peines portées et si
souvent réitérées\,; conclut enfin avec le même feu d'éloquence à
remédier enfin à un aussi grand mal et si préjudiciable, par des moyens
efficaces, mais sans en expliquer ni en proposer aucun, apparemment pour
éviter la haine du beau sexe. On opina là-dessus, ou plutôt on verbiagea
sans rien dire plus que des mots. Quand ce fut à mon tour, je louai fort
le zèle que témoignait le duc de Noailles pour le soutien des
manufactures de France, et contre l'abus de porter des étoffes
défendues. J'insistai particulièrement sur celui de porter des toiles
peintes, et j'ajoutai même là-dessus à ce que le duc de Noailles en
avait dit. Je fis remarquer avec beaucoup de gravité toute l'importance
d'arrêter une mode si générale, et un mépris des lois porté si loin par
toutes les femmes de tous états\,; que cela ne se pouvait sans une
rigueur proportionnée au besoin, qui fût suivie, et qui fît exemple pour
toutes\,; qu'ainsi mon avis était qu'après avoir renouvelé les défenses,
M\textsuperscript{me} la duchesse d'Orléans et M\textsuperscript{me} la
Duchesse fussent mises au carcan, s'il leur arrivait d'en porter. Le
sérieux du préambule et le sarcasme de la fin causèrent un éclat de rire
universel, et une confusion au duc de Noailles qu'il ne put cacher le
reste du conseil, dont il montra en sortant qu'il était outré.

Je ne manquais guère les occasions de divertir ainsi à ses dépens moi et
les autres, à quoi il ne pouvait s'accoutumer. Nous remarquâmes, M. le
comte de Toulouse et moi, qu'il rapportait les affaires de finances sans
en apporter aucunes pièces, quoiqu'il y eût beaucoup de ces affaires qui
étaient contentieuses. Cela lui donnait lieu de dire ce qu'il voulait
sans craindre d'être contredit. Nous résolûmes de ne pas souffrir cet
abus davantage. Dès le premier conseil pour finance, d'après cette
résolution, j'interrompis le duc de Noailles, et lui demandai où étaient
les pièces de l'affaire qu'il rapportait. Il balbutia, se fâcha et ne
sut que répondre. Je regardai la compagnie, puis le régent, et lui
adressant la parole, je lui dis que quelque confiance qu'on voulut bien
avoir, il était fâcheux de juger sur parole, et qu'en mon particulier
j'avais raison de n'être pas si confiant. Le feu monta au visage du duc
de Noailles, qui voulut parler. Je l'interrompis encore, et lui dis que
je ne proposais rien en cela qui ne fût en usage dans tous les
tribunaux, et qui de plus ne fût à la décharge et au soulagement du
rapporteur. Il voulut grommeler encore\,; je regardai le régent en
haussant fortement les épaules. Le comte de Toulouse dit qu'il ne voyait
pas quelle pouvait être la difficulté d'apporter les pièces. Noailles, à
ce mot, se tut, se mit la tête entre les épaules, continua son rapport,
qu'il abrégea tant qu'il put, et au conseil suivant pour finance,
apporta un grand sac plein de papiers.

Pour ses péchés, son rang le mettait toujours auprès de moi, parce
qu'alors il n'y avait de pair entre nous deux que le maréchal de
Villeroy, qui, par conséquent, ne pouvait être de mon côté, les jours de
finance non plus que les autres. Quand Noailles voulut parler\,: «\,Et
les pièces\,? lui dis-je. --- Voilà mon sac où elles sont, me
répondit-il. --- Je le vois ce sac, répliquai-je, mais point du tout les
pièces. Mettez donc sur la table celles de l'affaire dont vous voulez
parler.\,» II ouvrit son sac, de colère, en prit les pièces, qu'il mit
devant lui, et tandis qu'il rapportait, me voilà à les feuilleter et à
me faire son évangéliste. On ne vit jamais un homme plus déconcerté, ni
avec plus de volonté de ne le pas paraître\,; car tout cela se démêlait
en lui. Il ne se cachait point après chez lui, où il revenait bouffant
et rempli de ces algarades, que je le désolais, et qu'il ne pouvait plus
y tenir\,; et moi d'en rire et de le tenir en haleine. Il m'est souvent
arrivé de le faire chercher dans les pièces la preuve de ce qu'il
avançait, de lire avec lui bas, tandis qu'il lisait haut dans les
pièces, comme me défiant de sa bonne foi, et n'étant pas fâché qu'on le
vît, et de lui en donner le dégoût, sans que jamais M. le duc d'Orléans
ait osé m'en rien dire, ni au conseil ni en particulier. Il m'est arrivé
aussi quelquefois de lui dicter l'arrêt tel qu'il venait d'être
prononcé, et de l'obliger de l'écrire sous ma dictée, en plein conseil,
et, par-ci par-là, de lui faire ôter ce qu'il y avait mis, ou ajouter ce
qu'il y avait omis, et faire changer les termes qu'il avait substitués à
ceux qui venaient d'être prononcés. En ces occasions, la rage lui
sortait par tous les pores\,; son visage enflammé et furieux le
décelait, ainsi que toute son attitude et ses mouvements\,; mais, de
peur de pis, il se contenait et ne disait jamais que l'indispensable. Je
lui volais dessus cependant comme un oiseau de proie, et le conseil
fini, j'en riais avec les uns et les autres, qui, au partir de là, ne
gardaient pas le secret des procédés. Ils couraient le monde, et, comme
Noailles n'y était ni aimé ni estimé, parce que son accès n'était ni
facile, ni doux, on en riait. Il le savait, car il voulait tout savoir,
et cela le mettait d'autant plus au désespoir que la répétition de ces
scènes était très fréquente. C'en est assez pour un échantillon\,; la
pièce ne vaut pas de s'y étendre davantage.

Je ne sais pourquoi il fut question ce carême de bénir la chapelle des
Tuileries, où le feu roi avait toujours ouï la Messe lorsqu'il avait
logé dans ce palais, et où le roi l'entendait tous les jours depuis son
retour de Vincennes. Cette bénédiction forma une question entre le
cardinal de Noailles, ordinaire\footnote{{[}37{]}}, et le cardinal de
Rohan, grand aumônier. La même s'était, comme on l'a vu en son temps,
présentée pour la chapelle neuve de Versailles, entre le même cardinal
de Noailles et le cardinal de Janson, grand aumônier. Elle avait été
décidée en faveur du cardinal de Noailles, et le fut de même pour la
chapelle des Tuileries, sur quoi le cardinal de Rohan fit des
protestations.

Le duc d'Ossone mourut à Paris dans un âge peu avancé. Il avait été
premier ambassadeur plénipotentiaire d'Espagne à Utrecht, et avait
demeuré avant et après assez longtemps aux Pays-Bas et en Hollande, où
ses dettes, des violences inconnues dans ces pays-ci, et de continuelles
débauches, avaient fort obscurci sa naissance, sa dignité et son
caractère. Le comte de Pinto, son frère, succéda à sa grandesse et à son
titre. Leur maison est Acuña y Giron. L'ambassadeur à Utrecht était
gendre du duc de Frias, connétable de Castille, de la maison de Velasco.

Le grand prieur, dont on a vu en son lieu le caractère et la conduite,
était, comme on l'a vu aussi, revenu aussitôt après la mort du roi,
considéré, même respecté de M. le duc d'Orléans, qui avait toujours été
le jaloux admirateur d'une si continuelle uniformité d'impiété, de
débauches et d'effronterie, en faveur desquelles il lui passait tout le
reste. Le grand prieur lui imposait au dernier point, quoique méprisé et
abandonné de tout le monde, et réduit à souper tous les soirs avec des
bandits sans état et sans nom. À l'abri du duc du Maine, il faisait le
prince du sang tant qu'il pouvait, et cela ne lui était pas difficile,
par le peu et l'espèce de gens qu'il voyait. Il se hasarda, par le même
appui, d'aller à l'adoration de la croix après les princes du sang, le
vendredi saint, à l'office où le roi était. Le maréchal de Villeroy y
fut surpris et s'en plaignit au régent, qui glissa. Encouragé par le
succès de l'entreprise, il en tenta d'autres, tant qu'enfin les princes
du sang d'une part, et les ducs de l'autre, s'en fâchèrent, et que M. le
duc d'Orléans lui défendit d'en plus hasarder. Je pense qu'il s'en prit
à moi, car un jour M. le duc d'Orléans me dit, avec assez d'embarras,
que le grand prieur avait remarqué que j'affectais de vouloir passer
devant lui au Palais-Royal, qui était le seul lieu où je le rencontrais
quelquefois, et qu'il s'en était plaint à lui. Je demandai au régent ce
qu'il lui avait répondu, et tout de suite j'ajoutai que je n'avais point
de ces petitesses-là\,; mais que, puisque le grand prieur croyait voir
ce qui n'était pas, et qu'il s'avisait de le trouver mauvais et de s'en
plaindre, je lui ferais dire vrai, et lui montrerais partout que je le
précédais et le devais précéder\,; et aussitôt après je changeai de
discours.

En effet, quelques jours après je trouvai le grand prieur au
Palais-Royal. Il me salua froidement\,; car nous n'avions jamais eu
aucun rapport ensemble\,; moi plus sèchement et plus courtement
encore\,; et quand il fut question de passer, dont je m'étais mis à
portée, j'entrai. Je remarquai qu'il mit quelqu'un entre lui et moi pour
entrer après. Il n'osa rien dire, et je n'en ouïs plus parler. Mais
quelque temps après, je sus qu'il faisait tous ses efforts pour entrer
au conseil de régence et y précéder les ducs. J'en fis honte au régent,
et lui demandai quel talent, hors l'escroquerie, et pis, la poltronnerie
et la plus infâme débauche, il trouvait dans le grand prieur pour
l'admettre dans le gouvernement, et quelle réputation lui-même espérait
d'un tel choix.

La négative peu assurée et l'embarras du régent me déclarèrent tout ce
qu'il y avait à craindre de sa faiblesse et de sa vénération pour le
grand prieur. Je parlai aux maréchaux de Villeroy et d'Harcourt, qui
étaient du conseil de régence\,; au maréchal de Villars, qui y venait
quand il s'agissait des affaires de la guerre\,; à d'autres encore\,;
puis, de concert avec eux, je déclarai au régent que, s'il faisait à
l'État, au conseil de régence, à lui-même, l'ignominie d'y faire entrer
le grand prieur, et aux ducs l'injustice de le leur faire précéder, il
pourrait le même jour disposer des places qu'il nous avait données en ce
conseil et dans tous les autres, et compter que, sans ménagement aucun,
nous nous expliquerions sur un si bon choix, et sur l'insulte que de
gaieté de cœur nous recevrions de sa main, que nous éprouvions déjà si
équitable et si bienfaisante à l'égard du parlement, dont apparemment la
séance au conseil lui semblerait plus utile que le travail, l'avis et
l'attachement de ses serviteurs. J'ajoutai que toutes ces mêmes paroles
dont je me servais m'étaient prescrites, et tous les lui disaient
exactement par ma bouche. L'étonnement du régent et son embarras le
tinrent quelque temps en silence. J'y demeurai aussi. Il essaya de
tergiverser. Je lui dis que cela était inutile\,; que notre parti était
bien pris et sans retour\,; qu'il était maître de faire ce qu'il lui
plairait là-dessus\,; mais qu'il ne l'était pas d'empêcher notre
retraite, nos discours et l'éclat qu'il causerait. Il faiblit, et me
chargea enfin de dire aux ducs qu'il n'y avait jamais pensé, et que le
grand prieur n'entrerait point dans le conseil, quoiqu'il l'en eût fort
pressé. Il n'ajoutait pas qu'il avait dit au grand prieur qu'il l'y
ferait entrer, et il craignait ses reproches, et encore plus notre
éclat. Cette courte conversation termina les espérances du grand prieur,
dont il ne fut plus question depuis.

La duchesse de Béthune mourut à Paris assez vieille. Elle était fille du
surintendant Fouquet, et mère du duc de Charost. C'était une femme de
beaucoup de mérite et de vertu, d'esprit très médiocre, toute sa vie
fort retirée, et qui avait toujours paru fort rarement à la cour. On a
vu en son lieu comment le malheur de son père fit la solide fortune de
son mari, et comment le quiétisme fit son fils capitaine des gardes du
corps. Elle était dès sa jeunesse dans cette doctrine, et allait toutes
les semaines, tête à tête avec M. de Noailles, entendre un M. Bertaut à
Montmartre, qui était le chef du petit troupeau qui s'y assemblait, et
qu'il dirigeait. Elle et le duc de Noailles étaient bien jeunes, et
néanmoins ces voyages réglés tête à tête passaient sans scandale. Ces
assemblées grossirent, firent du bruit\,; la doctrine parut au moins
très suspecte\,; on les dissipa, et le docteur Bertaut fut vivement
tancé. Le Noailles, qui vit l'orage, appuyé de la cour, ne se crut pas
destiné au martyre\,; il tourna sa dévotion plus humainement, et
abandonna pour toujours ce petit troupeau, dont il avait été une des
brebis choisies. M\textsuperscript{me} de Béthune fut plus fidèle à la
doctrine et au docteur, tellement que, bien des années après, cette même
doctrine ayant reparu avec plus d'art et de brillant avec
M\textsuperscript{me} Guyon, elle les joignit bientôt l'une à l'autre,
et fit de M\textsuperscript{me} de Béthune la disciple la plus estimée
et la plus favorite de M\textsuperscript{me} Guyon, et de là l'amie
intime de l'archevêque de Cambrai, et de MM. et de
M\textsuperscript{me}s de Chevreuse et de Beauvilliers, et des duchesses
de Guiche et de Mortemart. Nulle tempête ne les sépara de leur
prophétesse ni de leur patriarche, et c'est ce qui a comblé la fortune
des Charost, par les routes qui ont été remarquées en leur temps, en
sorte que le malheur du père de M\textsuperscript{me} de Béthune, dont
M. Colbert fut le principal instrument pour se revêtir de sa dépouille,
et celui de sa prophétesse qui fit et qui rendit intime cette fille de
Fouquet avec les filles de Colbert qui l'avait perdu, ont fait des
Charost tout ce que nous les voyons, sans que la duchesse de Béthune
soit presque jamais sortie de son oratoire.

L'abbé de Vassé, duquel j'ai suffisamment parlé à propos du refus qu'il
fit de l'évêché du Mans, mourut fort vieux en même temps, ainsi que le
chevalier du Rosel, lieutenant général, commandeur de Saint-Louis,
excellent homme de guerre et très galant homme, dont j'ai parlé plus
d'une fois\,; et Fiennes, lieutenant général assez distingué, qui était
gendre d'Étampes, chevalier de l'ordre et capitaine des gardes de feu
Monsieur. Le père de Fiennes s'appelait M. de Lumbres, mort aussi
lieutenant général. C'étaient des gentilshommes fort ordinaires devers
la Flandre, qui n'étaient rien moins que de la maison de Fiennes,
éteinte depuis longtemps.

Valbelle mourut aussi fort vieux, fort riche et point marié. Il s'était
distingué à la guerre par des actions heureuses et brillantes, d'une
grande valeur, et avait quitté depuis longtemps, pour n'avoir pas été
avancé comme il avait espéré de l'être. C'était un très honnête homme,
mais que j'ai vu longtemps traîner à la cour, sans savoir pourquoi, où
il ne bougeait de chez M. de La Rochefoucauld et de peu d'autres
maisons. Rottembourg, maréchal de camp en Alsace, {[}mourut aussi{]}. Il
était gendre du feu maréchal Rosen, et père de Rottembourg, dès lors
envoyé du roi en Prusse, qui s'est fait depuis beaucoup de réputation en
diverses ambassades, et est mort chevalier de l'ordre, très riche, sans
avoir été marié.

Le duc de Perth, attaqué depuis longtemps de la pierre, fut taillé fort
vieux à Saint-Germain, et en mourut. Il était grand-chancelier d'Écosse
lors de la révolution d'Angleterre. Il signala sa fidélité\,; il fut
gouverneur du roi Jacques III, et Jacques II l'avait fait en France duc
et chevalier de la Jarretière.

La Vieuville, qui venait presque de perdre sa femme, dame d'atours de
M\textsuperscript{me} la duchesse de Berry, épousa en troisièmes noces
une Froulay, veuve de Breteuil, conseiller au parlement.

Il y avait souvent des scènes entre M. {[}le prince{]} et
M\textsuperscript{me} la princesse de Conti, laquelle ne s'en
contraignait guère, et qui lui disait devant le monde, qu'il n'avait que
faire de vouloir tant montrer son autorité sur elle, parce qu'il était
bon qu'il sût qu'il ne pouvait pas faire un prince du sang sans elle, au
lieu qu'elle en pouvait faire sans lui. Ils se querellèrent à souper à
l'Ile-Adam. La chose alla fort loin. Crèvecœur, qui avec ce beau nom
n'était qu'un assez plat gentilhomme, et sa femme, qui étaient à eux,
s'y trouvèrent mêlés et si offensés qu'ils furent sur-le-champ chassés,
et qu'ils s'en allèrent à pied coucher où ils purent. Cette aventure fit
grand bruit sur le prince et la princesse.

M\textsuperscript{me} la duchesse de Berry, qui vivait de la façon qui a
été expliquée, voulut apparemment pouvoir passer des nuits d'été dans le
jardin de Luxembourg en liberté. Elle en fit murer les portes, et ne
conserva que celle de la grille du bas de l'escalier du milieu du
palais. Ce jardin, de tout temps public, était la promenade de tout le
faubourg Saint-Germain, qui s'en trouva privé. M. le Duc fit ouvrir
aussitôt celui de l'hôtel de Condé, et le rendit public en contraste. Le
bruit fut grand et les propos peu mesurés sur la raison de cette
clôture. Elle se trouva aussi importunée des deuils. Les marchands
d'étoffes en saisirent le moment, et la prièrent d'obtenir de M. le duc
d'Orléans de les abréger\,; ce qu'il lit avec sa facilité ordinaire, de
façon qu'on porte le deuil de tout ce qui n'est point parent, tant il y
{[}a{]} d'éloignement, même souvent d'incertitude, et qu'on ne le porte
presque plus des plus proches, avec la dernière indécence. Mais comme le
mauvais dure toujours plus que le bon, ce retranchement des feuils est
l'unique règlement de la régence qui subsiste encore aujourd'hui. Cela
arriva à l'occasion de celui de la reine mère de Suède.

Elle fut aussi, avec toute sa gloire, la première fille de France qui
ait permis aux dames d'honneur des princesses du sang d'entrer dans sa
loge et de s'y mettre derrière leurs princesses. Il est vrai que ce fut
dans sa petite loge à l'Opéra\,; mais ce fut un pied pris qui, sur ce
léger fondement, a su depuis se soutenir.

Les nouveaux goûts de cette princesse lui firent chercher à récompenser
leurs anciens pour s'en défaire honnêtement. Vittement, qui avait été
lecteur des princes père et oncles du roi, et on a vu en son temps par
quelle occasion, fut nommé sous-précepteur du roi. À cette occasion,
M\textsuperscript{me} la duchesse de Berry voulut que La Haye, qui avait
perdu la charge qu'elle lui avait fait donner chez M. le duc de Berry,
eût une place de gentilhomme de la manche, qui vaut six mille francs par
an. Le roi en avait deux, et il n'y en avait jamais eu davantage. Ce
troisième fit donc difficulté. Pour la lever, on souffla à la duchesse
de Ventadour d'en demander un quatrième, moyennant quoi La Haye passa ;
et le roi en eut quatre.

Elle acheta, ou plutôt le roi pour elle, une petite maison à l'entrée du
bois de Boulogne, qui était jolie, avec tout le bois devant et un beau
et grand jardin derrière, qui appartenait à la charge de capitaine des
chasses de Boulogne et des plaines des environs. Catelan qui l'était
l'avait fait accommodée, et avait vendu à Armenonville\,; cela s'appelle
la Muette\footnote{}, que le roi a prise depuis et fort augmentée.
Armenonville fut payé grassement, conserva la capitainerie, eut quatre
cent mille livres de brevet de retenue sur sa charge de secrétaire
d'État, dont il n'avait pas payé davantage au chancelier, et presque
tout le château de Madrid et tous ses jardins pour sa maison de
campagne, réparée à son gré aux dépens du roi, et son fils en survivance
de cet usage et de la capitainerie. M\textsuperscript{me} la princesse
de Conti première douairière acheta aussi Choisy de la succession de
M\textsuperscript{me} de Louvois\,; c'est la même {[}maison{]} que le
roi acheta aussi de la sienne, et où il a fait et fait encore tous les
jours tant d'augmentations et d'embellissements.

M. le duc d'Orléans acheta six cent mille livres, pour le chevalier
d'Orléans, la charge de général des galères du maréchal de Tessé, qui y
gagna deux cent mille livres\,; et fit donner par le roi à M. le comte
de Charolais une pension de soixante mille livres. Ç'avait toujours été
la pension la plus forte, qui ne se donnait presque jamais qu'au premier
prince du sang. Je dis presque jamais, parce que je n'en sais d'exemple
avant la régence que celui de Chamillart, quand le roi le r envoya comme
malgré lui. Le régent prodiguait ainsi les grâces à des gens qu'il ne
gagnait pas, et qui s'en moquaient de lui\,: témoin La Feuillade, Tessé
et tant d'autres.

Il avait eu la complaisance de faire venir une troupe de comédiens
italiens, à la persuasion de Rouillé, conseiller d'État, dont j'ai parlé
plus d'une fois, et qui faisait tout dans les finances. On a vu en son
temps que le feu roi les avait chassés pour avoir joué à découvert
M\textsuperscript{me} de Maintenon, sous le nom de \emph{la Fausse
perdue}. Ces comédiens revinrent donc, desquels Rouillé fut le
protecteur, et le modérateur de leurs pièces\,; et pour qu'il le
demeurât indépendamment des premiers gentilhommes de la chambre, ils
n'eurent point la qualité de comédiens italiens du roi, mais de M. le
duc d'Orléans, qui fut à leur première représentation, où tout le monde
accourut, dans la salle de l'Opéra. Ils jouèrent quelque temps sur ce
théâtre, en attendant qu'on leur eût raccommodé leur hôtel de Bourgogne,
où ils étaient quand le feu roi les chassa. La nouveauté et la
protection les mirent fort à la mode\,; mais peu à peu les honnêtes gens
se dégoûtèrent de leurs ordures, et ils tombèrent. Ils sont demeurés
jusqu'à présent, et jouent toujours à l'hôtel de Bourgogne.

\hypertarget{chapitre-xx.}{%
\chapter{CHAPITRE XX.}\label{chapitre-xx.}}

1716

~

{\textsc{Berwick va commander en Guyenne au lieu de Montrevel, qui va en
Alsace et qui s'en prend à moi.}} {\textsc{- Berwick fait réformer sa
patente, et n'est sous les ordres de personne, contre la tentative du
duc du Maine.}} {\textsc{- Le parlement s'oppose au rétablissement des
charges de grand maître des postes et de surintendant des bâtiments.}}
{\textsc{- Ses vues, sa conduite, ses appuis.}} {\textsc{- Vues et
intérêts de ses appuis.}} {\textsc{- Je me dégoûte d'en parler au
régent.}} {\textsc{- Je lui en prédis le succès, et je reste là-dessus
dans le silence.}} {\textsc{- Law, dit Las\,; sa banque.}} {\textsc{-
Mon avis là-dessus, tant au régent en particulier qu'au conseil de
régence.}} {\textsc{- Elle y passe et au parlement.}} {\textsc{- Le
régent me met, malgré moi, en commerce réglé avec Law, qui dure jusqu'à
sa chute.}} {\textsc{- Vue de Law à mon égard.}} {\textsc{- Évêchés et
autres grâces.}} {\textsc{- Arouet, poète, depuis Voltaire, exilé.}}
{\textsc{- Un frère du roi de Portugal à Paris\,; va servir en
Hongrie.}} {\textsc{- Mort de M\textsuperscript{me} de Courtaumer et de
M\textsuperscript{me} de Villacerf\,; de la comtesse d'Egmont en
Flandre\,; sa famille.}} {\textsc{- Mort de la maréchale de Bellefonds
et de la marquise d'Harcourt.}} {\textsc{- Le maréchal d'Harcourt, en
apoplexie, perd la parole pour toujours.}} {\textsc{- Le roi, revenant
de l'Observatoire, visite en passant le chancelier de Pontchartrain.}}
{\textsc{- M\textsuperscript{me} de Nassau remise en liberté.}}
{\textsc{- MM. le Duc et prince de Conti ont la petite vérole.}}
{\textsc{- Naissance de la dernière fille de M\textsuperscript{me} la
duchesse d'Orléans.}} {\textsc{- Mort de l'électeur palatin.}}

~

Le maréchal de Montrevel commandait toujours en Guyenne, il y escroquait
et prenait tant qu'il pouvait, et faisait toutes sortes de sottises.
C'était un homme fort court, fort impertinent, tout au maréchal de
Villeroy et au bel air de la vieille cour, et fort peu sûr, par
conséquent, pour M. le duc d'Orléans. Il était à Paris et sur le point
de s'en retourner à Bordeaux. Le maréchal de Berwick eut le commandement
de Guyenne, et Montrevel celui d'Alsace, où il ne pouvait pas être
dangereux. Quand le régent l'eut déclaré, Montrevel vint lui dire qu'il
serait toujours content de tout ce qu'il lui ordonnerait, et ajouta\,:
«\,Mais, monsieur, le public en sera-t-il content pour moi\,? --- Oui,
monsieur, lui répondit le régent, il le sera, je vous en réponds.\,» Ces
sortes de fatuités, destituées comme celle-ci de tout mérite, n'allaient
point au régent, qui d'un mot prompt et court les mettait au net dans
tout leur ridicule. Montrevel fut outré. Tout vieux qu'il était, il
était fou d'une M\textsuperscript{me} de L'Église, femme d'un conseiller
du parlement de Bordeaux, et depuis tant d'années que le feu roi l'y
avait mis il avait là toutes ses habitudes. Il imagina que c'était moi
qui l'avais fait déplacer. Il en fit partout ses plaintes, et me les
envoya faire par Biron. Le maréchal Montrevel et moi n'avions pas ouï
parler l'un de l'autre depuis le règlement que le feu roi avait fait
entre nous et dont j'ai parlé en son temps, depuis lequel il n'avait osé
se mêler de quoi que ce soit du gouvernement de Blaye\,; ainsi rien qui
me fût plus indifférent que son commandement en Guyenne. Je n'avais pas
pensé un moment à lui, et M. le duc d'Orléans ne m'en parla qu'après
qu'il l'eut résolu. Je répondis donc à Biron qu'il pouvait assurer
Montrevel que, depuis que nous n'avions plus rien de commun, ni à
démêler ensemble, je n'avais pas songé s'il était au monde\,; que je
n'avais su son déplacement que lorsque M. le duc d'Orléans me l'avait
appris\,; et qu'il pouvait s'ôter de la tête que j'y eusse la moindre
part, parce que rien au monde ne m'était plus indifférent, depuis que le
feu roi avait confirmé et réglé ma très parfaite indépendance, qui ne me
pouvait plus être troublée. Je ne sais si Biron osa lui rendre
fidèlement ma réponse\,; mais il continua à se plaindre de moi, et moi à
me moquer de lui. Nous verrons bientôt qu'il ne sortit point de Paris,
et qu'il mourut de peur ou de rage.

L'affaire du duc de Berwick ne fut pas sitôt consommée. Il s'aperçut que
sa patente pour commander en Guyenne le soumettait aux ordres du comte
d'Eu, qui, comme devenu prince du sang, prétendait faire de Paris les
fonctions de gouverneur de Guyenne. Cela s'était évité avec Montrevel,
qui y avait été envoyé du vivant du duc de Chevreuse, et avant qu'il fût
question des dernières apothéoses de ces bâtards\,; d'ailleurs point
d'exemple à l'égard des princes du sang sur les maréchaux de France,
commandant dans leurs gouvernements\,; mais c'était le temps des
entreprises, surtout des princes du sang et des bâtards comme tels.
Berwick renvoya la patente. Le régent en brassière, amateur du poison
des \emph{mezzo-termine}, qui toujours désespèrent celle\footnote{{[}39{]}}
qui a raison, et ne contente pas celle qui a tort, fit ce qu'il put pour
concilier les choses. Berwick, sans s'en embarrasser, ne mollit point,
dit qu'il ne connaissait point de milieu entre être ou n'être pas aux
ordres d'un autre, se renferma à déclarer qu'il n'avait point demandé ce
commandement, et qu'il ne l'accepterait point à une condition nouvelle
et déshonorante. Quelque mouvement que les bâtards, et même, pour ce
fait particulier, que les princes du sang se pussent donner, parce qu'il
les regardait également, il en fallut passer par où le maréchal voulut.
Le régent comptait sur lui dans une province jalouse, et si proche de
l'Espagne\,: la patente fut réformée\,; il n'y fut pas fait la moindre
mention du comte d'Eu. Les maréchaux de France, qui avaient doucement
laissé démêler la fusée à leur confrère, furent fort contents, lui
beaucoup davantage\,; et le rare fut que M. du Maine, y ayant perdu sans
réserve tout ce qu'il avait prétendu, voulut paraître content aussi.

Le parlement persistait à ne vouloir point enregistrer les deux édits
d'érection de grand maître des postes et de surintendant des bâtiments.
Il prétendait que {[}ces charges{]} ayant été supprimées, et la
suppression enregistrée avec clause de ne pouvoir être rétablies, ils
les devaient rejeter. Ce n'était pas que cela intéressât ni eux ni le
peuple en aucune manière, encore moins s'il se pouvait l'État\,; mais
cette compagnie voulait figurer, se rendre considérable, faire compter
avec elle\,; elle ne le pouvait que par la lutte, et de propos délibéré
elle n'en perdait aucune occasion. Elle avait sondé le régent, puis
tâté\,; les succès répondaient de sa faiblesse. Il était environné
d'ennemis qui lui imposaient, et qui, avec bien moins d'esprit et de
lumière que lui, le trompaient et s'en moquaient, et qui s'étaient liés
avec le parlement qui avait les bâtards à lui et qui tenait les princes
du sang en mesure. Tels étaient\,: le maréchal de Villeroy, à qui les
conversations sur les Mémoires du cardinal de Retz et de Joly, qui
étaient lors fort à la mode, et que tout le monde se piquait de lire,
avaient tourné la tête, et qui voulait être comme le duc de Beaufort,
chef de la Fronde, roi des halles et de Paris, l'appui du parlement\,;
d'Effiat, son ami et du duc du Maine, à qui de longue main il avait
vendu son maître et qui trouvait son compte à figurer et à négocier
entre son maître et le parlement\,; Besons, plat robin, quoique maréchal
de France, qui s'était mis sous la tutelle d'Effiat\,; Canillac, par les
prestiges du feu président de Maisons, et que sa veuve, qui cabalait
encore tant qu'elle pouvait chez elle, entretenait toujours, avec
autorité sur son esprit quoiqu'elle n'en eût point, et il lui rendait
compte de ce qu'il pompait du régent sur le parlement\,; le duc de
Noailles qui l'avait flatté par ses trahisons, qui, pour les rendre
complètes, en avait fait peur au régent, et qui lui-même en mourait de
frayeur sur son administration des finances, uni d'ailleurs avec
d'Effiat par Dubois, trop petit garçon encore pour oser les contredire,
ce Noailles, ravi de partager les négociations avec le parlement, et de
voir naître du trouble pour se rendre nécessaire\,; Huxelles enfin, ami
intime du premier président, et dont le thème auprès du régent était la
nécessité de l'intelligence avec le parlement pour le pouvoir contenir
sur les matières de la constitution et de Rome\,; un Broglio, un Nocé,
d'autres petits compagnons, instruits par les autres ou par leurs
propres liaisons à placer leur mot à propos. Ainsi, tantôt sur une
matière, tantôt sur une autre, cette lutte se multiplia, se fortifia,
s'échauffa, et conduisit, comme on le verra, les choses au bord du
précipice.

Je m'étais dépité à cet égard par une infinité de raisons\,; la défiance
et la faiblesse du régent se réunissaient contre tout ce que je lui
pouvais dire là-dessus. Je lui déclarai â la fin que je me lavais les
mains de tout ce qui lui pouvait arriver de la misère de sa conduite
avec le parlement, de l'audace des entreprises de cette compagnie, de la
friponnerie de gens qui l'environnaient, qui avaient mis le grappin sur
lui, qu'il comblait d'amitiés, de confiance, de grâces, et qui étaient
ses ennemis et le vendaient à leurs intérêts, à leurs vues et au
parlement. J'ajoutai que je ne lui parlerais de ma vie de rien qui eut
rapport au parlement, et que je saurais mettre à leur aise ses soupçons
sur la haine qu'il me croyait contre le parlement\,; mais que je lui
prédisais et le priais de s'en bien souvenir, qu'il n'irait pas loin
sans que les choses n'en vinssent entre lui et cette compagnie au point
qu'il se verrait forcé de lui abandonner toute l'autorité et tout
l'exercice de la régence, ou d'avoir recours à des coups de force très
dangereux. Je lui tins exactement parole\,; on verra en son temps ce qui
en arriva.

Il avait alors une affaire à éclore, dont on se servit beaucoup pour le
rendre si docile à l'égard du parlement. Un Écossais, de je ne sais
quelle naissance\textsuperscript{{[}{[}40{]}{]}}{[}{[}40{]}{]}, grand
joueur et grand combinateur, et qui avait gagné fort gros en divers pays
où il avait été, était venu en France dans les derniers temps du feu
roi. Il s'appelait Law\,; mais quand il fut plus connu, on s'accoutuma
si bien à l'appeler Las, que son nom de Law disparut. On parla de lui à
M. le duc d'Orléans comme d'un homme profond dans les matières de
banque, de commerce, de mouvement d'argent, de monnaie et de finances\,;
cela lui donna curiosité de le voir. Il l'entretint plusieurs fois, et
il en fut si content qu'il en parla à Desmarets comme d'un homme de qui
il pourrait tirer des lumières. Je me souviens aussi que ce prince m'en
parla dans ce même temps. Desmarets manda Law, et fut longtemps avec lui
et plusieurs reprises\,; je n'ai point su ce qui se passa entre eux, ni
ce qui en résulta, sinon que, Desmarets en fut content, et prit pour lui
quelque estime.

M. le duc d'Orléans après cela ne le vit plus que de loin à loin\,; mais
après les premiers débouchés des affaires qui suivirent la mort du roi,
Law, qui avait fait au Palais-Royal des connaissances subalternes et
quelque liaison avec l'abbé Dubois, se présenta de nouveau devant M. le
duc d'Orléans, bientôt après l'entretint en particulier et lui proposa
des plans de finances. Il le fit travailler avec le duc de Noailles,
avec Rouillé, avec Amelot, ce dernier pour le commerce. Les deux
premiers eurent peur d'un intrus de la main du régent dans leur
administration, de manière qu'il fut longtemps ballotté, mais toujours
porté par M. le duc d'Orléans. À la fin le projet de banque plut tant à
ce prince qu'il voulut qu'il eût lieu. Il en parla en particulier aux
principaux des finances, en qui il trouva une grande opposition. Il m'en
avait souvent parlé, et je m'étais contenté de l'écouter sur une matière
que je n'ai jamais aimée, ni par conséquent bien entendue, et dont la
résolution me paraissait éloignée. Quand il eut tout à fait pris son
parti, il fit une assemblée de finance et de commerce, où Law expliqua
tout le plan de la banque qu'il proposait d'établir. On l'écouta tant
qu'il voulut. Quelques-uns, qui virent le régent presque déclaré,
acquiescèrent\,; mais le très grandnombre s'y opposa\footnote{{[}41{]}}.

Law ne se rebuta point. On parla à la plupart un peu français à
l'oreille. On refit à peu près la même assemblée, où en présence du
régent, Law expliqua encore ce projet. À cette fois peu y contredirent,
et faiblement. Le duc de Noailles n'avait osé soutenir la gageure, comme
eût voulu le maréchal de Villeroy qui allait toujours à contrecarrer M.
le duc d'Orléans, sans autre raison\,; car il n'entendait ni en
finances, ni en autres affaires\,; aussi n'opinait-il jamais au conseil
qu'en deux mots, ou si très rarement il voulait dire plus sur une
affaire qu'il savait qu'on y voulait traiter, il apportait une petite
feuille de papier, et quand ce venait à lui d'opiner, mettait ses
lunettes, lisait tout de suite les cinq ou six lignes qui étaient
écrites. Je ne l'ai jamais vu opiner autrement, et de cette dernière
façon quatre ou cinq fois au plus. La banque passée de la sorte, il la
fallut proposer au conseil de régence.

M. le duc d'Orléans prit la peine d'instruire en particulier chaque
membre de ce conseil, et de lui faire doucement entendre qu'il désirait
que la banque ne trouvât point d'opposition. Il m'en parla à fond\,;
alors il fallut bien répondre. Je lui dis\,: que je ne cachais point mon
ignorance, ni mon dégoût de toute matière de finance, que néanmoins ce
qu'il venait de m'expliquer me paraissait bon en soi, en ce que sans
levée, sans frais, et sans faire tort ni embarras à personne, l'argent
se doublait tout d'un coup par les billets de cette banque, et devenait
portatif avec la plus grande facilité\,; mais qu'à cet avantage je
trouvais deux inconvénients\,: le premier de gouverner la banque avec
assez de prévoyance et de sagesse pour ne faire pas plus de billets
qu'il ne fallait, afin d'être toujours au-dessus de ses forces, et de
pouvoir faire hardiment face à tout, et payer tous ceux qui viendraient
demander l'argent des billets dont ils seraient porteurs\,; l'autre, que
ce qui était excellent dans une république ou dans une monarchie où la
finance est entièrement populaire, comme est l'Angleterre, était d'un
pernicieux usage dans une monarchie absolue, telle que la France, où la
nécessité d'une guerre mal entreprise et mal soutenue, l'avidité d'un
premier ministre, d'un favori, d'une maîtresse, le luxe, les folles
dépenses, la prodigalité d'un roi ont bientôt épuisé une banque, et
ruiné tous les porteurs de billets, c'est-à-dire culbuté le royaume. M.
le duc d'Orléans en convint, mais en même temps me soutint qu'un roi
aurait un intérêt si grand et si essentiel à ne jamais toucher ni
laisser toucher ministre, maîtresse ni favoris à la banque, que cet
inconvénient capital ne pouvait jamais être à craindre. C'est sur quoi
nous disputâmes longtemps sans nous persuader l'un l'autre, de façon
que, lorsque quelques jours après il proposa la banque au conseil de
régence, j'opinai tout au long comme je viens de l'expliquer, mais avec
plus de force et d'étendue\,; et je conclus à rejeter la banque comme
l'appât le plus funeste dans un pays absolu, qui dans un pays libre
serait un très bon et très sage établissement.

Peu osèrent être de cet avis\,; la banque passa. M. le duc d'Orléans me
fit de petits reproches, mais doux, de m'être autant étendu. Je m'en
excusai sur ce que je croyais de mon devoir, honneur et conscience,
d'opiner suivant ma persuasion, après y avoir bien pensé, et de
m'expliquer suffisamment pour bien faire entendre mon avis, et les
raisons que j'avais de le prendre. Incontinent après, l'édit en fut
enregistré au parlement sans difficulté\footnote{{[}42{]}}. Cette
compagnie savait quelquefois complaire de bonne grâce au régent pour se
raidir après contre lui avec plus d'efficace.

Quelque temps après, pour le raconter tout de suite, M. le duc d'Orléans
voulut que je visse Law, qu'il m'expliquât ses plans, et me le demanda
comme une complaisance. Je lui représentai mon ineptie en toute matière
de finance\,; que Law aurait beau jeu avec moi à me parler un langage ou
je ne comprendrais rien\,; que ce serait nous faire perdre fort
inutilement notre temps l'un à l'autre. Je m'en excusai tant que je pus.
Le régent revint plusieurs fois à la charge, et à la fin l'exigea. Law
vint donc chez moi. Quoique avec beaucoup d'étranger dans son maintien,
dans ses expressions et dans son accent, il s'exprimait en fort bons
termes, avec beaucoup de clarté et de netteté. Il m'entretint fort au
long sur sa banque qui, en effet, était une excellente chose en
elle-même, mais pour un autre pays que la France, et avec un prince
moins facile que le régent. Law n'eut d'autre solution à me donner à ces
deux objections que celles que le régent m'avait données lui-même, qui
ne me satisfirent pas. Mais comme l'affaire était passée, et qu'il
n'était plus question que de la bien gouverner, ce fut principalement
là-dessus que notre conversation roula. Je lui fis sentir, tant que je
pus, l'importance de ne pas montrer assez de facilité pour qu'on en pût
abuser avec un régent aussi bon, aussi facile, aussi ouvert, aussi
environné. Je masquai le mieux que je pus ce que je voulais lui faire
entendre là-dessus\,; et j'appuyai surtout sur la nécessité de se tenir
en état de faire face sur-le-champ, et partout, à tout porteur de
billets de banque qui en demanderait le payement, d'où dépendait tout le
crédit ou la culbute de la banque. Law en sortant me pria de trouver bon
qu'il vînt quelquefois m'entretenir\,; nous nous séparâmes fort
satisfaits l'un de l'autre, dont le régent le fut encore plus.

Law vint quelques autres fois chez moi\,; il me montra beaucoup de désir
de lier avec moi. Je me tins sur les civilités, parce que la finance ne
m'entrait point dans la tête, et que je regardais comme perdues toutes
ces conversations. Quelque temps après, le régent, qui me parlait assez
souvent de Law avec grand engouement, me dit qu'il avait à me demander,
même à exiger de moi une complaisance\,; c'était de recevoir règlement
une visite de Law par semaine. Je lui représentai la parfaite inutilité
de ces entretiens, dans lesquels j'étais incapable de rien apprendre, et
plus encore d'éclairer Law sur des matières qu'il possédait, auxquelles
je n'entendais rien. J'eus beau m'en défendre, il le voulut
absolument\,; il fallut obéir. Law, averti par le régent, vint donc chez
moi. Il m'avoua de bonne grâce que c'était lui qui avait demandé cela au
régent, n'osant me le demander à moi-même. Force compliments suivirent
de part et d'autre\,; et nous convînmes qu'il viendrait chez moi tous
les mardis matin sur les dix heures, et que ma porte serait fermée à
tout le monde tant qu'il y demeurerait. Cette visite ne fut point mêlée
d'affaires. Le mardi matin suivant, il vint au rendez-vous, et y est
exactement venu ainsi jusqu'à sa déconfiture. Une heure et demie, très
souvent deux heures, était le temps ordinaire de nos conversations. Il
avait toujours soin de m'instruire de la faveur que prenait sa banque en
France et dans les pays étrangers, de son produit, de ses vues, de sa
conduite, des contradictions qu'il essuyait des principaux des finances
et de la magistrature, de ses raisons, et surtout de son bilan, pour me
convaincre qu'il était bien plus qu'en état de faire face à tous
porteurs de billets, quelques sommes qu'ils eussent à demander.

Je connus bientôt que, si Law avait désiré ces visites réglées chez moi,
ce n'était pas qu'il eût compté faire de moi un habile financier\,; mais
qu'en homme d'esprit, et il en avait beaucoup, il avait songé à
s'approcher d'un serviteur du régent qui avait la plus véritable part en
sa confiance, et qui de longue main s'était mis en possession de lui
parler de tout et de tous avec la plus grande franchise et la plus
entière liberté, de tâcher par cette fréquence de commerce, de gagner
mon amitié, de s'instruire par moi de la qualité intrinsèque de ceux
dont il ne voyait que l'écorce, et peu à peu de pouvoir venir au conseil
à moi sur les traverses qu'il essuyait, et sur les gens à qui il avait
affaire, enfin de profiter de mon inimitié pour le duc de Noailles, qui
en l'embrassant tous les jours, mourait de jalousie et de dépit, lui
suscitait sous main tous les obstacles et tous les embarras possibles,
et eût bien voulu l'étouffer. La banque en train et florissante, je crus
nécessaire de la soutenir. Je me prêtai à ces instructions que Law
s'était proposées, et bientôt nous nous parlâmes avec une confiance dont
je n'ai jamais eu lieu de me repentir. Je n'entrerai point dans le
détail de cette banque, des autres vues qui la suivirent, des opérations
faites en conséquence. Cette matière de finances pourrait faire des
volumes nombreux. Je n'en parlerai que par rapport à l'historique du
temps, ou à ce qui a pu me regarder en particulier. J'ai dit les
raisons, vers les temps de la mort du roi, qui m'ont fait prendre le
parti de décharger ces Mémoires des détails immenses des affaires des
finances et de celles de la constitution. On les trouvera traitées par
ceux qui n'auront eu que ces objets en vue beaucoup plus exactement, et
mieux que je n'aurais pu le faire, et que je n'aurais fait qu'en me
détournant trop longuement et trop fréquemment de l'histoire de mon
temps, que je me suis seulement proposée. Je pourrais ajouter ici quel
fut Law. Je le diffère à un temps où cette curiosité se trouvera mieux
en sa place.

M. le duc d'Orléans donna l'évêché de Vannes à l'abbé de Tressan, son
premier aumônier\,; celui de Rodez à l'abbé de Tourouvre, à la prière du
cardinal de Noailles, et celui de Saint-Papoul à l'abbé de Choiseul à la
mienne, qui ne l'a su que plus de quinze ans après, et qui est
présentement évêque de Mende. Je ne lui avais jamais parlé, et personne
ne m'avait parlé de lui\,; mais je le savais homme de bien et pauvre. Le
ressort qui me fit agir fut la mémoire du maréchal de Choiseul, dont il
était neveu, et tout jeune, lorsque j'en entendis dire un jour au
maréchal qu'il l'aimait. La même raison me fit obtenir de M. le duc
d'Orléans des assistances pécuniaires pour le chevalier de Peseu, que je
ne connaissais point, puis avancements, commandements et subsistances
qui l'ont conduit jusqu'à la fin de sa vie à d'autres. Il le sut parce
que cela ne se put cacher, et en a toujours été reconnaissant, ainsi que
M. de Mende. Peseu était fils d'une soeur du maréchal de Choiseul, dont
je savais qu'il avait fort aimé et aidé les enfants, à qui jamais je
n'avais eu occasion de parler.

Arouet, fils d'un notaire qui l'a été de mon père et de moi jusqu'à sa
mort, fut exilé et envoyé à Tulle, pour des vers fort satiriques et fort
impudents. Je ne m'amuserais pas à marquer une si petite bagatelle, si
ce même Arouet, devenu grand poète et académicien, sous le nom de
Voltaire, n'était devenu, à travers force aventures tragiques, une
manière de personnage dans la république des lettres, et même une
manière d'important parmi un certain monde.

Le prince Emmanuel, qui n'avait pas encore dix-neuf ans, dernier des
frères du roi de Portugal, arriva a Paris chez l'ambassadeur de sa
nation, où il logea. Le roi son frère, dont la conduite était fort
singulière, pour en parler plus que mesurément, l'avait frappé dans un
emportement. Le prince fut outré, et ne se crut plus en sûreté en
Portugal. On ne se mit nullement en peine de le recevoir, sous prétexte
de l'incognito. L'Angleterre dominait en Portugal, y trouvait son compte
pour son commerce\,; et, pour cela, le roi d'Angleterre complaisait en
tout au roi de Portugal. La considération des Anglais entra donc pour
beaucoup dans le peu de cas qu'on fit ici du prince Emmanuel. M. le duc
d'Orléans fut encore bien aise de s'épargner la dépense et l'importunité
personnelle d'une réception convenable. Il aima donc mieux tout
supprimer, jusqu'à la plus grande indécence. Ce prince ne vit ni le roi,
ni le régent, ni les filles de France, ni les princes et princesses du
sang. Il vécut à Paris tout en particulier, et n'y vit encore que
mauvaise compagnie. Aussi s'en lassa-t-il bientôt\,; et, au bout de six
semaines ou deux mois, partit malgré toutes les instances de
l'ambassadeur de Portugal, et s'en alla à Vienne, et servit volontaire
en Hongrie avec beaucoup de valeur.

Le duc de La Force perdit sa sœur, M\textsuperscript{me} de Courtaumer,
de la petite vérole. Le calvinisme avait fait ce mariage, ainsi que
celui de son père. M\textsuperscript{me} de Villacerf en mourut aussi\,;
elle était Saint-Nectaire et son mari avait été premier maître d'hôtel
de M\textsuperscript{me} la duchesse de Bourgogne. La comtesse d'Egmont
mourut à Bruxelles. Elle était sœur du duc d'Aremberg, père de celui
d'aujourd'hui et de la princesse d'Auvergne, à qui le cardinal de
Bouillon avait fait épouser Mesy, son écuyer, pour devenir maître de ses
biens, comme je l'ai rapporté en son temps. Cette comtesse d'Egmont
avait d'abord épousé le marquis de Grana, gouverneur des Pays-Bas dont
le duc d'Aremberg son frère avait épousé la fille, dont la comtesse
d'Egmont était ainsi belle-mère et belle-sœur. Elle épousa ensuite le
frère aîné du comte d'Egmont, dernier de cette illustre maison d'Egmont
dont la mort a été marquée en son temps, arrivée en Espagne, à qui
M\textsuperscript{me} des Ursins, lors en France duchesse de Bracciano,
avait fait épouser M\textsuperscript{lle} de Cosnac, nièce de
l'archevêque d'Aix, qui était sa parente et logeait chez elle. Ces deux
frères n'eurent point d'enfants.

La maréchale de Bellefonds-Fouquet, parente éloignée du surintendant,
mourut fort âgée et fort retirée à Vincennes\,; et la marquise
d'Harcourt, fille du duc de Villeroy, nouvelle mariée, toute jeune, à
Paris, sans enfants, dont les deux familles furent fort affligées. Peu
de jours après, le maréchal d'Harcourt eut une nouvelle attaque
d'apoplexie qui lui ôta l'usage de la parole pour toujours.

Le maréchal de Villeroy mena le roi voir l'Observatoire. Il était de
tout temps ami du chancelier de Pontchartrain retiré lors à
l'Institution\footnote{{[}43{]}}, c'est-à-dire dans une maison
joignante, qui y avait des entrées sans sortir. Des Tuileries à
l'Observatoire, il fallait nécessairement passer devant sa porte, et il
était à Paris. Le maréchal se souvint que les princes, ses
petits{[}-fils{]}\footnote{}, allant voir Paris de Versailles, le roi
ordonna au duc de Beauvilliers de les mener chez le vieux Beringhen,
pour leur faire voir un homme qu'il aimait, qui avait fait une étrange
fortune, et qui avait su sans rien quitter, faire justice à son âge en
ne sortant plus de chez lui à Paris parmi ses amis et avec sa famille.
Villeroy pour cette fois pensa très dignement qu'il était bon de faire
voir au roi un homme qui, vert et sain, et en état de corps et d'esprit
de figurer encore longtemps avec réputation dans le ministère et dans la
place de chancelier et de garde des sceaux sans dégoût et sans crainte,
avait su quitter tout pour mettre un sage et saint intervalle entre la
vie et la mort, dans une parfaite retraite où il ne voulait voir
personne, et n'était plus du tout occupé que de son salut sans aucun
délassement, et accoutumer le roi à honorer la vertu. Il manda donc de
l'Observatoire au chancelier de Pontchartrain qu'en repassant le roi
entrerait chez lui et lui ferait une visite. Rien de plus simple que de
recevoir cet honneur extraordinaire auquel il était bien loin de
songer\,; mais Pontchartrain, solidement modeste et détaché, mit ordre
d'être averti à temps, et se trouva sur sa porte dans la rue comme le
roi arrivait chez lui. Il fit inutilement tout ce qu'il put pour
empêcher le roi de mettre pied à terre\,; mais il réussit, à force
d'esprit, d'opiniâtreté et de respects à faire que la visite se passât
ainsi dans la rue, qui ne laissa pas de durer un quart d'heure jusqu'à
ce que le roi remonta en carrosse. Pontchartrain le vit partir et rentra
aussitôt dans sa chère modestie, où son parfait renoncement lui fit
oublier aussitôt l'extraordinaire honneur de la visite, et la pieuse
adresse qui lui en avait évité tout ce qu'il avait pu. Tout le monde qui
le sut l'admira, et loua fort aussi le maréchal de Villeroy d'une pensée
si honnête et si convenablement exécutée.

M\textsuperscript{me} de Nassau qui, pour d'étranges affaires avec son
mari, avait été longtemps à la Bastille, puis dans un couvent à Rethel,
eut permission de revenir à Paris chez le marquis de Nesle son frère,
par le consentement de son mari.

M. le Duc et M. le prince de Conti eurent la petite vérole à peu de
distance l'un de l'autre\,; et M\textsuperscript{me} la duchesse
d'Orléans accoucha d'une fille, qui est morte princesse de Conti, dont
elle a laissé un fils unique appelé comte de La Marche.

L'électeur palatin Guillaume-Joseph mourut à Dusseldorf sans enfants\,;
il était frère de l'impératrice épouse de l'empereur Léopold, de la
reine de Portugal, mère du roi Jean d'aujourd'hui, de la reine d'Espagne
seconde femme de Charles II, qui a été si longtemps à Bayonne, de la
duchesse de Parme mère de la reine d'Espagne, seconde femme de Philippe
V, et de l'épouse de Jacques Sobieski, fils aîné du célèbre roi de
Pologne. Cet électeur ne laissa point d'enfants de ses deux femmes,
l'une fille de l'empereur Ferdinand III, l'autre de
M\textsuperscript{me} la grande-duchesse, morte en France, fille de
Gaston, frire de Louis XIII. Charles-Philippe son frère, gouverneur du
Tyrol, lui succéda. Il était veuf d'Anne Radziwil, puis d'une
Lubomirski, dont il n'eut point de garçons, et fit depuis un troisième
mariage d'inclination si inégal qu'il n'en a jamais osé parler, et que
les enfants qu'il en aurait ne succéderaient point. Charles-Philippe
était frère de l'évêque d'Augsbourg, tombé en enfance, et du grand
maître de l'ordre Teutonique dont on a parlé sur Trèves et Mayence, dont
il eut les deux coadjutoreries.

\hypertarget{chapitre-xxi.}{%
\chapter{CHAPITRE XXI.}\label{chapitre-xxi.}}

1716

~

{\textsc{Soupçons et propos publics contre la reine d'Espagne et
Albéroni.}} {\textsc{- Dégoût et licence del Giudice.}} {\textsc{-
Triste état et emploi des finances.}} {\textsc{- Dégoût d'Albéroni sur
Hersent.}} {\textsc{- Incertitudes d'Albéroni au dehors.}} {\textsc{- Le
Prétendant tire quelques secours de lui, se retire à Avignon faute
d'autre asile.}} {\textsc{- Les puissances maritimes offrent des
vaisseaux à l'Espagne.}} {\textsc{- Leur intérêt.}} {\textsc{-
Indiscrète réponse d'Albéroni.}} {\textsc{- Plaintes.}} {\textsc{-
Frayeur de l'Italie du Turc et de l'empereur.}} {\textsc{- Albéroni
trompe Aldovrandi, attrape les décimes et se moque de lui.}} {\textsc{-
Ses vues.}} {\textsc{- Offres de l'Angleterre à l'Espagne contre la
grandeur de l'empereur en Italie.}} {\textsc{- L'Angleterre se plaint
d'Albéroni et le dupe sur l'empereur.}} {\textsc{- Le roi d'Angleterre
veut aller à Hanovre.}} {\textsc{- Wismar rendu.}} {\textsc{- Frayeur
des Hollandais de l'empereur.}} {\textsc{- Hauteurs partout des
Impériaux.}} {\textsc{- Vues et adresses des Hollandais.}} {\textsc{-
Hardiesse et scélératesse de Stairs.}} {\textsc{- Imprudence du
régent.}} {\textsc{- Sagesse de Cellamare.}} {\textsc{- Canal de
Mardick.}} {\textsc{- Naissance d'un fils à l'empereur.}} {\textsc{-
Folle catastrophe de Langallerie.}} {\textsc{- Scélératesse
ecclésiastique et temporelle de Bentivoglio.}} {\textsc{- Situation et
inquiétudes d'Albéroni.}} {\textsc{- Parlements d'Angleterre rendus
septénaires.}} {\textsc{- Vue et conduite des ministres anglais et de la
Hollande à l'égard de la France et de l'empereur.}} {\textsc{- Albéroni
inquiet se prête un peu à l'Angleterre.}} {\textsc{- Ses haines, ses
fourberies, ses adresses, son insolence.}} {\textsc{- Albéroni veut
savoir à quoi s'en tenir avec l'Angleterre\,; ne tire de Stanhope que du
vague, dont Monteléon voudrait que l'Espagne se contentât.}} {\textsc{-
Souplesses de l'Angleterre pour l'Espagne.}} {\textsc{- Friponnerie et
faussetés de Stanhope pour se défaire de Monteléon, qu'il trouvait trop
clairvoyant.}} {\textsc{- Albéroni, dupe de Stanhope et même de Riperda,
ne songe qu'au chapeau.}} {\textsc{- Triste état du gouvernement
d'Espagne.}} {\textsc{- Scandaleux pronostics du médecin Burlet sur les
enfants de la feue reine.}} {\textsc{- L'Angleterre tâche de détourner
la guerre de Hongrie.}} {\textsc{- Artifices contre la France.}}
{\textsc{- Ligue défensive signée entre l'empereur et l'Angleterre, qui
y veulent attirer la Hollande.}} {\textsc{- Conditions.}} {\textsc{-
Prié gouverneur général des Pays-Bas.}} {\textsc{- Juste alarme du roi
de Sicile.}} {\textsc{- Souplesses et artifices de l'Angleterre pour
calmer l'Espagne sur cette ligue.}} {\textsc{- Albéroni change
subitement d'avis et ne veut d'aucun traité.}} {\textsc{- Albéroni
flatte le pape\,; promet {[}des secours{]}\,; envoie Aldovrandi
subitement à Rome pour ajuster les difficultés entre les deux cours, en
effet pour presser son chapeau.}} {\textsc{- Bentivoglio et Cellamare,
l'un en méchant fou, l'autre en ministre sage, avertissent leur cour du
détail de la ligue traitée entre la France et l'Angleterre.}} {\textsc{-
Confidences de Stairs à Penterrieder.}} {\textsc{- Quel était ce
secrétaire impérial.}} {\textsc{- Considérations diverses.}} {\textsc{-
Manège infâme de Stairs.}} {\textsc{- Dure hauteur de l'empereur sur
l'Espagne et la Bavière aux Pays-Bas.}} {\textsc{- Le roi de Prusse à
Clèves.}} {\textsc{- Aldovrandi mal reçu à Rome, pénétré, blâmé.}}
{\textsc{- Avis au pape sur le chapeau d'Albéroni.}} {\textsc{- Cour
d'Espagne déplorable.}} {\textsc{- Jalousies et craintes d'Albéroni.}}
{\textsc{- {[}Il{]} rassure la reine.}} {\textsc{- Ce qu'il pense de son
caractère.}} {\textsc{- Bruits à Madrid fâcheux sur le voyage
d'Aldovrandi.}} {\textsc{- Demandes du roi d'Espagne au pape.}}
{\textsc{- Courte réflexion sur le joug de Rome et du clergé.}}
{\textsc{- Vues et mesures de l'Espagne sur ses anciens domaines
d'Italie.}} {\textsc{- Sage avis du duc de Parme.}} {\textsc{- Fol et
faux raffinement de politique d'Albéroni.}} {\textsc{- Manèges étranges
du ministère anglais sur le traité à faire avec la France.}} {\textsc{-
Horreurs de Stairs.}} {\textsc{- Rare omission au projet communiqué de
ce traité par les Anglais.}} {\textsc{- Fâcheuse situation intérieure de
la Grande-Bretagne et de la cour d'Angleterre.}} {\textsc{- Vues du roi
de Prusse.}} {\textsc{- Mauvaise foi de Stairs.}} {\textsc{- Intrigues
de la cour d'Angleterre.}}

~

L'Espagne, mécontente à l'excès du gouvernement, qui était entièrement
entre les mains de la reine et d'Albéroni, ne leur épargnait ni ses
soupçons ni ses discours\,; on n'y doutait point qu'Albéroni n'eût tiré
de grandes sommes des Anglais pour sa complaisance à leur passer
l'\emph{asiento} des nègres, et un traité de commerce aussi avantageux
pour eux que celui dont il avait procuré la signature\,; et les chasses
outrées par le froid de la fin de mars au pied des montagnes glacées de
l'Escurial où le prince des Asturies si jeune et si délicat suivait
toujours le roi son père, y donnaient un vaste champ, d'autant plus que
l'indiscrétion de Burlet, premier médecin du roi, semblait préparer à
quelque chose de funeste, en publiant que ce prince était fort menacé du
même mal dont la reine sa mère était morte, quoiqu'il soit vrai qu'il
n'en a jamais eu la moindre atteinte. Les vues d'Albéroni sur le
cardinalat étaient devenues publiques. Les différends avec la cour de
Rome demeuraient toujours au même état. Albéroni était accusé de les
suspendre pour forcer le pape à lui donner le chapeau. Acquaviva, qui
d'ailleurs passait pour un homme peu sûr, et qui pourtant avait à Rome
toute la confiance du roi d'Espagne, était abandonné aux volontés
d'Albéroni, et son fidèle agent. Giudice, dont les dégoûts augmentaient
à proportion du crédit d'Albéroni, ne tenait que des propos de retraite
et d'un mécontentement qui ne ménage rien. Il est vrai que le désordre
et l'épuisement des finances était extrême, que l'évêque de Cadix qui
les administrait avait ordre de fournir tout l'argent qu'Albéroni lui
demandait, qui n'était libéral que de celui qui était nécessaire pour
les voyages et les chasses, en quoi consistaient tous les plaisirs du
roi d'Espagne. Albéroni voulut retrancher sur la dépense de sa
garde-robe. Hersent qui en était chargé, et qui depuis l'affaire de la
réforme ne pouvait, comme on l'a vu, souffrir Albéroni, lui résista,
parla au roi d'Espagne avec la liberté d'un ancien domestique, et
l'emporta si bien que les dépenses de la garde-robe, au lieu d'être
retranchées, furent augmentées par ordre du roi.

Parmi ces occupations domestiques qui n'étaient pas les moindres
d'Albéroni, il était chargé de toutes celles du dehors\,; il négociait
seul avec les ministres que la Hollande et l'Angleterre tenaient à
Madrid, et il entretenait un commerce direct avec le pensionnaire de
Hollande, qui plus versé que lui en affaires lui fit accroire qu'il
redoutait autant que l'Espagne la puissance de l'empereur, et qu'il
était jaloux de celle de l'Angleterre. Albéroni leur avait proposé une
ligue défensive\,; il craignait en même temps que ces puissances n'en
voulussent une offensive, qui, étant sûrement contre la France, ne
pouvait convenir à l'Espagne. En même temps il se ravisa sur le
Prétendant\,; il crut de l'intérêt de l'Espagne de ne le pas abandonner
absolument, et lui fit toucher quelque argent. Ce malheureux prince
avait été à Commercy. Le duc de Lorraine l'y alla voir incontinent, et
le pria civilement de sortir de ses États\,; ce qu'il ne tarda pas de
faire, et, faute d'autre asile, alla à Avignon. Le duc de Lorraine
dépêcha à Londres pour y faire valoir cette conduite, et on y fut
content de lui.

Les puissances maritimes, bien informées du triste état de la marine
d'Espagne, du secours de vaisseaux qu'elle avait promis au pape sans en
avoir elle-même, et de son embarras pour faire partir la flotte des
Indes, au départ de laquelle elles avaient grand intérêt, lui en
offrirent. Albéroni répondit avec une singulière hardiesse que le roi
d'Espagne ne manquerait pas de vaisseaux, mais que s'il en voulait,
c'était acheter, non pas emprunter ou louer\,; et que si l'argent lui
manquait, il donnerait des hypothèques sur les Indes. Une déclaration si
indiscrète faite au secrétaire d'Angleterre à Madrid, qui avait le
dernier offert des vaisseaux, lui fit ouvrir les oreilles, et remontrer
à Londres tout l'avantage d'un pareil moyen pour négocier directement
aux Indes. Le pape en attendant mourait de peur des Turcs. Sa crainte de
l'empereur lui avait fait demander des vaisseaux au lieu de troupes,
dont l'arrivée en Italie aurait blessé la cour de Vienne, et les
Vénitiens, qui en désiraient pour leur sûreté, y renoncèrent sur ce que
l'Espagne ne leur en voulut envoyer que par terre\,; cependant le nonce
Aldovrandi se plaignait de l'inutilité de son séjour à Madrid où il ne
finissait aucune affaire\,; et le roi de Sicile se plaignait bien haut
de n'être pas protégé fortement à Rome par l'Espagne pendant le besoin
que cette cour avait des forces du roi d'Espagne. Ce besoin y parut si
pressant que le pape accorda au roi d'Espagne les mêmes levées que les
rois ses prédécesseurs et lui-même avaient faites sur le clergé
d'Espagne, mais dont le temps était expiré. Le roi d'Espagne prétendait
de plus les sommes qu'il aurait levées depuis l'expiration du temps de
cette permission. Rome s'en défendait sur ce que la charge serait trop
pesante, toutefois sans refus positif. La concession allait à quatre
millions d'écus\,; la prétention était de trois autres. L'intention du
pape était de terminer en même temps ses différends avec l'Espagne, et
avait laissé ce moyen à la discrétion d'Aldovrandi pour s'en servir à
propos. Albéroni le sut si bien pomper qu'il lui fit déclarer ses
ordres, en l'assurant que rien n'avancerait tant la conclusion de tout
que cette grâce faite au roi d'Espagne\,; puis lui fit déclarer par le
conseil que le roi ne devait de remerciements au pape que ceux de lui
avoir fait justice, qui n'était pas une raison pour qu'il se relâchât
sur les droits de sa couronne dans les différends qu'il avait avec Rome.

Ce fut ainsi qu'Albéroni se moqua d'Aldovrandi. Il voulait se réserver
le mérite de finir ces différends pour son cardinalat, et les laisser
durer tant qu'il ne le verrait pas prochain. Il était tellement maître
que tout s'adressait à lui, et qu'il remplissait à découvert le
personnage de premier ministre. Il s'applaudissait d'avoir la confiance
des étrangers et de son commerce direct avec le pensionnaire de Hollande
et avec Stanhope. Ce dernier l'assurait que l'Angleterre était prête à
faire une ligue défensive avec l'Espagne pour la neutralité de l'Italie,
et plus encore si les ministres allemands ne détournaient le roi Georges
de tout engagement capable de lui faire perdre l'occasion de profiter
des dépouilles de la Suède. Le secrétaire d'Angleterre à Madrid donna
les mêmes assurances à l'ambassadeur que le roi de Sicile y tenait.

Avec toute cette intelligence entre l'Espagne et l'Angleterre, Albéroni,
qui n'avait pas pardonné au duc de Saint-Aignan de s'être voulu mêler de
l'affaire de sa réforme des troupes, ne trouvait pas meilleure celle
qu'il voyait entre cet ambassadeur et le secrétaire d'Angleterre, qui de
concert agissaient pour l'intérêt des marchands français et anglais,
accablés d'injustices, qu'il n'était pas dans le dessein de faire
cesser. Sa lenteur à terminer ce qui restait encore à régler sur
l'\emph{asiento} des nègres\footnote{On a déjà dit que ce traite cédait
  aux Anglais le droit de faire exclusivement la traite des nègres dans
  l'Amérique espagnole pendant un temps déterminé.}, quoique accordée,
lui attirait des plaintes du ministre d'Angleterre\,; il se détermina
donc à leur faire une proposition sur l'envoi de leur warrant\footnote{Ce
  mot, qui est en abrégé dans le manuscrit de Saint-Simon, a été omis
  dans les anciennes éditions. C'est le terme anglais pour désigner un
  brevet, un diplôme royal.} de permission et sur le lieu et le temps de
la tenue des foires aux Indes, et du débit des Anglais qu'il crut
convenir également aux intérêts de l'Espagne et de l'Angleterre,
laquelle semblait s'éloigner des dispositions qu'elle avait témoignées
d'union avec la France. Les Impériaux n'oubliaient rien pour engager le
roi Georges à favoriser leur dessein sur l'Italie\,; et Monteléon sut
certainement qu'un bibliothécaire allemand du roi d'Angleterre
travaillait à un traité pour établir les droits de la maison d'Autriche
sur la Toscane.

Le désir de revoir son pays et de s'assurer de son larcin sur la Suède
persuada au roi Georges que l'Angleterre se trouvait désormais assez
calme pour qu'il pût faire un voyage à Hanovre. Le czar lui avait fait
part de ses projets. Le roi de Danemark le pressait de se déclarer comme
roi d'Angleterre contre le roi de Suède, qui était entré en Norvège\,;
enfin Wismar s'était rendu le 15 avril, qui restait unique au roi de
Suède au deçà de la mer.

Les Hollandais avaient une telle crainte de s'engager dans une nouvelle
guerre que Duywenworde, leur ambassadeur à Londres, qui s'était offert
pour moyenner une alliance entre la France, l'Angleterre et ses maîtres,
s'en ralentit tout à coup, et que les ministres de France et d'Espagne à
Londres lui ayant demandé si les Hollandais souffriraient tranquillement
que l'empereur violât la neutralité d'Italie et s'en rendît le maître,
il répondit nettement qu'ils ne feraient jamais rien qui pût déplaire a
ce prince.

L'incertitude de la guerre de Hongrie durait toujours. L'empereur, selon
sa coutume, parlait haut partout par ses ministres\,: à la Porte, par la
paix de Carlowitz, qui l'obligeait à s'armer en faveur des Vénitiens\,;
en effet, parce qu'il craignait que les Turcs ne s'étendissent dans la
Dalmatie\,; en France, que si on secourait le pape de troupes, elles
auraient plus affaire aux Impériaux qu'aux Turcs\,; en Angleterre, des
mépris de leur froideur\,; en Hollande, beaucoup de mécontentement sur
les prolongations de l'exécution du traité de la Barrière, quoiqu'ils la
voulussent flatter\,; c'est qu'avant de finir, les États généraux
voulaient s'assurer du terrain que l'empereur leur céderait\,; ce qui
dépendait du succès de la députation que la province de Flandre avait
envoyée à Vienne, qui répandait des listes des forces impériales à cent
soixante-douze mille sept cent quatre-vingt-dix hommes, et qui essaya
inutilement d'engager le régent à faire sortir de France le prince
Ragotzi qui, retiré aux Camaldules dans la plus sincère dévotion, ne
songeait à rien moins qu'à travailler à troubler l'empereur.

Stairs ne laissa pas de chercher encore à inquiéter sa cour sur la
France par rapport au Prétendant, quoique lui-même vît bien qu'il n'y
avait rien à en craindre\,; mais il prit un ombrage plus effectif de la
marche de quarante bataillons en Languedoc et en Guyenne sous un
commandant qui tenait de si près au Prétendant. Il en parla au régent
qui lui répondit que ces quarante bataillons n'étaient que dix, et
n'étaient envoyés que pour la consommation des denrées\,; que cela ne
regardait en rien l'Angleterre, à laquelle il était prêt de donner
toutes sortes de sûretés pour le maintien d'une parfaite intelligence.
Il ajouta un peu légèrement qu'il était vrai aussi qu'il était bien aise
d'avoir sur la frontière d'Espagne des troupes dont il fût assuré.
Stairs accoutumé à tourner tout en poison, ne pouvant là-dessus alarmer
l'Angleterre, fit à Cellamare confidence de ce propos, qu'il assaisonna
de toutes les réflexions les plus propres à l'inquiéter et à aigrir
l'Espagne. Heureusement il eut affaire à un homme sage qui se contentait
d'avoir les yeux bien ouverts, mais qui le connaissait, qui rabattit
toutes ses réflexions par les siennes, et qui manda en Espagne que si le
régent avait eu des desseins, il ne se serait pas privé, par la grande
réforme qu'il avait faite, des troupes nécessaires pour les exécuter.

Stairs, flatté de la réponse, que le régent lui avait faite avec tant
d'ouverture, espéra bientôt de parvenir à une explication formelle sur
Dunkerque, qui était le point sensible des Anglais. Le roi Georges se
proposait de l'obtenir comme préliminaire essentiel du traité que la
France proposait. Walpole voyait que les États généraux auprès desquels
il était désiraient, par crainte de toute apparence de guerre, qu'on
prît des mesures avec la France, en même temps que leur alliance
s'achèverait avec l'Angleterre et l'empereur, et le roi d'Angleterre
pressait la conclusion de cette alliance défensive\,; il assurait les
Hollandais que, dès qu'elle serait signée, il concourrait sûrement et
honorablement avec la France pour la garantie réciproque de leurs
successions, pourvu qu'elle consentît à dissiper toute inquiétude sur le
Prétendant, et à mettre le canal de Mardick hors d'état d'y pouvoir
naviguer.

La naissance d'un fils de l'empereur rehaussa encore le ton de ses
ministres dans toutes les cours, qui ne s'en promettaient pas moins que
la réunion de la monarchie d'Espagne à la maison d'Autriche sous le
règne du père ou du fils, et qui osaient s'en expliquer tout
ouvertement.

On a vu en son lieu la désertion de Langallerie, lieutenant général en
l'armée d'Italie, qui, recherché pour ses horribles concussions, passa
aux ennemis, qui lui conservèrent son grade dans les troupes impériales,
où il se distingua à l'attaque des lignes de Turin. Son père était
lieutenant général, mais pour gentilhomme c'était bien tout au plus.
Celui-{[}ci{]} était gueux, pillard et fort borné, ambitieux et plein de
son mérite. Il ne le crut pas suffisamment récompensé à Vienne et se mit
au service du czar, duquel il ne fut pas plus content. Il se retira donc
à Amsterdam, ou son peu de fortune lui tourna le peu de tête qu'il
avait. Il se fit protestant, et subsista quelque temps des charités de
cette ville. Un autre aventurier se joignit à lui sous un grand nom\,:
il se faisait appeler le comte de Linage, et disait avoir servi dans la
marine de France. Ils s'engagèrent à un officier turc ou soi-disant,
pour commander en chef, l'un par terre, l'autre par mer, pour établir
une nouvelle religion et une nouvelle république aux dépens de la Porte
et de l'empereur, qui les fit arrêter et exécuter à mort.

Bentivoglio, non content de n'oublier rien pour embraser la France du
feu de la discorde et du schisme, avertit le pape que les huguenots
recevaient toutes sortes de faveurs en France\,; que le régent était
prêt de conclure un traité de garantie mutuelle des successions de
France et d'Angleterre avec les puissances maritimes, au préjudice du
roi d'Espagne et du Prétendant, et de l'importance dont il était que le
pape le traversât efficacement. Il n'oublia pas d'exciter Cellamare, qui
avertit sa cour, laquelle, peu attentive aux affaires, excitait par sa
lenteur les plaintes du dehors et du dedans, qui retombaient à plomb sur
Albéroni, dont l'autorité et la confiance étaient à un point unique, et
les soupçons fort grands sur l'alliance prête à conclure entre les
puissances maritimes et l'empereur.

Le bill qui rendait les parlements septénaires avait enfin passé, et le
roi d'Angleterre songeait tout de bon à s'en aller à Hanovre. Quelque
assurance qu'il reçût du régent de la bonne intelligence qu'il voulait
conserver avec lui, il n'y voulait point ajouter foi\,; et quoique
Stairs même commençât à changer de langage et que les ministres anglais
fussent persuadés, ils voulaient entretenir les alarmes de leur nation.
Eux et les Hollandais sentaient leur faiblesse, et ne voulaient pas
renouveler la guerre ni prendre avec l'empereur, qui s'en plaignait, des
engagements qui pussent les y conduire, tandis que pour entretenir les
Anglais dans leur animosité contre la France, ils laissaient exprès
semer des bruits d'une guerre prochaine avec cette couronne, qui
protégeait toujours le Prétendant. La Hollande, plus franche, et qui
n'avait point ces intérêts particuliers à ménager, appuyait sur un
traité à faire avec la France, mais voulait auparavant conclure avec
l'empereur pour le ménager avec soin, malgré les contestations qu'ils
avaient avec lui par rapport à l'exécution de leur traité de la
Barrière.

Albéroni, de mauvaise humeur de voir l'Angleterre offrir à toutes les
puissances de traiter avec elles, ne laissa pas de se charger de finir
avec elle les difficultés qui restaient dans leurs derniers traités sur
l'\emph{asiento} des nègres et quelques points de commerce. Il se
moquait des bruits répandus contre lui sur les présents pécuniaires, et
tirait avantage du profit des décimes que la pointillerie du conseil
d'État aurait laissé perdre. Il regardait le duc de Saint-Aignan comme
le fauteur des plus fâcheux bruits qui couraient sur son compte, et le
prince Pio, qui commandait en Catalogne, comme son ennemi et l'ami des
censeurs de son gouvernement. L'arrivée de Scotti, de la part du duc de
Parme, qu'il n'avait pu empêcher, lui avait donné de grandes alarmes.
Pour le tenir de court et l'éclairer de plus près, il l'avait accablé
d'amitiés et logé chez lui. Il se fit communiquer ses instructions, et
s'en débarrassa le plus promptement qu'il put, avec des présents
considérables qu'il lui procura et une pension de cinq cents pistoles du
roi d'Espagne, avec quoi il s'en retourna à la cour de Parme. En même
temps il se faisait de misérables mérites auprès du régent d'avoir
détourné de fâcheux avis donnés au roi d'Espagne sur les troupes
envoyées en Languedoc et en Guyenne sous le duc de Berwick, et
l'exhortait à une liaison parfaite avec le roi d'Espagne, et à une
confiance entière en ses intentions et en sa probité.

En même temps il voulut savoir quels seraient les engagements que
l'Angleterre prendrait pour une ligue offensive, et les conditions qui
lui seraient offertes pour y engager l'Espagne, surtout pour ce qui
regardait la neutralité de l'Italie. Stanhope entortilla sa réponse {[}à
Albéroni{]} de force compliments, se tint dans le vague, lui voulut
persuader que la seule alliance défensive arrêterait les Impériaux sur
l'Italie\,; qu'en exprimer la neutralité dans le traité serait s'exposer
à en troubler le repos\,; qu'il n'était pas temps d'en faire une
stipulation expresse, et, de là, se mit à charger les artifices des
Impériaux, et alléguer des propositions qu'ils avaient faites à
l'Angleterre, qui n'avait pas voulu y entrer. Il s'étendit sur les
avantages que l'Espagne tirerait de cette alliance défensive qui, en
même temps, ferait renouveler les anciens traités\,; enfin que, pour
assurance de la neutralité de l'Italie, on conviendrait d'un article
séparé, dans les termes les plus forts, qui serait signé de part et
d'autre. Monteléon, qui aurait voulu des engagements plus forts et plus
précis, ne laissa pas de presser sa cour d'accepter ses offres qui, tant
que l'engagement durerait, empêcheraient l'Angleterre d'en prendre de
contraires à l'Espagne, et qui étaient une ouverture pour des vues plus
considérables au roi d'Espagne, en cas d'un malheur, en France. En même
temps l'Angleterre n'oubliait rien pour que l'Espagne fût contente de sa
conduite. Les menaces qu'un vice-amiral anglais avait faites à Cadix sur
les injustices dont les marchands de sa nation se plaignaient furent
désavouées, et la liaison là-dessus du secrétaire que l'Angleterre
tenait à Madrid avec le duc de Saint-Aignan blâmée. Stanhope, en même
temps qu'il accablait Monteléon d'amitiés, de distinctions, d'apparente
confiance, le trouvait trop clairvoyant\,; il demandait son rappel comme
d'un ministre vendu à la France, espion du régent, et dépendant du
dernier ministère français, qui gouvernait en Espagne. C'était, en deux
mots, tout ce qui pouvait le plus aliéner de lui le soupçonneux
Albéroni, à qui il écrivait directement de tout avec tant d'art et de
flatterie, qu'il lui persuadait tout ce qu'il voulait en se moquant de
lui, jusque-là qu'Albéroni, sur la parole de Stanhope, était intimement
assuré que jamais l'Angleterre ne permettrait aucun agrandissement de
l'empereur en Italie. Il était dans la même duperie sur les Hollandais,
sur ce que leur ambassadeur Riperda, qui avait gagné sa confiance, et
qui pourtant n'avait ni crédit, ni considération, ni estime dans sa
patrie, l'avait assuré que ses maîtres déclareraient la guerre à
l'empereur s'il entrait en Italie. Le roi et la reine d'Espagne
n'étaient du tout occupés que de la chasse, Albéroni uniquement de leur
plaire et de son chapeau. Tel était le gouvernement de l'Espagne, et le
ressort unique qui y conduisait tout. Les funestes et impertinents
pronostics de Burlet sur la santé de tous les enfants de la feue reine
continuaient à faire horreur, et à donner lieu aux discours et aux
bruits les plus scandaleux, et qui à la fin se trouvèrent les plus faux.

Le ministère anglais, persuadé qu'il était de l'intérêt de cette
couronne que l'empereur fût toujours libre de pouvoir attaquer la
France, et qu'il n'y avait d'alliance utile à l'Angleterre qu'avec
l'empereur, n'oubliait rien à Constantinople pour détourner la guerre.
Le grand vizir répondit ambigument, mais hautement, à l'ambassadeur
d'Angleterre, consentant toutefois à ce que le roi d'Angleterre fût
médiateur, s'il le voulait être, qui y consentit aussitôt, et dépêcha à
Venise, à Vienne et à Constantinople au plus tôt. En même temps,
persuadé que la France pénétrait leurs intentions, et ferait son
possible pour empêcher les États généraux d'entrer dans l'alliance
défensive qui leur était proposée par l'empereur et les Anglais, il
n'était rien que ces derniers ne fissent pour décrier la France en
Hollande. Stairs, toujours le même, empoisonnait les réponses les plus
gracieuses qu'il recevait du régent, et les démarches qu'il l'engageait
de faire à Rome pour faire sortir le Prétendant d'Avignon, et ne cessait
de prêter des desseins secrets à Son Altesse Royale, dont l'Angleterre
devait s'alarmer.

Enfin le 3 juin le traité de ligue défensive fut signé entre l'empereur
et le roi d'Angleterre. Les Hollandais n'y entrèrent pas encore, mais
l'empereur se promettait tout là-dessus de l'industrie de Prié qu'il
envoyait en même temps gouverner en chef les Pays-Bas\,; et le roi
d'Angleterre, de son autorité en personne, à son passage pour aller à
Hanovre. Les condition de ce traité ne furent pas d'abord toutes
publiques, mais on sut qu'il y avait une promesse mutuelle de douze
mille hommes, évalués en vaisseaux si l'empereur l'aimait mieux, et une
garantie réciproque des possessions dont les deux parties jouissaient
alors, et de celles qui pourraient leur accroître par voie de
négociation. En même temps le roi d'Angleterre facilita à l'empereur un
emprunt à Londres de deux cent mille livres sterling, dont il se rendit
comme garant. Il n'était pas difficile de voir que la Sicile était
l'objet qu'on se proposait dans un traité qui laissait à l'empereur le
choix de vaisseaux au lieu de troupes, et qui portait une garantie
réciproque des possessions non seulement actuelles, mais de celles qui
pourraient accroître par voie de négociation. Trivié en parla fortement
à Stanhope. Il n'en reçut que des reproches sur les ménagements
prétendus de sa cour pour le Prétendant, à quoi il en ajouta d'autres
sur la conduite du roi de Sicile à l'égard de l'empereur. Parmi ces
hauteurs, Stanhope alla chez Monteléon l'assurer que le gouverneur de la
Jamaïque était rappelé pour quelques pirateries contre la flotte du
Pérou, qu'il avait souffertes, et un autre envoyé à sa place, avec ordre
de faire rendre aux Espagnols tout ce qui leur avait été pris. Il lui
protesta que le traité n'engageait qu'à une mutuelle défense en cas
d'attaque des États actuellement possédés par les parties
contractantes\,; qu'il n'y avait point d'article secret ni rien qui pût
préjudicier aux intérêts de l'Espagne. Monteléon avait trop répondu de
l'Angleterre pour n'en pas répondre jusqu'au bout. Il ne voulut pas
qu'on crût en Espagne qu'il se fût laissé tromper. Il se trouva donc
intéressé au dernier point à faire valoir les assurances que lui donnait
Stanhope pour véritables, et se plaignit à sa cour de la négligence qui
l'avait privée du fruit de traiter la première avec l'Angleterre, depuis
tant de temps que cette couronne l'en pressait. Albéroni, peu ferme dans
ses principes, avait changé d'avis\,; sa chaleur pour l'Angleterre était
refroidie\,; il avait pris opinion que le roi d'Espagne, retiré par la
situation de l'Espagne, dans un coin du monde, devait demeurer quelque
temps simple spectateur de ce qu'il s'y passerait sans prendre
d'engagement, et ne songer principalement qu'à remettre l'ordre dans le
commerce des Indes et dans ses finances, et mettre à part quelques
millions pour les occasions\,: chose d'autant plus aisée qu'il était le
seul prince de l'Europe libre de toutes dettes, parce que dans les temps
qu'il avait eu besoin d'emprunter il n'en avait pas eu le crédit. Le roi
d'Espagne ne dissimulait point son mécontentement du traité de
l'Angleterre avec l'empereur.

Il fit redoubler les soins et la diligence à travailler à l'escadre
destinée au secours du pape, se relâcha de quelques demandes que le
conseil voulait qu'il lui fit, et en obtint aussi quelques-unes.
Albéroni voulait plaire au pape et avancer son cardinalat. Aldovrandi
l'avait habilement ménagé, malgré la tromperie qu'il en avait essuyée,
et le concert entre eux fut poussé si loin que le nonce s'offrit d'aller
lui-même aplanir les difficultés qui arrêtaient l'accommodement des deux
cours. Albéroni fit un projet pour donner, l'année suivante, un plus
grand secours au pape, moyennant quelque imposition sur le clergé
d'Espagne et des Indes, et en chargea Aldovrandi, qui partit subitement
dans un carrosse du roi d'Espagne, qui le mena à Cadix, d'où il gagna
l'Italie sur les vaisseaux de Sa Majesté Catholique. On comprit aisément
qu'Albéroni n'avait pas oublié ses intérêts personnels dans une démarche
aussi singulière que l'envoi d'un nonce à Rome à l'insu de cette cour,
et la curiosité était grande sur les secrets dont pouvait être chargé un
courrier aussi extraordinaire. On crut que ce qui se passait en France
sur la constitution avait fait préférer la mer à Aldovrandi. Bentivoglio
y soufflait le feu tant qu'il pouvait, et tâchait d'irriter le pape de
toutes les chimères dont il pouvait s'aviser. Comme il avait des gens à
lui dans le secret du régent, il fut averti de tout le détail de la
ligue qui se traitait entre la France et l'Angleterre. Il se hâta d'en
informer le pape en l'assaisonnant de tout le venin qu'il y put jeter.
Il l'attribuait au désir qu'il imputait au régent de venir à la
couronne, faisait peur au pape de cette union avec les ennemis de
l'Église, et l'exhortait à les empêcher de la détruire en prenant des
liaisons avec ceux qui pouvaient l'empêcher. Cellamare avertit sa cour
que la principale condition du traité était la garantie réciproque des
successions aux couronnes de France et d'Angleterre, suivant la paix
d'Utrecht\,; que de plus les ouvrages du canal de Mardick cesseraient,
et que le Prétendant sortirait d'Avignon\,; il se plaignait aussi bien
que Monteléon de la négligence de l'Espagne qui laissait faire aux
autres des liaisons qu'elle aurait pu prendre avant eux, et qui lui
auraient été utiles.

Penterrieder, secrétaire de la cour impériale à Paris, ne pouvait
concilier l'alliance prête à se faire entre la France et l'Angleterre
avec la ligue nouvellement signée entre l'empereur et le roi Georges.
Stairs lui faisait confidence des ordres de sa cour, et des réponses
qu'il recevait du régent, et il tenait alors le traité pour conclu,
parce qu'il semblait que la signature ne dépendait plus que de la sortie
du Prétendant d'Avignon, et la garantie réciproque des successions
semblait à Penterrieder incompatible avec l'engagement pris par
l'Angleterre de sou tenir les droits de l'empereur. Penterrieder était
une manière de géant qui avait plus de sept pieds de haut, avec un
visage et une voix de châtré, comme on le croyait être aussi, et la
corpulence à peu près de sa taille, dont il était toujours honteux et
embarrassé. Il avait été petit scribe dans les bureaux de Vienne\,; son
esprit, très supérieur à son petit état, l'avait conduit à être
secrétaire de Zinzendorf, chancelier de la cour de Vienne, et ministre
de conférence, qui est ce que nous appelons ici être ministre d'État et
avoir les affaires étrangères. Zinzendorf, fort content de lui, l'avait
poussé au secrétariat de quelques conseils, et enfin l'avait fait
employer dans l'empire, puis dans les principales cours, et toujours
avec grande satisfaction partout. Ce secrétaire, poli, fort en sa place,
mais pétri des maximes et des hauteurs autrichiennes, sans avoir comme
de soi rien que de très modeste et de mesuré, avec beaucoup de savoir,
d'esprit, d'insinuation et de langage, remarquait bien les ménagements
réciproques de l'Espagne et de l'Angleterre, et le grand intérêt de la
dernière à conserver les avantages qu'elle avait obtenus de la première
pour son commerce, et il réfléchissait beaucoup sur l'espérance qui se
montrait trop en France d'engager la Hollande à traiter séparément de
l'Angleterre, si cette couronne ne finissait point, fondée sur le
mécontentement de la Hollande de la ligue conclue sans elle entre
l'Angleterre et l'empereur. On soupçonnait que cette dernière union
fondée sur l'intérêt commun de ces deux puissances, s'étendait jusqu'à
la garantie des États qu'ils pourraient acquérir par des traités, et que
le Portugal y entrait en troisième\,; et on s'aperçut que depuis la
signature de ce traité, l'Angleterre ménagea moins le roi de Sicile.
Elle n'avait alors de considération que pour l'empereur et l'Espagne,
laquelle pouvant aisément entrer en défiance de ce traité avec
l'empereur, l'Angleterre eut grand soin de l'assurer qu'il ne la
regardait en aucune sorte, mais la France seulement\,; et Stairs même
avec qui le régent traitait ne s'en cachait pas, dans le temps même que
le régent l'assurait être en état et en volonté actuelle de faire sortir
le Prétendant d'Avignon. En même temps tout fut en désordre dans les
Pays-Bas, où il n'y avait aucune sorte d'autorité ni de gouvernement, en
attendant le marquis de Prié, nommé gouverneur général de ces provinces.
Il y vint un ordre de confisquer les biens de tous ceux qui étaient au
service d'Espagne, et des menaces à tous ceux qui tenaient des pensions,
des emplois, des titres et des honneurs, tant du roi d'Espagne que de
l'électeur de Bavière.

Le voyage du roi de Prusse, si attentif à son agrandissement, inquiéta
également les États généraux et la cour de Vienne. Ce nouveau monarque,
aussitôt après la mort de l'électeur palatin, était allé à Clèves\,; ce
qui leur fit craindre une entreprise sur Juliers\,; et à Vienne, les
forces et les desseins de ce prince, et ses négociations avec la France.

Aldovrandi ne trouva pas à Rome ce qu'il y avait espéré, quoique son bon
ami Aubenton eût tâché de prévenir le pape que son voyage n'était que
pour concerter avec lui les moyens de lui procurer pour l'année suivante
de plus grands secours d'Espagne, et pour lui rendre compte de sa
négociation en ce pays-là. Le pape, très mécontent de voir arriver son
nonce sans avoir pu s'y attendre, trouva qu'il devait rendre compte de
sa négociation par ses dépêches, et comprit que les plus grands secours
d'Espagne ne lui seraient offerts qu'à des conditions de grâces qu'il ne
pourrait accorder. On jugeait à Rome qu'Aldovrandi voulait obtenir le
gouvernement de cette ville, et servir Albéroni pour le cardinalat. Ceux
à qui le pape s'ouvrait là-dessus, et qui ne voulait lui accorder le
chapeau que par la nomination d'Espagne, l'en détournaient. Ils lui
conseillaient de ne pas souffrir qu'Albéroni s'en adressât à autre qu'à
Sa Sainteté, qui le devait amuser par la cour de Parme\,; lui cacher à
jamais ses véritables dispositions, et que si elle ne pouvait terminer
ses différends honorablement avec l'Espagne que par ce chapeau, ce
serait alors bien fait de le jeter à Albéroni. Cet ambitieux voyait avec
un extrême dépit sa faveur s'ombrager par celle d'Aubenton, à qui le roi
d'Espagne confiait plusieurs affaires du gouvernement et même des
finances, et de la liaison de ce jésuite avec Mejorada. Le roi et la
reine s'étaient disputés et querellés. On croit aisément les changements
qu'on désire dans un gouvernement sans ordre et sans règle, et dans une
cour ténébreuse, pleine de confusion, où la fausseté et la calomnie
était ce qui approchait le plus près de Leurs Majestés Catholiques, et
où chacun se croyait tout permis, et se promettait tout des plus
mauvaises voies, en sorte que les bruits les plus inquiétants se
trouvaient les plus répandus. Albéroni commençait à craindre. La reine
l'avertit que le roi avait beaucoup de soupçons contre lui, et
qu'elle-même ne voulait plus se fatiguer du gouvernement. Quelques
représentations qu'Albéroni lui sût faire, elle ne les goûtait point. Il
la connaissait incapable des affaires, susceptible de mauvais conseils,
peu touchée de se conserver ceux qui lui donnaient de bons avis, prête à
les abandonner et à les oublier à la moindre difficulté qu'elle
trouverait à les soutenir, et facile à se laisser conduire par ceux qui
l'environnaient. Il redoutait surtout deux hommes de rien que la reine
avait connus à Parme, et qu'elle voulait toujours faire venir en
Espagne\,; et il ménagea si bien le duc de Parme qu'il fit en sorte que
ce prince les empêcha de sortir de ses États. On avait pénétré à Madrid
qu'Aldovrandi avait emporté un mémoire de la main du roi d'Espagne, et
là-dessus on bâtissait des chimères en faveur des enfants de la reine au
préjudice du prince des Asturies. Ce mémoire ne contenait rien moins. Le
roi d'Espagne y demandait au pape la moitié du \emph{sussidio y
excusado}\footnote{Le mot espagnol \emph{subsidio} ou \emph{sussidio}
  désigne d'une manière générale toute espèce d'impôt. On appelait
  \emph{excusado} un tribut spécial que le roi d'Espagne levait sur les
  revenus du clergé avec l'autorisation du pape.}**, qui est une
imposition sur le clergé dont il ne jouissait pas depuis cinq ans, et le
même aux Indes\,; un délai de quelque temps de nommer aux vacances des
archevêchés et des évêchés d'Espagne, pour en amasser les revenus et les
employer à l'armement de mer que le pape désirait pour l'année suivante,
ainsi que les libéralités que le clergé voudrait bien faire, suivant les
brefs d'exhortation que Sa Sainteté avait envoyés, et remettre ces
sommes au commissaire del cruzade\footnote{On appelait \emph{crusade,
  cruzade} ou \emph{cruzada}, le droit que le pape Jules II avait
  accorde, en 1509, aux rois d'Espagne de percevoir un impôt sur les
  biens du clergé pour faire la guerre aux infidèles. Il y avait un
  conseil particulier de la cruzade, dont le président portait le nom de
  \emph{commissaire de la cruzade}.}, qu'on comptait devoir être
suffisantes pour armer douze vaisseaux et six galères. On peut réfléchir
en passant sur la dureté du joug que le clergé exerce sur les plus
grands rois qui ont eu la faiblesse de se le laisser imposer, et qui ne
peuvent le secouer que par des extrémités qui les séparent de l'Église,
comme il est arrivé à la moitié de l'Europe, que Rome et leur clergé a
mieux aimé perdre\,: Rome par sa tyrannique domination qui n'avait de
fondement que son usurpation contre les préceptes si formels de
Jésus-Christ\,; le clergé par son insolence et son indépendance.

Il est vrai que ces demandes ne méritaient pas pour courrier un nonce
dépêché à l'insu du pape, qui avait eu tant de peine à le faire recevoir
comme que ce fût à Madrid. On se persuada donc qu'il s'agissait de
former une ligue entre l'Espagne et les princes d'Italie, et même de
prendre des mesures avec le pape sur les événements qui pouvaient
arriver en France. Le roi d'Espagne avait toujours été entretenu dans le
désir de recouvrer les États qu'il avait cédés en Italie par la paix,
beaucoup plus depuis son second mariage. Ce dessein ne se pouvait
effectuer que par une ligue des princes d'Italie dont le roi de Sicile
serait le chef comme le plus puissant, et Villamayor, ambassadeur
d'Espagne à Turin, avait ordre d'y travailler sous l'inspection du duc
de Parme. Ce prince, qui sentait toutes les difficultés d'amener à ce
point un souverain aussi sage, aussi clairvoyant, aussi défiant, aussi
mal prévenu d'estime pour le gouvernement d'Espagne, et aussi fortement
de crainte de la puissance et des desseins de l'empereur, et dont toute
la conduite inspirait aussi peu de confiance, voulait que l'Espagne,
suivant sa première pensée, engageât l'Angleterre à faire une ligue avec
elle pour la neutralité de l'Italie, dont le premier intérêt était d'en
détourner la guerre. C'était aussi dans cette vue que l'Espagne avait eu
tant de facilité en accordant à l'Angleterre un traité de commerce si
avantageux, et l'\emph{asiento} des nègres. Elle était sur le point d'en
recueillir le fruit qu'elle s'en était proposé, quand tout à coup, et
sans aucun changement de conjonctures, Albéroni changea lui-même d'avis
tout à coup, et se mit à désirer que l'empereur contrevînt à la
neutralité de l'Italie, dans l'idée que les Impériaux ne pourraient
exécuter leur projet si promptement que l'Espagne n'eût part aux
mouvements de l'Italie\,; et que, s'il arrivait alors que le roi
d'Angleterre eût besoin de l'Espagne, il serait facile d'obtenir par lui
les avantages qu'elle pourrait désirer. C'était sur ce fondement ruineux
et chimérique qu'Albéroni avait rejeté l'alliance d'Angleterre pour la
neutralité d'Italie, qu'il avait tant souhaitée, et qu'il pouvait alors
conclure\,; et il le devait d'autant plus qu'il aurait par là
contrebalancé celle que l'Angleterre venait de signer avec l'empereur.

Telle était l'habileté et la capacité de ce ministre qui gouvernait
absolument l'Espagne. Il disait à ses amis qu'il fallait bien vivre avec
la France, écarter tout sujet d'ombrage et de jalousie, mais se tenir
doucement et sans bruit en État d'agir quand le besoin et l'occasion le
demanderaient, ou que si le roi d'Espagne prenait le parti d'abandonner
des vues éloignées, il devait tirer de ceux qui profiteraient de ce
sacrifice des engagements à soutenir ses droits en Italie. Albéroni
ajoutait à ces raisonnements des lamentations sur l'inaction du roi
d'Espagne, tandis que le régent n'oubliait rien pour se fortifier au cas
qu'il arrivât en France ouverture à succession.

Les manèges du ministère anglais étaient infinis sur ce traité avec la
France, quoiqu'ils en sentissent la nécessité par rapport à la
tranquillité intérieure de la Grande-Bretagne et à leurs vues au dehors.
Ils l'éludaient pour le prolonger, afin d'entretenir la défiance de leur
nation à l'égard de la France, et de se conserver le prétexte d'avoir
des troupes en Angleterre et des subsides du parlement. Ainsi ils
transférèrent la négociation de Paris à la Haye, où ils firent
communiquer le traité au pensionnaire, à Duywenworde qui revenait de
l'ambassade de Londres, et à l'ambassadeur de France, bien moins pour en
faciliter la conclusion que pour intéresser les Hollandais dans les
demandes de l'Angleterre. Stairs, piqué de se voir enlever la conclusion
d'une négociation commencée par lui et si avancée, se mit à déclamer
contre les ministres de France, qui, à l'entendre, avaient changé toutes
les dispositions si favorables que le régent lui avait témoignées\,; et
ne cessa de mander au roi d'Angleterre de se défier de ce prince qui ne
voulait que le tromper et favoriser le Prétendant. Le singulier de ce
projet de traité envoyé à la Haye fut qu'il n'y était pas fait la
moindre mention du traité d'Utrecht, ni des garanties réciproques des
successions aux couronnes de France et d'Angleterre, deux articles
néanmoins qui devaient être la base d'une alliance à faire pour
maintenir le repos de l'Europe. On soupçonna que c'était l'effet des
avantages obtenus par les derniers traités de commerce faits entre
l'Espagne et l'Angleterre, que celle-ci ne voulait perdre pour rien, et
que c'était pour la même raison que Stanhope n'avait pas témoigné le
moindre chagrin à Monteléon, lorsque, après avoir vivement poursuivi la
conclusion d'une alliance avec l'Angleterre, l'ambassadeur espagnol
avait cessé tout à coup d'en parler.

Les mécontents se multipliaient en Angleterre, la fermentation générale
menaçait d'une révolution, la division de la famille royale était
extrême. On a vu en son lieu l'aventure de l'épouse du roi Georges
longtemps avant qu'il fût électeur et roi, et la catastrophe terrible du
comte de Kœnigsmarck. Le roi Georges ne pouvait souffrir le prince de
Galles qu'il ne croyait pas son fils, et l'aversion était réciproque.
Prêt à passer la mer, il laissait ce prince régent avec toute
l'apparence de l'autorité, sans aucune en effet par ses ordres et ses
instructions secrètes, en sorte que le prince de Galles n'eut pas le
pouvoir de conférer ni de changer les charges, ni de convoquer ou de
séparer le parlement. Une telle limitation lui fit refuser la régence.
Son père le menaça de faire venir d'Allemagne son frère l'évêque
d'Osnabrück, et de la lui donner, ce qui engagea le fils à l'accepter.
On était surpris avec raison que dans une conjoncture où les Anglais
eux-mêmes s'attendaient à voir chez eux les plus étranges scènes, le
régent préférât une alliance avec eux au parti de fomenter un feu qui
pouvait embraser l'Angleterre.

La surprise était pareille de voir dans ces temps si critiques le roi
Georges faire le voyage d'Allemagne. Lui et le roi de Prusse, son
gendre, étaient inquiets des projets l'un de l'autre. Le dernier visait
à s'emparer des duchés de Berg et de Juliers, si l'électeur palatin
venait a manquer, parce que l'inégalité de son mariage exclurait les
enfants qu'il en pourrait laisser des fiefs et des dignités de l'empire.
Il comptait que la France aimerait mieux ces États entre ses mains qu'en
la disposition de l'empereur. Il semblait aussi se détacher de l'intérêt
de ses alliés dont il n'approuvait pas les entreprises sur le pays de
Schonen. Il aurait vu avec jalousie son beau-père réussir à faire
stathouder de Hollande l'évêque d'Osnabrück son frère, à quoi il
craignait qu'il ne travaillât\,; et en même temps qu'il cultivait
bassement l'empereur, il en était mécontent et déclarait qu'il n'avait
aucune négociation avec lui. Penterrieder profitait de la mauvaise
humeur de Stairs et de ses confidences pour tenir les ministres
impériaux avertis de l'état de la négociation de la France avec
l'Angleterre, qu'ils traversaient de tout leur pouvoir.

Stairs en l'entamant n'avait jamais eu dessein de la conclure. Ses
protecteurs à Londres avaient trop d'intérêt à montrer toujours le
fantôme du Prétendant secrètement appuyé des secours et des desseins de
la France, pour conserver une armée en Angleterre et une source assurée
de subsides. Ils n'avaient osé s'opposer de front à la négociation\,;
mais ils n'en voulaient pas la conclusion, et ils en étaient bien
assurés entre les mains de Stairs. Le transport de la négociation en
Hollande leur fut donc, et à lui, également sensible, et Stairs n'oublia
rien pour la traverser.

La disgrâce du duc d'Argyle, favori et premier gentilhomme de la chambre
du prince de Galles, retarda le départ du roi d'Angleterre. Il fit
demander à ce duc la démission de ses charges de général de
l'infanterie, de colonel du régiment des gardes bleus, et de son
gouvernement de Minorque, qu'il envoya sur-le-champ. Le roi avait compté
qu'après cet éclat le prince de Galles n'oserait ne pas demander au même
duc la démission de sa charge de premier gentilhomme de sa chambre\,;
non seulement il ne le fit pas, mais il se piqua d'honneur de le
soutenir dans sa disgrâce. Le duc de Marlborough, qui végétait encore
parmi ses apoplexies, ennemi d'Argyle, et qui voulait élever sur ses
ruines Cadogan sa créature, poussait le roi. On crut que la princesse de
Galles y entra aussi contre Argyle, confident des galanteries de son
époux. Le comte d'Isla, frère d'Argyle, fut enveloppé dans sa disgrâce.
Le prince de Galles se prit aux ministres de son père, jura leur perte,
et résolut de se réunir aux torys. Stairs, instruit de la situation
intérieure de l'Angleterre, en craignit les suites et redoubla de
mensonges et d'artifices pour empêcher le traité avec la France,
laquelle aurait dû en être bien dégoûtée\,; mais le régent ne voyait que
par Noailles, Canillac et Dubois, lequel bâtissait tous ses desseins
personnels sur l'Angleterre, dont par conséquent, il voulait, à quelque
prix que ce fût, l'alliance étroite avec la France, où il nous faut
présentement retourner.

\hypertarget{note-i.-protestation-des-ducs-et-pairs-uxe0-la-suxe9ance-du-parlement.}{%
\chapter{NOTE I. PROTESTATION DES DUCS ET PAIRS À LA SÉANCE DU
PARLEMENT.}\label{note-i.-protestation-des-ducs-et-pairs-uxe0-la-suxe9ance-du-parlement.}}

Le procès-verbal imprimé de la séance du parlement (2 septembre 1715) ne
parle pas de la protestation des ducs et pairs\footnote{Voy. ce
  procès-verbal dans les \emph{Anciennes lois françaises}, t. XXI, pages
  2 et suivantes.}\,; mais un manuscrit de la Bibliothèque impériale du
Louvre, provenant de la famille de Caumartin (F. n° 401), contient
quelques annotations marginales qui confirment et complètent, avec de
légères modifications, le récit de Saint-Simon relativement à cette
protestation. La note est connue en ces termes\,:

«\,Il faut remarquer qu'avant de se lever, MM. les ducs de Saint-Simon
et de La Force ont demanda au parlement acte des protestations qu'ils
faisaient que rien de tout ce qui venait d'être fait ne pouvait leur
préjudicier. M. le premier président leur a répondu qu'ils pouvaient
présenter au greffe leurs protestations, et se pourvoir ainsi qu'ils
aviseraient. Ils ont répliqué qu'ils demandaient acte, et que si on leur
refusait, ils avaient amené un notaire pour verbaliser. M. le président
de Novion leur a répondu\,: \emph{Vous reconnaissez donc la cour pour
juge\,?} Ils ont répondu que non. Il leur a répliqué\,: \emph{Il n'y a
donc que le roi, messieurs, qui puisse vous juger\,; il faut attendre
qu'il soit en âge}. M. le duc d'Orléans a répliqué à cela qu'il
déciderait toutes ces contestations. M. le président de Novion a
répondu\,: \emph{Non pas, monsieur, s'il vous plaît\,; le roi seul en
sera juge}. M. le maréchal de Villars a pris la parole, et a dit à M. le
premier président que le roi défunt lui avait souvent dit que les ducs
avaient raison\,: \emph{Et moi}, a répondu M. le premier président,
\emph{le roi m'a dit tout le contraire}. Sur quoi chacun s'est levé.\,»

\hypertarget{note-ii.-muxe9pris-pour-les-anciens-usages-pendant-la-ruxe9gence.}{%
\chapter{NOTE II. MÉPRIS POUR LES ANCIENS USAGES PENDANT LA
RÉGENCE.}\label{note-ii.-muxe9pris-pour-les-anciens-usages-pendant-la-ruxe9gence.}}

La facilité avec laquelle le régent abandonna les anciens usages est
bien caractérisée dans le passage suivant des Mémoires inédits du
marquis d'Argenson\,:

«\,Saint-Cernain demanda à S. A. R. M. le duc d'Orléans, régent,
l'honneur de porter l'habit à brevet\,; il l'obtint et alla remercier.
Le régent répondit\,: \emph{Je souhaite, monsieur, que votre tailleur
vous le donne d'aussi bon cœur que moi}. Ledit Saint-Cernain est pauvre
et glorieux\,; au reste, brave et ambitieux. Il se pique de ressembler
au maréchal de Villars\,; il le copie\,; il prétend qu'il fera une aussi
grande fortune que lui. On l'a trouvé une fois s'exerçant à signer\,: le
maréchal-duc de Saint-Cernain. En attendant il va à pied. L'habit à
brevet allait mal à ce train-là. Le régent était non seulement fait à
multiplier les grâces ci-devant singulières, mais il avait une secrète
malice pour avilir tout ce que le feu roi avait eu à cœur d'illustrer\,:
cela provenait d'avoir été maltraité sur la fin du règne et par le
testament. Ajoutez à cela qu'une cour moderne se pique de tourner en
ridicule et de traiter avec une supériorité indiscrète tout ouvrage,
manière et respects de l'ancienne cour.\,»

\hypertarget{note-iii.-le-maruxe9chal-de-noailles-adrien-maurice.}{%
\chapter{NOTE III. LE MARÉCHAL DE NOAILLES
(ADRIEN-MAURICE).}\label{note-iii.-le-maruxe9chal-de-noailles-adrien-maurice.}}

Saint-Simon exprime souvent contre le duc de Noailles des sentiments de
haine et de mépris qui s'expliquent surtout par l'influence que le duc
de Noailles, devenu maréchal de France, exerça pendant une grande partie
du règne de Louis XV. Ceux qui voudront apprécier sérieusement le rôle
du maréchal de Noailles devront étudier non seulement les Mémoires
imprimés sous son nom, mais surtout ses nombreux manuscrits, dispersés
dans les bibliothèques de Paris. La Bibliothèque impériale seule possède
près de quarante volumes in-folio de correspondance et Mémoires du
maréchal de Noailles\footnote{B. I. ms. suppl. fr. 2232 nos 22 et suiv.}.
Ce n'est pas ici le lieu d'examiner, d'après ces papiers, quel a été le
véritable caractère du maréchal de Noailles. Je me bornerai à extraire
des Mémoires inédits du marquis d'Argenson une série de notes qui
montrent à la fois la puissance du maréchal de Noailles et la jalousie
qu'il excitait à l'époque même où Saint-Simon écrivait ses Mémoires. Ce
qu'il y a de plus curieux dans ces extraits est la lettre remise par
Louis XIV mourant à M\textsuperscript{me} de Maintenon, et par elle au
maréchal de Noailles qui ne devait la donner qu'au nouveau roi. On en
trouvera l'analyse dans l'article qui porte la date du 9 avril 1743.

~

{\textsc{«\,14 novembre 1740.}} {\textsc{- Les Noailles sont
actuellement dans l'intrigue la plus violente. Comme M. de Charost se
meurt, il s'agit de sa place de chef du conseil royal et d'une place de
ministre au conseil d'État. À cette occasion, le maréchal de Noailles
remue ciel et terre pour cela. Il a enfourné l'affaire des bâtards pour
faire régler le rang de M. de Penthièvre avant de le marier, et cela lui
retombera sur le corps. Son fils le duc d'Ayen}}

~

\footnote{Louis de Noailles, fils aîné d'Adrien-Maurice.} fait
l'amoureux de M\textsuperscript{me} de Vintimille \footnote{Pauline-Félicité
  de Nesle, née en 1712, morte en 1741.} , sœur de M\textsuperscript{me}
de Mailly \footnote{Louise-Julie de Nesle, née le 1er mars 1712, morte
  le 30 mars 1751.} . Par ses conseils, elle cherche à supplanter sa
sœur, et toutes les confidences du roi vont à elle\,; on ne sait ce qui
en sera.\,»

«\,18 décembre 1740. --- Le parti du cardinal Tencin travaille à force
et avec grande apparence de succès. M\textsuperscript{me} de Vintimille
étant au grand bien avec le duc d'Ayen, elle est pour qu'on prenne ce
premier ministre\,; et M\textsuperscript{me} de Mailly, étant fort
gouvernée par sa sœur, commence, dit-on, à entrer dans ce maudit projet.
La grosse faction des Noailles et des légitimés y coopère de toutes ses
forces.\,»

«\,16 septembre 1741. --- On se pique de prôner les Noailles, et de leur
donner un grand crédit apparent depuis la mort de M\textsuperscript{me}
de Vintimille\footnote{M\textsuperscript{me} de Vintimille était morte
  au commencement de septembre 1741.}. On manda d'abord le maréchal de
Noailles à Saint-Léger, pour travailler aux intérêts de
M\textsuperscript{me} de Mailly, en vue de la mort du petit du Luc, et
il travailla deux heures avec le roi. Ses fils et M\textsuperscript{lle}
de Noailles ne quittent pas le roi. Le crédit de M\textsuperscript{me}
la comtesse de Toulouse paraît accru.\,»

«\,19 mars 1743. --- Voilà le maréchal de Noailles général de toutes nos
forces de France depuis le Rhin jusques à la mer, et maître d'y mouvoir
nos forces arbitrairement pour la dépense de la frontière. Voilà M. de
Belle-Isle tout à fait disgracie, etc. La sagesse ne consiste pas
seulement dans l'abstention des folies, ni même dans celle des desseins
trop élevés\,; elle demande plus de sagacité dans des temps difficiles
que n'en ont les Noailles, les Orry, les Amelot,\,» etc.

«\,9 avril 1743. --- La survenue du maréchal de Noailles dans le conseil
rend la vie très dure aux ministres. Ce n'est pas un premier ministre,
mais c'est un inspecteur importun qui leur a été donné et qui se mêle de
tout, quoiqu'il ne soit le naître de rien. On assure que cela a été
inspiré au roi par M. Orry ou par Bachelier.\,»

--- «\,Le maréchal de Noailles a rendu au roi, quelques jours après la
mort du cardinal {[}de Fleury{]}, une lettre de Louis XIV, lettre très
longue, toute écrite par ce monarque, et peu de jours avant l'extrémité
de la maladie dont il mourut. Cette lettre avait été remise à
M\textsuperscript{me} de Maintenon, pour la rendre par quelqu'un de sûr
au roi son petit-fils et successeur.

«\,Il lui disait que cette lettre ne lui devait être rendue que quand il
pourrait l'entendre, et quand il commencerait à gouverner réellement par
lui-même. Louis XIV y disait qu'ayant longtemps gouverné, il pouvait lui
donner des avis tirés d'une profonde expérience\,; qu'il avait fait
plusieurs grandes choses, mais qu'il avait fait quantité de sottises\,;
qu'il lui donnait avis de s'appliquer principalement au choix des
ministres\,; que quand il commencerait à gouverner, il laissât quelque
temps en place les ministres qu'il y trouverait, pour les mieux
connaître et faire ensuite des choix plus sûrs\,; qu'il se gardât bien
de prendre jamais de premier ministre\,; que, dans les commencements, il
composât son conseil de plusieurs personnes habiles, et qu'il n'y
craignit point la multitude\,; que même \emph{les gens
d'imagination}\footnote{Souligné dans le manuscrit.} y seraient utiles,
pourvu qu'ils fussent gens de probité, parce qu'ils feraient naître des
idées.\,»

«\,Cette lettre ayant été transmise de M\textsuperscript{me} de
Maintenon au maréchal de Noailles, c'est par là que celui-ci a été
choisi pour ministre, son caractère se trouvant quasi désigné par ce
dernier trait.\,»

«\,21 mai 1743. --- Lui {[}Belle-Isle{]} et le maréchal de Noailles se
sont tout à fait raccommodés ensemble par l'entremise de
Bachelier\footnote{Premier valet de chambre du roi.}. Le Noailles est un
bon homme\,; il n'y en a point de meilleur\,; mais il est bilboquet\,;
il sera bien avec tout le monde, et ne décidera jamais de rien.\,»

«\,30 juillet 1743. --- Le duc de Grammont et la timidité du duc de
Noailles a rendu notre honte irrémédiable à Dettingen\footnote{La
  bataille de Dettingen fut livrée le 27 juillet 1743.}. Nous sommes
sans ressources et à la merci de nos ennemis, qui n'ont plus à mesurer
notre destruction que sur leurs désirs.\,»

«\,5 août 1744. --- Le roi se trouve actuellement à la tête de trente
mille hommes destinés à joindre l'armée du maréchal de Coigny, et M. le
duc d'Harcourt, à la tête de dix-huit mille hommes avant-coureur de Sa
Majesté, se trouve sous Phalsbourg.

«\,Il y aura scission entre les généraux\,; mais la présence du roi et
des ministres les décidera\,; le maréchal de Noailles achèvera de tomber
de cette affaire-ci. La place de secrétaire d'État des affaires
étrangères ne se donne point. Cette interruption de ministère continue
toujours. On disait que c'était la haute faveur de M. de Noailles qui en
était cause. Mon frère\footnote{Le comte d'Argenson, ministre de la
  guerre.} me dit en partant que c'était la perle du ministère\,; que
les seigneurs et favoris le détruisaient.\,»

\hypertarget{note-iv.-conseil-extraordinaire-de-finances-tenu-le-24-octobre-1715-pour-linstitution-de-la-banque-de-law.}{%
\chapter{NOTE IV. CONSEIL EXTRAORDINAIRE DE FINANCES TENU LE 24 OCTOBRE
1715 POUR L'INSTITUTION DE LA BANQUE DE
LAW.}\label{note-iv.-conseil-extraordinaire-de-finances-tenu-le-24-octobre-1715-pour-linstitution-de-la-banque-de-law.}}

~

~

Les détails du conseil de finances mentionnés par Saint-Simon se
trouvent dans les papiers du duc de Noailles \footnote{Bibl. imp., ms.
  S. F. 2232, t. XXIII. - Délibérations du conseil particulier des
  finances du 20 septembre 1715 au 15 mai 1716.} . Les membres
ordinaires du conseil des finances étaient le duc d'Orléans\,; le
maréchal de Villeroy, chef du conseil\,; le duc de Noailles,
président\,; le marquis d'Effiat, vice-président\,; Le Pelletier des
Forts, Rouillé du Coudray, Le Fèvre d'Ormesson, Gilbert de Voisins, de
Gaumont. Taschereau de Baudry, Dodun, conseillers\,: Lefèvre et de La
Blinière, secrétaires. Outre ces membres ordinaires du conseil, le
régent appela à celui du 24 octobre MM. Pelletier \footnote{Ce
  Pelletier, ou Le Pelletier, est appelé de La Houssaye, pour le
  distinguer des autres personnages du même nom. Il fut contrôleur
  général du 10 décembre 1720 au 10 avril 1722. Voy. Saint-Simon à
  l'année 1720.} , Amelot, Bignon, d'Argenson, conseillers d'État\,; Le
Blanc et de Saint-Contest, maîtres des requêtes, et d'Aguesseau,
procureur général. Voici le procès-verbal de cette séance, dans laquelle
le système de Law se produisit pour la première fois en public, et fut
apprécié par des hommes d'État\,:

«\,M. Fagon\footnote{Saint-Simon parle souvent de ce personnage qui
  était conseiller d'État.} a proposé le projet du sieur Lass d'établir
une banque à Paris. Il en a exposé la nature et la constitution\,; il a
fait voir d'un côté tous les avantages, et de l'autre tous les
inconvénients, par objections et par réponses.

«\,L'idée de cette banque est de faire porter tous les revenus du roi à
la banque\,; de donner aux receveurs généraux et fermiers des billets de
dix écus, cent écus et mille écus, poids et titres de ce jour, qui
seront nommés \emph{billets de banque\,;} lesquels billets seront portés
ensuite par lesdits receveurs et fermiers au trésor royal, qui leur
expédiera dos quittances comptables. Tous ceux à qui il est dû par le
roi ne recevront au trésor royal que des billets de banque, dont ils
pourront aller sur-le-champ recevoir la valeur a la banque, sans que
personne soit tenue ni de les garder, ni de les recevoir dans le
commerce. Mais le sieur Lass prétend que l'utilité en sera telle que
tout le monde sera charmé d'avoir des billets de banque plutôt que de
l'argent, par la facilité qu'on aura à faire les payements en papier, et
par l'assurance d'en recevoir le payement toutes les fois que l'on
voudra. Il ajoute qu'il sera impossible qu'il puisse jamais y avoir plus
de billets que d'argent, parce qu'on ne fera de billets qu'au prorata de
l'argent, et que par ce moyen on évitera les frais de remise, le danger
des voitures, la multiplicité des commis, etc.

«\,Son Altesse Royale a jugé à propos d'entendre sur ce sujet des
négociants et banquiers qu'elle a fait entrer pour avoir leurs avis. Ces
négociants étant entrés au nombre de treize avec le sieur Lass, ils se
sont expliqués et ont proposé trois avis\,:

«\,Le premier, que l'établissement de la banque serait utile dès à
présent. --- Fénelon, Tourton, Guygner et Pion.

«\,Le second, que cet établissement pouvait être utile dans un autre
temps que celui-ci, mais qu'il serait nuisible dans la conjoncture
présente. --- Auisson.

«\,Le troisième, que cela devait être entièrement rejeté. --- Bernard,
Heusch, Moras, Le Couteux et quatre autres.

«\,Ces négociants retirés, Son Altesse Royale a pris les voix.

«\,Le Pelletier (de La Houssaye) a été d'avis d'établir la banque en
donnant quelque profit sur les billets pour les accréditer\,; mais il a
ajouté que la conjoncture n'était pas propre, et qu'il fallait attendre.

«\,Dodun\footnote{Charles-Gaspard Dodun, ancien président aux enquêtes
  du parlement de Paris, devint plus tard contrôleur général des
  finances.} croit la banque bonne sans donner un profit aux billets,
parce que cela chargerait l'État\,; mais qu'il faut attendre que la
confiance dans le gouvernement soit rétablie.

«\,M. de Saint-Contest ne croit pas que la banque puisse jamais avoir de
solidité dans le royaume, parce que l'autorité y règne toujours et que
le besoin y est souvent\,; ainsi il n'y aurait jamais de sûreté ni de
solidité\footnote{Cette opinion est à peu près celle que Saint-Simon
  lui-même a exprimée en appréciant la banque de Law\,: «\,Tout bon que
  pût être cet établissement en soi, il ne pouvait l'être que dans une
  république ou dans une monarchie telle qu'est l'Angleterre, dont les
  finances se gouvernent absolument par ceux-là seuls qui les
  fournissent et qui n'en fournissent qu'autant et que comme il leur
  plaît\,; mais dans un État léger, changeant, plus qu'absolu, tel
  qu'est la France, la solidité y manquait nécessairement, par
  conséquent la confiance.\,»}.

«\,M. Gilbert\footnote{Pierre Gilbert de Voisins avait été reçu maître
  des requêtes en 1711\,; il devint avocat général au parlement de Paris
  en 1718.} est persuadé que l'établissement de la banque est avantageux
en soi par la circulation et la multiplication des espèces\,; mais il ne
pense pas qu'on puisse présentement l'établir sans de grands
inconvénients, et il ajoute que l'incertitude du succès va à décréditer
le gouvernement, et qu'il serait fâcheux présentement de hasarder un
projet qui pourrait ne pas roussir.

«\,M. de Gaumont\footnote{Jean-Baptiste de Gaumont, intendant des
  finances.}, qu'on ne doit pas risquer cet établissement dans le
présent, et que cela influerait sur le gouvernement.

«\,M. Baudry\footnote{Gabriel Taschereau, seigneur de Baudry, devint
  dans la suite lieutenant de police.} croit cet établissement bon, mais
ne croit pas que, dans les circonstances présentes, le public puisse y
donner sa confiance\,; que c'est cependant ce qui doit l'accréditer,
sans quoi la banque tomberait d'elle-même. Ainsi il juge qu'il faut
attendre, pour ne pas donner comme un remède ce qui serait visiblement
un mal.

«\,M. d'Argenson\footnote{Marc-René Le Voyer de Paulmy, marquis
  d'Argenson, qui fut garde des sceaux et contrôleur général des
  finances. Saint-Simon parle souvent de ce personnage dans ses
  mémoires.} ne regarde la banque que comme la caisse des revenus du
roi, ne trouve aucun inconvénient à l'établir, on supposant que la
fidélité en sera toujours exacte, et croit qu'on doit tenter cette voie
innocente pour rattraper la confiance.

«\,M. d'Effiat\footnote{Antoine Ruzé, marquis d'Effiat, conseiller
  d'État et membre du conseil de régence.} en croit l'établissement
utile, mais non pas à présent, et que cela ferait présentement resserrer
l'argent encore plus qu'il ne l'est.

«\,M. le duc de Noailles\footnote{Adrien-Maurice, duc de Noailles, fut
  nommé maréchal de France en 1734. Voy. sur ce personnage l'article
  précédent.} est persuadé de l'utilité d'une banque, mais que les temps
ne conviennent pas, la défiance étant générale\,; que, de plus,
l'opposition des négociants\,; dont la confiance est essentielle pour
l'accréditement de la banque, la ferait échouer\,; qu'il faut la leur
faire désirer avant que de l'établir, et commencer par supprimer toutes
les dépenses inutiles pour payer les dettes de l'État\,; que rien ne
sera plus propre à regagner la confiance, par l'attention qu'on verra à
Son Altesse Royale pour le bien public, dont on est déjà très persuadé
par les premiers arrangements qu'elle a faits\,; et afin que l'on ne
soit pas plus longtemps dans l'incertitude, qu'on doit déclarer dès
aujourd'hui que la banque n'aura pas lieu.

«\,M. Fagon, de même avis\,; ajoute que le papier répandu dans le public
est ce qui cause le discrédit, et qu'en arrangeant le papier, on
regagnera la confiance.

«\,M. d'Aguesseau, que pour rétablir la confiance, Son Altesse Royale
n'a qu'à continuer à travailler comme elle le fait pour le bien public,
et de l'avis de M. de Noailles en tout.

«\,M. Le Blanc\footnote{Claude Le Blanc, conseiller d'État, devint, dans
  la suite, ministre de la guerre. Il est souvent question de Le Blanc
  dans les Mémoires de Saint-Simon.}, de l'avis de M. de Noailles en
tout.

«\,M. Rouillé\footnote{Hilaire Rouillé du Coudray, directeur des
  finances.}, que l'on doit prendre l'avis du public sur ce qui le
concerne, et que le public y est opposé\,; qu'il n'y a qu'à persévérer
dans le bien pour faire revenir la confiance.

«\,M. d'Ormesson\footnote{Henri-François-de-Paule Le Fèvre d'Ormesson,
  seigneur d'Amboille, intendant des finances.}, tout comme M. de
Noailles.

«\,M. Amelot\footnote{Michel Amelot, marquis de Gournay, conseiller
  d'État.}, que le public a parlé par la bouche des banquiers\,; de
l'avis de M. de Noailles.

«\,M. des Forts\footnote{Michel-Robert Le Pelletier des Forts fut, dans
  la suite, contrôleur général des finances.}, en tout de l'avis de M.
de Noailles.

«\,M. le maréchal de Villeroy\footnote{François de Neufville, duc de
  Villeroy, maréchal de France, chef du conseil des finances. Les
  \emph{Mémoires de Saint-Simon} abondent en détails sur le maréchal-duc
  de Villeroy.}, qu'on n'en pourrait tirer présentement aucun profit, et
que l'arrangement des rentes et des troupes, suivi de l'arrangement des
billets, ramènera la confiance. Au reste, de l'avis entier de M. le duc
de Noailles.

«\,Son Altesse Royale a dit qu'elle était entrée persuadée que la banque
devait avoir lieu\,; mais qu'après ce qu'elle venait d'entendre, elle
était de l'avis entier de M. le duc de Noailles, et qu'il fallait
annoncer à tout le monde, dès aujourd'hui, que la banque était
manquée\footnote{Voy. sur ce conseil de finances l'ouvrage de M.
  Levasseur, Recherches historiques sur le système de Law, p.~39 et
  suiv.}.\,»

\end{document}
